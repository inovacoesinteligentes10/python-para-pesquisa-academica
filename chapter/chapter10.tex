% =============================================================================
% CAPÍTULO 10: VISUALIZAÇÕES INTERATIVAS E DASHBOARDS
% =============================================================================

\chapter{Visualizações Interativas e Dashboards}

\lettrine{A}{comunicação} efetiva de resultados científicos vai além de gráficos estáticos. Na era digital, visualizações interativas e dashboards permitem que pesquisadores explorem dados dinamicamente, facilitando a descoberta de padrões e a comunicação de insights complexos. Este capítulo explora ferramentas modernas para criar interfaces interativas que transformam dados em experiências envolventes.

\section{Plotly e Visualizações Dinâmicas}

\subsection{Fundamentos do Plotly}

O Plotly é uma biblioteca poderosa para criação de visualizações interativas que podem ser incorporadas em notebooks, aplicações web ou relatórios. Oferece interatividade nativa com zoom, pan, hover e seleção de dados.

\begin{pythonbox}
\begin{lstlisting}[language=Python]
import plotly.express as px
import plotly.graph_objects as go
from plotly.subplots import make_subplots
import plotly.offline as pyo
import pandas as pd
import numpy as np
from datetime import datetime, timedelta
\end{lstlisting}
\end{pythonbox}

\begin{pythonbox}
\begin{lstlisting}[language=Python]
# Configuração para exibição em notebooks
pyo.init_notebook_mode(connected=True)

# Criando dados de exemplo para pesquisa em ciências sociais
np.random.seed(42)
n_participantes = 500

# Simulando dados de um estudo sobre bem-estar e produtividade
dados_estudo = pd.DataFrame({
    'participante_id': range(1, n_participantes + 1),
    'idade': np.random.normal(35, 12, n_participantes).astype(int),
    'bem_estar': np.random.normal(7, 1.5, n_participantes),
    'produtividade': np.random.normal(75, 15, n_participantes),
    'horas_trabalho': np.random.normal(8, 2, n_participantes),
    'satisfacao_trabalho': np.random.normal(6.5, 1.8, n_participantes),
    'area': np.random.choice(['Tecnologia', 'Educação', 'Saúde', 'Finanças'], n_participantes),
    'experiencia': np.random.randint(1, 20, n_participantes)
})

# Adicionando correlações realistas
dados_estudo['produtividade'] += dados_estudo['bem_estar'] * 3 + np.random.normal(0, 5, n_participantes)
dados_estudo['satisfacao_trabalho'] += dados_estudo['bem_estar'] * 0.3 + np.random.normal(0, 0.5, n_participantes)

# Limitando valores aos ranges apropriados
dados_estudo['bem_estar'] = np.clip(dados_estudo['bem_estar'], 1, 10)
dados_estudo['satisfacao_trabalho'] = np.clip(dados_estudo['satisfacao_trabalho'], 1, 10)
dados_estudo['produtividade'] = np.clip(dados_estudo['produtividade'], 30, 100)
dados_estudo['horas_trabalho'] = np.clip(dados_estudo['horas_trabalho'], 4, 12)

print("Dataset criado com sucesso!")
print(dados_estudo.head())
print(f"\nEstatísticas descritivas:")
print(dados_estudo.describe())
\end{lstlisting}
\end{pythonbox}

\subsection{Gráficos Interativos Básicos}

\begin{examplebox}
Vamos criar visualizações interativas que permitem explorar as relações entre bem-estar, produtividade e satisfação no trabalho:
\end{examplebox}

\begin{pythonbox}
\begin{lstlisting}[language=Python]
# Gráfico de dispersão interativo
fig_scatter = px.scatter(
    dados_estudo, 
    x='bem_estar', 
    y='produtividade',
    color='area',
    size='experiencia',
    hover_data=['idade', 'horas_trabalho', 'satisfacao_trabalho'],
    title='Relação entre Bem-estar e Produtividade por Área de Atuação',
    labels={
        'bem_estar': 'Índice de Bem-estar (1-10)',
        'produtividade': 'Índice de Produtividade (30-100)',
        'area': 'Área de Atuação',
        'experiencia': 'Anos de Experiência'
    }
)

# Personalizando o layout
fig_scatter.update_layout(
    width=800,
    height=600,
    font=dict(size=12),
    hovermode='closest'
)

# Adicionando linha de tendência
fig_scatter.add_scatter(
    x=dados_estudo['bem_estar'],
    y=np.poly1d(np.polyfit(dados_estudo['bem_estar'], dados_estudo['produtividade'], 1))(dados_estudo['bem_estar']),
    mode='lines',
    name='Tendência Linear',
    line=dict(color='red', dash='dash')
)

fig_scatter.show()
\end{lstlisting}
\end{pythonbox}

\begin{pythonbox}
\begin{lstlisting}[language=Python]
# Histograma interativo com seleção de variável
from ipywidgets import interact, Dropdown

def criar_histograma_interativo(variavel='bem_estar'):
    fig = px.histogram(
        dados_estudo, 
        x=variavel,
        color='area',
        nbins=20,
        title=f'Distribuição de {variavel.replace("_", " ").title()}',
        barmode='overlay',
        opacity=0.7
    )
    
    fig.update_layout(
        width=700,
        height=500,
        bargap=0.1
    )
    
    return fig.show()

# Lista de variáveis disponíveis
variaveis_numericas = ['bem_estar', 'produtividade', 'horas_trabalho', 
                      'satisfacao_trabalho', 'idade', 'experiencia']

print("Histograma Interativo - selecione uma variável:")
print("Disponível em ambiente Jupyter com widgets")
\end{lstlisting}
\end{pythonbox}

\subsection{Visualizações Multidimensionais}

\begin{pythonbox}
\begin{lstlisting}[language=Python]
# Matriz de correlação interativa
correlacoes = dados_estudo[variaveis_numericas].corr()

fig_heatmap = go.Figure(data=go.Heatmap(
    z=correlacoes.values,
    x=correlacoes.columns,
    y=correlacoes.columns,
    colorscale='RdBu',
    zmid=0,
    text=correlacoes.round(2).values,
    texttemplate="%{text}",
    textfont={"size": 10},
    hoverongaps=False
))

fig_heatmap.update_layout(
    title='Matriz de Correlação Interativa - Variáveis do Estudo',
    width=600,
    height=600,
    font=dict(size=12)
)

fig_heatmap.show()
\end{lstlisting}
\end{pythonbox}

\begin{pythonbox}
\begin{lstlisting}[language=Python]
# Gráfico de coordenadas paralelas
fig_parallel = go.Figure(data=go.Parcoords(
    line=dict(
        color=dados_estudo['bem_estar'],
        colorscale='Viridis',
        showscale=True,
        colorbar=dict(title="Bem-estar")
    ),
    dimensions=[
        dict(
            range=[dados_estudo['idade'].min(), dados_estudo['idade'].max()],
            label='Idade',
            values=dados_estudo['idade']
        ),
        dict(
            range=[dados_estudo['experiencia'].min(), dados_estudo['experiencia'].max()],
            label='Experiência',
            values=dados_estudo['experiencia']
        ),
        dict(
            range=[dados_estudo['horas_trabalho'].min(), dados_estudo['horas_trabalho'].max()],
            label='Horas Trabalho',
            values=dados_estudo['horas_trabalho']
        ),
        dict(
            range=[dados_estudo['bem_estar'].min(), dados_estudo['bem_estar'].max()],
            label='Bem-estar',
            values=dados_estudo['bem_estar']
        ),
        dict(
            range=[dados_estudo['produtividade'].min(), dados_estudo['produtividade'].max()],
            label='Produtividade',
            values=dados_estudo['produtividade']
        ),
        dict(
            range=[dados_estudo['satisfacao_trabalho'].min(), dados_estudo['satisfacao_trabalho'].max()],
            label='Satisfação',
            values=dados_estudo['satisfacao_trabalho']
        )
    ]
))

fig_parallel.update_layout(
    title='Coordenadas Paralelas - Análise Multivariada',
    font=dict(size=12)
)

fig_parallel.show()
\end{lstlisting}
\end{pythonbox}

\subsection{Animações e Séries Temporais}

\begin{pythonbox}
\begin{lstlisting}[language=Python]
# Criando dados temporais para demonstrar animações
datas = pd.date_range(start='2020-01-01', end='2023-12-31', freq='M')
np.random.seed(42)

# Simulando evolução temporal de indicadores por área
dados_temporais = []
for area in dados_estudo['area'].unique():
    for data in datas:
        # Simulando tendências e sazonalidade
        base_bem_estar = 6 + np.sin(data.month * 2 * np.pi / 12) * 0.5
        base_produtividade = 70 + np.sin(data.month * 2 * np.pi / 12) * 5
        
        # Adicionando variação por área
        if area == 'Tecnologia':
            base_bem_estar += 0.5
            base_produtividade += 5
        elif area == 'Saúde':
            base_bem_estar += 0.3
            base_produtividade += 3
        elif area == 'Educação':
            base_bem_estar += 0.1
            base_produtividade -= 2
        
        # Adicionando ruído e tendência temporal
        tendencia = (data.year - 2020) * 0.1
        ruido_bem_estar = np.random.normal(0, 0.3)
        ruido_produtividade = np.random.normal(0, 3)
        
        dados_temporais.append({
            'data': data,
            'area': area,
            'bem_estar_medio': base_bem_estar + tendencia + ruido_bem_estar,
            'produtividade_media': base_produtividade + tendencia * 2 + ruido_produtividade,
            'ano': data.year,
            'mes': data.month
        })

df_temporal = pd.DataFrame(dados_temporais)

print("Dados temporais criados:")
print(df_temporal.head())
\end{lstlisting}
\end{pythonbox}

\begin{pythonbox}
\begin{lstlisting}[language=Python]
# Gráfico animado mostrando evolução temporal
fig_animado = px.scatter(
    df_temporal,
    x='bem_estar_medio',
    y='produtividade_media',
    color='area',
    animation_frame='ano',
    animation_group='area',
    size_max=20,
    range_x=[5, 8],
    range_y=[60, 85],
    title='Evolução Temporal: Bem-estar vs Produtividade por Área',
    labels={
        'bem_estar_medio': 'Bem-estar Médio',
        'produtividade_media': 'Produtividade Média',
        'area': 'Área de Atuação'
    }
)

fig_animado.update_layout(
    width=800,
    height=600,
    font=dict(size=12)
)

fig_animado.show()
\end{lstlisting}
\end{pythonbox}

\section{Streamlit para Aplicações de Pesquisa}

\subsection{Fundamentos do Streamlit}

O Streamlit permite criar aplicações web interativas com Python puro, sem necessidade de conhecimento em HTML, CSS ou JavaScript. É ideal para criar dashboards de pesquisa e ferramentas analíticas.

\begin{pythonbox}
\begin{lstlisting}[language=Python]
import streamlit as st
import pandas as pd
import numpy as np
import plotly.express as px
import plotly.graph_objects as go
from datetime import datetime, timedelta

# Nota: Este código deve ser salvo em um arquivo .py separado
# e executado com: streamlit run nome_do_arquivo.py
\end{lstlisting}
\end{pythonbox}

\begin{examplebox}
Vamos criar uma aplicação Streamlit para análise interativa de dados de pesquisa. Salve o código a seguir como \texttt{app\_pesquisa.py}:
\end{examplebox}

\begin{pythonbox}
\begin{lstlisting}[language=Python]
# app_pesquisa.py - Aplicação Streamlit para Análise de Dados de Pesquisa

def main():
    st.set_page_config(
        page_title="Dashboard de Pesquisa",
        page_icon="��",
        layout="wide",
        initial_sidebar_state="expanded"
    )
    
    st.title("�� Dashboard Interativo de Análise de Pesquisa")
    st.markdown("---")
    
    # Sidebar para controles
    st.sidebar.header("�� Controles de Análise")
    
    # Upload de dados ou usar dados de exemplo
    opcao_dados = st.sidebar.radio(
        "Fonte de Dados:",
        ["Usar dados de exemplo", "Upload de arquivo CSV"]
    )
    
    if opcao_dados == "Upload de arquivo CSV":
        uploaded_file = st.sidebar.file_uploader(
            "Escolha um arquivo CSV",
            type="csv"
        )
        
        if uploaded_file is not None:
            dados = pd.read_csv(uploaded_file)
            st.sidebar.success("Arquivo carregado com sucesso!")
        else:
            st.warning("Por favor, faça upload de um arquivo CSV.")
            return
    else:
        # Usando dados de exemplo (mesmo dataset anterior)
        dados = carregar_dados_exemplo()
    
    # Exibindo informações básicas
    col1, col2, col3, col4 = st.columns(4)
    
    with col1:
        st.metric("Total de Participantes", len(dados))
    with col2:
        st.metric("Bem-estar Médio", f"{dados['bem_estar'].mean():.1f}")
    with col3:
        st.metric("Produtividade Média", f"{dados['produtividade'].mean():.1f}")
    with col4:
        st.metric("Satisfação Média", f"{dados['satisfacao_trabalho'].mean():.1f}")
    
    st.markdown("---")

if __name__ == "__main__":
    main()
\end{lstlisting}
\end{pythonbox}

\section{Jupyter Widgets e Interfaces Interativas}

\subsection{Fundamentos dos Jupyter Widgets}

Os Jupyter Widgets (ipywidgets) permitem criar interfaces interativas diretamente nos notebooks Jupyter, facilitando a exploração de dados e parâmetros de modelos.

\begin{pythonbox}
\begin{lstlisting}[language=Python]
import ipywidgets as widgets
from ipywidgets import interact, interactive, fixed, interact_manual
from IPython.display import display, clear_output
import matplotlib.pyplot as plt
import seaborn as sns
\end{lstlisting}
\end{pythonbox}

\begin{pythonbox}
\begin{lstlisting}[language=Python]
# Widget básico para exploração de dados
def explorar_distribuicoes(dados):
    """Cria widgets para explorar distribuições de variáveis"""
    
    # Definindo opções
    variaveis = ['bem_estar', 'produtividade', 'satisfacao_trabalho', 'idade', 'experiencia']
    areas = ['Todas'] + list(dados['area'].unique())
    tipos_grafico = ['Histograma', 'Box Plot', 'Density Plot']
    
    # Criando widgets
    widget_variavel = widgets.Dropdown(
        options=variaveis,
        value='bem_estar',
        description='Variável:'
    )
    
    widget_area = widgets.Dropdown(
        options=areas,
        value='Todas',
        description='Área:'
    )
    
    widget_tipo = widgets.Dropdown(
        options=tipos_grafico,
        value='Histograma',
        description='Tipo:'
    )
    
    widget_bins = widgets.IntSlider(
        value=20,
        min=5,
        max=50,
        step=5,
        description='Bins:'
    )
    
    # Interface interativa
    interface = interactive(
        plotar_distribuicao,
        variavel=widget_variavel,
        area=widget_area,
        tipo_grafico=widget_tipo,
        bins=widget_bins
    )
    
    return interface
\end{lstlisting}
\end{pythonbox}

\section{Dashboards com Dash}

\subsection{Introdução ao Dash}

O Dash é um framework Python para criar aplicações web analíticas interativas. É especialmente útil para dashboards corporativos e painéis de monitoramento científico.

\begin{pythonbox}
\begin{lstlisting}[language=Python]
import dash
from dash import dcc, html, Input, Output, State, callback_context
import dash_bootstrap_components as dbc
import plotly.express as px
import plotly.graph_objects as go
from datetime import datetime, timedelta
import json
\end{lstlisting}
\end{pythonbox}

\begin{examplebox}
Vamos criar um dashboard completo para monitoramento de indicadores de pesquisa. Salve o código a seguir como \texttt{dashboard\_pesquisa.py}:
\end{examplebox}

\begin{pythonbox}
\begin{lstlisting}[language=Python]
# dashboard_pesquisa.py - Dashboard completo com Dash

# Configuração da aplicação
app = dash.Dash(__name__, external_stylesheets=[dbc.themes.BOOTSTRAP])
app.title = "Dashboard de Pesquisa Acadêmica"

# Carregando dados (mesmo dataset anterior)
def carregar_dados():
    np.random.seed(42)
    n_participantes = 1000
    
    dados = pd.DataFrame({
        'participante_id': range(1, n_participantes + 1),
        'idade': np.random.normal(35, 12, n_participantes).astype(int),
        'bem_estar': np.random.normal(7, 1.5, n_participantes),
        'produtividade': np.random.normal(75, 15, n_participantes),
        'horas_trabalho': np.random.normal(8, 2, n_participantes),
        'satisfacao_trabalho': np.random.normal(6.5, 1.8, n_participantes),
        'area': np.random.choice(['Tecnologia', 'Educação', 'Saúde', 'Finanças'], n_participantes),
        'experiencia': np.random.randint(1, 20, n_participantes),
        'data_coleta': pd.date_range(start='2023-01-01', periods=n_participantes, freq='H')
    })
    
    # Correlações realistas
    dados['produtividade'] += dados['bem_estar'] * 3 + np.random.normal(0, 5, n_participantes)
    dados['satisfacao_trabalho'] += dados['bem_estar'] * 0.3 + np.random.normal(0, 0.5, n_participantes)
    
    # Limitando valores
    dados['bem_estar'] = np.clip(dados['bem_estar'], 1, 10)
    dados['satisfacao_trabalho'] = np.clip(dados['satisfacao_trabalho'], 1, 10)
    dados['produtividade'] = np.clip(dados['produtividade'], 30, 100)
    dados['horas_trabalho'] = np.clip(dados['horas_trabalho'], 4, 12)
    
    return dados

dados_globais = carregar_dados()
\end{lstlisting}
\end{pythonbox}

\begin{pythonbox}
\begin{lstlisting}[language=Python]
# Layout básico do dashboard
app.layout = html.Div([
    html.H1("Dashboard de Pesquisa Acadêmica"),
    dcc.Graph(id="exemplo-grafico"),
    html.P("Dashboard funcional com Dash")
])

# Executando a aplicação
if __name__ == '__main__':
    app.run_server(debug=True, port=8050)
\end{lstlisting}
\end{pythonbox}

\section{Considerações Finais do Capítulo}

As visualizações interativas e dashboards representam uma evolução natural na comunicação científica. Elas permitem que pesquisadores e stakeholders explorem dados de forma intuitiva, descobrindo padrões que poderiam passar despercebidos em análises estáticas.

### Principais Benefícios:

\begin{itemize}
\item \textbf{Engajamento}: Interfaces interativas mantêm o usuário engajado na exploração dos dados
\item \textbf{Flexibilidade}: Permitem análises sob diferentes perspectivas sem necessidade de novo código
\item \textbf{Acessibilidade}: Tornam análises complexas acessíveis a não-programadores
\item \textbf{Colaboração}: Facilitam o compartilhamento e discussão de resultados
\item \textbf{Iteração}: Permitem refinamento rápido de hipóteses e análises
\end{itemize}

### Escolha da Ferramenta:

\begin{itemize}
\item \textbf{Plotly}: Ideal para gráficos interativos em notebooks e relatórios
\item \textbf{Streamlit}: Perfeito para protótipos rápidos e aplicações simples
\item \textbf{Jupyter Widgets}: Excelente para exploração paramétrica em notebooks
\item \textbf{Dash}: Melhor opção para dashboards corporativos e aplicações robustas
\end{itemize}

\begin{warningbox}
\textbf{Boas Práticas:} Mantenha a simplicidade, teste a usabilidade com usuários reais, documente as funcionalidades, implemente controle de acesso quando necessário, e sempre valide os dados antes da visualização.
\end{warningbox}

\begin{researchbox}
\textbf{Aplicações Práticas:} Dashboards de monitoramento de experimentos, interfaces para coleta de dados de pesquisa, painéis de acompanhamento de indicadores acadêmicos, ferramentas de análise colaborativa, e sistemas de visualização para apresentações interativas.
\end{researchbox}