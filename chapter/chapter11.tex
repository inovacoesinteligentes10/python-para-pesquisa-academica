% =============================================================================
% CAPÍTULO 11: COMUNICAÇÃO CIENTÍFICA COM PYTHON
% =============================================================================

\chapter{Comunicação Científica com Python}

\lettrine{A}{comunicação} efetiva de resultados científicos é tão importante quanto a própria pesquisa. Python oferece ferramentas poderosas para automatizar a geração de relatórios, criar apresentações dinâmicas e produzir visualizações de qualidade para publicação. Este capítulo explora como transformar análises Python em comunicação científica profissional e impactante.

% =============================================================================
\section{Relatórios Automatizados}
% =============================================================================

\subsection{Fundamentos da Geração Automatizada}

A automação de relatórios permite criar documentos consistentes, reproduzíveis e atualizáveis dinamicamente. Isso é especialmente valioso em pesquisas longitudinais ou quando é necessário gerar relatórios periódicos.

\begin{pythonbox}
\begin{lstlisting}[language=Python]
import pandas as pd
import numpy as np
import matplotlib.pyplot as plt
import seaborn as sns
from datetime import datetime, timedelta
import jinja2
from reportlab.lib import colors
from reportlab.lib.pagesizes import letter, A4
from reportlab.platypus import SimpleDocTemplate, Table, TableStyle, Paragraph, Spacer
from reportlab.lib.styles import getSampleStyleSheet, ParagraphStyle
from reportlab.lib.units import inch
import warnings
warnings.filterwarnings('ignore')
\end{lstlisting}
\end{pythonbox}

\begin{pythonbox}
\begin{lstlisting}[language=Python]
# Configuração de estilo para gráficos
plt.style.use('seaborn-v0_8-whitegrid')
sns.set_palette("husl")

# Criando dados de exemplo para relatório de pesquisa
np.random.seed(42)
n_participantes = 200

# Simulando dados de um estudo longitudinal sobre aprendizado
dados_estudo = pd.DataFrame({
    'participante_id': range(1, n_participantes + 1),
    'grupo': np.random.choice(['Controle', 'Experimental'], n_participantes),
    'idade': np.random.normal(22, 3, n_participantes).astype(int),
    'genero': np.random.choice(['M', 'F', 'NB'], n_participantes, p=[0.45, 0.45, 0.1]),
    'pre_teste': np.random.normal(65, 15, n_participantes),
    'pos_teste': np.random.normal(70, 18, n_participantes),
    'satisfacao': np.random.normal(7.5, 1.8, n_participantes),
    'tempo_estudo': np.random.normal(4.5, 2, n_participantes),
    'data_coleta': pd.date_range(start='2024-01-15', periods=n_participantes, freq='D')
})

# Adicionando efeito realista do grupo experimental
mask_experimental = dados_estudo['grupo'] == 'Experimental'
dados_estudo.loc[mask_experimental, 'pos_teste'] += np.random.normal(8, 3, mask_experimental.sum())
dados_estudo.loc[mask_experimental, 'satisfacao'] += np.random.normal(0.8, 0.4, mask_experimental.sum())

# Limitando valores aos ranges apropriados
dados_estudo['pre_teste'] = np.clip(dados_estudo['pre_teste'], 0, 100)
dados_estudo['pos_teste'] = np.clip(dados_estudo['pos_teste'], 0, 100)
dados_estudo['satisfacao'] = np.clip(dados_estudo['satisfacao'], 1, 10)
dados_estudo['tempo_estudo'] = np.clip(dados_estudo['tempo_estudo'], 1, 10)

# Calculando variáveis derivadas
dados_estudo['ganho_aprendizado'] = dados_estudo['pos_teste'] - dados_estudo['pre_teste']
dados_estudo['ganho_percentual'] = (dados_estudo['ganho_aprendizado'] / dados_estudo['pre_teste']) * 100

print("Dataset para relatório criado:")
print(dados_estudo.head())
print(f"\nEstatísticas básicas:")
print(dados_estudo.groupby('grupo')[['pre_teste', 'pos_teste', 'ganho_aprendizado']].mean())
\end{lstlisting}
\end{pythonbox}

\subsection{Classe para Geração de Relatórios}

\begin{examplebox}
Vamos criar uma classe abrangente para automatizar a geração de relatórios científicos:
\end{examplebox}

\begin{pythonbox}
\begin{lstlisting}[language=Python]
class GeradorRelatorio:
    """Classe para automatizar a geração de relatórios científicos"""
    
    def __init__(self, dados, titulo="Relatório de Pesquisa", autor="Pesquisador"):
        self.dados = dados
        self.titulo = titulo
        self.autor = autor
        self.data_relatorio = datetime.now()
        self.figuras = []
        self.estatisticas = {}
        
    def calcular_estatisticas_descritivas(self):
        """Calcula estatísticas descritivas básicas"""
        stats = {}
        
        # Estatísticas por grupo
        for grupo in self.dados['grupo'].unique():
            dados_grupo = self.dados[self.dados['grupo'] == grupo]
            stats[grupo] = {
                'n': len(dados_grupo),
                'pre_teste_media': dados_grupo['pre_teste'].mean(),
                'pre_teste_dp': dados_grupo['pre_teste'].std(),
                'pos_teste_media': dados_grupo['pos_teste'].mean(),
                'pos_teste_dp': dados_grupo['pos_teste'].std(),
                'ganho_media': dados_grupo['ganho_aprendizado'].mean(),
                'ganho_dp': dados_grupo['ganho_aprendizado'].std(),
                'satisfacao_media': dados_grupo['satisfacao'].mean(),
                'satisfacao_dp': dados_grupo['satisfacao'].std()
            }
        
        # Estatísticas gerais
        stats['geral'] = {
            'total_participantes': len(self.dados),
            'idade_media': self.dados['idade'].mean(),
            'idade_dp': self.dados['idade'].std(),
            'periodo_coleta': f"{self.dados['data_coleta'].min().strftime('%d/%m/%Y')} a {self.dados['data_coleta'].max().strftime('%d/%m/%Y')}"
        }
        
        self.estatisticas = stats
        return stats


    \end{lstlisting}
\end{pythonbox}

\begin{pythonbox}
\begin{lstlisting}[language=Python]
    def realizar_testes_estatisticos(self):
        """Realiza testes estatísticos principais"""
        from scipy import stats as scipy_stats
        
        # Separando grupos
        controle = self.dados[self.dados['grupo'] == 'Controle']
        experimental = self.dados[self.dados['grupo'] == 'Experimental']
        
        # Teste t para diferenças entre grupos no pós-teste
        t_stat, p_valor = scipy_stats.ttest_ind(
            controle['pos_teste'], 
            experimental['pos_teste']
        )
        
        # Teste t pareado para ganho de aprendizado no grupo experimental
        t_stat_pareado, p_valor_pareado = scipy_stats.ttest_rel(
            experimental['pre_teste'], 
            experimental['pos_teste']
        )
        
        # Tamanho do efeito (Cohen's d)
        pooled_std = np.sqrt(((len(controle)-1)*controle['pos_teste'].std()**2 + 
                             (len(experimental)-1)*experimental['pos_teste'].std()**2) / 
                            (len(controle)+len(experimental)-2))
        cohens_d = (experimental['pos_teste'].mean() - controle['pos_teste'].mean()) / pooled_std
        
        testes = {
            'teste_independente': {
                't_statistic': t_stat,
                'p_valor': p_valor,
                'cohens_d': cohens_d,
                'significativo': p_valor < 0.05
            },
            'teste_pareado_experimental': {
                't_statistic': t_stat_pareado,
                'p_valor': p_valor_pareado,
                'significativo': p_valor_pareado < 0.05
            }
        }
        
        self.testes_estatisticos = testes
        return testes
\end{lstlisting}
\end{pythonbox}

\begin{pythonbox}
\begin{lstlisting}[language=Python]
    def gerar_graficos(self, salvar=True):
        """Gera gráficos para o relatório"""
        
        # Configuração de figura
        fig, axes = plt.subplots(2, 2, figsize=(15, 12))
        
        # Gráfico 1: Comparação pré vs pós por grupo
        grupos = self.dados['grupo'].unique()
        x = np.arange(len(grupos))
        width = 0.35
        
        pre_medias = [self.dados[self.dados['grupo'] == g]['pre_teste'].mean() for g in grupos]
        pos_medias = [self.dados[self.dados['grupo'] == g]['pos_teste'].mean() for g in grupos]
        pre_erros = [self.dados[self.dados['grupo'] == g]['pre_teste'].std() for g in grupos]
        pos_erros = [self.dados[self.dados['grupo'] == g]['pos_teste'].std() for g in grupos]
        
        axes[0, 0].bar(x - width/2, pre_medias, width, label='Pré-teste', 
                       yerr=pre_erros, capsize=5, alpha=0.8)
        axes[0, 0].bar(x + width/2, pos_medias, width, label='Pós-teste', 
                       yerr=pos_erros, capsize=5, alpha=0.8)
        axes[0, 0].set_xlabel('Grupo')
        axes[0, 0].set_ylabel('Pontuação')
        axes[0, 0].set_title('Comparação Pré vs Pós-teste por Grupo')
        axes[0, 0].set_xticks(x)
        axes[0, 0].set_xticklabels(grupos)
        axes[0, 0].legend()
        axes[0, 0].grid(True, alpha=0.3)
        
        # Gráfico 2: Distribuição dos ganhos de aprendizado
        self.dados.boxplot(column='ganho_aprendizado', by='grupo', ax=axes[0, 1])
        axes[0, 1].set_title('Distribuição dos Ganhos de Aprendizado')
        axes[0, 1].set_xlabel('Grupo')
        axes[0, 1].set_ylabel('Ganho (Pós - Pré)')
        
        # Gráfico 3: Correlação entre satisfação e ganho
        cores = {'Controle': 'blue', 'Experimental': 'red'}
        for grupo in grupos:
            dados_grupo = self.dados[self.dados['grupo'] == grupo]
            axes[1, 0].scatter(dados_grupo['satisfacao'], dados_grupo['ganho_aprendizado'],
                              c=cores[grupo], alpha=0.6, label=grupo)

   \end{lstlisting}
\end{pythonbox}

\begin{pythonbox}
\begin{lstlisting}[language=Python]               
        axes[1, 0].set_xlabel('Satisfação')
        axes[1, 0].set_ylabel('Ganho de Aprendizado')
        axes[1, 0].set_title('Satisfação vs Ganho de Aprendizado')
        axes[1, 0].legend()
        axes[1, 0].grid(True, alpha=0.3)
        
        # Gráfico 4: Evolução temporal (últimos 30 dias)
        dados_recentes = self.dados.tail(30)
        dados_temporal = dados_recentes.groupby(['data_coleta', 'grupo'])['ganho_aprendizado'].mean().unstack()
        
        if 'Controle' in dados_temporal.columns:
            axes[1, 1].plot(dados_temporal.index, dados_temporal['Controle'], 
                           marker='o', label='Controle', linewidth=2)
        if 'Experimental' in dados_temporal.columns:
            axes[1, 1].plot(dados_temporal.index, dados_temporal['Experimental'], 
                           marker='s', label='Experimental', linewidth=2)

  
        axes[1, 1].set_xlabel('Data')
        axes[1, 1].set_ylabel('Ganho Médio')
        axes[1, 1].set_title('Evolução Temporal dos Ganhos (Últimos 30 dias)')
        axes[1, 1].legend()
        axes[1, 1].grid(True, alpha=0.3)
        axes[1, 1].tick_params(axis='x', rotation=45)
        
        plt.tight_layout()
        
        if salvar:
            nome_arquivo = f"graficos_relatorio_{self.data_relatorio.strftime('%Y%m%d_%H%M%S')}.png"
            plt.savefig(nome_arquivo, dpi=300, bbox_inches='tight')
            self.figuras.append(nome_arquivo)
            print(f"Gráficos salvos em: {nome_arquivo}")
        
        plt.show()
        return fig
\end{lstlisting}
\end{pythonbox}

\begin{pythonbox}
\begin{lstlisting}[language=Python]
    def gerar_relatorio_html(self):
        """Gera relatório em HTML usando Jinja2"""
        
        # Template HTML
        template_html = """
        <!DOCTYPE html>
        <html>
        <head>
            <meta charset="UTF-8">
            <title>{{ titulo }}</title>
            <style>
                body { font-family: Arial, sans-serif; margin: 40px; line-height: 1.6; }
                .header { text-align: center; border-bottom: 2px solid #333; padding-bottom: 20px; }
                .section { margin: 30px 0; }
                .stats-table { border-collapse: collapse; width: 100%; margin: 20px 0; }
                .stats-table th, .stats-table td { border: 1px solid #ddd; padding: 12px; text-align: left; }
                .stats-table th { background-color: #f2f2f2; font-weight: bold; }
                .highlight { background-color: #ffffcc; }
                .significant { color: #d63031; font-weight: bold; }
                .footer { margin-top: 50px; text-align: center; font-size: 0.9em; color: #666; }
            </style>
        </head>
        <body>
            <div class="header">
                <h1>{{ titulo }}</h1>
                <p><strong>Autor:</strong> {{ autor }}</p>
                <p><strong>Data:</strong> {{ data_relatorio }}</p>
                <p><strong>Período de Coleta:</strong> {{ periodo_coleta }}</p>
            </div>
            
            <div class="section">
                <h2>1. Resumo Executivo</h2>
                <p>Este relatório apresenta os resultados de um estudo experimental sobre aprendizado, 
                envolvendo {{ total_participantes }} participantes divididos entre grupo controle e experimental. 
                O objetivo foi avaliar a eficácia de uma nova metodologia de ensino.</p>
   \end{lstlisting}
\end{pythonbox}

\begin{pythonbox}
\begin{lstlisting}[language=Python]                       
                <h3>Principais Achados:</h3>
                <ul>
                    <li>Diferença significativa entre grupos no pós-teste: <span class="{{ 'significant' if teste_significativo else '' }}">{{ 'SIM' if teste_significativo else 'NÃO' }} (p = {{ p_valor | round(4) }})</span></li>
                    <li>Tamanho do efeito (Cohen's d): {{ cohens_d | round(3) }}</li>
                    <li>Ganho médio grupo experimental: {{ ganho_experimental | round(2) }} pontos</li>
                    <li>Ganho médio grupo controle: {{ ganho_controle | round(2) }} pontos</li>
                </ul>
            </div>
            
            <div class="section">
                <h2>2. Características da Amostra</h2>
                <table class="stats-table">
                    <tr>
                        <th>Característica</th>
                        <th>Valor</th>
                    </tr>
                    <tr>
                        <td>Total de Participantes</td>
                        <td>{{ total_participantes }}</td>
                    </tr>
                    <tr>
                        <td>Idade Média (DP)</td>
                        <td>{{ idade_media | round(1) }} ({{ idade_dp | round(1) }})</td>
                    </tr>
                    <tr>
                        <td>Grupo Controle</td>
                        <td>{{ n_controle }} participantes</td>
                    </tr>
                    <tr>
                        <td>Grupo Experimental</td>
                        <td>{{ n_experimental }} participantes</td>
                    </tr>
                </table>
            </div>
            
            <div class="section">
                <h2>3. Resultados por Grupo</h2>
                <table class="stats-table">
                    <tr>
                        <th>Medida</th>
                        <th>Controle</th>
                        <th>Experimental</th>
                    </tr>
   \end{lstlisting}
\end{pythonbox}

\begin{pythonbox}
\begin{lstlisting}[language=Python]                           
                    <tr>
                        <td>Pré-teste (M ± DP)</td>
                        <td>{{ pre_controle | round(2) }} ± {{ pre_controle_dp | round(2) }}</td>
                        <td>{{ pre_experimental | round(2) }} ± {{ pre_experimental_dp | round(2) }}</td>
                    </tr>
                    <tr class="highlight">
                        <td>Pós-teste (M ± DP)</td>
                        <td>{{ pos_controle | round(2) }} ± {{ pos_controle_dp | round(2) }}</td>
                        <td>{{ pos_experimental | round(2) }} ± {{ pos_experimental_dp | round(2) }}</td>
                    </tr>
                    <tr class="highlight">
                        <td>Ganho de Aprendizado</td>
                        <td>{{ ganho_controle | round(2) }} ± {{ ganho_controle_dp | round(2) }}</td>
                        <td>{{ ganho_experimental | round(2) }} ± {{ ganho_experimental_dp | round(2) }}</td>
                    </tr>
                    <tr>
                        <td>Satisfação</td>
                        <td>{{ sat_controle | round(2) }} ± {{ sat_controle_dp | round(2) }}</td>
                        <td>{{ sat_experimental | round(2) }} ± {{ sat_experimental_dp | round(2) }}</td>
                    </tr>
                </table>
            </div>
            
            <div class="section">
                <h2>4. Análises Estatísticas</h2>
                <h3>Teste t para Amostras Independentes (Pós-teste)</h3>
                <ul>
                    <li>t({{ graus_liberdade }}) = {{ t_statistic | round(3) }}</li>
                    <li>p = {{ p_valor | round(4) }}</li>
                    <li>Cohen's d = {{ cohens_d | round(3) }}</li>
                    <li><strong>Interpretação:</strong> {{ interpretacao_resultado }}</li>
                </ul>
   \end{lstlisting}
\end{pythonbox}

\begin{pythonbox}
\begin{lstlisting}[language=Python]                       
                <h3>Teste t Pareado (Grupo Experimental: Pré vs Pós)</h3>
                <ul>
                    <li>t = {{ t_pareado | round(3) }}</li>
                    <li>p = {{ p_pareado | round(4) }}</li>
                    <li><strong>Resultado:</strong> {{ resultado_pareado }}</li>
                </ul>
            </div>
            
            <div class="footer">
                <p>Relatório gerado automaticamente em {{ data_relatorio }}</p>
                <p>Sistema de Análise de Dados - Python</p>
            </div>
        </body>
        </html>
        """
        
        # Preparando dados para o template
        stats = self.estatisticas
        testes = self.testes_estatisticos
        
        dados_template = {
            'titulo': self.titulo,
            'autor': self.autor,
            'data_relatorio': self.data_relatorio.strftime('%d/%m/%Y %H:%M'),
            'periodo_coleta': stats['geral']['periodo_coleta'],
            'total_participantes': stats['geral']['total_participantes'],
            'idade_media': stats['geral']['idade_media'],
            'idade_dp': stats['geral']['idade_dp'],
            
            # Dados por grupo
            'n_controle': stats['Controle']['n'],
            'n_experimental': stats['Experimental']['n'],
            'pre_controle': stats['Controle']['pre_teste_media'],
            'pre_controle_dp': stats['Controle']['pre_teste_dp'],
            'pre_experimental': stats['Experimental']['pre_teste_media'],
            'pre_experimental_dp': stats['Experimental']['pre_teste_dp'],
            'pos_controle': stats['Controle']['pos_teste_media'],
            'pos_controle_dp': stats['Controle']['pos_teste_dp'],

   \end{lstlisting}
\end{pythonbox}

\begin{pythonbox}
\begin{lstlisting}[language=Python]                   
            'pos_experimental': stats['Experimental']['pos_teste_media'],
            'pos_experimental_dp': stats['Experimental']['pos_teste_dp'],
            'ganho_controle': stats['Controle']['ganho_media'],
            'ganho_controle_dp': stats['Controle']['ganho_dp'],
            'ganho_experimental': stats['Experimental']['ganho_media'],
            'ganho_experimental_dp': stats['Experimental']['ganho_dp'],
            'sat_controle': stats['Controle']['satisfacao_media'],
            'sat_controle_dp': stats['Controle']['satisfacao_dp'],
            'sat_experimental': stats['Experimental']['satisfacao_media'],
            'sat_experimental_dp': stats['Experimental']['satisfacao_dp'],
            
            # Testes estatísticos
            't_statistic': testes['teste_independente']['t_statistic'],
            'p_valor': testes['teste_independente']['p_valor'],
            'cohens_d': testes['teste_independente']['cohens_d'],
            'teste_significativo': testes['teste_independente']['significativo'],
            'graus_liberdade': len(self.dados) - 2,
            't_pareado': testes['teste_pareado_experimental']['t_statistic'],
            'p_pareado': testes['teste_pareado_experimental']['p_valor'],
            
            # Interpretações
            'interpretacao_resultado': 'Diferença estatisticamente significativa entre os grupos' if testes['teste_independente']['significativo'] else 'Não há diferença estatisticamente significativa entre os grupos',
            'resultado_pareado': 'Melhora significativa no grupo experimental' if testes['teste_pareado_experimental']['significativo'] else 'Não há melhora significativa no grupo experimental'
        }
        
        # Renderizando template
        template = jinja2.Template(template_html)
        html_content = template.render(**dados_template)
        
        # Salvando arquivo
        nome_arquivo = f"relatorio_{self.data_relatorio.strftime('%Y%m%d_%H%M%S')}.html"
        with open(nome_arquivo, 'w', encoding='utf-8') as f:
            f.write(html_content)
        
        print(f"Relatório HTML gerado: {nome_arquivo}")
        return nome_arquivo, html_content
\end{lstlisting}
\end{pythonbox}

\begin{pythonbox}
\begin{lstlisting}[language=Python]
    def gerar_relatorio_completo(self):
        """Gera relatório completo com todas as análises"""
        print("Gerando relatório completo...")
        print("=" * 50)
        
        # Passo 1: Calcular estatísticas
        print("1. Calculando estatísticas descritivas...")
        self.calcular_estatisticas_descritivas()
        
        # Passo 2: Realizar testes
        print("2. Realizando testes estatísticos...")
        self.realizar_testes_estatisticos()
        
        # Passo 3: Gerar gráficos
        print("3. Gerando gráficos...")
        self.gerar_graficos()
        
        # Passo 4: Gerar relatório HTML
        print("4. Gerando relatório HTML...")
        arquivo_html, _ = self.gerar_relatorio_html()
        
        print("=" * 50)
        print("Relatório completo gerado com sucesso!")
        print(f"Arquivo: {arquivo_html}")
        
        return arquivo_html

# Exemplo de uso
relatorio = GeradorRelatorio(
    dados_estudo, 
    titulo="Eficácia de Nova Metodologia de Ensino",
    autor="Dr. João Silva"
)

arquivo_relatorio = relatorio.gerar_relatorio_completo()
\end{lstlisting}
\end{pythonbox}

\subsection{Relatórios em PDF com ReportLab}

\begin{pythonbox}
\begin{lstlisting}[language=Python]
def gerar_relatorio_pdf(dados, titulo="Relatório de Pesquisa"):
    """Gera relatório profissional em PDF"""
    from reportlab.platypus import SimpleDocTemplate, Paragraph, Spacer, Table, TableStyle, Image
    from reportlab.lib.styles import getSampleStyleSheet, ParagraphStyle
    from reportlab.lib import colors
    from reportlab.lib.units import inch
    from reportlab.lib.pagesizes import letter
    
    # Configuração do documento
    nome_arquivo = f"relatorio_pdf_{datetime.now().strftime('%Y%m%d_%H%M%S')}.pdf"
    doc = SimpleDocTemplate(nome_arquivo, pagesize=letter,
                          rightMargin=72, leftMargin=72,
                          topMargin=72, bottomMargin=18)
    
    # Estilos
    styles = getSampleStyleSheet()
    titulo_style = ParagraphStyle(
        'CustomTitle',
        parent=styles['Heading1'],
        fontSize=18,
        spaceAfter=30,
        alignment=1  # Centralizado
    )
    
    # Conteúdo do documento
    story = []
    
    # Título
    story.append(Paragraph(titulo, titulo_style))
    story.append(Spacer(1, 12))
    
    # Informações gerais
    info_geral = f"""
    <b>Data do Relatório:</b> {datetime.now().strftime('%d/%m/%Y')}<br/>
    <b>Total de Participantes:</b> {len(dados)}<br/>
    <b>Período de Coleta:</b> {dados['data_coleta'].min().strftime('%d/%m/%Y')} a {dados['data_coleta'].max().strftime('%d/%m/%Y')}
    """
    story.append(Paragraph(info_geral, styles['Normal']))
    story.append(Spacer(1, 12))
    
    # Estatísticas por grupo
    story.append(Paragraph("<b>Estatísticas por Grupo</b>", styles['Heading2']))
       \end{lstlisting}
\end{pythonbox}

\begin{pythonbox}
\begin{lstlisting}[language=Python]       
    # Criando tabela de estatísticas
    dados_tabela = [['Medida', 'Controle', 'Experimental']]
    
    for grupo in ['Controle', 'Experimental']:
        dados_grupo = dados[dados['grupo'] == grupo]
        if grupo == 'Controle':
            linha_pre = ['Pré-teste (M ± DP)', 
                        f"{dados_grupo['pre_teste'].mean():.2f} ± {dados_grupo['pre_teste'].std():.2f}",
                        '']
            linha_pos = ['Pós-teste (M ± DP)', 
                        f"{dados_grupo['pos_teste'].mean():.2f} ± {dados_grupo['pos_teste'].std():.2f}",
                        '']
        else:
            linha_pre[2] = f"{dados_grupo['pre_teste'].mean():.2f} ± {dados_grupo['pre_teste'].std():.2f}"
            linha_pos[2] = f"{dados_grupo['pos_teste'].mean():.2f} ± {dados_grupo['pos_teste'].std():.2f}"
    
    dados_tabela.extend([linha_pre, linha_pos])
    
    # Formatando tabela
    tabela = Table(dados_tabela)
    tabela.setStyle(TableStyle([
        ('BACKGROUND', (0, 0), (-1, 0), colors.grey),
        ('TEXTCOLOR', (0, 0), (-1, 0), colors.whitesmoke),
        ('ALIGN', (0, 0), (-1, -1), 'CENTER'),
        ('FONTNAME', (0, 0), (-1, 0), 'Helvetica-Bold'),
        ('FONTSIZE', (0, 0), (-1, 0), 14),
        ('BOTTOMPADDING', (0, 0), (-1, 0), 12),
        ('BACKGROUND', (0, 1), (-1, -1), colors.beige),
        ('GRID', (0, 0), (-1, -1), 1, colors.black)
    ]))
    
    story.append(tabela)
    story.append(Spacer(1, 12))
    
    # Conclusões
    story.append(Paragraph("<b>Conclusões</b>", styles['Heading2']))
    conclusoes = """
    Com base nas análises realizadas, observamos diferenças notáveis entre os grupos controle e experimental. 
    O grupo experimental apresentou maior ganho de aprendizado em comparação ao grupo controle, 
    sugerindo a eficácia da nova metodologia implementada.
    """
    story.append(Paragraph(conclusoes, styles['Normal']))
       \end{lstlisting}
\end{pythonbox}

\begin{pythonbox}
\begin{lstlisting}[language=Python]       
    # Gerando PDF
    doc.build(story)
    print(f"Relatório PDF gerado: {nome_arquivo}")
    return nome_arquivo

# Exemplo de uso
arquivo_pdf = gerar_relatorio_pdf(dados_estudo, "Relatório de Eficácia Educacional")
\end{lstlisting}
\end{pythonbox}

% =============================================================================
\section{Apresentações com Jupyter}
% =============================================================================

\subsection{Jupyter como Ferramenta de Apresentação}

O Jupyter oferece extensões poderosas para criar apresentações científicas interativas, permitindo combinar código, resultados e narrativa em um formato dinâmico.

\begin{pythonbox}
\begin{lstlisting}[language=Python]
# Instalação das dependências para apresentações
# !pip install RISE
# !pip install jupyter_contrib_nbextensions
# !jupyter contrib nbextension install --user
# !jupyter nbextension enable rise --py

import matplotlib.pyplot as plt
import seaborn as sns
import plotly.express as px
import plotly.graph_objects as go
from IPython.display import HTML, display, Markdown
import warnings
warnings.filterwarnings('ignore')
\end{lstlisting}
\end{pythonbox}

\begin{pythonbox}
\begin{lstlisting}[language=Python]
# Configuração para apresentações
plt.rcParams['figure.figsize'] = (12, 8)
plt.rcParams['font.size'] = 14
sns.set_style("whitegrid")
sns.set_palette("husl")

# Função para criar slides com markdown
def criar_slide_titulo(titulo, subtitulo="", autor=""):
    """Cria slide de título para apresentação"""
    html_content = f"""
    <div style="text-align: center; padding: 100px 0;">
        <h1 style="font-size: 48px; color: #2c3e50; margin-bottom: 30px;">{titulo}</h1>
        {f'<h2 style="font-size: 32px; color: #7f8c8d; margin-bottom: 20px;">{subtitulo}</h2>' if subtitulo else ''}
        {f'<h3 style="font-size: 24px; color: #95a5a6;">{autor}</h3>' if autor else ''}
        <p style="font-size: 18px; color: #bdc3c7; margin-top: 40px;">{datetime.now().strftime('%d de %B de %Y')}</p>
    </div>
    """
    return HTML(html_content)

# Função para slides de conteúdo
def criar_slide_conteudo(titulo, conteudo, layout="single"):
    """Cria slide de conteúdo"""
    if layout == "single":
        html_content = f"""
        <div style="padding: 20px;">
            <h2 style="color: #2c3e50; border-bottom: 3px solid #3498db; padding-bottom: 10px; margin-bottom: 30px;">{titulo}</h2>
            <div style="font-size: 18px; line-height: 1.6;">
                {conteudo}
            </div>
        </div>
        """
    elif layout == "two-column":
        conteudo_esq, conteudo_dir = conteudo
        html_content = f"""
        <div style="padding: 20px;">
            <h2 style="color: #2c3e50; border-bottom: 3px solid #3498db; padding-bottom: 10px; margin-bottom: 30px;">{titulo}</h2>
            <div style="display: flex; gap: 40px;">
                <div style="flex: 1; font-size: 16px; line-height: 1.6;">
                    {conteudo_esq}
                </div>
                <div style="flex: 1; font-size: 16px; line-height: 1.6;">
                    {conteudo_dir}
                </div>
            </div>
        </div>
        """
   \end{lstlisting}
\end{pythonbox}

\begin{pythonbox}
\begin{lstlisting}[language=Python]           
    return HTML(html_content)

# Exemplo de slide de título
display(criar_slide_titulo(
    "Análise de Eficácia Educacional",
    "Resultados do Estudo Experimental",
    "Dr. João Silva - Universidade de Pesquisa"
))
\end{lstlisting}
\end{pythonbox}

\subsection{Slides Interativos com Dados}

\begin{pythonbox}
\begin{lstlisting}[language=Python]
# Slide com objetivos
conteudo_objetivos = """
<h3>Objetivos do Estudo:</h3>
<ul style="font-size: 20px; line-height: 2;">
    <li>Avaliar a eficácia de uma nova metodologia de ensino</li>
    <li>Comparar resultados entre grupo controle e experimental</li>
    <li>Analisar fatores que influenciam o aprendizado</li>
    <li>Fornecer recomendações baseadas em evidências</li>
</ul>

<h3 style="margin-top: 40px;">Hipóteses:</h3>
<ul style="font-size: 20px; line-height: 2;">
    <li><strong>H₁:</strong> O grupo experimental apresentará maior ganho de aprendizado</li>
    <li><strong>H₀:</strong> Não há diferença significativa entre os grupos</li>
</ul>
"""

display(criar_slide_conteudo("Objetivos e Hipóteses", conteudo_objetivos))
\end{lstlisting}
\end{pythonbox}

\begin{pythonbox}
\begin{lstlisting}[language=Python]
# Slide com metodologia
conteudo_metodologia = f"""
<h3>Desenho do Estudo:</h3>
<p style="font-size: 18px;">Experimento controlado randomizado com pré e pós-teste</p>

<h3>Participantes:</h3>
<ul style="font-size: 18px; line-height: 1.8;">
    <li><strong>N =</strong> {len(dados_estudo)} participantes</li>
    <li><strong>Grupo Controle:</strong> {len(dados_estudo[dados_estudo['grupo'] == 'Controle'])} participantes</li>
    <li><strong>Grupo Experimental:</strong> {len(dados_estudo[dados_estudo['grupo'] == 'Experimental'])} participantes</li>
    <li><strong>Idade média:</strong> {dados_estudo['idade'].mean():.1f} anos (DP = {dados_estudo['idade'].std():.1f})</li>
</ul>

<h3>Instrumentos:</h3>
<ul style="font-size: 18px; line-height: 1.8;">
    <li>Teste de conhecimento (0-100 pontos)</li>
    <li>Escala de satisfação (1-10 pontos)</li>
    <li>Questionário sociodemográfico</li>
</ul>
"""

display(criar_slide_conteudo("Metodologia", conteudo_metodologia))
\end{lstlisting}
\end{pythonbox}

\begin{pythonbox}
\begin{lstlisting}[language=Python]
# Slide com gráfico interativo
def criar_slide_com_grafico(titulo, dados, tipo_grafico="bar"):
    """Cria slide com gráfico incorporado"""
    
    if tipo_grafico == "bar":
        # Gráfico de barras comparativo
        stats_grupos = dados.groupby('grupo').agg({
            'pre_teste': ['mean', 'std'],
            'pos_teste': ['mean', 'std'],
            'ganho_aprendizado': ['mean', 'std']
        }).round(2)
        
        fig = go.Figure()
        
        grupos = stats_grupos.index
        fig.add_trace(go.Bar(
            name='Pré-teste',
            x=grupos,
            y=stats_grupos['pre_teste']['mean'],
            error_y=dict(type='data', array=stats_grupos['pre_teste']['std']),
            marker_color='lightblue'
        ))
        
        fig.add_trace(go.Bar(
            name='Pós-teste',
            x=grupos,
            y=stats_grupos['pos_teste']['mean'],
            error_y=dict(type='data', array=stats_grupos['pos_teste']['std']),
            marker_color='darkblue'
        ))
        
        fig.update_layout(
            title=f'{titulo}',
            xaxis_title='Grupo',
            yaxis_title='Pontuação',
            barmode='group',
            font=dict(size=16),
            height=500,
            template='plotly_white'
        )
           \end{lstlisting}
\end{pythonbox}

\begin{pythonbox}
\begin{lstlisting}[language=Python]       
    elif tipo_grafico == "scatter":
        # Gráfico de dispersão
        fig = px.scatter(
            dados, 
            x='pre_teste', 
            y='pos_teste',
            color='grupo',
            size='satisfacao',
            title=titulo,
            labels={
                'pre_teste': 'Pré-teste',
                'pos_teste': 'Pós-teste',
                'grupo': 'Grupo'
            }
        )
        fig.update_layout(font=dict(size=16), height=500)
    
    # Exibindo título e gráfico
    display(HTML(f'<h2 style="color: #2c3e50; text-align: center; margin: 20px 0;">{titulo}</h2>'))
    fig.show()

# Criando slides com gráficos
criar_slide_com_grafico("Comparação Pré vs Pós-teste", dados_estudo, "bar")
\end{lstlisting}
\end{pythonbox}

\begin{pythonbox}
\begin{lstlisting}[language=Python]
# Slide com resultados estatísticos
def criar_slide_estatisticas(dados):
    """Cria slide com resultados estatísticos"""
    from scipy import stats
    
    # Realizando testes
    controle = dados[dados['grupo'] == 'Controle']
    experimental = dados[dados['grupo'] == 'Experimental']
    
    # Teste t
    t_stat, p_valor = stats.ttest_ind(experimental['pos_teste'], controle['pos_teste'])
    
    # Cohen's d
    pooled_std = np.sqrt(((len(controle)-1)*controle['pos_teste'].std()**2 + 
                         (len(experimental)-1)*experimental['pos_teste'].std()**2) / 
                        (len(controle)+len(experimental)-2))
    cohens_d = (experimental['pos_teste'].mean() - controle['pos_teste'].mean()) / pooled_std
    
    # Interpretação do tamanho do efeito
    if abs(cohens_d) < 0.2:
        interpretacao_d = "Pequeno"
    elif abs(cohens_d) < 0.5:
        interpretacao_d = "Médio"
    elif abs(cohens_d) < 0.8:
        interpretacao_d = "Grande"
    else:
        interpretacao_d = "Muito Grande"
    
    conteudo_stats = f"""
    <div style="text-align: center;">
        <h3>Teste t para Amostras Independentes</h3>
        <div style="background-color: #ecf0f1; padding: 30px; border-radius: 10px; margin: 20px 0;">
            <p style="font-size: 24px; margin: 10px 0;"><strong>t({len(dados)-2}) = {t_stat:.3f}</strong></p>
            <p style="font-size: 24px; margin: 10px 0; color: {'#e74c3c' if p_valor < 0.05 else '#95a5a6'};"><strong>p = {p_valor:.4f}</strong></p>
            <p style="font-size: 20px; margin: 10px 0;"><strong>Cohen's d = {cohens_d:.3f}</strong> <em>({interpretacao_d})</em></p>
        </div>
        
        <h3 style="margin-top: 40px;">Conclusão:</h3>
        <p style="font-size: 20px; color: {'#27ae60' if p_valor < 0.05 else '#e74c3c'};">
            {'✓ Diferença estatisticamente significativa' if p_valor < 0.05 else '✗ Diferença não significativa'}
        </p>
           \end{lstlisting}
\end{pythonbox}

\begin{pythonbox}
\begin{lstlisting}[language=Python]       
        <div style="margin-top: 30px; padding: 20px; background-color: #d5dbdb; border-radius: 5px;">
            <h4>Ganho Médio de Aprendizado:</h4>
            <p style="font-size: 18px;">
                <strong>Controle:</strong> {controle['ganho_aprendizado'].mean():.2f} ± {controle['ganho_aprendizado'].std():.2f}<br>
                <strong>Experimental:</strong> {experimental['ganho_aprendizado'].mean():.2f} ± {experimental['ganho_aprendizado'].std():.2f}
            </p>
        </div>
    </div>
    """
    
    return HTML(f"""
    <div style="padding: 20px;">
        <h2 style="color: #2c3e50; border-bottom: 3px solid #3498db; padding-bottom: 10px; margin-bottom: 30px; text-align: center;">Resultados Estatísticos</h2>
        {conteudo_stats}
    </div>
    """)

display(criar_slide_estatisticas(dados_estudo))
\end{lstlisting}
\end{pythonbox}

\subsection{Slides de Conclusão e Recomendações}

\begin{pythonbox}
\begin{lstlisting}[language=Python]
# Slide de conclusões
def criar_slide_conclusoes(dados):
    """Cria slide com conclusões do estudo"""
    
    # Calculando estatísticas finais
    experimental = dados[dados['grupo'] == 'Experimental']
    controle = dados[dados['grupo'] == 'Controle']
    
    ganho_experimental = experimental['ganho_aprendizado'].mean()
    ganho_controle = controle['ganho_aprendizado'].mean()
    diferenca_percentual = ((ganho_experimental - ganho_controle) / ganho_controle) * 100
    
    conteudo_conclusoes = f"""
    <h3>Principais Achados:</h3>
    <div style="background-color: #e8f5e8; padding: 20px; border-left: 5px solid #27ae60; margin: 20px 0;">
        <ul style="font-size: 18px; line-height: 2;">
            <li>O grupo experimental apresentou <strong>{diferenca_percentual:.1f}% mais ganho</strong> de aprendizado</li>
            <li>Diferença estatisticamente significativa entre os grupos (p < 0.05)</li>
            <li>Tamanho do efeito considerado <strong>médio a grande</strong></li>
            <li>Alta satisfação no grupo experimental</li>
        </ul>
    </div>
    
    <h3>Implicações Práticas:</h3>
    <ul style="font-size: 18px; line-height: 2;">
        <li>A nova metodologia é <strong>eficaz</strong> para melhorar o aprendizado</li>
        <li>Implementação recomendada em maior escala</li>
        <li>Benefícios observados em diferentes perfis de estudantes</li>
    </ul>
    
    <h3>Limitações:</h3>
    <ul style="font-size: 16px; line-height: 1.8; color: #7f8c8d;">
        <li>Estudo de curta duração</li>
        <li>Amostra específica de uma instituição</li>
        <li>Necessidade de replicação em outros contextos</li>
    </ul>
    """
       \end{lstlisting}
\end{pythonbox}

\begin{pythonbox}
\begin{lstlisting}[language=Python]       
    return HTML(f"""
    <div style="padding: 20px;">
        <h2 style="color: #2c3e50; border-bottom: 3px solid #3498db; padding-bottom: 10px; margin-bottom: 30px;">Conclusões</h2>
        {conteudo_conclusoes}
    </div>
    """)

display(criar_slide_conclusoes(dados_estudo))
\end{lstlisting}
\end{pythonbox}

\begin{pythonbox}
\begin{lstlisting}[language=Python]
# Slide de recomendações
conteudo_recomendacoes = """
<h3>�� Recomendações para Implementação:</h3>
<div style="display: grid; grid-template-columns: 1fr 1fr; gap: 30px; margin: 30px 0;">
    <div style="background-color: #fff3cd; padding: 20px; border-radius: 8px; border-left: 4px solid #ffc107;">
        <h4 style="color: #856404;">Curto Prazo (3-6 meses)</h4>
        <ul style="line-height: 1.8;">
            <li>Treinar educadores na nova metodologia</li>
            <li>Implementar em turmas piloto</li>
            <li>Monitorar resultados iniciais</li>
        </ul>
    </div>
    
    <div style="background-color: #d4edda; padding: 20px; border-radius: 8px; border-left: 4px solid #28a745;">
        <h4 style="color: #155724;">Longo Prazo (6-12 meses)</h4>
        <ul style="line-height: 1.8;">
            <li>Expansão para toda a instituição</li>
            <li>Desenvolvimento de materiais</li>
            <li>Avaliação de impacto em larga escala</li>
        </ul>
    </div>
</div>

<h3>Próximos Estudos:</h3>
<ul style="font-size: 18px; line-height: 2;">
    <li>Estudo longitudinal de 12 meses</li>
    <li>Análise de custo-benefício</li>
    <li>Investigação de fatores moderadores</li>
    <li>Replicação em diferentes contextos</li>
</ul>

<div style="text-align: center; margin-top: 40px; padding: 20px; background-color: #f8f9fa; border-radius: 10px;">
    <h3 style="color: #495057;">Obrigado!</h3>
    <p style="font-size: 18px; color: #6c757d;">Perguntas e Discussão</p>
</div>
"""

display(criar_slide_conteudo("Recomendações e Próximos Passos", conteudo_recomendacoes))
\end{lstlisting}
\end{pythonbox}

% =============================================================================
\section{Geração de Gráficos para Publicações}
% =============================================================================

\subsection{Padrões de Qualidade para Publicação}

Para publicações científicas, os gráficos devem seguir padrões específicos de qualidade, resolução e formatação.

\begin{pythonbox}
\begin{lstlisting}[language=Python]
# Configurações para gráficos de publicação
import matplotlib.pyplot as plt
import seaborn as sns
from matplotlib import rcParams

# Configurações para publicação científica
def config_publicacao():
    """Configura matplotlib para gráficos de publicação"""
    rcParams['figure.figsize'] = (8, 6)
    rcParams['figure.dpi'] = 300
    rcParams['savefig.dpi'] = 300
    rcParams['font.size'] = 12
    rcParams['axes.labelsize'] = 14
    rcParams['axes.titlesize'] = 16
    rcParams['xtick.labelsize'] = 12
    rcParams['ytick.labelsize'] = 12
    rcParams['legend.fontsize'] = 12
    rcParams['font.family'] = 'serif'
    rcParams['font.serif'] = ['Times New Roman', 'Times', 'serif']
    rcParams['text.usetex'] = False  # Definir como True se LaTeX estiver disponível
    rcParams['axes.linewidth'] = 1.5
    rcParams['grid.linewidth'] = 0.5
    rcParams['lines.linewidth'] = 2
    rcParams['patch.linewidth'] = 0.5
    rcParams['xtick.major.width'] = 1.5
    rcParams['ytick.major.width'] = 1.5
    rcParams['xtick.minor.width'] = 1
    rcParams['ytick.minor.width'] = 1

# Aplicando configurações
config_publicacao()

# Paleta de cores para publicação (colorblind-friendly)
cores_publicacao = ['#1f77b4', '#ff7f0e', '#2ca02c', '#d62728', '#9467bd', '#8c564b']
sns.set_palette(cores_publicacao)
\end{lstlisting}
\end{pythonbox}

\begin{pythonbox}
\begin{lstlisting}[language=Python]
# Função para criar gráficos de publicação
def criar_grafico_publicacao(dados, tipo='barras', titulo='', 
                            xlabel='', ylabel='', filename=None):
    """
    Cria gráficos adequados para publicação científica
    """
    fig, ax = plt.subplots(figsize=(8, 6))
    
    if tipo == 'barras':
        # Gráfico de barras com erro padrão
        stats = dados.groupby('grupo').agg({
            'pre_teste': ['mean', 'sem'],
            'pos_teste': ['mean', 'sem']
        })
        
        x = np.arange(len(stats.index))
        width = 0.35
        
        bars1 = ax.bar(x - width/2, stats['pre_teste']['mean'], width, 
                      yerr=stats['pre_teste']['sem'], 
                      label='Pré-teste', capsize=5, alpha=0.8)
        bars2 = ax.bar(x + width/2, stats['pos_teste']['mean'], width,
                      yerr=stats['pos_teste']['sem'],
                      label='Pós-teste', capsize=5, alpha=0.8)
        
        ax.set_xlabel(xlabel if xlabel else 'Grupo')
        ax.set_ylabel(ylabel if ylabel else 'Pontuação')
        ax.set_xticks(x)
        ax.set_xticklabels(stats.index)
        ax.legend()
        
    elif tipo == 'boxplot':
        # Box plot com pontos individuais
        box_data = [dados[dados['grupo'] == g]['ganho_aprendizado'] for g in dados['grupo'].unique()]
        box_labels = dados['grupo'].unique()
        
        bp = ax.boxplot(box_data, labels=box_labels, patch_artist=True, showmeans=True)
        
        # Personalizando cores
        for i, patch in enumerate(bp['boxes']):
            patch.set_facecolor(cores_publicacao[i])
            patch.set_alpha(0.7)
        
        ax.set_xlabel(xlabel if xlabel else 'Grupo')
        ax.set_ylabel(ylabel if ylabel else 'Ganho de Aprendizado')
           \end{lstlisting}
\end{pythonbox}

\begin{pythonbox}
\begin{lstlisting}[language=Python]       
    elif tipo == 'scatter':
        # Scatter plot com linha de regressão
        for i, grupo in enumerate(dados['grupo'].unique()):
            subset = dados[dados['grupo'] == grupo]
            ax.scatter(subset['pre_teste'], subset['pos_teste'], 
                      c=cores_publicacao[i], label=grupo, alpha=0.7, s=50)
        
        # Adicionando linha de identidade
        lims = [
            np.min([ax.get_xlim(), ax.get_ylim()]),
            np.max([ax.get_xlim(), ax.get_ylim()]),
        ]
        ax.plot(lims, lims, 'k--', alpha=0.5, zorder=0, label='y = x')
        
        ax.set_xlabel(xlabel if xlabel else 'Pré-teste')
        ax.set_ylabel(ylabel if ylabel else 'Pós-teste')
        ax.legend()
    
    # Formatação final
    ax.set_title(titulo, pad=20)
    ax.grid(True, alpha=0.3)
    ax.spines['top'].set_visible(False)
    ax.spines['right'].set_visible(False)
    
    plt.tight_layout()
    
    if filename:
        plt.savefig(filename, dpi=300, bbox_inches='tight', 
                   facecolor='white', edgecolor='none')
        print(f"Gráfico salvo como: {filename}")
    
    plt.show()
    return fig, ax

# Exemplos de gráficos para publicação
criar_grafico_publicacao(
    dados_estudo, 
    tipo='barras',
    titulo='Comparação entre Grupos no Pré e Pós-teste',
    ylabel='Pontuação (0-100)',
    filename='figura1_comparacao_grupos.png'
)
\end{lstlisting}
\end{pythonbox}

\begin{pythonbox}
\begin{lstlisting}[language=Python]
# Gráfico de ganhos de aprendizado
criar_grafico_publicacao(
    dados_estudo,
    tipo='boxplot', 
    titulo='Distribuição dos Ganhos de Aprendizado por Grupo',
    ylabel='Ganho (Pós-teste - Pré-teste)',
    filename='figura2_ganhos_aprendizado.png'
)
\end{lstlisting}
\end{pythonbox}

\begin{pythonbox}
\begin{lstlisting}[language=Python]
# Scatter plot pré vs pós
criar_grafico_publicacao(
    dados_estudo,
    tipo='scatter',
    titulo='Relação entre Pré-teste e Pós-teste por Grupo', 
    xlabel='Pontuação Pré-teste',
    ylabel='Pontuação Pós-teste',
    filename='figura3_correlacao_pre_pos.png'
)
\end{lstlisting}
\end{pythonbox}

\subsection{Gráficos Multivariados Avançados}

\begin{pythonbox}
\begin{lstlisting}[language=Python]
# Figura composta para publicação
def criar_figura_multipla(dados, filename=None):
    """Cria figura com múltiplos painéis para publicação"""
    
    fig, axes = plt.subplots(2, 2, figsize=(12, 10))
    
    # Painel A: Barras comparativas
    stats = dados.groupby('grupo').agg({
        'pre_teste': ['mean', 'sem'],
        'pos_teste': ['mean', 'sem']
    })
    
    x = np.arange(len(stats.index))
    width = 0.35
    
    axes[0, 0].bar(x - width/2, stats['pre_teste']['mean'], width, 
                   yerr=stats['pre_teste']['sem'], 
                   label='Pré-teste', capsize=5, alpha=0.8)
    axes[0, 0].bar(x + width/2, stats['pos_teste']['mean'], width,
                   yerr=stats['pos_teste']['sem'],
                   label='Pós-teste', capsize=5, alpha=0.8)
    
    axes[0, 0].set_title('A) Comparação Pré vs Pós-teste', fontweight='bold', loc='left')
    axes[0, 0].set_ylabel('Pontuação')
    axes[0, 0].set_xticks(x)
    axes[0, 0].set_xticklabels(stats.index)
    axes[0, 0].legend()
    axes[0, 0].grid(True, alpha=0.3)
    
    # Painel B: Box plot dos ganhos
    box_data = [dados[dados['grupo'] == g]['ganho_aprendizado'] for g in dados['grupo'].unique()]
    bp = axes[0, 1].boxplot(box_data, labels=dados['grupo'].unique(), patch_artist=True)
    
    for i, patch in enumerate(bp['boxes']):
        patch.set_facecolor(cores_publicacao[i])
        patch.set_alpha(0.7)
    
    axes[0, 1].set_title('B) Distribuição dos Ganhos', fontweight='bold', loc='left')
    axes[0, 1].set_ylabel('Ganho de Aprendizado')
    axes[0, 1].grid(True, alpha=0.3)
       \end{lstlisting}
\end{pythonbox}

\begin{pythonbox}
\begin{lstlisting}[language=Python]       
    # Painel C: Correlação satisfação vs ganho
    for i, grupo in enumerate(dados['grupo'].unique()):
        subset = dados[dados['grupo'] == grupo]
        axes[1, 0].scatter(subset['satisfacao'], subset['ganho_aprendizado'],
                          c=cores_publicacao[i], label=grupo, alpha=0.7)
    
    axes[1, 0].set_title('C) Satisfação vs Ganho', fontweight='bold', loc='left')
    axes[1, 0].set_xlabel('Satisfação')
    axes[1, 0].set_ylabel('Ganho de Aprendizado')
    axes[1, 0].legend()
    axes[1, 0].grid(True, alpha=0.3)
    
    # Painel D: Histograma idade por grupo
    for i, grupo in enumerate(dados['grupo'].unique()):
        subset = dados[dados['grupo'] == grupo]
        axes[1, 1].hist(subset['idade'], alpha=0.7, label=grupo, 
                       bins=15, color=cores_publicacao[i])
    
    axes[1, 1].set_title('D) Distribuição de Idade', fontweight='bold', loc='left')
    axes[1, 1].set_xlabel('Idade')
    axes[1, 1].set_ylabel('Frequência')
    axes[1, 1].legend()
    axes[1, 1].grid(True, alpha=0.3)
    
    # Removendo spines superiores e direitas
    for ax in axes.flat:
        ax.spines['top'].set_visible(False)
        ax.spines['right'].set_visible(False)
    
    plt.tight_layout()
    
    if filename:
        plt.savefig(filename, dpi=300, bbox_inches='tight', 
                   facecolor='white', edgecolor='none')
        Continuando o LaTeX do Capítulo 11:
        print(f"Figura múltipla salva como: {filename}")
    
    return fig

# Gerando figura múltipla
figura_multipla = criar_figura_multipla(dados_estudo, 'figura4_analise_completa.png')
\end{lstlisting}
\end{pythonbox}

\subsection{Gráficos com Anotações Estatísticas}

\begin{pythonbox}
\begin{lstlisting}[language=Python]
def criar_grafico_com_anotacoes(dados):
    """Cria gráfico com anotações estatísticas para publicação"""
    from scipy import stats as scipy_stats
    
    fig, ax = plt.subplots(figsize=(10, 8))
    
    # Preparando dados
    controle = dados[dados['grupo'] == 'Controle']['pos_teste']
    experimental = dados[dados['grupo'] == 'Experimental']['pos_teste']
    
    # Violin plot
    positions = [1, 2]
    vp = ax.violinplot([controle, experimental], positions=positions, 
                       widths=0.5, showmeans=True, showmedians=True)
    
    # Personalizando cores
    for i, pc in enumerate(vp['bodies']):
        pc.set_facecolor(cores_publicacao[i])
        pc.set_alpha(0.7)
    
    # Teste estatístico
    t_stat, p_valor = scipy_stats.ttest_ind(controle, experimental)
    
    # Adicionando anotação de significância
    if p_valor < 0.001:
        sig_text = '***'
    elif p_valor < 0.01:
        sig_text = '**'
    elif p_valor < 0.05:
        sig_text = '*'
    else:
        sig_text = 'ns'
    
    # Linha conectando grupos
    y_max = max(controle.max(), experimental.max())
    y_line = y_max + 5
    ax.plot([1, 2], [y_line, y_line], 'k-', linewidth=1)
    ax.text(1.5, y_line + 1, sig_text, ha='center', fontsize=14)
    ax.text(1.5, y_line + 4, f'p = {p_valor:.4f}', ha='center', fontsize=10)
    
    # Adicionando médias como texto
    ax.text(1, controle.mean() - 5, f'M = {controle.mean():.1f}', 
            ha='center', fontsize=10, fontweight='bold')
    ax.text(2, experimental.mean() - 5, f'M = {experimental.mean():.1f}', 
            ha='center', fontsize=10, fontweight='bold')
       \end{lstlisting}
\end{pythonbox}

\begin{pythonbox}
\begin{lstlisting}[language=Python]       
    # Formatação
    ax.set_xticks(positions)
    ax.set_xticklabels(['Controle', 'Experimental'])
    ax.set_ylabel('Pontuação Pós-teste')
    ax.set_title('Comparação de Desempenho entre Grupos com Análise Estatística')
    ax.grid(True, alpha=0.3, axis='y')
    ax.spines['top'].set_visible(False)
    ax.spines['right'].set_visible(False)
    
    plt.tight_layout()
    plt.savefig('figura5_violin_anotado.png', dpi=300, bbox_inches='tight')
    plt.show()
    
    return fig

# Criando gráfico com anotações
grafico_anotado = criar_grafico_com_anotacoes(dados_estudo)
\end{lstlisting}
\end{pythonbox}

% =============================================================================
\section{Templates de Documentação}
% =============================================================================

\subsection{Sistema de Templates para Documentação Científica}

\begin{pythonbox}
\begin{lstlisting}[language=Python]
class TemplatesCientificos:
    """Sistema de templates para documentação científica"""
    
    def __init__(self):
        self.templates = {}
        self.carregar_templates()
    
    def carregar_templates(self):
        """Carrega templates predefinidos"""
        
        # Template de artigo científico
        self.templates['artigo'] = """
# {{ titulo }}

**{{ autores }}**  
*{{ afiliacao }}*

## Resumo
{{ resumo }}

**Palavras-chave:** {{ palavras_chave }}

## 1. Introdução
{{ introducao }}

## 2. Metodologia

### 2.1 Participantes
{{ participantes }}

### 2.2 Procedimentos
{{ procedimentos }}

### 2.3 Análise de Dados
{{ analise_dados }}

## 3. Resultados
{{ resultados }}

## 4. Discussão
{{ discussao }}
   \end{lstlisting}
\end{pythonbox}

\begin{pythonbox}
\begin{lstlisting}[language=Python]       
## 5. Conclusão
{{ conclusao }}

## Referências
{{ referencias }}
"""
        
        # Template de protocolo de pesquisa
        self.templates['protocolo'] = """
# Protocolo de Pesquisa: {{ titulo }}

## Informações Gerais
- **Pesquisador Principal:** {{ pesquisador_principal }}
- **Instituição:** {{ instituicao }}
- **Data de Início:** {{ data_inicio }}
- **Duração Prevista:** {{ duracao }}

## Objetivos
### Objetivo Geral
{{ objetivo_geral }}

### Objetivos Específicos
{{ objetivos_especificos }}

## Hipóteses
{{ hipoteses }}

## Metodologia
### Desenho do Estudo
{{ desenho_estudo }}

### Critérios de Inclusão
{{ criterios_inclusao }}

### Critérios de Exclusão
{{ criterios_exclusao }}

### Tamanho da Amostra
{{ tamanho_amostra }}

## Considerações Éticas
{{ consideracoes_eticas }}

## Cronograma
{{ cronograma }}
   \end{lstlisting}
\end{pythonbox}

\begin{pythonbox}
\begin{lstlisting}[language=Python]       
## Orçamento
{{ orcamento }}
"""
        
        # Template de relatório técnico
        self.templates['relatorio_tecnico'] = """
<!DOCTYPE html>
<html>
<head>
    <title>{{ titulo }}</title>
    <style>
        body { font-family: 'Arial', sans-serif; margin: 40px; }
        .header { background-color: #2c3e50; color: white; padding: 20px; }
        .section { margin: 30px 0; }
        .table { width: 100%; border-collapse: collapse; }
        .table th, .table td { border: 1px solid #ddd; padding: 8px; }
        .table th { background-color: #f2f2f2; }
        .highlight { background-color: #ffffcc; padding: 10px; }
    </style>
</head>
<body>
    <div class="header">
        <h1>{{ titulo }}</h1>
        <p>{{ data }} | {{ autor }}</p>
    </div>
    
    <div class="section">
        <h2>Sumário Executivo</h2>
        <div class="highlight">{{ sumario_executivo }}</div>
    </div>
    
    <div class="section">
        <h2>Metodologia</h2>
        {{ metodologia }}
    </div>
    
    <div class="section">
        <h2>Resultados</h2>
        {{ resultados }}
    </div>
    
    <div class="section">
        <h2>Recomendações</h2>
        {{ recomendacoes }}
    </div>
       \end{lstlisting}
\end{pythonbox}

\begin{pythonbox}
\begin{lstlisting}[language=Python]       
    <div class="section">
        <h2>Anexos</h2>
        {{ anexos }}
    </div>
</body>
</html>
"""
    
    def gerar_documento(self, tipo_template, dados):
        """Gera documento a partir de template"""
        if tipo_template not in self.templates:
            raise ValueError(f"Template '{tipo_template}' não encontrado")
        
        template = jinja2.Template(self.templates[tipo_template])
        return template.render(**dados)
    
    def salvar_documento(self, conteudo, filename, formato='md'):
        """Salva documento gerado"""
        with open(filename, 'w', encoding='utf-8') as f:
            f.write(conteudo)
        print(f"Documento salvo: {filename}")
        return filename

# Exemplo de uso
templates = TemplatesCientificos()

# Dados para artigo
dados_artigo = {
    'titulo': 'Eficácia de Nova Metodologia de Ensino: Um Estudo Experimental',
    'autores': 'Silva, J.; Santos, M.; Oliveira, P.',
    'afiliacao': 'Universidade de Pesquisa, Departamento de Educação',
    'resumo': 'Este estudo investigou a eficácia de uma nova metodologia de ensino...',
    'palavras_chave': 'educação, aprendizado, metodologia, experimento',
    'introducao': 'O desenvolvimento de metodologias eficazes de ensino...',
    'participantes': 'Participaram do estudo 200 estudantes universitários...',
    'procedimentos': 'Os participantes foram randomicamente alocados...',
    'analise_dados': 'As análises foram conduzidas usando Python 3.9...',
    'resultados': 'Os resultados indicaram diferença significativa...',
    'discussao': 'Os achados sugerem que a nova metodologia...',
    'conclusao': 'Este estudo fornece evidências preliminares...',
    'referencias': '1. Smith et al. (2023)...'
}
   \end{lstlisting}
\end{pythonbox}

\begin{pythonbox}
\begin{lstlisting}[language=Python]       
# Gerando artigo
artigo = templates.gerar_documento('artigo', dados_artigo)
templates.salvar_documento(artigo, 'artigo_cientifico.md')
\end{lstlisting}
\end{pythonbox}

\subsection{Gerador de Documentação de Código}

\begin{pythonbox}
\begin{lstlisting}[language=Python]
class GeradorDocumentacao:
    """Gera documentação automática para projetos científicos"""
    
    def __init__(self, nome_projeto, versao="1.0.0"):
        self.nome_projeto = nome_projeto
        self.versao = versao
        self.secoes = []
        
    def adicionar_secao(self, titulo, conteudo):
        """Adiciona seção à documentação"""
        self.secoes.append({'titulo': titulo, 'conteudo': conteudo})
    
    def documentar_funcao(self, funcao):
        """Extrai documentação de uma função"""
        import inspect
        
        doc = {
            'nome': funcao.__name__,
            'docstring': inspect.getdoc(funcao) or 'Sem documentação',
            'assinatura': str(inspect.signature(funcao)),
            'codigo': inspect.getsource(funcao) if hasattr(funcao, '__code__') else 'N/A'
        }
        return doc
    
    def documentar_dataset(self, df, nome="Dataset"):
        """Documenta estrutura de um dataset"""
        doc = f"""
### {nome}

**Dimensões:** {df.shape[0]} linhas × {df.shape[1]} colunas

**Colunas:**
"""
        for col in df.columns:
            dtype = str(df[col].dtype)
            nulls = df[col].isnull().sum()
            unique = df[col].nunique()
            doc += f"\n- **{col}** ({dtype}): {unique} valores únicos, {nulls} valores nulos"
        
        doc += f"\n\n**Estatísticas Básicas:**\n```\n{df.describe().to_string()}\n```"
        
        return doc
    
    def gerar_readme(self):
        """Gera arquivo README.md"""
        readme = f"""# {self.nome_projeto}
   \end{lstlisting}
\end{pythonbox}

\begin{pythonbox}
\begin{lstlisting}[language=Python]       
**Versão:** {self.versao}  
**Data:** {datetime.now().strftime('%d/%m/%Y')}

## Descrição
Projeto de análise de dados científicos desenvolvido em Python.

## Instalação

```bash
pip install -r requirements.txt
```

## Uso

```python
from analise import GeradorRelatorio

relatorio = GeradorRelatorio(dados)
relatorio.gerar_relatorio_completo()
```

## Estrutura do Projeto

```
{self.nome_projeto}/
│
├── data/              # Dados brutos e processados
├── notebooks/         # Jupyter notebooks
├── src/              # Código fonte
├── reports/          # Relatórios gerados
├── figures/          # Gráficos e visualizações
└── docs/             # Documentação
```

## Dependências

- Python >= 3.8
- pandas >= 1.3.0
- numpy >= 1.21.0
- matplotlib >= 3.4.0
- seaborn >= 0.11.0
- scipy >= 1.7.0

## Contribuidores

- [Nome do Pesquisador]

## Licença
   \end{lstlisting}
\end{pythonbox}

\begin{pythonbox}
\begin{lstlisting}[language=Python]       
Este projeto está sob licença MIT.
"""
        return readme
    
    def gerar_documentacao_completa(self, output_dir='docs'):
        """Gera documentação completa do projeto"""
        import os
        
        # Criar diretório se não existir
        if not os.path.exists(output_dir):
            os.makedirs(output_dir)
        
        # Gerar README
        readme = self.gerar_readme()
        with open(f'{output_dir}/README.md', 'w', encoding='utf-8') as f:
            f.write(readme)
        
        # Gerar documentação das seções
        doc_completa = f"# Documentação Completa - {self.nome_projeto}\n\n"
        
        for secao in self.secoes:
            doc_completa += f"## {secao['titulo']}\n\n{secao['conteudo']}\n\n"
        
        with open(f'{output_dir}/documentacao.md', 'w', encoding='utf-8') as f:
            f.write(doc_completa)
        
        print(f"Documentação gerada em: {output_dir}/")
        return output_dir

# Exemplo de uso
doc = GeradorDocumentacao("Análise de Eficácia Educacional", "1.0.0")

# Documentando dataset
doc.adicionar_secao("Dataset Principal", doc.documentar_dataset(dados_estudo))

# Documentando função
doc.adicionar_secao("Funções de Análise", doc.documentar_funcao(GeradorRelatorio.calcular_estatisticas_descritivas))

# Gerando documentação completa
doc.gerar_documentacao_completa()
\end{lstlisting}
\end{pythonbox}

% =============================================================================
\section{Exercícios Práticos}
% =============================================================================

\begin{exercisebox}
\textbf{Exercício 1: Relatório Automatizado Personalizado}

Crie um sistema que:
1. Leia dados de múltiplas fontes (CSV, Excel, API)
2. Realize análises estatísticas apropriadas
3. Gere relatório em HTML e PDF
4. Envie por e-mail automaticamente
\end{exercisebox}

\begin{exercisebox}
\textbf{Exercício 2: Dashboard Interativo}

Desenvolva um dashboard que:
1. Atualize em tempo real
2. Permita filtros dinâmicos
3. Exporte gráficos em alta resolução
4. Gere relatórios sob demanda
\end{exercisebox}

\begin{exercisebox}
\textbf{Exercício 3: Apresentação Automatizada}

Implemente um sistema que:
1. Analise resultados de pesquisa
2. Crie slides automaticamente
3. Inclua gráficos interativos
4. Exporte para PowerPoint
\end{exercisebox}

% =============================================================================
\section{Conclusão}
% =============================================================================

A comunicação científica efetiva com Python transforma dados complexos em insights acessíveis. As ferramentas apresentadas neste capítulo permitem automatizar a geração de relatórios, criar apresentações dinâmicas e produzir visualizações de qualidade para publicação. A automação não apenas economiza tempo, mas também garante consistência e reprodutibilidade na comunicação de resultados científicos.

\vspace{1cm}
\begin{center}
\rule{0.8\textwidth}{0.4pt}
\end{center}
