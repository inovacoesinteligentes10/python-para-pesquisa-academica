% Python para Pesquisa Acadêmica - Ebook Completo
\documentclass[11pt,a4paper,oneside]{book}

% =============================================================================
% PACOTES E CONFIGURAÇÕES
% =============================================================================

% Configuração de página
\usepackage[
    top=3cm,
    bottom=3cm,
    left=3cm,
    right=2.5cm,
    headheight=25pt,
    headsep=25pt,
    footskip=25pt
]{geometry}
\usepackage{eso-pic}
% Pacotes essenciais
\usepackage[utf8]{inputenc}    % suporte a UTF-8
\usepackage[T1]{fontenc}       % encoding da fonte
\usepackage[portuguese]{babel} % idioma
\usepackage{lmodern}           % fonte melhorada
\usepackage{microtype}         % tipografia

% Matemática e símbolos
\usepackage{amsmath,amssymb,amsfonts}
\usepackage{physics}
\usepackage{siunitx}

% Tabelas e figuras
\usepackage{booktabs}
\usepackage{tabularx}
\usepackage{longtable}
\usepackage{multirow}
\usepackage{graphicx}
\usepackage{float}
\usepackage{subcaption}

% Código e listagens (com suporte a UTF-8 nos acentos)
\usepackage{listingsutf8}
\usepackage{fancyvrb}

% Caixas coloridas
\usepackage{tcolorbox}
\tcbuselibrary{most}

% Cores personalizadas
\usepackage{xcolor}
\definecolor{pythonblue}{RGB}{52, 101, 164}
\definecolor{pythonyellow}{RGB}{255, 212, 59}
\definecolor{codegreen}{RGB}{34, 139, 34}
\definecolor{codegray}{RGB}{128, 128, 128}
\definecolor{codepurple}{RGB}{148, 0, 211}
\definecolor{backcolour}{RGB}{248, 248, 248}
\definecolor{sectioncolor}{RGB}{25, 50, 82}
\definecolor{warningred}{RGB}{220, 53, 69}

% Links e referências
\usepackage[
    colorlinks=true,
    linkcolor=pythonblue,
    urlcolor=pythonblue,
    citecolor=pythonblue,
    filecolor=pythonblue
]{hyperref}

% Cabeçalhos
\usepackage{fancyhdr}
\pagestyle{fancy}
\fancyhf{}
\fancyhead[L]{\small\textit{Python para Pesquisa Acadêmica}}
\fancyhead[R]{\small\thepage}
\renewcommand{\headrulewidth}{0.4pt}
\renewcommand{\chaptermark}[1]{}
\renewcommand{\sectionmark}[1]{}

% Espaçamento
\usepackage{setspace}
\onehalfspacing

% Letrina
\usepackage{lettrine}

% =============================================================================
% METADADOS DO LIVRO
% =============================================================================

\newcommand{\goldenratio}{1.618}

\newcommand{\authorname}{Seu Nome}
\newcommand{\translatorname}{Editor Técnico}
\newcommand{\booktitle}{Python para Pesquisa Acadêmica}
\newcommand{\subtitle}{Guia Prático com Exemplos Reais}
\newcommand{\publisher}{Editora Acadêmica}
\newcommand{\editionyear}{2025}
\newcommand{\isbn}{978-0-123456-78-9}

% =============================================================================
% ESTILOS PERSONALIZADOS
% =============================================================================

% Configuração de código Python
\lstdefinestyle{pythonstyle}{
    backgroundcolor=\color{white},   
    commentstyle=\color{codegreen},
    keywordstyle=\color{pythonblue},
    numberstyle=\tiny\color{codegray},
    stringstyle=\color{codepurple},
    basicstyle=\ttfamily\footnotesize,
    breakatwhitespace=false,         
    breaklines=true,                 
    captionpos=b,                    
    keepspaces=true,                 
    numbers=left,                    
    numbersep=5pt,                  
    showspaces=false,                
    showstringspaces=false,
    showtabs=false,                  
    tabsize=2,
    frame=none,
    framesep=0pt,
    xleftmargin=15pt,
    xrightmargin=0pt,
    aboveskip=10pt,
    belowskip=10pt,
    inputencoding=utf8
}

\lstset{style=pythonstyle}


\lstset{
  style=pythonstyle,
  inputencoding=utf8,
  extendedchars=true,
  literate=
    {á}{{\'a}}1 {à}{{\`a}}1 {ã}{{\~a}}1 {â}{{\^a}}1
    {Á}{{\'A}}1 {À}{{\`A}}1 {Ã}{{\~A}}1 {Â}{{\^A}}1
    {é}{{\'e}}1 {ê}{{\^e}}1
    {É}{{\'E}}1 {Ê}{{\^E}}1
    {í}{{\'i}}1
    {Í}{{\'I}}1
    {ó}{{\'o}}1 {õ}{{\~o}}1 {ô}{{\^o}}1
    {Ó}{{\'O}}1 {Õ}{{\~O}}1 {Ô}{{\^O}}1
    {ú}{{\'u}}1
    {Ú}{{\'U}}1
    {ç}{{\c{c}}}1
    {Ç}{{\c{C}}}1
}

% Caixas personalizadas
\newtcolorbox{pythonbox}[1][]{
    colback=white,
    colframe=white,
    boxrule=0pt,
    sharp corners,
    #1
}

\newtcolorbox{examplebox}[1][]{
    colback=white,
    colframe=white,
    boxrule=0pt,
    sharp corners,
    fonttitle=\bfseries,
    title=Exemplo Prático,
    #1
}

\newtcolorbox{researchbox}[1][]{
    colback=white,
    colframe=white,
    boxrule=0pt,
    sharp corners,
    fonttitle=\bfseries,
    title=Aplicação em Pesquisa,
    #1
}

\newtcolorbox{warningbox}[1][]{
    colback=white,
    colframe=white,
    boxrule=0pt,
    sharp corners,
    fonttitle=\bfseries,
    title=Atenção,
    #1
}

% Títulos de capítulos e seções
\usepackage{titlesec}

\titleformat{\chapter}[display]
{\normalfont\huge\bfseries\color{sectioncolor}}
{\chaptertitlename\ \thechapter}{20pt}{\Huge}

\titleformat{\section}
{\normalfont\Large\bfseries\color{sectioncolor}}
{\thesection}{1em}{}

\titleformat{\subsection}
{\normalfont\large\bfseries\color{sectioncolor}}
{\thesubsection}{1em}{}

% =============================================================================
% INFORMAÇÕES DO DOCUMENTO
% =============================================================================
% Adiciona imagem de fundo apenas na primeira página
% \AddToShipoutPictureBG*{%
%   \includegraphics[width=\paperwidth,height=\paperheight]{frontmatter/ChatGPT.png}
% }
\title{
    \vspace{-2cm}
    {\Huge\bfseries\color{sectioncolor} Python para}\\[0.5cm]
    {\Huge\bfseries\color{sectioncolor} Pesquisa Acadêmica}\\[1cm]
    {\Large\bfseries\color{sectioncolor} Guia Prático com Exemplos Reais}
}

\author{
    \Large Seu Nome\\[0.3cm]
    \normalsize Universidade/Instituição\\[0.5cm]
    \normalsize Versão 1.0 - 2025
}

\date{}

% =============================================================================
% DOCUMENTO
% =============================================================================

\begin{document}
% Página dedicada apenas para a imagem de fundo
\newpage
\thispagestyle{empty}
\AddToShipoutPictureBG*{%
  \includegraphics[width=\paperwidth,height=\paperheight]{frontmatter/ChatGPT.png}
}
\mbox{}  % Página vazia com apenas a imagem
\newpage

\frontmatter

% Página de título
\maketitle
\thispagestyle{empty}

% Página de copyright
\newpage
\thispagestyle{empty}
\vspace*{\fill}
\begin{center}
\small
\textbf{\booktitle}\\
\subtitle\\[1cm]

\textcopyright\
\editionyear\ \authorname\\
\publisher\\[0.5cm]

ISBN: \isbn\\[1cm]

Este documento foi compilado com pdfLaTeX\\
Versão: \editionyear\\[0.5cm]

Todos os direitos reservados. Nenhuma parte desta publicação pode ser\\
reproduzida, distribuída ou transmitida de qualquer forma ou por qualquer meio,\\
incluindo fotocópia, gravação ou outros métodos eletrônicos ou mecânicos,\\
sem a permissão prévia por escrito do editor.
\end{center}
\vspace*{\fill}

% Prefácio
\chapter*{Prefácio}
\addcontentsline{toc}{chapter}{Prefácio}

Este livro nasceu da necessidade de democratizar o uso do Python na pesquisa acadêmica. Durante anos, observei pesquisadores lutando com ferramentas inadequadas ou gastando tempo excessivo em tarefas que poderiam ser automatizadas. Python oferece uma solução elegante, mas a curva de aprendizado pode ser intimidadora sem orientação adequada.

O objetivo desta obra é servir como ponte entre o conhecimento teórico e a aplicação prática. Cada capítulo foi cuidadosamente estruturado para apresentar conceitos fundamentais seguidos de exemplos reais de diferentes áreas acadêmicas. Não se trata apenas de ensinar Python, mas de mostrar como pensar computacionalmente sobre problemas de pesquisa.

Agradeço aos colegas pesquisadores que compartilharam seus casos de uso, aos estudantes que testaram os exemplos, e à comunidade Python que continua construindo ferramentas extraordinárias para a ciência.

\vspace{1cm}
\begin{flushright}
\authorname\\
\editionyear
\end{flushright}

% Sumário
\tableofcontents

% ===== CONTEÚDO PRINCIPAL =====
\mainmatter
\pagestyle{fancy}

% =============================================================================
% CAPÍTULO 1: INTRODUÇÃO AO PYTHON PARA PESQUISA
% =============================================================================

\chapter{Python na Pesquisa Acadêmica: Uma Revolução Silenciosa}

\lettrine{N}{os últimos anos}, Python tornou-se a linguagem de programação mais utilizada na pesquisa acadêmica, revolucionando a forma como cientistas coletam, analisam e interpretam dados. Esta transformação não é acidental: Python combina simplicidade sintática com poder computacional, oferecendo um ecossistema rico de bibliotecas especializadas que atendem às necessidades específicas da pesquisa científica.

\section{Por que Python Conquistou a Academia?}

A adoção massiva do Python na academia pode ser atribuída a vários fatores convergentes. Primeiro, sua sintaxe limpa e intuitiva permite que pesquisadores se concentrem na lógica científica em vez de complexidades técnicas. Segundo, o ecossistema de bibliotecas científicas (NumPy, SciPy, pandas, matplotlib) oferece ferramentas prontas para uso que antes exigiam implementações custosas.

\begin{examplebox}
\textbf{Exemplo Real:} Um biólogo que precisava analisar 10.000 sequências de DNA levaria semanas escrevendo código em C++. Com Python e BioPython, o mesmo trabalho pode ser feito em algumas horas:

\begin{lstlisting}[language=Python]
from Bio import SeqIO
import pandas as pd

# Carregar e analisar sequências de DNA
sequences = []
for record in SeqIO.parse("sequences.fasta", "fasta"):
    gc_content = (record.seq.count("G") + record.seq.count("C")) / len(record.seq)
    sequences.append({
        "id": record.id,
        "length": len(record.seq),
        "gc_content": gc_content
    })

df = pd.DataFrame(sequences)
print(f"GC médio: {df['gc_content'].mean():.2%}")
\end{lstlisting}
\end{examplebox}

\section{O Ecossistema Científico Python}

O poder do Python para pesquisa não reside apenas na linguagem, mas no seu ecossistema. Bibliotecas como NumPy fornecem computação numérica eficiente, pandas facilita manipulação de dados, matplotlib e seaborn criam visualizações publicáveis, e scikit-learn oferece algoritmos de machine learning prontos para uso.

\subsection{Principais Bibliotecas por Área}

\begin{table}[H]
\centering
\begin{tabular}{@{}ll@{}}
\toprule
\textbf{Área de Pesquisa} & \textbf{Bibliotecas Principais} \\
\midrule
Análise de Dados & pandas, NumPy, SciPy \\
Visualização & matplotlib, seaborn, plotly \\
Machine Learning & scikit-learn, TensorFlow, PyTorch \\
Bioinformática & BioPython, scikit-bio \\
Astronomia & AstroPy \\
Psicologia/Neurociência & PsychoPy, MNE, nilearn \\
Física & SymPy, QuTiP \\
Economia & statsmodels, arch \\
Geografia & GeoPandas, Folium \\
\bottomrule
\end{tabular}
\caption{Bibliotecas Python especializadas por área de pesquisa}
\end{table}

\section{Casos de Sucesso na Academia}

Python tem sido fundamental em descobertas científicas recentes. O projeto de detecção de ondas gravitacionais LIGO utilizou Python extensivamente para análise de dados\footnote{LIGO Scientific Collaboration. "Observation of Gravitational Waves from a Binary Black Hole Merger." Physical Review Letters 116.6 (2016): 061102.}. A primeira imagem de um buraco negro foi processada usando Python. Estudos sobre COVID-19 dependeram fortemente de análises em Python para modelagem epidemiológica.

\begin{researchbox}
\textbf{Caso Real - Análise de Redes Sociais:}

Um pesquisador em sociologia precisava analisar padrões de interação em redes sociais durante crises políticas. Com Python:

\begin{lstlisting}[language=Python]
import networkx as nx
import pandas as pd
from textblob import TextBlob

# Carregar dados de tweets
tweets_df = pd.read_csv('political_tweets.csv')

# Análise de sentimento
tweets_df['sentiment'] = tweets_df['text'].apply(
    lambda x: TextBlob(x).sentiment.polarity
)

# Criar rede de menções
G = nx.DiGraph()
for _, tweet in tweets_df.iterrows():
    mentions = extract_mentions(tweet['text'])
    for mention in mentions:
        G.add_edge(tweet['user'], mention, sentiment=tweet['sentiment'])

# Métricas de rede
print(f"Usuários mais influentes:")
centrality = nx.betweenness_centrality(G)
top_users = sorted(centrality.items(), key=lambda x: x[1], reverse=True)[:10]
for user, score in top_users:
    print(f"{user}: {score:.3f}")
\end{lstlisting}

Este código permite identificar usuários influentes, padrões de polarização e propagação de sentimentos em tempo real.
\end{researchbox}

\section{Reproducibilidade e Transparência}

Uma das maiores contribuições do Python para a pesquisa é promover reproducibilidade. Jupyter Notebooks permitem combinar código, resultados e narrativa em um único documento, facilitando a verificação e replicação de estudos.

\begin{pythonbox}
\begin{lstlisting}[language=Python]
# Exemplo de análise reproduzível
import pandas as pd
import numpy as np
import matplotlib.pyplot as plt
from scipy import stats
import seaborn as sns

# Configurar seed para reprodutibilidade
np.random.seed(42)

# Simular dados de experimento
control_group = np.random.normal(100, 15, 50)
treatment_group = np.random.normal(110, 15, 50)

# Teste estatístico
t_stat, p_value = stats.ttest_ind(control_group, treatment_group)

# Visualização
fig, (ax1, ax2) = plt.subplots(1, 2, figsize=(12, 5))

# Histogramas
ax1.hist(control_group, alpha=0.7, label='Controle', bins=15)
ax1.hist(treatment_group, alpha=0.7, label='Tratamento', bins=15)
ax1.legend()
ax1.set_title('Distribuição dos Grupos')

# Box plot
data_combined = pd.DataFrame({
    'Grupo': ['Controle']*50 + ['Tratamento']*50,
    'Valor': np.concatenate([control_group, treatment_group])
})
sns.boxplot(data=data_combined, x='Grupo', y='Valor', ax=ax2)
ax2.set_title('Comparação entre Grupos')

plt.tight_layout()
plt.savefig('results_analysis.png', dpi=300, bbox_inches='tight')
plt.show()

# Resultados
print(f"Estatística t: {t_stat:.3f}")
print(f"Valor p: {p_value:.3f}")
print(f"Diferença significativa: {'Sim' if p_value < 0.05 else 'Não'}")
\end{lstlisting}
\end{pythonbox}

\section{Desafios e Limitações}

Apesar de suas vantagens, Python apresenta desafios para pesquisadores. A velocidade de execução pode ser limitante para computações intensivas, embora isso seja mitigado por bibliotecas otimizadas e paralelização. A curva de aprendizado, embora suave, ainda requer investimento de tempo que nem todos os pesquisadores podem fazer.

\begin{warningbox}
\textbf{Armadilhas Comuns para Pesquisadores:}
\begin{itemize}
    \item \textbf{Dependências:} Projetos podem quebrar com atualizações de bibliotecas
    \item \textbf{Performance:} Loops Python puros são lentos para grandes datasets
    \item \textbf{Memória:} Carregar datasets inteiros pode esgotar RAM
    \item \textbf{Reprodutibilidade:} Diferentes versões podem gerar resultados distintos
\end{itemize}
\end{warningbox}

\section{O Futuro do Python na Pesquisa}

O futuro do Python na pesquisa acadêmica é promissor. Desenvolvimentos em compilação just-in-time (como Numba), computação distribuída (Dask), e integração com GPUs (CuPy) estão expandindo as fronteiras do possível. A crescente integração com ferramentas de big data e cloud computing posiciona Python como a linguagem central para a ciência de dados do futuro.

\subsection{Tendências Emergentes}

\begin{enumerate}
    \item \textbf{Automated Machine Learning (AutoML):} Bibliotecas como TPOT e Auto-sklearn democratizam ML
    \item \textbf{Computação Quântica:} Qiskit e Cirq trazem quantum computing para Python
    \item \textbf{Análise em Tempo Real:} Streaming de dados com Kafka e PyFlink
    \item \textbf{Colaboração Global:} Plataformas como Google Colab facilitam pesquisa colaborativa
\end{enumerate}

\section{Estrutura deste Livro}

Este livro foi estruturado para maximizar o aprendizado prático. Cada capítulo combina teoria fundamental com exemplos reais de pesquisa, casos de estudo detalhados e exercícios práticos. Os exemplos cobrem diversas disciplinas, desde ciências sociais até física teórica, demonstrando a versatilidade do Python.

Nos próximos capítulos, exploraremos desde conceitos básicos de programação até técnicas avançadas de análise de dados, sempre com foco em aplicações acadêmicas reais. Você aprenderá não apenas a usar Python, mas a pensar como um pesquisador que usa Python para resolver problemas complexos.

\begin{examplebox}
\textbf{O que você conseguirá fazer após este livro:}
\begin{itemize}
    \item Automatizar coleta e limpeza de dados de qualquer fonte
    \item Realizar análises estatísticas robustas e interpretá-las corretamente
    \item Criar visualizações que comuniquem resultados efetivamente
    \item Implementar modelos de machine learning para suas pesquisas
    \item Escrever código reproduzível e compartilhável
    \item Otimizar workflows de pesquisa usando Python
\end{itemize}
\end{examplebox}

A jornada começa agora. Python não é apenas uma ferramenta; é uma forma de pensar sobre problemas de pesquisa de maneira mais eficiente, rigorosa e colaborativa. Prepare-se para transformar sua prática de pesquisa.


% =============================================================================
% CAPÍTULO 2: FUNDAMENTOS DO PYTHON PARA PESQUISADORES
% =============================================================================

\chapter{Fundamentos do Python para Pesquisadores}

Antes de mergulharmos em aplicações específicas, é fundamental estabelecer uma base sólida nos conceitos do Python que são essenciais para pesquisa acadêmica. Este capítulo não é uma introdução tradicional à programação, mas sim um guia focado nas necessidades específicas de pesquisadores que precisam processar dados, automatizar análises e garantir reproducibilidade.

\section{Configurando seu Ambiente de Pesquisa}

O primeiro passo para usar Python efetivamente em pesquisa é configurar um ambiente apropriado. Diferentemente do desenvolvimento de software comercial, a pesquisa acadêmica tem necessidades específicas: reproducibilidade, documentação integrada e capacidade de compartilhamento.

\subsection{Anaconda: A Distribuição Científica}

Anaconda é a distribuição Python padrão para ciência de dados. Ela inclui Python, centenas de bibliotecas científicas e ferramentas de gerenciamento de ambiente que são cruciais para pesquisa reproducível.

\begin{examplebox}
\textbf{Instalação e Configuração Básica:}

\begin{lstlisting}[language=bash,breaklines=true,postbreak=\mbox{\textcolor{red}{$\hookrightarrow$}\space}]
# Baixar Anaconda em https://www.anaconda.com/
# Após instalação, criar ambiente para projeto específico
conda create -n minha_pesquisa python=3.9
conda activate minha_pesquisa

# Instalar bibliotecas essenciais
conda install numpy pandas matplotlib scipy scikit-learn jupyter

# Listar ambientes
conda env list

# Exportar ambiente para reproducibilidade
conda env export > environment.yml
\end{lstlisting}
\end{examplebox}

\subsection{Jupyter Notebooks: Pesquisa Interativa}

Jupyter Notebooks revolucionaram a pesquisa computacional ao permitir combinar código, resultados, visualizações e narrativa em um único documento. Isso é fundamental para a ciência aberta e reproducível.

\begin{researchbox}
\textbf{Estrutura de um Notebook de Pesquisa:}

Um notebook bem estruturado deve conter:
\begin{enumerate}
    \item \textbf{Introdução}: Contexto e objetivos da análise
    \item \textbf{Importações}: Todas as bibliotecas necessárias
    \item \textbf{Carregamento de dados}: Com verificações de integridade
    \item \textbf{Análise exploratória}: Entendimento inicial dos dados
    \item \textbf{Processamento}: Limpeza e transformação
    \item \textbf{Análise principal}: Métodos estatísticos/algoritmos
    \item \textbf{Visualização}: Gráficos interpretativos
    \item \textbf{Conclusões}: Interpretação dos resultados
\end{enumerate}
\end{researchbox}

\section{Tipos de Dados Essenciais para Pesquisa}

Python oferece estruturas de dados flexíveis que se adaptam perfeitamente às necessidades de pesquisa. Compreender quando e como usar cada tipo é crucial para escrever código eficiente e legível.

\subsection{Listas: Coletas Ordenadas}

\begin{examplebox}
\textbf{Exemplo: Análise de Dados Experimentais}

\begin{lstlisting}[language=Python,breaklines=true,postbreak=\mbox{\textcolor{red}{$\hookrightarrow$}\space}]
# Dados de tempo de reação em experimento psicológico
tempos_reacao = [0.45, 0.52, 0.38, 0.61, 0.49, 0.55, 0.42]
condicoes = ["controle", "tratamento_a", "controle", 
            "tratamento_b", "controle", "tratamento_a", "tratamento_b"]

# Operações básicas
print(f"Número de observações: {len(tempos_reacao)}")
print(f"Tempo médio: {sum(tempos_reacao)/len(tempos_reacao):.3f}s")
print(f"Tempo mínimo: {min(tempos_reacao):.3f}s")
print(f"Tempo máximo: {max(tempos_reacao):.3f}s")

# Filtrar dados por condição
tempos_controle = [t for t, c in zip(tempos_reacao, condicoes) 
                   if c == "controle"]
print(f"Tempos no grupo controle: {tempos_controle}")
\end{lstlisting}
\end{examplebox}

\subsection{Dicionários: Dados Estruturados}

\begin{examplebox}
\textbf{Exemplo: Base de Dados de Participantes}

\begin{lstlisting}[language=Python,breaklines=true,postbreak=\mbox{\textcolor{red}{$\hookrightarrow$}\space}]
# Representação de dados de participantes
participantes = [
    {"id": "P001", "idade": 25, "genero": "F", "grupo": "experimental",
     "pre_teste": 78, "pos_teste": 85, "data_coleta": "2024-03-15"},
    {"id": "P002", "idade": 31, "genero": "M", "grupo": "controle",
     "pre_teste": 72, "pos_teste": 74, "data_coleta": "2024-03-16"}
]

# Calcular melhoria média por grupo
def calcular_melhoria_grupo(dados, grupo_alvo):
    melhorias = []
    for p in dados:
        if p["grupo"] == grupo_alvo:
            melhorias.append(p["pos_teste"] - p["pre_teste"])
    return sum(melhorias)/len(melhorias) if melhorias else 0

melhoria_exp = calcular_melhoria_grupo(participantes, "experimental")
melhoria_ctrl = calcular_melhoria_grupo(participantes, "controle")

print(f"Melhoria grupo experimental: {melhoria_exp:.1f} pontos")
print(f"Melhoria grupo controle: {melhoria_ctrl:.1f} pontos")
\end{lstlisting}
\end{examplebox}

\section{Funções: Modularizando sua Pesquisa}

\begin{examplebox}
\textbf{Exemplo: Função para Análise Estatística}

\begin{lstlisting}[language=Python,breaklines=true,postbreak=\mbox{\textcolor{red}{$\hookrightarrow$}\space}]
import math

def teste_t_pareado(antes, depois, alpha=0.05):
    """
    Realiza teste t pareado para amostras dependentes.
    Retorna dict com t, p-valor e conclusão.
    """
    if len(antes) != len(depois):
        raise ValueError("Listas devem ter mesmo tamanho")
    if len(antes) < 3:
        raise ValueError("Tamanho amostral muito pequeno (n < 3)")

    diferencas = [d - a for a, d in zip(antes, depois)]
    n = len(diferencas)
    media_diff = sum(diferencas)/n
    var_diff = sum((d - media_diff)**2 for d in diferencas)/(n-1)
    erro_padrao = math.sqrt(var_diff/n)
    t_stat = media_diff/erro_padrao
    t_critico = 2.262 if n-1 <= 9 else 1.96
    p_valor = 0.05 if abs(t_stat) > t_critico else 0.10
    conclusao = "Diferença significativa" if p_valor<alpha else "Diferença não significativa"
    return {"t_statistic": t_stat, "p_valor": p_valor, "conclusao": conclusao}

dados_pre = [72, 68, 75, 71, 69, 74, 70]
dados_pos = [78, 72, 80, 76, 74, 79, 75]
resultado = teste_t_pareado(dados_pre, dados_pos)
print(f"t: {resultado['t_statistic']:.3f}, p: {resultado['p_valor']:.3f}, {resultado['conclusao']}")
\end{lstlisting}
\end{examplebox}

\section{Trabalhando com Arquivos e Dados}

\subsection{Lendo Dados CSV}

\begin{examplebox}
\textbf{Exemplo: Processamento de Dados de Survey}

\begin{lstlisting}[language=Python,breaklines=true,postbreak=\mbox{\textcolor{red}{$\hookrightarrow$}\space}]
import csv
from collections import defaultdict

def analisar_survey(arquivo_csv):
    dados = []
    with open(arquivo_csv,'r',encoding='utf-8') as f:
        leitor = csv.DictReader(f)
        for linha in leitor:
            linha['idade'] = int(linha['idade'])
            linha['satisfacao'] = float(linha['satisfacao'])
            dados.append(linha)
    stats_departamento = defaultdict(list)
    for pessoa in dados:
        dept = pessoa['departamento']
        stats_departamento[dept].append({'idade':pessoa['idade'],'satisfacao':pessoa['satisfacao']})
    resultados = {}
    for dept,pessoas in stats_departamento.items():
        idades = [p['idade'] for p in pessoas]
        satisfacoes = [p['satisfacao'] for p in pessoas]
        resultados[dept] = {'n':len(pessoas),
                            'idade_media':sum(idades)/len(idades),
                            'satisfacao_media':sum(satisfacoes)/len(satisfacoes),
                            'idade_min':min(idades),'idade_max':max(idades)}
    return resultados

# Criar arquivo exemplo
dados_exemplo = ["id,idade,satisfacao,departamento",
                 "1,25,7.5,TI",
                 "2,32,8.2,RH",
                 "3,28,6.8,TI",
                 "4,45,9.1,Financeiro",
                 "5,31,7.9,RH"]
with open('survey_exemplo.csv','w') as f: f.write('\n'.join(dados_exemplo))

resultados = analisar_survey('survey_exemplo.csv')
for dept, stats in resultados.items():
    print(f"\n{dept}: N={stats['n']}, Idade média={stats['idade_media']:.1f}, Satisfação média={stats['satisfacao_media']:.1f}")
\end{lstlisting}
\end{examplebox}

\section{Tratamento de Erros e Boas Práticas}

\begin{examplebox}
\textbf{Exemplo: Limpeza Robusta de Dados}

\begin{lstlisting}[language=Python,breaklines=true,postbreak=\mbox{\textcolor{red}{$\hookrightarrow$}\space}]
def limpar_dados_numericos(dados, nome_coluna, valor_min=None, valor_max=None):
    dados_limpos = []
    problemas = []
    for i, registro in enumerate(dados):
        try:
            valor = float(registro[nome_coluna])
            if valor_min is not None and valor<valor_min:
                problemas.append(f"Linha {i+1}: {valor} abaixo do mínimo {valor_min}")
                continue
            if valor_max is not None and valor>valor_max:
                problemas.append(f"Linha {i+1}: {valor} acima do máximo {valor_max}")
                continue
            registro_limpo = registro.copy()
            registro_limpo[nome_coluna] = valor
            dados_limpos.append(registro_limpo)
        except ValueError:
            problemas.append(f"Linha {i+1}: valor '{registro[nome_coluna]}' não é numérico")
        except KeyError:
            problemas.append(f"Linha {i+1}: coluna '{nome_coluna}' não encontrada")
    relatorio = {'total_original':len(dados),'total_limpo':len(dados_limpos),
                 'removidos':len(dados)-len(dados_limpos),'problemas':problemas}
    return dados_limpos, relatorio

# Exemplo
dados_raw = [{'id':'1','score':'85'},
             {'id':'2','score':'92'},
             {'id':'3','score':'NA'},
             {'id':'4','score':'150'},
             {'id':'5','score':'78'},
             {'id':'6','score':'-5'}]
dados_limpos, relatorio = limpar_dados_numericos(dados_raw,'score',0,100)
print(relatorio)
\end{lstlisting}
\end{examplebox}

\section{Documentação e Comentários}

\begin{warningbox}
\textbf{Princípios de Documentação para Pesquisadores:}
\begin{enumerate}
    \item Explique o “porquê”, não apenas o “como”
    \item Documente suposições e parâmetros
    \item Cite fontes e referências
    \item Registre limitações do código
\end{enumerate}
\end{warningbox}

\begin{examplebox}
\textbf{Exemplo: Documentação Científica Apropriada}

\begin{lstlisting}[language=Python,breaklines=true,postbreak=\mbox{\textcolor{red}{$\hookrightarrow$}\space}]
def calcular_cohen_d(grupo1, grupo2):
    """
    Calcula Cohen's d (tamanho do efeito) entre dois grupos.
    Retorna valor float.
    """
    import math
    media1 = sum(grupo1)/len(grupo1)
    media2 = sum(grupo2)/len(grupo2)
    var1 = sum((x-media1)**2 for x in grupo1)/(len(grupo1)-1)
    var2 = sum((x-media2)**2 for x in grupo2)/(len(grupo2)-1)
    n1,n2 = len(grupo1),len(grupo2)
    dp_pooled = math.sqrt(((n1-1)*var1 + (n2-1)*var2)/(n1+n2-2))
    return (media1-media2)/dp_pooled

experimental = [78,82,85,79,88,84,81]
controle = [72,75,71,74,73,76,70]
d = calcular_cohen_d(experimental,controle)
print(f"Cohen's d={d:.3f}")
\end{lstlisting}
\end{examplebox}

\section{Exercícios Práticos}

\begin{examplebox}
\textbf{Exercício 1: Estatísticas Descritivas}  
Crie função que calcule média, mediana, desvio padrão e quartis para uma lista, tratando valores ausentes.

\textbf{Exercício 2: Leitor de Dados Robusto}  
Escreva função que leia CSV e detecte tipo de cada coluna (numérico, categórico, data).

\textbf{Exercício 3: Validador de Dados Experimentais}  
Implemente função que valide dados verificando duplicatas, outliers e padrões suspeitos.
\end{examplebox}


Este capítulo estabeleceu as bases fundamentais do Python para pesquisa. No próximo capítulo, exploraremos como trabalhar efetivamente com as principais bibliotecas científicas (NumPy, pandas, Matplotlib) que são essenciais para análise de dados em pesquisa acadêmica.

% =============================================================================
% CAPÍTULO 3: BIBLIOTECAS CIENTÍFICAS ESSENCIAIS
% =============================================================================

\chapter{Bibliotecas Científicas Essenciais: NumPy, Pandas e Matplotlib}

Embora Python puro seja poderoso, o verdadeiro potencial para pesquisa acadêmica surge quando combinamos a linguagem com suas bibliotecas científicas especializadas. Este capítulo explora as três bibliotecas fundamentais que formam a espinha dorsal de praticamente qualquer projeto de análise de dados: NumPy para computação numérica, pandas para manipulação de dados estruturados, e matplotlib para visualização.

\section{NumPy: Computação Numérica Eficiente}

NumPy (Numerical Python) é a base de todo o ecossistema científico Python. Ele fornece arrays multidimensionais eficientes e operações matemáticas otimizadas que são ordens de magnitude mais rápidas que listas Python puras.

\subsection{Por que NumPy é Crucial para Pesquisa}

A diferença de performance entre listas Python e arrays NumPy pode ser a diferença entre uma análise que leva minutos versus horas, especialmente com datasets grandes comuns em pesquisa moderna.

\begin{examplebox}
\textbf{Comparação de Performance: Python vs NumPy}

\begin{lstlisting}[language=Python]
import numpy as np
import time

# Dados de exemplo: medicoes de 1 milhao de participantes
n = 1000000

# Metodo 1: Listas Python puras
dados_python = list(range(n))
inicio = time.time()
resultado_python = [x**2 for x in dados_python]
tempo_python = time.time() - inicio

# Metodo 2: Arrays NumPy
dados_numpy = np.arange(n)
inicio = time.time()
resultado_numpy = dados_numpy**2
tempo_numpy = time.time() - inicio

print(f"Tempo Python puro: {tempo_python:.3f} segundos")
print(f"Tempo NumPy: {tempo_numpy:.3f} segundos")
print(f"NumPy e {tempo_python/tempo_numpy:.1f}x mais rapido")

# Verificar se resultados sao iguais
print(f"Resultados identicos: {np.array_equal(resultado_python, resultado_numpy)}")
\end{lstlisting}
\end{examplebox}

\subsection{Arrays NumPy: Fundamentos}

Arrays NumPy são estruturas homogêneas (todos elementos do mesmo tipo) que permitem operações vetorizadas eficientes.

\begin{examplebox}
\textbf{Criação e Manipulação Básica de Arrays}

\begin{lstlisting}[language=Python]
import numpy as np

# Diferentes maneiras de criar arrays
dados_experimentais = np.array([1.2, 1.5, 1.3, 1.7, 1.1])
zeros = np.zeros(10)  # Array de zeros
uns = np.ones((3, 4))  # Matriz 3x4 de uns
sequencia = np.arange(0, 10, 0.5)  # De 0 a 10 com passo 0.5
linear = np.linspace(0, 100, 50)  # 50 pontos igualmente espaçados

# Propriedades importantes
print(f"Forma: {dados_experimentais.shape}")
print(f"Tipo de dados: {dados_experimentais.dtype}")
print(f"Numero de dimensoes: {dados_experimentais.ndim}")
print(f"Tamanho total: {dados_experimentais.size}")

# Operacoes basicas (vetorizadas)
print(f"Media: {np.mean(dados_experimentais):.3f}")
print(f"Desvio padrao: {np.std(dados_experimentais):.3f}")
print(f"Maximo: {np.max(dados_experimentais):.3f}")
print(f"Minimo: {np.min(dados_experimentais):.3f}")

# Operacoes elemento a elemento
dados_normalizados = (dados_experimentais - np.mean(dados_experimentais)) / np.std(dados_experimentais)
print(f"Dados normalizados: {dados_normalizados}")
\end{lstlisting}
\end{examplebox}

\subsection{Arrays Multidimensionais para Dados Complexos}

Dados de pesquisa frequentemente têm múltiplas dimensões: participantes × condições × medições temporais.

\begin{researchbox}
\textbf{Exemplo: Análise de Dados de EEG}

\begin{lstlisting}[language=Python]
# Simular dados de EEG: 64 eletrodos, 1000 pontos temporais, 100 trials
n_eletrodos = 64
n_tempos = 1000
n_trials = 100

# Criar dados simulados (normalmente carregados de arquivo)
np.random.seed(42)
dados_eeg = np.random.randn(n_eletrodos, n_tempos, n_trials)

print(f"Forma dos dados: {dados_eeg.shape}")
print(f"Memoria ocupada: {dados_eeg.nbytes / 1024**2:.1f} MB")

# Calcular media por eletrodo atraves dos trials
media_eletrodos = np.mean(dados_eeg, axis=2)  # media na dimensao trials
print(f"Forma da media: {media_eletrodos.shape}")

# Encontrar eletrodo com maior atividade media
atividade_media = np.mean(np.abs(media_eletrodos), axis=1)
eletrodo_max = np.argmax(atividade_media)
print(f"Eletrodo mais ativo: {eletrodo_max} (atividade: {atividade_media[eletrodo_max]:.3f})")
\end{lstlisting}
\end{researchbox}

\begin{researchbox}
\textbf{Continuação: Análise de Dados de EEG}

\begin{lstlisting}[language=Python]
# Calcular potencial evocado (media atraves de trials)
potencial_evocado = np.mean(dados_eeg, axis=2)

# Encontrar pico máximo no tempo
tempo_max = np.unravel_index(np.argmax(np.abs(potencial_evocado)), 
                            potencial_evocado.shape)
print(f"Pico maximo no eletrodo {tempo_max[0]}, tempo {tempo_max[1]}")
\end{lstlisting}
\end{researchbox}

\subsection{Indexação Avançada e Máscaras Booleanas}

Uma das características mais poderosas do NumPy é a capacidade de filtrar e selecionar dados baseado em condições complexas.

\begin{examplebox}
\textbf{Filtragem Avançada de Dados Experimentais - Parte 1}

\begin{lstlisting}[language=Python]
# Dados de um experimento: tempo de reacao e acuracia
n_participantes = 1000
np.random.seed(123)

tempos_reacao = np.random.normal(500, 100, n_participantes)  # ms
acuracia = np.random.beta(8, 2, n_participantes)  # proporção entre 0-1
idades = np.random.randint(18, 65, n_participantes)
grupos = np.random.choice(['controle', 'experimental'], n_participantes)

# Criar mascara booleana para participantes validos
# (tempo de reacao entre 200-1000ms e acuracia > 50%)
mascara_validos = (tempos_reacao >= 200) & (tempos_reacao <= 1000) & (acuracia > 0.5)

print(f"Participantes totais: {n_participantes}")
print(f"Participantes validos: {np.sum(mascara_validos)}")
print(f"Taxa de exclusao: {(1 - np.mean(mascara_validos))*100:.1f}%")

# Aplicar filtros
tempos_validos = tempos_reacao[mascara_validos]
acuracia_valida = acuracia[mascara_validos]
idades_validas = idades[mascara_validos]
grupos_validos = grupos[mascara_validos]
\end{lstlisting}
\end{examplebox}

\begin{examplebox}
\textbf{Filtragem Avançada de Dados Experimentais - Parte 2}

\begin{lstlisting}[language=Python]
# Aalise por grupo usando máscaras
mascara_controle = grupos_validos == 'controle'
mascara_experimental = grupos_validos == 'experimental'

print(f"\nGrupo Controle (n={np.sum(mascara_controle)}):")
print(f"  Tempo reacao medio: {np.mean(tempos_validos[mascara_controle]):.1f} ms")
print(f"  Acuracia media: {np.mean(acuracia_valida[mascara_controle])*100:.1f}%")

print(f"\nGrupo Experimental (n={np.sum(mascara_experimental)}):")
print(f"  Tempo reacao medio: {np.mean(tempos_validos[mascara_experimental]):.1f} ms")
print(f"  Acuracia media: {np.mean(acuracia_valida[mascara_experimental])*100:.1f}%")

# Analise por faixa etária
jovens = idades_validas < 30
adultos = (idades_validas >= 30) & (idades_validas < 50)
idosos = idades_validas >= 50

print(f"\nPor faixa etaria:")
for nome, mascara in [('Jovens', jovens), ('Adultos', adultos), ('Idosos', idosos)]:
    if np.sum(mascara) > 0:
        print(f"  {nome}: TR = {np.mean(tempos_validos[mascara]):.1f} ms")
\end{lstlisting}
\end{examplebox}

\section{Pandas: Manipulação de Dados Estruturados}

Pandas é construído sobre NumPy mas oferece estruturas de dados mais flexíveis e intuitivas para trabalhar com dados heterogêneos, como planilhas e bancos de dados.

\subsection{DataFrames: Planilhas Inteligentes}

DataFrames são a estrutura central do pandas, similares a planilhas Excel mas com capacidades muito mais avançadas.

\begin{examplebox}
\textbf{Criação e Exploração Básica de DataFrames}

\begin{lstlisting}[language=Python]
import pandas as pd
import numpy as np

# Criar DataFrame a partir de dicionario
dados_participantes = {
    'id': range(1, 101),
    'idade': np.random.randint(18, 65, 100),
    'genero': np.random.choice(['M', 'F'], 100),
    'grupo': np.random.choice(['controle', 'tratamento'], 100),
    'pre_teste': np.random.normal(50, 10, 100),
    'pos_teste': np.random.normal(55, 12, 100),
    'satisfacao': np.random.randint(1, 11, 100)
}

df = pd.DataFrame(dados_participantes)

# Exploracao inicial
print("Informacoes basicas:")
print(df.info())
print(f"\nForma: {df.shape}")
print(f"\nPrimeiras 5 linhas:")
print(df.head())

print(f"\nEstatisticas descritivas:")
print(df.describe())

# Verificar valores ausentes
print(f"\nValores ausentes por coluna:")
print(df.isnull().sum())
\end{lstlisting}
\end{examplebox}

\subsection{Operações de Agrupamento e Análise}

Pandas excela em operações group-by que são fundamentais para análise experimental.

\begin{researchbox}
\textbf{Análise de Eficácia de Tratamento - Parte 1}

\begin{lstlisting}[language=Python]
# Calcular melhoria (diferença pré-pós teste)
df['melhoria'] = df['pos_teste'] - df['pre_teste']

# Analise por grupo
analise_grupo = df.groupby('grupo').agg({
    'idade': ['mean', 'std', 'count'],
    'pre_teste': ['mean', 'std'],
    'pos_teste': ['mean', 'std'],
    'melhoria': ['mean', 'std'],
    'satisfacao': ['mean', 'std']
}).round(2)

print("Análise por grupo:")
print(analise_grupo)

# Análise mais detalhada com múltiplas variáveis
analise_detalhada = df.groupby(['grupo', 'genero']).agg({
    'melhoria': ['count', 'mean', 'std'],
    'satisfacao': 'mean'
}).round(2)

print(f"\nAnálise por grupo e gênero:")
print(analise_detalhada)
\end{lstlisting}
\end{researchbox}

\begin{researchbox}
\textbf{Análise de Eficácia de Tratamento - Parte 2}

\begin{lstlisting}[language=Python]
# Calcular tamanho do efeito simples
def cohen_d(grupo1, grupo2):
    """Calcula Cohen's d entre dois grupos"""
    n1, n2 = len(grupo1), len(grupo2)
    s1, s2 = grupo1.std(), grupo2.std()
    
    # Pooled standard deviation
    s_pooled = np.sqrt(((n1-1)*s1**2 + (n2-1)*s2**2) / (n1+n2-2))
    
    return (grupo1.mean() - grupo2.mean()) / s_pooled

controle = df[df['grupo'] == 'controle']['melhoria']
tratamento = df[df['grupo'] == 'tratamento']['melhoria']

d = cohen_d(tratamento, controle)
print(f"\nTamanho do efeito (Cohen's d): {d:.3f}")

# Interpretacao
if abs(d) < 0.2:
    interpretacao = "trivial"
elif abs(d) < 0.5:
    interpretacao = "pequeno"
elif abs(d) < 0.8:
    interpretacao = "médio"
else:
    interpretacao = "grande"

print(f"Interpretação: efeito {interpretacao}")
\end{lstlisting}
\end{researchbox}

\subsection{Limpeza e Transformação de Dados}

Dados reais sempre precisam de limpeza. Pandas oferece ferramentas poderosas para isso.

\begin{examplebox}
\textbf{Pipeline Completo de Limpeza de Dados - Parte 1}

\begin{lstlisting}[language=Python]
# Simular dados com problemas comuns
np.random.seed(42)
dados_sujos = pd.DataFrame({
    'participante_id': range(1, 201),
    'idade': np.random.randint(16, 80, 200),
    'score': np.random.normal(75, 15, 200),
    'grupo': np.random.choice(['A', 'B', 'C'], 200),
    'data_coleta': pd.date_range('2024-01-01', periods=200, freq='D')
})

# Introduzir problemas nos dados
# 1. Valores ausentes
indices_na = np.random.choice(200, 20, replace=False)
dados_sujos.loc[indices_na, 'score'] = np.nan

# 2. Outliers extremos
indices_outliers = np.random.choice(200, 5, replace=False)
dados_sujos.loc[indices_outliers, 'score'] = [200, -50, 300, -100, 250]

# 3. Idades impossíveis
dados_sujos.loc[190:195, 'idade'] = [150, 5, 200, 0, -10]

print("Dados antes da limpeza:")
print(f"Valores ausentes: {dados_sujos.isnull().sum().sum()}")
print(f"Forma: {dados_sujos.shape}")
print(f"\nEstatísticas do score:")
print(dados_sujos['score'].describe())
\end{lstlisting}
\end{examplebox}

\begin{examplebox}
\textbf{Pipeline Completo de Limpeza de Dados - Parte 2}

\begin{lstlisting}[language=Python]
# Pipeline de limpeza
dados_limpos = dados_sujos.copy()

# 1. Filtrar idades válidas (18-65)
mask_idade_valida = (dados_limpos['idade'] >= 18) & (dados_limpos['idade'] <= 65)
dados_limpos = dados_limpos[mask_idade_valida]

# 2. Remover outliers extremos do score (fora de 3 desvios padrão)
mean_score = dados_limpos['score'].mean()
std_score = dados_limpos['score'].std()
limite_inferior = mean_score - 3 * std_score
limite_superior = mean_score + 3 * std_score

mask_score_valido = (dados_limpos['score'] >= limite_inferior) & \
                   (dados_limpos['score'] <= limite_superior)
dados_limpos = dados_limpos[mask_score_valido | dados_limpos['score'].isnull()]

# 3. Tratar valores ausentes - Imputar pela média do grupo
dados_limpos['score'] = dados_limpos.groupby('grupo')['score'].transform(
    lambda x: x.fillna(x.mean())
)
\end{lstlisting}
\end{examplebox}

\begin{examplebox}
\textbf{Pipeline Completo de Limpeza de Dados - Parte 3}

\begin{lstlisting}[language=Python]
# 4. Criar variáveis derivadas
dados_limpos['faixa_etaria'] = pd.cut(dados_limpos['idade'], 
                                     bins=[0, 30, 50, 100], 
                                     labels=['Jovem', 'Adulto', 'Idoso'])

dados_limpos['score_categoria'] = pd.cut(dados_limpos['score'],
                                        bins=[0, 60, 80, 100],
                                        labels=['Baixo', 'Médio', 'Alto'])

print(f"\nDados após limpeza:")
print(f"Valores ausentes: {dados_limpos.isnull().sum().sum()}")
print(f"Forma: {dados_limpos.shape}")
print(f"Registros removidos: {len(dados_sujos) - len(dados_limpos)}")

print(f"\nDistribuição por faixa etária:")
print(dados_limpos['faixa_etaria'].value_counts())

print(f"\nDistribuição por categoria de score:")
print(dados_limpos['score_categoria'].value_counts())
\end{lstlisting}
\end{examplebox}

\section{Matplotlib: Visualização Científica}

Visualização é crucial para comunicar resultados de pesquisa. Matplotlib oferece controle completo sobre cada aspecto dos gráficos.

\subsection{Gráficos Básicos para Publicação}

\begin{examplebox}
\textbf{Gráficos Prontos para Publicação - Configuração Inicial}

\begin{lstlisting}[language=Python]
import matplotlib.pyplot as plt
import seaborn as sns

# Configurar estilo para publicação
plt.style.use('seaborn-v0_8-whitegrid')
plt.rcParams.update({
    'font.size': 12,
    'axes.labelsize': 14,
    'axes.titlesize': 16,
    'xtick.labelsize': 12,
    'ytick.labelsize': 12,
    'legend.fontsize': 12,
    'figure.titlesize': 18
})

# Dados para visualização
grupos = ['Controle', 'Tratamento A', 'Tratamento B']
pre_teste = [72.5, 73.1, 72.8]
pos_teste = [74.2, 81.3, 85.7]
erros_pre = [2.1, 2.3, 2.0]
erros_pos = [2.5, 2.8, 3.1]

# Criar figura com múltiplos subplots
fig, ((ax1, ax2), (ax3, ax4)) = plt.subplots(2, 2, figsize=(14, 10))
\end{lstlisting}
\end{examplebox}

\begin{examplebox}
\textbf{Gráficos Prontos para Publicação - Gráfico de Barras}

\begin{lstlisting}[language=Python]
# 1. Gráfico de barras com barras de erro
x_pos = np.arange(len(grupos))
largura = 0.35

barras1 = ax1.bar(x_pos - largura/2, pre_teste, largura, 
                  yerr=erros_pre, label='Pré-teste', alpha=0.8)
barras2 = ax1.bar(x_pos + largura/2, pos_teste, largura,
                  yerr=erros_pos, label='Pós-teste', alpha=0.8)

ax1.set_xlabel('Grupos')
ax1.set_ylabel('Score Médio')
ax1.set_title('Comparação Pré vs Pós-teste')
ax1.set_xticks(x_pos)
ax1.set_xticklabels(grupos)
ax1.legend()
ax1.set_ylim(65, 90)

# Adicionar valores nas barras
for barra in barras1:
    altura = barra.get_height()
    ax1.text(barra.get_x() + barra.get_width()/2., altura + 1,
             f'{altura:.1f}', ha='center', va='bottom')

for barra in barras2:
    altura = barra.get_height()
    ax1.text(barra.get_x() + barra.get_width()/2., altura + 1,
             f'{altura:.1f}', ha='center', va='bottom')
\end{lstlisting}
\end{examplebox}

\begin{examplebox}
\textbf{Gráficos Prontos para Publicação - Boxplot e Outros}

\begin{lstlisting}[language=Python]
# 2. Boxplot
dados_boxplot = [
    np.random.normal(72.5, 8, 50),  # Controle
    np.random.normal(81.3, 9, 50),  # Tratamento A
    np.random.normal(85.7, 7, 50)   # Tratamento B
]

bp = ax2.boxplot(dados_boxplot, labels=grupos, patch_artist=True)
ax2.set_ylabel('Score')
ax2.set_title('Distribuição dos Scores por Grupo')

# Colorir boxplots
cores = ['lightblue', 'lightgreen', 'lightcoral']
for patch, cor in zip(bp['boxes'], cores):
    patch.set_facecolor(cor)

# 3. Gráfico de linha temporal
tempos = np.arange(0, 11)  # 0 a 10 semanas
grupo_controle = 72 + 0.2 * tempos + np.random.normal(0, 1, len(tempos))
grupo_trat = 72 + 1.5 * tempos + np.random.normal(0, 1.5, len(tempos))

ax3.plot(tempos, grupo_controle, 'o-', label='Controle', linewidth=2, markersize=6)
ax3.plot(tempos, grupo_trat, 's-', label='Tratamento', linewidth=2, markersize=6)
ax3.set_xlabel('Semanas')
ax3.set_ylabel('Score')
ax3.set_title('Evolução Temporal')
ax3.legend()
ax3.grid(True, alpha=0.3)
\end{lstlisting}
\end{examplebox}

\begin{examplebox}
\textbf{Gráficos Prontos para Publicação - Scatter Plot Final}

\begin{lstlisting}[language=Python]
# 4. Scatter plot com linha de tendência
np.random.seed(42)
idade = np.random.randint(20, 60, 100)
score = 60 + 0.5 * idade + np.random.normal(0, 5, 100)

ax4.scatter(idade, score, alpha=0.6, s=50)

# Adicionar linha de tendência
z = np.polyfit(idade, score, 1)
p = np.poly1d(z)
ax4.plot(idade, p(idade), "r--", alpha=0.8, linewidth=2)

# Calcular e mostrar correlação
correlacao = np.corrcoef(idade, score)[0, 1]
ax4.text(0.05, 0.95, f'r = {correlacao:.3f}', transform=ax4.transAxes,
         bbox=dict(boxstyle="round", facecolor='wheat', alpha=0.8))

ax4.set_xlabel('Idade')
ax4.set_ylabel('Score')
ax4.set_title('Relação Idade vs Performance')

plt.tight_layout()
plt.savefig('analise_completa.png', dpi=300, bbox_inches='tight')
plt.show()
\end{lstlisting}
\end{examplebox}

\subsection{Visualizações Avançadas para Pesquisa}

\begin{researchbox}
\textbf{Heatmap de Correlações - Preparação dos Dados}

\begin{lstlisting}[language=Python]
# Criar dataset multivariado simulado
np.random.seed(123)
n_participantes = 200

# Variáveis correlacionadas para simular dados reais
dados_pesquisa = pd.DataFrame({
    'idade': np.random.randint(18, 65, n_participantes),
    'educacao_anos': np.random.randint(8, 20, n_participantes),
    'renda': np.random.normal(50000, 15000, n_participantes),
    'stress_percebido': np.random.randint(1, 11, n_participantes),
    'satisfacao_vida': np.random.randint(1, 11, n_participantes),
    'horas_exercicio': np.random.exponential(3, n_participantes),
    'qualidade_sono': np.random.randint(1, 11, n_participantes)
})

# Introduzir correlações realistas
dados_pesquisa['renda'] += dados_pesquisa['educacao_anos'] * 2000
dados_pesquisa['satisfacao_vida'] = 10 - dados_pesquisa['stress_percebido'] + \
                                   np.random.normal(0, 1, n_participantes)
dados_pesquisa['qualidade_sono'] = 10 - dados_pesquisa['stress_percebido'] * 0.5 + \
                                  dados_pesquisa['horas_exercicio'] * 0.3 + \
                                  np.random.normal(0, 1, n_participantes)

# Limitar valores aos ranges apropriados
dados_pesquisa['satisfacao_vida'] = np.clip(dados_pesquisa['satisfacao_vida'], 1, 10)
dados_pesquisa['qualidade_sono'] = np.clip(dados_pesquisa['qualidade_sono'], 1, 10)
dados_pesquisa['horas_exercicio'] = np.clip(dados_pesquisa['horas_exercicio'], 0, 15)
\end{lstlisting}
\end{researchbox}

\begin{researchbox}
\textbf{Heatmap de Correlações - Visualizações 1-3}

\begin{lstlisting}[language=Python]
# Criar visualização complexa
fig = plt.figure(figsize=(16, 12))

# 1. Heatmap de correlações
ax1 = plt.subplot(2, 3, 1)
correlacoes = dados_pesquisa.corr()
mask = np.triu(np.ones_like(correlacoes, dtype=bool))
sns.heatmap(correlacoes, mask=mask, annot=True, cmap='coolwarm', center=0,
            square=True, linewidths=.5, cbar_kws={"shrink": .8})
plt.title('Matriz de Correlações')

# 2. Distribuição de idades por quartis de satisfação
ax2 = plt.subplot(2, 3, 2)
dados_pesquisa['quartil_satisfacao'] = pd.qcut(dados_pesquisa['satisfacao_vida'], 
                                              4, labels=['Q1', 'Q2', 'Q3', 'Q4'])
dados_pesquisa.boxplot(column='idade', by='quartil_satisfacao', ax=ax2)
plt.title('Idade por Quartil de Satisfação')
plt.suptitle('')  # Remove título automático do pandas

# 3. Scatter plot matriz
ax3 = plt.subplot(2, 3, 3)
scatter = ax3.scatter(dados_pesquisa['stress_percebido'], 
                     dados_pesquisa['qualidade_sono'],
                     c=dados_pesquisa['horas_exercicio'], 
                     s=dados_pesquisa['idade'],
                     alpha=0.6, cmap='viridis')
ax3.set_xlabel('Stress Percebido')
ax3.set_ylabel('Qualidade do Sono')
ax3.set_title('Stress vs Sono (cor=exercício, tamanho=idade)')
cbar = plt.colorbar(scatter, ax=ax3)
cbar.set_label('Horas de Exercício')
\end{lstlisting}
\end{researchbox}

\begin{researchbox}
\textbf{Heatmap de Correlações - Visualizações 4-6}

\begin{lstlisting}[language=Python]
# 4. Histograma de distribuições
ax4 = plt.subplot(2, 3, 4)
dados_pesquisa[['stress_percebido', 'satisfacao_vida', 'qualidade_sono']].hist(
    bins=10, alpha=0.7, ax=ax4)
plt.title('Distribuições das Variáveis Principais')

# 5. Gráfico de violin
ax5 = plt.subplot(2, 3, 5)
# Categorizar exercício em grupos
dados_pesquisa['grupo_exercicio'] = pd.cut(dados_pesquisa['horas_exercicio'],
                                          bins=3, labels=['Baixo', 'Médio', 'Alto'])
sns.violinplot(data=dados_pesquisa, x='grupo_exercicio', y='satisfacao_vida', ax=ax5)
plt.title('Satisfação por Nível de Exercício')

# 6. Regressão linear
ax6 = plt.subplot(2, 3, 6)
sns.regplot(data=dados_pesquisa, x='educacao_anos', y='renda', ax=ax6, scatter_kws={'alpha':0.6})
plt.title('Educação vs Renda')

plt.tight_layout()
plt.savefig('analise_multivariada.png', dpi=300, bbox_inches='tight')
plt.show()
\end{lstlisting}
\end{researchbox}

\begin{researchbox}
\textbf{Heatmap de Correlações - Relatório Final}

\begin{lstlisting}[language=Python]
# Relatório estatístico
print("RELATÓRIO DE ANÁLISE MULTIVARIADA")
print("="*50)
print(f"N = {len(dados_pesquisa)} participantes")
print(f"\nCorrelações mais fortes:")
correlacoes_abs = correlacoes.abs()
np.fill_diagonal(correlacoes_abs.values, 0)
maior_corr = correlacoes_abs.stack().nlargest(3)
for i, (vars, corr) in enumerate(maior_corr.items(), 1):
    print(f"{i}. {vars[0]} vs {vars[1]}: r = {correlacoes.loc[vars[0], vars[1]]:.3f}")
\end{lstlisting}
\end{researchbox}

\section{Integração das Três Bibliotecas}

O verdadeiro poder surge quando combinamos NumPy, pandas e Matplotlib em um workflow integrado.

\begin{examplebox}
\textbf{Workflow Completo - Configuração e Coleta}

\begin{lstlisting}[language=Python]
# Simulação de pipeline completo de pesquisa
def pipeline_completo_pesquisa():
    """
    Demonstra workflow típico: dados -> limpeza -> análise -> visualização
    """
    print("PIPELINE COMPLETO DE ANÁLISE DE PESQUISA")
    print("="*50)
    
    # 1. COLETA DE DADOS (simulada)
    print("1. Coletando dados...")
    np.random.seed(42)
    
    dados_brutos = pd.DataFrame({
        'participante': range(1, 301),
        'grupo': np.random.choice(['controle', 'experimental'], 300),
        'pre_teste': np.random.normal(50, 10, 300),
        'pos_teste': np.random.normal(55, 12, 300),
        'idade': np.random.randint(18, 65, 300),
        'genero': np.random.choice(['M', 'F'], 300),
        'tempo_reacao': np.random.lognormal(6, 0.3, 300)  # Distribuição realista
    })
    
    # Introduzir efeito realista do tratamento
    mask_experimental = dados_brutos['grupo'] == 'experimental'
    dados_brutos.loc[mask_experimental, 'pos_teste'] += np.random.normal(8, 3, mask_experimental.sum())
    
    print(f"   Dados coletados: {len(dados_brutos)} participantes")
    return dados_brutos
\end{lstlisting}
\end{examplebox}

\begin{examplebox}
\textbf{Workflow Completo - Limpeza e Análise}

\begin{lstlisting}[language=Python]
def pipeline_limpeza_analise(dados_brutos):
    # 2. LIMPEZA DE DADOS
    print("2. Limpando dados...")
    
    # Remover outliers extremos no tempo de reação (> 3 DP)
    tr_mean = dados_brutos['tempo_reacao'].mean()
    tr_std = dados_brutos['tempo_reacao'].std()
    mask_tr_valido = np.abs(dados_brutos['tempo_reacao'] - tr_mean) <= 3 * tr_std
    
    dados_limpos = dados_brutos[mask_tr_valido].copy()
    print(f"   Outliers removidos: {len(dados_brutos) - len(dados_limpos)}")
    
    # 3. ANALISE ESTATISTICA
    print("3. Realizando análises...")
    
    # Calcular melhoria
    dados_limpos['melhoria'] = dados_limpos['pos_teste'] - dados_limpos['pre_teste']
    
    # Analise por grupo
    resultados_grupo = dados_limpos.groupby('grupo')['melhoria'].agg(['count', 'mean', 'std'])
    print(f"   Resultados por grupo:")
    print(resultados_grupo)
    
    return dados_limpos, resultados_grupo
\end{lstlisting}
\end{examplebox}

\begin{examplebox}
\textbf{Workflow Completo - Testes Estatísticos}

\begin{lstlisting}[language=Python]
def pipeline_testes_estatisticos(dados_limpos):
    # Teste t
    from scipy import stats
    controle = dados_limpos[dados_limpos['grupo'] == 'controle']['melhoria']
    experimental = dados_limpos[dados_limpos['grupo'] == 'experimental']['melhoria']
    
    t_stat, p_valor = stats.ttest_ind(experimental, controle)
    
    # Tamanho do efeito
    def cohen_d(grupo1, grupo2):
        n1, n2 = len(grupo1), len(grupo2)
        s_pooled = np.sqrt(((n1-1)*grupo1.var() + (n2-1)*grupo2.var()) / (n1+n2-2))
        return (grupo1.mean() - grupo2.mean()) / s_pooled
    
    d = cohen_d(experimental, controle)
    
    print(f"   Teste t: t = {t_stat:.3f}, p = {p_valor:.3f}")
    print(f"   Cohen's d = {d:.3f}")
    
    return controle, experimental, t_stat, p_valor, d
\end{lstlisting}
\end{examplebox}

\begin{examplebox}
\textbf{Workflow Completo - Visualizações 1-2}

\begin{lstlisting}[language=Python]
def pipeline_visualizacoes_1_2(dados_limpos, controle, experimental):
    # 4. VISUALIZACAO
    print("4. Criando visualizações...")
    
    fig, ((ax1, ax2), (ax3, ax4)) = plt.subplots(2, 2, figsize=(14, 10))
    
    # Grafico 1: Boxplot da melhoria por grupo
    dados_limpos.boxplot(column='melhoria', by='grupo', ax=ax1)
    ax1.set_title('Melhoria por Grupo')
    ax1.set_xlabel('Grupo')
    ax1.set_ylabel('Melhoria (pontos)')
    
    # Gráfico 2: Histograma sobreposto
    ax2.hist(controle, alpha=0.7, label='Controle', bins=20)
    ax2.hist(experimental, alpha=0.7, label='Experimental', bins=20)
    ax2.set_xlabel('Melhoria')
    ax2.set_ylabel('Frequência')
    ax2.set_title('Distribuição da Melhoria')
    ax2.legend()
    
    return fig, ax1, ax2, ax3, ax4
\end{lstlisting}
\end{examplebox}

\begin{examplebox}
\textbf{Workflow Completo - Visualizações 3-4}

\begin{lstlisting}[language=Python]
def pipeline_visualizacoes_3_4(dados_limpos, ax3, ax4):
    # Gráfico 3: Pré vs Pós por grupo
    grupos = ['Controle', 'Experimental']
    pre_medias = [dados_limpos[dados_limpos['grupo'] == g]['pre_teste'].mean() 
                  for g in ['controle', 'experimental']]
    pos_medias = [dados_limpos[dados_limpos['grupo'] == g]['pos_teste'].mean() 
                  for g in ['controle', 'experimental']]
    
    x = np.arange(len(grupos))
    largura = 0.35
    
    ax3.bar(x - largura/2, pre_medias, largura, label='Pré-teste', alpha=0.8)
    ax3.bar(x + largura/2, pos_medias, largura, label='Pós-teste', alpha=0.8)
    ax3.set_xlabel('Grupo')
    ax3.set_ylabel('Score Médio')
    ax3.set_title('Pré vs Pós-teste')
    ax3.set_xticks(x)
    ax3.set_xticklabels(grupos)
    ax3.legend()
    
    # Gráfico 4: Correlação idade vs melhoria
    ax4.scatter(dados_limpos['idade'], dados_limpos['melhoria'], alpha=0.6)
    
    # Linha de tendência
    z = np.polyfit(dados_limpos['idade'], dados_limpos['melhoria'], 1)
    p = np.poly1d(z)
    ax4.plot(dados_limpos['idade'], p(dados_limpos['idade']), "r--", alpha=0.8)
    
    # Correlação
    r = np.corrcoef(dados_limpos['idade'], dados_limpos['melhoria'])[0, 1]
    ax4.text(0.05, 0.95, f'r = {r:.3f}', transform=ax4.transAxes,
             bbox=dict(boxstyle="round", facecolor='wheat', alpha=0.8))
    
    ax4.set_xlabel('Idade')
    ax4.set_ylabel('Melhoria')
    ax4.set_title('Idade vs Melhoria')
    
    plt.tight_layout()
    plt.savefig('pipeline_completo.png', dpi=300, bbox_inches='tight')
    plt.show()
\end{lstlisting}
\end{examplebox}

\begin{examplebox}
\textbf{Workflow Completo - Relatório Final e Execução}

\begin{lstlisting}[language=Python]
def pipeline_relatorio_final(dados_limpos, controle, experimental, t_stat, p_valor, d):
    # 5. RELATÓRIO FINAL
    print("\n5. RELATÓRIO FINAL")
    print("="*30)
    print(f"Amostra final: N = {len(dados_limpos)}")
    print(f"Melhoria Controle: M = {controle.mean():.2f}, DP = {controle.std():.2f}")
    print(f"Melhoria Experimental: M = {experimental.mean():.2f}, DP = {experimental.std():.2f}")
    print(f"Teste t: t({len(dados_limpos)-2}) = {t_stat:.3f}, p = {p_valor:.3f}")
    print(f"Tamanho do efeito: d = {d:.3f}")
    
    if p_valor < 0.05:
        significancia = "significativa"
    else:
        significancia = "não significativa"
    
    if abs(d) < 0.2:
        interpretacao_d = "trivial"
    elif abs(d) < 0.5:
        interpretacao_d = "pequeno"
    elif abs(d) < 0.8:
        interpretacao_d = "médio"
    else:
        interpretacao_d = "grande"
    
    print(f"\nCONCLUSÃO: Diferença {significancia} com efeito {interpretacao_d}")
    
    return dados_limpos

# Executar pipeline completo
dados_brutos = pipeline_completo_pesquisa()
dados_limpos, resultados = pipeline_limpeza_analise(dados_brutos)
controle, experimental, t_stat, p_valor, d = pipeline_testes_estatisticos(dados_limpos)
fig, ax1, ax2, ax3, ax4 = pipeline_visualizacoes_1_2(dados_limpos, controle, experimental)
pipeline_visualizacoes_3_4(dados_limpos, ax3, ax4)
dados_finais = pipeline_relatorio_final(dados_limpos, controle, experimental, t_stat, p_valor, d)
\end{lstlisting}
\end{examplebox}

\section{Dicas de Performance e Otimização}

Quando trabalhando com datasets grandes, performance se torna crucial.

\begin{warningbox}
\textbf{Armadilhas Comuns de Performance:}

\begin{enumerate}
    \item \textbf{Loops Python em dados grandes}: Use operações vetorizadas
    \item \textbf{Cópias desnecessárias}: Use views quando possível
    \item \textbf{Tipos de dados inadequados}: int64 vs int32 pode economizar 50\% da memória
    \item \textbf{Operações não otimizadas}: Use métodos nativos do pandas/numpy
    \item \textbf{Carregamento de dados ineficiente}: Use chunks para arquivos muito grandes
\end{enumerate}
\end{warningbox}

\begin{examplebox}
\textbf{Otimização de Performance - Comparações 1-2}

\begin{lstlisting}[language=Python]
import time

# Demonstração de diferenças de performance
def comparar_performance():
    """Compara diferentes abordagens para operações comuns"""
    
    # Dados de teste
    n = 1000000
    dados = pd.DataFrame({
        'grupo': np.random.choice(['A', 'B'], n),
        'valor': np.random.randn(n)
    })
    
    print("COMPARAÇÃO DE PERFORMANCE")
    print("="*40)
    
    # 1. Loop vs Vetorização
    print("1. Calcular quadrado dos valores")
    
    # Método lento: loop Python
    inicio = time.time()
    resultado_loop = []
    for valor in dados['valor']:
        resultado_loop.append(valor ** 2)
    tempo_loop = time.time() - inicio
    
    # Método rápido: vetorização NumPy
    inicio = time.time()
    resultado_numpy = dados['valor'] ** 2
    tempo_numpy = time.time() - inicio
    
    print(f"   Loop Python: {tempo_loop:.3f}s")
    print(f"   NumPy: {tempo_numpy:.3f}s")
    print(f"   Speedup: {tempo_loop/tempo_numpy:.1f}x")
    
    return dados, n
\end{lstlisting}
\end{examplebox}

\begin{examplebox}
\textbf{Otimização de Performance - Comparação 2-3}

\begin{lstlisting}[language=Python]
def comparar_performance_continuacao(dados):
    # 2. Agrupamento eficiente
    print("\n2. Calcular média por grupo")
    
    # Método lento: loop manual
    inicio = time.time()
    resultado_manual = {}
    for grupo in ['A', 'B']:
        mask = dados['grupo'] == grupo
        resultado_manual[grupo] = dados.loc[mask, 'valor'].mean()
    tempo_manual = time.time() - inicio
    
    # Método rápido: groupby
    inicio = time.time()
    resultado_groupby = dados.groupby('grupo')['valor'].mean()
    tempo_groupby = time.time() - inicio
    
    print(f"   Loop manual: {tempo_manual:.3f}s")
    print(f"   GroupBy: {tempo_groupby:.3f}s")
    print(f"   Speedup: {tempo_manual/tempo_groupby:.1f}x")
    
    return resultado_manual, resultado_groupby
\end{lstlisting}
\end{examplebox}

\begin{examplebox}
\textbf{Otimização de Performance - Otimização de Memória}

\begin{lstlisting}[language=Python]
def otimizacao_memoria(n):
    # 3. Otimização de tipos
    print("\n3. Otimização de memória")
    
    # DataFrame com tipos não otimizados
    df_original = pd.DataFrame({
        'int_col': np.random.randint(0, 100, n),  # int64 por padrão
        'float_col': np.random.randn(n),  # float64 por padrão
        'cat_col': np.random.choice(['A', 'B', 'C'], n)  # object por padrão
    })
    
    # DataFrame com tipos otimizados
    df_otimizado = pd.DataFrame({
        'int_col': np.random.randint(0, 100, n).astype('int8'),  # Suficiente para 0-100
        'float_col': np.random.randn(n).astype('float32'),  # Precisão suficiente
        'cat_col': pd.Categorical(np.random.choice(['A', 'B', 'C'], n))  # Categórico
    })
    
    memoria_original = df_original.memory_usage(deep=True).sum() / 1024**2
    memoria_otimizada = df_otimizado.memory_usage(deep=True).sum() / 1024**2
    
    print(f"   Memória original: {memoria_original:.1f} MB")
    print(f"   Memória otimizada: {memoria_otimizada:.1f} MB")
    print(f"   Economia: {(1 - memoria_otimizada/memoria_original)*100:.1f}%")

# Executar comparação completa
dados, n = comparar_performance()
resultado_manual, resultado_groupby = comparar_performance_continuacao(dados)
otimizacao_memoria(n)
\end{lstlisting}
\end{examplebox}

\section{Exercícios Práticos}

Para dominar essas bibliotecas, pratique com estes exercícios baseados em cenários reais de pesquisa:

\begin{examplebox}
\textbf{Exercício 1: Análise de Dados Longitudinais}

Crie um dataset simulando medições mensais de 100 pacientes ao longo de 12 meses. Cada paciente tem: grupo (controle/tratamento), idade, e scores de ansiedade/depressão. Analise a evolução temporal por grupo e identifique fatores preditivos.

\textbf{Exercício 2: Análise de Questionário Complexo}

Simule dados de um questionário com 50 perguntas (escalas Likert 1-7) respondido por 500 pessoas. Realize análise fatorial exploratória para identificar dimensões latentes e crie visualizações das correlações entre fatores.

\textbf{Exercício 3: Dados de Experimento Psicofísico}

Simule dados de tempo de reação em diferentes condições experimentais (2 grupos × 3 condições × 100 trials por participante). Analise efeitos principais e interações, removendo outliers e criando visualizações apropriadas.
\end{examplebox}

Este capítulo estabeleceu o domínio das três bibliotecas fundamentais do ecossistema científico Python. No próximo capítulo, exploraremos métodos estatísticos avançados e como implementá-los usando essas ferramentas, incluindo testes de hipóteses robustos, análise de regressão e métodos de reamostragem.
% =============================================================================
% CAPÍTULO 4: ANÁLISE ESTATÍSTICA AVANÇADA COM PYTHON
% =============================================================================

\chapter{Análise Estatística Avançada com Python}

A análise estatística é o coração da pesquisa empírica. Este capítulo vai além das estatísticas descritivas básicas, explorando métodos inferenciais robustos, análise de regressão, métodos não-paramétricos e técnicas de reamostragem. Aprenderemos não apenas a executar testes, mas a interpretar resultados criticamente e reportar achados de forma apropriada para publicação.

\section{Fundamentos da Inferência Estatística}

Antes de aplicar testes complexos, é crucial entender os pressupostos subjacentes e quando cada método é apropriado.

\subsection{Verificação de Pressupostos}

A maioria dos testes estatísticos assume normalidade, homogeneidade de variâncias e independência das observações. Violações desses pressupostos podem invalidar conclusões.

\begin{examplebox}
\textbf{Teste de Pressupostos Estatísticos - Configuração Inicial}

\begin{lstlisting}[language=Python]
import numpy as np
import pandas as pd
from scipy import stats
import matplotlib.pyplot as plt
import seaborn as sns
from statsmodels.stats.diagnostic import lilliefors

def verificar_pressupostos(dados, grupo_col=None, variavel_dependente=None):
    """
    Verifica pressupostos estatisticos fundamentais
    
    Parametros:
    -----------
    dados : DataFrame
        Dados para analise
    grupo_col : str, optional
        Nome da coluna com grupos (para teste de homogeneidade)
    variavel_dependente : str
        Nome da variavel dependente a ser testada
    """
    
    print("VERIFICACAO DE PRESSUPOSTOS ESTATISTICOS")
    print("="*50)
    
    return dados, grupo_col, variavel_dependente
\end{lstlisting}
\end{examplebox}

\begin{examplebox}
\textbf{Teste de Pressupostos - Normalidade}

\begin{lstlisting}[language=Python]
def testar_normalidade(dados, grupo_col, variavel_dependente):
    """Testa normalidade dos dados"""
    # 1. TESTE DE NORMALIDADE
    print("1. TESTE DE NORMALIDADE")
    print("-" * 25)
    
    if grupo_col:
        # Testar normalidade por grupo
        grupos = dados[grupo_col].unique()
        for grupo in grupos:
            subset = dados[dados[grupo_col] == grupo][variavel_dependente]
            
            # Shapiro-Wilk (melhor para n < 50)
            if len(subset) < 50:
                stat_sw, p_sw = stats.shapiro(subset)
                teste_usado = "Shapiro-Wilk"
                stat_final, p_final = stat_sw, p_sw
            else:
                # Lilliefors (modificacao do KS para normalidade)
                stat_lf, p_lf = lilliefors(subset)
                teste_usado = "Lilliefors"
                stat_final, p_final = stat_lf, p_lf
            
            print(f"   {grupo}: {teste_usado}")
            print(f"     Estatistica: {stat_final:.4f}")
            print(f"     p-valor: {p_final:.4f}")
            print(f"     Normal: {'Sim' if p_final > 0.05 else 'Nao'}")
    else:
        # Teste geral de normalidade
        if len(dados[variavel_dependente]) < 50:
            stat, p = stats.shapiro(dados[variavel_dependente])
            teste = "Shapiro-Wilk"
        else:
            stat, p = lilliefors(dados[variavel_dependente])
            teste = "Lilliefors"
        
        print(f"   {teste}: estatistica = {stat:.4f}, p = {p:.4f}")
        print(f"   Distribuicao normal: {'Sim' if p > 0.05 else 'Nao'}")
        p_final = p
    
    return p_final
\end{lstlisting}
\end{examplebox}

\begin{examplebox}
\textbf{Teste de Pressupostos - Homogeneidade de Variâncias}

\begin{lstlisting}[language=Python]
def testar_homogeneidade(dados, grupo_col, variavel_dependente):
    """Testa homogeneidade de variancias"""
    p_levene = None
    if grupo_col:
        print(f"\n2. HOMOGENEIDADE DE VARIANCIAS")
        print("-" * 30)
        
        # Levene's test (robusto a desvios da normalidade)
        grupos_dados = [dados[dados[grupo_col] == g][variavel_dependente] 
                       for g in dados[grupo_col].unique()]
        
        stat_levene, p_levene = stats.levene(*grupos_dados)
        print(f"   Teste de Levene:")
        print(f"     Estatistica: {stat_levene:.4f}")
        print(f"     p-valor: {p_levene:.4f}")
        print(f"     Variancias homogeneas: {'Sim' if p_levene > 0.05 else 'Nao'}")
        
        # Bartlett (mais sensivel, assume normalidade)
        stat_bartlett, p_bartlett = stats.bartlett(*grupos_dados)
        print(f"   Teste de Bartlett:")
        print(f"     Estatistica: {stat_bartlett:.4f}")
        print(f"     p-valor: {p_bartlett:.4f}")
        print(f"     Variancias homogeneas: {'Sim' if p_bartlett > 0.05 else 'Nao'}")
    
    return p_levene
\end{lstlisting}
\end{examplebox}

\begin{examplebox}
\textbf{Teste de Pressupostos - Diagnósticos Visuais}

\begin{lstlisting}[language=Python]
def diagnosticos_visuais(dados, grupo_col, variavel_dependente):
    """Cria diagnosticos visuais"""
    print(f"\n3. DIAGNOSTICOS VISUAIS")
    print("-" * 20)
    
    fig, axes = plt.subplots(2, 2, figsize=(12, 10))
    
    # Q-Q plot para normalidade
    stats.probplot(dados[variavel_dependente], dist="norm", plot=axes[0,0])
    axes[0,0].set_title('Q-Q Plot (Normalidade)')
    axes[0,0].grid(True, alpha=0.3)
    
    # Histograma com curva normal
    axes[0,1].hist(dados[variavel_dependente], bins=20, density=True, alpha=0.7)
    mu, sigma = dados[variavel_dependente].mean(), dados[variavel_dependente].std()
    x = np.linspace(dados[variavel_dependente].min(), dados[variavel_dependente].max(), 100)
    axes[0,1].plot(x, stats.norm.pdf(x, mu, sigma), 'r-', linewidth=2, label='Normal teorica')
    axes[0,1].set_title('Histograma vs Normal')
    axes[0,1].legend()
    
    # Boxplot por grupo (se aplicavel)
    if grupo_col:
        dados.boxplot(column=variavel_dependente, by=grupo_col, ax=axes[1,0])
        axes[1,0].set_title('Boxplot por Grupo')
    else:
        axes[1,0].boxplot(dados[variavel_dependente])
        axes[1,0].set_title('Boxplot Geral')
    
    return axes
\end{lstlisting}
\end{examplebox}

\begin{examplebox}
\textbf{Teste de Pressupostos - Detecção de Outliers e Exemplo}

\begin{lstlisting}[language=Python]
def detectar_outliers(dados, variavel_dependente, axes):
    """Detecta e visualiza outliers"""
    # Teste de outliers
    q1 = dados[variavel_dependente].quantile(0.25)
    q3 = dados[variavel_dependente].quantile(0.75)
    iqr = q3 - q1
    outliers = dados[(dados[variavel_dependente] < q1 - 1.5*iqr) | 
                    (dados[variavel_dependente] > q3 + 1.5*iqr)]
    
    axes[1,1].scatter(range(len(dados)), dados[variavel_dependente], alpha=0.6)
    if len(outliers) > 0:
        outlier_indices = outliers.index
        axes[1,1].scatter(outlier_indices, outliers[variavel_dependente], 
                         color='red', s=50, label=f'{len(outliers)} outliers')
        axes[1,1].legend()
    axes[1,1].set_title('Deteccao de Outliers')
    axes[1,1].set_xlabel('Indice')
    axes[1,1].set_ylabel(variavel_dependente)
    
    plt.tight_layout()
    plt.show()
    
    print(f"   Outliers detectados: {len(outliers)}")
    if len(outliers) > 0:
        print(f"   Indices dos outliers: {list(outliers.index)}")
    
    return outliers

# Exemplo de uso
np.random.seed(42)
dados_exemplo = pd.DataFrame({
    'grupo': np.repeat(['controle', 'experimental'], 50),
    'score': np.concatenate([
        np.random.normal(50, 10, 50),  # controle
        np.random.normal(58, 12, 50)   # experimental
    ])
})

# Adicionar alguns outliers
dados_exemplo.loc[5, 'score'] = 100  # outlier
dados_exemplo.loc[55, 'score'] = 20  # outlier

# Executar verificacao completa
dados, grupo_col, var_dep = verificar_pressupostos(dados_exemplo, 'grupo', 'score')
p_final = testar_normalidade(dados, grupo_col, var_dep)
p_levene = testar_homogeneidade(dados, grupo_col, var_dep)
axes = diagnosticos_visuais(dados, grupo_col, var_dep)
outliers = detectar_outliers(dados, var_dep, axes)
\end{lstlisting}
\end{examplebox}

\section{Testes de Hipóteses Robustos}

Quando os pressupostos dos testes paramétricos são violados, precisamos de alternativas robustas.

\subsection{Comparação de Grupos: Paramétrico vs Não-paramétrico}

\begin{examplebox}
\textbf{Arsenal de Testes - Configuração e Dois Grupos}

\begin{lstlisting}[language=Python]
from scipy.stats import ttest_ind, mannwhitneyu, welch_ttest
from scipy.stats import ttest_rel, wilcoxon
from scipy.stats import f_oneway, kruskal
import pingouin as pg

def bateria_testes_comparacao(dados, grupo_col, variavel_dep, pareado=False):
    """
    Executa bateria completa de testes de comparacao entre grupos
    """
    
    print("BATERIA DE TESTES DE COMPARACAO")
    print("="*40)
    
    grupos = dados[grupo_col].unique()
    
    if len(grupos) == 2:
        # DOIS GRUPOS
        grupo1 = dados[dados[grupo_col] == grupos[0]][variavel_dep]
        grupo2 = dados[dados[grupo_col] == grupos[1]][variavel_dep]
        
        print(f"Comparando: {grupos[0]} vs {grupos[1]}")
        print(f"N1 = {len(grupo1)}, N2 = {len(grupo2)}")
        print(f"M1 = {grupo1.mean():.3f}, M2 = {grupo2.mean():.3f}")
        print(f"DP1 = {grupo1.std():.3f}, DP2 = {grupo2.std():.3f}")
        
        return grupo1, grupo2, grupos
    
    return None, None, grupos
\end{lstlisting}
\end{examplebox}

\begin{examplebox}
\textbf{Arsenal de Testes - Amostras Independentes}

\begin{lstlisting}[language=Python]
def testes_independentes(grupo1, grupo2):
    """Executa testes para amostras independentes"""
    # TESTES PARA AMOSTRAS INDEPENDENTES
    print(f"\nTESTES PARA AMOSTRAS INDEPENDENTES:")
    print("-" * 35)
    
    # 1. Teste t de Student (assume variancias iguais)
    t_stat, p_t = ttest_ind(grupo1, grupo2)
    print(f"1. Teste t de Student:")
    print(f"   t = {t_stat:.3f}, p = {p_t:.3f}")
    
    # 2. Teste t de Welch (nao assume variancias iguais)
    t_welch, p_welch = ttest_ind(grupo1, grupo2, equal_var=False)
    print(f"2. Teste t de Welch:")
    print(f"   t = {t_welch:.3f}, p = {p_welch:.3f}")
    
    # 3. Mann-Whitney U (nao-parametrico)
    u_stat, p_u = mannwhitneyu(grupo1, grupo2, alternative='two-sided')
    print(f"3. Mann-Whitney U:")
    print(f"   U = {u_stat:.3f}, p = {p_u:.3f}")
    
    # 4. Teste robusto (usando pingouin)
    resultado_robusto = pg.ttest(grupo1, grupo2, paired=False)
    print(f"4. Teste robusto (Yuen):")
    print(f"   T = {resultado_robusto['T'].iloc[0]:.3f}")
    print(f"   p = {resultado_robusto['p-val'].iloc[0]:.3f}")
    
    return t_stat
\end{lstlisting}
\end{examplebox}

\begin{examplebox}
\textbf{Arsenal de Testes - Amostras Pareadas}

\begin{lstlisting}[language=Python]
def testes_pareados(grupo1, grupo2):
    """Executa testes para amostras pareadas"""
    # TESTES PARA AMOSTRAS PAREADAS
    print(f"\nTESTES PARA AMOSTRAS PAREADAS:")
    print("-" * 30)
    
    # 1. Teste t pareado
    t_paired, p_paired = ttest_rel(grupo1, grupo2)
    print(f"1. Teste t pareado:")
    print(f"   t = {t_paired:.3f}, p = {p_paired:.3f}")
    
    # 2. Wilcoxon signed-rank
    w_stat, p_w = wilcoxon(grupo1, grupo2)
    print(f"2. Wilcoxon signed-rank:")
    print(f"   W = {w_stat:.3f}, p = {p_w:.3f}")
    
    return t_paired
\end{lstlisting}
\end{examplebox}

\begin{examplebox}
\textbf{Arsenal de Testes - Tamanhos de Efeito}

\begin{lstlisting}[language=Python]
def calcular_tamanhos_efeito(grupo1, grupo2, t_stat):
    """Calcula diferentes tamanhos de efeito"""
    # TAMANHOS DE EFEITO
    print(f"\nTAMANHOS DE EFEITO:")
    print("-" * 18)
    
    # Cohen's d
    pooled_std = np.sqrt(((len(grupo1)-1)*grupo1.var() + 
                         (len(grupo2)-1)*grupo2.var()) / 
                        (len(grupo1)+len(grupo2)-2))
    cohens_d = (grupo1.mean() - grupo2.mean()) / pooled_std
    print(f"Cohen's d = {cohens_d:.3f}")
    
    # Glass's delta (usa DP do grupo controle)
    glass_delta = (grupo1.mean() - grupo2.mean()) / grupo2.std()
    print(f"Glass's delta = {glass_delta:.3f}")
    
    # r de Pearson (a partir do teste t)
    r_pearson = t_stat / np.sqrt(t_stat**2 + (len(grupo1) + len(grupo2) - 2))
    print(f"r de Pearson = {r_pearson:.3f}")
    
    # Interpretacao do tamanho do efeito
    if abs(cohens_d) < 0.2:
        interpretacao = "trivial"
    elif abs(cohens_d) < 0.5:
        interpretacao = "pequeno"
    elif abs(cohens_d) < 0.8:
        interpretacao = "medio"
    else:
        interpretacao = "grande"
    
    print(f"Interpretacao: efeito {interpretacao}")
    
    return cohens_d
\end{lstlisting}
\end{examplebox}

\begin{examplebox}
\textbf{Arsenal de Testes - Múltiplos Grupos (ANOVA)}

\begin{lstlisting}[language=Python]
def testes_multiplos_grupos(dados, grupo_col, variavel_dep):
    """Executa testes para multiplos grupos"""
    grupos = dados[grupo_col].unique()
    print(f"Comparando {len(grupos)} grupos: {list(grupos)}")
    
    # Extrair dados por grupo
    grupos_dados = [dados[dados[grupo_col] == g][variavel_dep] 
                   for g in grupos]
    
    print(f"\nESTATISTICAS DESCRITIVAS:")
    print("-" * 25)
    for i, grupo in enumerate(grupos):
        subset = grupos_dados[i]
        print(f"{grupo}: N={len(subset)}, M={subset.mean():.3f}, DP={subset.std():.3f}")
    
    print(f"\nTESTES OMNIBUS:")
    print("-" * 15)
    
    # 1. ANOVA one-way
    f_stat, p_anova = f_oneway(*grupos_dados)
    print(f"1. ANOVA one-way:")
    print(f"   F = {f_stat:.3f}, p = {p_anova:.3f}")
    
    # 2. Kruskal-Wallis (nao-parametrico)
    h_stat, p_kruskal = kruskal(*grupos_dados)
    print(f"2. Kruskal-Wallis:")
    print(f"   H = {h_stat:.3f}, p = {p_kruskal:.3f}")
    
    # 3. ANOVA robusta (usando pingouin)
    dados_long = dados.copy()
    anova_robusta = pg.welch_anova(data=dados_long, dv=variavel_dep, between=grupo_col)
    print(f"3. ANOVA de Welch (robusta):")
    print(f"   F = {anova_robusta['F'].iloc[0]:.3f}")
    print(f"   p = {anova_robusta['p-unc'].iloc[0]:.3f}")
    
    return grupos_dados, p_anova, dados_long
\end{lstlisting}
\end{examplebox}

\begin{examplebox}
\textbf{Arsenal de Testes - Tamanho de Efeito ANOVA e Post-hoc}
\begin{lstlisting}[language=Python]
def anova_efeito_posthoc(dados, grupo_col, variavel_dep, grupos_dados, p_anova):
   """Calcula tamanho de efeito e testes post-hoc"""
   print(f"\nTAMANHO DE EFEITO:")
   print("-" * 17)
   ss_between = sum(len(g) * (g.mean() - dados[variavel_dep].mean())**2 
                   for g in grupos_dados)
   ss_total = sum((dados[variavel_dep] - dados[variavel_dep].mean())**2)
   eta_squared = ss_between / ss_total
   print(f"Eta-squared = {eta_squared:.3f}")
   if eta_squared < 0.01:
       interpretacao = "trivial"
   elif eta_squared < 0.06:
       interpretacao = "pequeno"
   elif eta_squared < 0.14:
       interpretacao = "medio"
   else:
       interpretacao = "grande"
   print(f"Interpretacao: efeito {interpretacao}")
   if p_anova < 0.05:
       print(f"\nTESTES POST-HOC:")
       print("-" * 15)
       posthoc = pg.pairwise_tukey(data=dados, dv=variavel_dep, between=grupo_col)
       print("Tukey HSD:")
       for _, row in posthoc.iterrows():
           print(f"   {row['A']} vs {row['B']}: p = {row['p-tukey']:.3f}")
np.random.seed(123)
dados_multi = pd.DataFrame({
   'grupo': np.repeat(['controle', 'trat1', 'trat2'], 30),
   'score': np.concatenate([
       np.random.normal(50, 8, 30),
       np.random.normal(55, 9, 30),
       np.random.normal(62, 10, 30)
   ])
})
grupo1, grupo2, grupos = bateria_testes_comparacao(dados_multi, 'grupo', 'score')
if grupo1 is not None:
   t_stat = testes_independentes(grupo1, grupo2)
   cohens_d = calcular_tamanhos_efeito(grupo1, grupo2, t_stat)
else:
   grupos_dados, p_anova, dados_long = testes_multiplos_grupos(dados_multi, 'grupo', 'score')
   anova_efeito_posthoc(dados_multi, 'grupo', 'score', grupos_dados, p_anova)
\end{lstlisting}
\end{examplebox}

\subsection{Análise de Regressão Robusta}

A regressão linear é fundamental, mas precisa de diagnósticos cuidadosos e alternativas robustas.

\begin{examplebox}
\textbf{Regressão Linear - Configuração e Modelo Clássico}

\begin{lstlisting}[language=Python]
import statsmodels.api as sm
from statsmodels.stats.diagnostic import het_breuschpagan, het_white
from statsmodels.stats.stattools import durbin_watson
from sklearn.linear_model import HuberRegressor
from sklearn.metrics import r2_score

def regressao_completa(dados, variavel_dep, preditores):
    """
    Executa regressao linear completa com diagnosticos
    """
    
    print("ANALISE DE REGRESSAO COMPLETA")
    print("="*35)
    
    # Preparar dados
    X = dados[preditores]
    y = dados[variavel_dep]
    X_const = sm.add_constant(X)  # Adicionar intercepto
    
    # 1. REGRESSAO LINEAR CLASSICA
    print("1. REGRESSAO LINEAR CLASSICA")
    print("-" * 28)
    
    modelo = sm.OLS(y, X_const).fit()
    print(modelo.summary())
    
    return modelo, X, y, X_const
\end{lstlisting}
\end{examplebox}

\begin{examplebox}
\textbf{Regressão Linear - Diagnósticos de Resíduos}

\begin{lstlisting}[language=Python]
def diagnosticos_residuos(modelo, X_const):
    """Executa diagnosticos completos dos residuos"""
    # 2. DIAGNOSTICOS DE RESIDUOS
    print(f"\n2. DIAGNOSTICOS DE RESIDUOS")
    print("-" * 27)
    
    residuos = modelo.resid
    valores_preditos = modelo.fittedvalues
    residuos_estudentizados = modelo.get_influence().resid_studentized_external
    
    # Teste de normalidade dos residuos
    stat_norm, p_norm = stats.shapiro(residuos)
    print(f"Normalidade dos residuos (Shapiro): p = {p_norm:.3f}")
    
    # Teste de homocedasticidade
    bp_stat, bp_p, _, _ = het_breuschpagan(residuos, X_const)
    print(f"Homocedasticidade (Breusch-Pagan): p = {bp_p:.3f}")
    
    white_stat, white_p, _, _ = het_white(residuos, X_const)
    print(f"Homocedasticidade (White): p = {white_p:.3f}")
    
    # Teste de independencia (Durbin-Watson)
    dw_stat = durbin_watson(residuos)
    print(f"Independencia (Durbin-Watson): {dw_stat:.3f}")
    print(f"  (Valores entre 1.5-2.5 indicam independencia)")
    
    # Deteccao de outliers
    outliers_residuos = np.where(np.abs(residuos_estudentizados) > 3)[0]
    print(f"Outliers detectados (|resid| > 3): {len(outliers_residuos)}")
    
    return residuos, valores_preditos, residuos_estudentizados, p_norm, bp_p, white_p, dw_stat, outliers_residuos
\end{lstlisting}
\end{examplebox}

\begin{examplebox}
\textbf{Regressão Linear - Visualizações Diagnósticas}

\begin{lstlisting}[language=Python]
def visualizacoes_diagnosticas(residuos, valores_preditos, residuos_estudentizados, dados, preditores):
    """Cria visualizacoes diagnosticas"""
    # 3. VISUALIZACOES DIAGNOSTICAS
    print(f"\n3. DIAGNOSTICOS VISUAIS")
    print("-" * 20)
    
    fig, axes = plt.subplots(2, 3, figsize=(15, 10))
    
    # Residuos vs valores preditos
    axes[0,0].scatter(valores_preditos, residuos, alpha=0.7)
    axes[0,0].axhline(y=0, color='r', linestyle='--')
    axes[0,0].set_xlabel('Valores Preditos')
    axes[0,0].set_ylabel('Residuos')
    axes[0,0].set_title('Residuos vs Preditos')
    
    # Q-Q plot dos residuos
    stats.probplot(residuos, dist="norm", plot=axes[0,1])
    axes[0,1].set_title('Q-Q Plot dos Residuos')
    
    # Scale-Location plot
    residuos_padronizados = np.sqrt(np.abs(residuos_estudentizados))
    axes[0,2].scatter(valores_preditos, residuos_padronizados, alpha=0.7)
    axes[0,2].set_xlabel('Valores Preditos')
    axes[0,2].set_ylabel('Residuos Padronizados')
    axes[0,2].set_title('Scale-Location Plot')
    
    # Residuos vs cada preditor
    for i, pred in enumerate(preditores[:3]):  # Maximo 3 preditores
        if i < len(preditores):
            axes[1,i].scatter(dados[pred], residuos, alpha=0.7)
            axes[1,i].axhline(y=0, color='r', linestyle='--')
            axes[1,i].set_xlabel(pred)
            axes[1,i].set_ylabel('Residuos')
            axes[1,i].set_title(f'Residuos vs {pred}')
    
    plt.tight_layout()
    plt.show()
\end{lstlisting}
\end{examplebox}

\begin{examplebox}
\textbf{Regressão Linear - Regressão Robusta e Interpretação}

\begin{lstlisting}[language=Python]
def regressao_robusta_interpretacao(modelo, X, y, preditores, p_norm, bp_p, white_p, dw_stat, outliers_residuos):
    """Executa regressao robusta e interpretacao final"""
    print(f"\n4. REGRESSAO ROBUSTA (Huber)")
    print("-" * 27)
    huber = HuberRegressor(epsilon=1.35, max_iter=1000)
    huber.fit(X, y)
    y_pred_huber = huber.predict(X)
    r2_huber = r2_score(y, y_pred_huber)
    print(f"R2 da regressao robusta: {r2_huber:.3f}")
    print(f"R2 da regressao classica: {modelo.rsquared:.3f}")
    print(f"\nCoeficientes:")
    print(f"{'Variavel':<15} {'OLS':<12} {'Huber':<12} {'Diferenca'}")
    print("-" * 45)
    print(f"{'Intercepto':<15} {modelo.params[0]:<12.3f} {huber.intercept_:<12.3f} {abs(modelo.params[0] - huber.intercept_):<12.3f}")
    for i, pred in enumerate(preditores):
        diff = abs(modelo.params[i+1] - huber.coef_[i])
        print(f"{pred:<15} {modelo.params[i+1]:<12.3f} {huber.coef_[i]:<12.3f} {diff:<12.3f}")
    print(f"\n5. INTERPRETACAO E RECOMENDACOES")
    print("-" * 32)
    pressupostos_ok = True
    if p_norm < 0.05:
        print("AVISO: Residuos nao seguem distribuicao normal")
        pressupostos_ok = False
    if bp_p < 0.05 or white_p < 0.05:
        print("AVISO: Heterocedasticidade detectada")
        pressupostos_ok = False
    if dw_stat < 1.5 or dw_stat > 2.5:
        print("AVISO: Possivel autocorrelacao nos residuos")
        pressupostos_ok = False
    if len(outliers_residuos) > 0:
        print(f"AVISO: {len(outliers_residuos)} outliers detectados")
        pressupostos_ok = False
    if pressupostos_ok:
        print("- Todos os pressupostos da regressao satisfeitos")
        print("- Resultados da regressao OLS sao confiaveis")
    else:
        print("\nRECOMENDACOES:")
        print("- Considere transformacao de variaveis")
        print("- Use regressao robusta (Huber mostrada acima)")
        print("- Investigue e possivelmente remova outliers")
        print("- Considere modelos nao-lineares")
    return huber, pressupostos_ok
np.random.seed(42)
n = 100
\end{lstlisting}
\end{examplebox}


\begin{examplebox}
\textbf{Regressão Linear - Regressão Robusta e Interpretação - Continuação}
\begin{lstlisting}[language=Python]
dados_reg = pd.DataFrame({
    'idade': np.random.randint(20, 60, n),
    'educacao': np.random.randint(8, 20, n),
    'experiencia': np.random.randint(0, 30, n)
})
dados_reg['salario'] = (dados_reg['idade'] * 500 + 
                       dados_reg['educacao'] * 2000 + 
                       dados_reg['experiencia'] * 800 + 
                       np.random.normal(0, 5000, n))
dados_reg.loc[5, 'salario'] = 150000  # outlier alto
dados_reg.loc[25, 'salario'] = 10000   # outlier baixo
modelo, X, y, X_const = regressao_completa(dados_reg, 'salario', ['idade', 'educacao', 'experiencia'])
residuos, valores_preditos, residuos_estudentizados, p_norm, bp_p, white_p, dw_stat, outliers_residuos = diagnosticos_residuos(modelo, X_const)
visualizacoes_diagnosticas(residuos, valores_preditos, residuos_estudentizados, dados_reg, ['idade', 'educacao', 'experiencia'])
huber, pressupostos_ok = regressao_robusta_interpretacao(modelo, X, y, ['idade', 'educacao', 'experiencia'], p_norm, bp_p, white_p, dw_stat, outliers_residuos)
\end{lstlisting}
\end{examplebox}

\section{Métodos de Reamostragem}

Quando os métodos tradicionais falham ou quando queremos estimativas mais robustas, métodos de reamostragem oferecem alternativas poderosas.

\subsection{Bootstrap: Estimando Distribuições}

\begin{researchbox}
\textbf{Bootstrap - Análise Básica}

\begin{lstlisting}[language=Python]
from sklearn.utils import resample

def bootstrap_analise(dados, estatistica_func, n_bootstrap=10000, alpha=0.05):
    """
    Realiza analise bootstrap para qualquer estatistica
    
    Parametros:
    -----------
    dados : array-like
        Dados originais
    estatistica_func : function
        Funcao que calcula a estatistica de interesse
    n_bootstrap : int
        Numero de amostras bootstrap
    alpha : float
        Nivel de significancia para IC
    """
    
    print("ANALISE BOOTSTRAP")
    print("="*20)
    
    # Estatistica original
    stat_original = estatistica_func(dados)
    print(f"Estatistica original: {stat_original:.4f}")
    
    # Gerar amostras bootstrap
    bootstrap_stats = []
    
    for i in range(n_bootstrap):
        # Reamostragem com reposicao
        amostra_boot = resample(dados, n_samples=len(dados), replace=True)
        stat_boot = estatistica_func(amostra_boot)
        bootstrap_stats.append(stat_boot)
    
    bootstrap_stats = np.array(bootstrap_stats)
    
    return stat_original, bootstrap_stats
\end{lstlisting}
\end{researchbox}

\begin{researchbox}
\textbf{Bootstrap - Intervalos de Confiança e Visualização}

\begin{lstlisting}[language=Python]
def bootstrap_intervalos_viz(stat_original, bootstrap_stats, alpha=0.05):
    """Calcula intervalos de confianca e cria visualizacoes"""
    # Calcular intervalos de confianca
    ci_lower = np.percentile(bootstrap_stats, (alpha/2) * 100)
    ci_upper = np.percentile(bootstrap_stats, (1 - alpha/2) * 100)
    
    # Bias e erro padrao
    bias = np.mean(bootstrap_stats) - stat_original
    erro_padrao = np.std(bootstrap_stats)
    
    print(f"Bias bootstrap: {bias:.4f}")
    print(f"Erro padrao bootstrap: {erro_padrao:.4f}")
    print(f"IC {(1-alpha)*100:.0f}%: [{ci_lower:.4f}, {ci_upper:.4f}]")
    
    # Visualizacao
    plt.figure(figsize=(12, 4))
    
    # Histograma das estatisticas bootstrap
    plt.subplot(1, 2, 1)
    plt.hist(bootstrap_stats, bins=50, alpha=0.7, density=True)
    plt.axvline(stat_original, color='red', linestyle='--', 
                label=f'Original: {stat_original:.3f}')
    plt.axvline(ci_lower, color='green', linestyle='--', alpha=0.7)
    plt.axvline(ci_upper, color='green', linestyle='--', alpha=0.7, 
                label=f'IC {(1-alpha)*100:.0f}%')
    plt.xlabel('Valor da Estatistica')
    plt.ylabel('Densidade')
    plt.title('Distribuicao Bootstrap')
    plt.legend()
    
    # Q-Q plot para verificar normalidade da distribuicao bootstrap
    plt.subplot(1, 2, 2)
    stats.probplot(bootstrap_stats, dist="norm", plot=plt)
    plt.title('Q-Q Plot: Distribuicao Bootstrap')
    plt.grid(True, alpha=0.3)
    
    plt.tight_layout()
    plt.show()
    
    return bias, erro_padrao, ci_lower, ci_upper
\end{lstlisting}
\end{researchbox}

\begin{researchbox}
\textbf{Bootstrap - Exemplos de Aplicação}

\begin{lstlisting}[language=Python]
# Exemplos de uso do bootstrap
np.random.seed(123)
dados_assimetricos = np.random.exponential(2, 100)  # Dados nao-normais

print("EXEMPLO 1: MEDIA DE DISTRIBUICAO ASSIMETRICA")
print("=" * 45)
stat_orig, boot_stats = bootstrap_analise(dados_assimetricos, np.mean)
bias, erro_padrao, ci_lower, ci_upper = bootstrap_intervalos_viz(stat_orig, boot_stats)

print("\nEXEMPLO 2: CORRELACAO")
print("=" * 20)
# Dados bivariados
x = np.random.normal(0, 1, 50)
y = 0.7*x + np.random.normal(0, 0.5, 50)
dados_correlacao = np.column_stack([x, y])

def correlacao_func(dados):
    return np.corrcoef(dados[:, 0], dados[:, 1])[0, 1]

stat_orig_corr, boot_stats_corr = bootstrap_analise(dados_correlacao, correlacao_func)
bias_corr, erro_padrao_corr, ci_lower_corr, ci_upper_corr = bootstrap_intervalos_viz(stat_orig_corr, boot_stats_corr)
\end{lstlisting}
\end{researchbox}

\begin{researchbox}
\textbf{Bootstrap - Diferença entre Grupos}

\begin{lstlisting}[language=Python]
print("\nEXEMPLO 3: DIFERENCA ENTRE MEDIAS")
print("=" * 30)
grupo1 = np.random.normal(50, 10, 30)
grupo2 = np.random.normal(55, 12, 25)

# Bootstrap para diferenca entre grupos (versao simplificada)
def bootstrap_grupos(grupo1, grupo2, n_bootstrap=10000):
    """Bootstrap especifico para diferenca entre grupos"""
    diffs = []
    for _ in range(n_bootstrap):
        boot_g1 = resample(grupo1)
        boot_g2 = resample(grupo2)
        diff = np.mean(boot_g1) - np.mean(boot_g2)
        diffs.append(diff)
    return np.array(diffs)

diffs_boot = bootstrap_grupos(grupo1, grupo2)
diff_original = np.mean(grupo1) - np.mean(grupo2)

print(f"Diferenca original: {diff_original:.3f}")
print(f"IC 95% bootstrap: [{np.percentile(diffs_boot, 2.5):.3f}, {np.percentile(diffs_boot, 97.5):.3f}]")

# Visualizar distribuicao das diferencas
plt.figure(figsize=(8, 5))
plt.hist(diffs_boot, bins=50, alpha=0.7, density=True)
plt.axvline(diff_original, color='red', linestyle='--', label=f'Diferenca observada: {diff_original:.3f}')
plt.axvline(np.percentile(diffs_boot, 2.5), color='green', linestyle='--', alpha=0.7)
plt.axvline(np.percentile(diffs_boot, 97.5), color='green', linestyle='--', alpha=0.7, label='IC 95%')
plt.xlabel('Diferenca entre Medias')
plt.ylabel('Densidade')
plt.title('Bootstrap: Diferenca entre Grupos')
plt.legend()
plt.grid(True, alpha=0.3)
plt.show()
\end{lstlisting}
\end{researchbox}

\subsection{Testes de Permutação}

Quando não podemos assumir distribuições específicas, testes de permutação oferecem uma alternativa exata.

\begin{examplebox}
\textbf{Testes de Permutação - Função Principal}

\begin{lstlisting}[language=Python]
def teste_permutacao(grupo1, grupo2, estatistica_func=None, n_perm=10000):
    """
    Realiza teste de permutacao para comparar dois grupos
    
    Parametros:
    -----------
    grupo1, grupo2 : array-like
        Dados dos dois grupos
    estatistica_func : function
        Funcao para calcular estatistica (default: diferenca de medias)
    n_perm : int
        Numero de permutacoes
    """
    
    if estatistica_func is None:
        estatistica_func = lambda g1, g2: np.mean(g1) - np.mean(g2)
    
    print("TESTE DE PERMUTACAO")
    print("=" * 20)
    
    # Estatistica observada
    stat_observada = estatistica_func(grupo1, grupo2)
    print(f"Estatistica observada: {stat_observada:.4f}")
    
    # Combinar todos os dados
    dados_combinados = np.concatenate([grupo1, grupo2])
    n1, n2 = len(grupo1), len(grupo2)
    
    return stat_observada, dados_combinados, n1, n2
\end{lstlisting}
\end{examplebox}

\begin{examplebox}
\textbf{Testes de Permutação - Distribuição Nula e P-valor}

\begin{lstlisting}[language=Python]
def permutacao_distribuicao_nula(stat_observada, dados_combinados, n1, n2, estatistica_func, n_perm=10000):
    """Gera distribuicao nula e calcula p-valor"""
    # Gerar distribuicao nula por permutacao
    stats_permutacao = []
    
    for i in range(n_perm):
        # Permutacao aleatoria
        dados_permutados = np.random.permutation(dados_combinados)
        
        # Dividir novamente nos grupos originais
        perm_g1 = dados_permutados[:n1]
        perm_g2 = dados_permutados[n1:]
        
        # Calcular estatistica para esta permutacao
        stat_perm = estatistica_func(perm_g1, perm_g2)
        stats_permutacao.append(stat_perm)
    
    stats_permutacao = np.array(stats_permutacao)
    
    # Calcular p-valor
    # Para teste bilateral
    p_valor = np.mean(np.abs(stats_permutacao) >= np.abs(stat_observada))
    
    print(f"P-valor (bilateral): {p_valor:.4f}")
    
    return stats_permutacao, p_valor
\end{lstlisting}
\end{examplebox}

\begin{examplebox}
\textbf{Testes de Permutação - Visualização}

\begin{lstlisting}[language=Python]
def permutacao_visualizacao(stat_observada, stats_permutacao):
    """Cria visualizacao do teste de permutacao"""
    # Visualizacao
    plt.figure(figsize=(10, 6))
    
    plt.hist(stats_permutacao, bins=50, alpha=0.7, density=True, 
             label='Distribuicao nula')
    plt.axvline(stat_observada, color='red', linestyle='--', linewidth=2,
                label=f'Estatistica observada: {stat_observada:.3f}')
    plt.axvline(-stat_observada, color='red', linestyle='--', linewidth=2, alpha=0.7)
    
    # Marcar regiao critica
    critico_superior = np.percentile(stats_permutacao, 97.5)
    critico_inferior = np.percentile(stats_permutacao, 2.5)
    
    x_fill = np.linspace(critico_superior, max(stats_permutacao), 100)
    plt.fill_between(x_fill, 0, stats.norm.pdf(x_fill, np.mean(stats_permutacao), 
                                              np.std(stats_permutacao)), 
                     alpha=0.3, color='red', label='Regiao critica (alpha=0.05)')
    
    x_fill2 = np.linspace(min(stats_permutacao), critico_inferior, 100)
    plt.fill_between(x_fill2, 0, stats.norm.pdf(x_fill2, np.mean(stats_permutacao), 
                                               np.std(stats_permutacao)), 
                     alpha=0.3, color='red')
    
    plt.xlabel('Estatistica de Teste')
    plt.ylabel('Densidade')
    plt.title('Teste de Permutacao: Distribuicao Nula')
    plt.legend()
    plt.grid(True, alpha=0.3)
    plt.show()

\end{lstlisting}
\end{examplebox}

\begin{examplebox}
\textbf{Testes de Permutação - Visualização - Continuação}

\begin{lstlisting}[language=Python]
# Exemplo de uso
np.random.seed(42)
controle = np.random.normal(50, 10, 25)
tratamento = np.random.normal(58, 12, 30)

print("EXEMPLO: COMPARACAO DE MEDIAS")
stat_obs, dados_comb, n1, n2 = teste_permutacao(controle, tratamento)
def diff_medias_func(g1, g2): return np.mean(g1) - np.mean(g2)
stats_perm, p_val = permutacao_distribuicao_nula(stat_obs, dados_comb, n1, n2, diff_medias_func)
permutacao_visualizacao(stat_obs, stats_perm)

# Exemplo com estatistica customizada (diferenca de medianas)
def diff_medianas(g1, g2):
    return np.median(g1) - np.median(g2)

print("\nEXEMPLO: COMPARACAO DE MEDIANAS")
stat_obs_med, dados_comb_med, n1_med, n2_med = teste_permutacao(controle, tratamento)
stats_perm_med, p_val_med = permutacao_distribuicao_nula(stat_obs_med, dados_comb_med, n1_med, n2_med, diff_medianas)
permutacao_visualizacao(stat_obs_med, stats_perm_med)
\end{lstlisting}
\end{examplebox}

\section{Análise Multivariada}

Pesquisas frequentemente envolvem múltiplas variáveis interrelacionadas, exigindo métodos multivariados.

\subsection{Análise de Componentes Principais (PCA)}

\begin{researchbox}
\textbf{PCA - Configuração e Análise Inicial}

\begin{lstlisting}[language=Python]
from sklearn.decomposition import PCA
from sklearn.preprocessing import StandardScaler
from sklearn.datasets import make_blobs

def analise_pca_completa(dados, n_componentes=None, padronizar=True):
    """
    Realiza PCA completa com interpretacao e visualizacao
    
    Parametros:
    -----------
    dados : DataFrame
        Dados numericos para PCA
    n_componentes : int, optional
        Numero de componentes a reter
    padronizar : bool
        Se deve padronizar as variaveis
    """
    
    print("ANALISE DE COMPONENTES PRINCIPAIS")
    print("=" * 35)
    
    # Preparar dados
    if padronizar:
        scaler = StandardScaler()
        dados_scaled = scaler.fit_transform(dados)
        print("Dados padronizados (media=0, dp=1)")
    else:
        dados_scaled = dados.values
        print("Dados originais (sem padronizacao)")
    
    # PCA inicial para ver todas as componentes
    pca_completo = PCA()
    pca_completo.fit(dados_scaled)
    
    # Variancia explicada
    var_explicada = pca_completo.explained_variance_ratio_
    var_cumulativa = np.cumsum(var_explicada)
    
    print(f"\nVariancia explicada por componente:")
    for i, var in enumerate(var_explicada[:min(10, len(var_explicada))]):
        print(f"  PC{i+1}: {var:.3f} ({var*100:.1f}%)")
    
    return dados_scaled, pca_completo, var_explicada, var_cumulativa
\end{lstlisting}
\end{researchbox}

\begin{researchbox}
\textbf{PCA - Seleção de Componentes e Análise Final}

\begin{lstlisting}[language=Python]
def pca_selecao_componentes(dados, dados_scaled, var_explicada, var_cumulativa, n_componentes=None):
    """Seleciona numero ideal de componentes e executa PCA final"""
    # Determinar numero ideal de componentes
    if n_componentes is None:
        # Criterio de Kaiser (eigenvalues > 1)
        eigenvalues = pca_completo.explained_variance_
        n_kaiser = np.sum(eigenvalues > 1)
        
        # Criterio de 95% da variancia
        n_95 = np.argmax(var_cumulativa >= 0.95) + 1
        
        print(f"\nCriterios para selecao de componentes:")
        print(f"  Kaiser (eigenvalue > 1): {n_kaiser} componentes")
        print(f"  95% da variancia: {n_95} componentes")
        
        n_componentes = min(n_kaiser, n_95)
        print(f"  Recomendado: {n_componentes} componentes")
    
    # PCA final com numero escolhido
    pca = PCA(n_components=n_componentes)
    dados_pca = pca.fit_transform(dados_scaled)
    
    print(f"\nPCA final com {n_componentes} componentes:")
    print(f"Variancia total explicada: {pca.explained_variance_ratio_.sum():.3f} ({pca.explained_variance_ratio_.sum()*100:.1f}%)")
    
    # Loadings (pesos das variaveis originais)
    loadings = pca.components_.T * np.sqrt(pca.explained_variance_)
    loadings_df = pd.DataFrame(loadings, 
                              index=dados.columns, 
                              columns=[f'PC{i+1}' for i in range(n_componentes)])
    
    print(f"\nLoadings das variaveis:")
    print(loadings_df.round(3))
    
    return pca, dados_pca, loadings_df, n_componentes
\end{lstlisting}
\end{researchbox}

\begin{researchbox}
\textbf{PCA - Interpretação dos Componentes}

\begin{lstlisting}[language=Python]
def pca_interpretacao(pca, loadings_df, n_componentes):
    """Interpreta os componentes principais"""
    # Interpretacao dos componentes
    print(f"\nINTERPRETACAO DOS COMPONENTES:")
    for i in range(n_componentes):
        print(f"\nPC{i+1} (variancia explicada: {pca.explained_variance_ratio_[i]:.3f}):")
        
        # Variaveis com maior peso (positivo e negativo)
        pc_loadings = loadings_df[f'PC{i+1}'].abs().sort_values(ascending=False)
        print(f"  Variaveis mais importantes:")
        
        for var in pc_loadings.head(3).index:
            loading = loadings_df.loc[var, f'PC{i+1}']
            print(f"    {var}: {loading:.3f}")
\end{lstlisting}
\end{researchbox}

\begin{researchbox}
\textbf{PCA - Visualizações}

\begin{lstlisting}[language=Python]
def pca_visualizacoes(dados, pca, dados_pca, loadings_df, var_explicada, var_cumulativa, n_componentes):
    """Cria visualizacoes do PCA"""
    # Visualizacoes
    fig, axes = plt.subplots(2, 2, figsize=(14, 10))
    
    # 1. Scree plot
    axes[0,0].plot(range(1, len(var_explicada[:10])+1), var_explicada[:10], 'bo-')
    axes[0,0].axhline(y=1/len(dados.columns), color='r', linestyle='--', 
                     label='Media da variancia')
    axes[0,0].set_xlabel('Componente')
    axes[0,0].set_ylabel('Variancia Explicada')
    axes[0,0].set_title('Scree Plot')
    axes[0,0].legend()
    axes[0,0].grid(True, alpha=0.3)
    
    # 2. Variancia cumulativa
    axes[0,1].plot(range(1, len(var_cumulativa[:10])+1), var_cumulativa[:10], 'go-')
    axes[0,1].axhline(y=0.95, color='r', linestyle='--', label='95%')
    axes[0,1].set_xlabel('Numero de Componentes')
    axes[0,1].set_ylabel('Variancia Cumulativa')
    axes[0,1].set_title('Variancia Cumulativa Explicada')
    axes[0,1].legend()
    axes[0,1].grid(True, alpha=0.3)
    
    return axes
\end{lstlisting}
\end{researchbox}

\begin{researchbox}
\textbf{PCA - Biplot e Heatmap}

\begin{lstlisting}[language=Python]
def pca_biplot_heatmap(dados, dados_pca, loadings_df, pca, n_componentes, axes):
    """Cria biplot e heatmap"""
    # 3. Biplot (se temos pelo menos 2 componentes)
    if n_componentes >= 2:
        loadings = loadings_df.values
        # Scatter dos scores
        axes[1,0].scatter(dados_pca[:, 0], dados_pca[:, 1], alpha=0.7)
        
        # Adicionar vetores das variaveis
        for i, var in enumerate(dados.columns):
            axes[1,0].arrow(0, 0, loadings[i, 0]*3, loadings[i, 1]*3, 
                           head_width=0.1, head_length=0.1, fc='red', ec='red')
            axes[1,0].text(loadings[i, 0]*3.2, loadings[i, 1]*3.2, var, 
                          fontsize=9, ha='center')
        
        axes[1,0].set_xlabel(f'PC1 ({pca.explained_variance_ratio_[0]:.1%})')
        axes[1,0].set_ylabel(f'PC2 ({pca.explained_variance_ratio_[1]:.1%})')
        axes[1,0].set_title('Biplot PC1 vs PC2')
        axes[1,0].grid(True, alpha=0.3)
    
    # 4. Heatmap dos loadings
    sns.heatmap(loadings_df.T, annot=True, cmap='RdBu_r', center=0, 
                ax=axes[1,1], fmt='.2f')
    axes[1,1].set_title('Heatmap dos Loadings')
    
    plt.tight_layout()
    plt.show()

# Dados de exemplo: simulando questionario psicologico
np.random.seed(42)
n_participantes = 200

# Simular fatores latentes
fator_ansiedade = np.random.normal(0, 1, n_participantes)
fator_depressao = np.random.normal(0, 1, n_participantes)
fator_autoestima = np.random.normal(0, 1, n_participantes)
\end{lstlisting}
\end{researchbox}

\begin{researchbox}
\textbf{PCA - Biplot e Heatmap - Continuação}
\begin{lstlisting}[language=Python]
# Criar variaveis observadas com ruido
dados_psico = pd.DataFrame({
    'ansiedade_1': fator_ansiedade + np.random.normal(0, 0.5, n_participantes),
    'ansiedade_2': fator_ansiedade + np.random.normal(0, 0.5, n_participantes),
    'ansiedade_3': fator_ansiedade + np.random.normal(0, 0.5, n_participantes),
    'depressao_1': fator_depressao + np.random.normal(0, 0.5, n_participantes),
    'depressao_2': fator_depressao + np.random.normal(0, 0.5, n_participantes),
    'depressao_3': fator_depressao + np.random.normal(0, 0.5, n_participantes),
    'autoestima_1': -fator_autoestima + np.random.normal(0, 0.5, n_participantes),
    'autoestima_2': -fator_autoestima + np.random.normal(0, 0.5, n_participantes),
})

# Executar PCA completo
dados_scaled, pca_completo, var_explicada, var_cumulativa = analise_pca_completa(dados_psico)
pca, dados_pca, loadings_df, n_componentes = pca_selecao_componentes(dados_psico, dados_scaled, var_explicada, var_cumulativa)
pca_interpretacao(pca, loadings_df, n_componentes)
axes = pca_visualizacoes(dados_psico, pca, dados_pca, loadings_df, var_explicada, var_cumulativa, n_componentes)
pca_biplot_heatmap(dados_psico, dados_pca, loadings_df, pca, n_componentes, axes)
\end{lstlisting}
\end{researchbox}

\section{Controle de Erro Tipo I em Múltiplas Comparações}

Quando realizamos múltiplos testes, a probabilidade de erro Tipo I inflaciona dramaticamente.

\begin{warningbox}
\textbf{O Problema das Múltiplas Comparações:}

Se realizarmos 20 testes independentes com alpha = 0.05, a probabilidade de pelo menos um falso positivo é:

$P(\text{pelo menos 1 falso positivo}) = 1 - (0.95)^{20} \approx 0.64$

Ou seja, 64\% de chance de erro Tipo I!
\end{warningbox}

\begin{examplebox}
\textbf{Correções para Múltiplas Comparações - Configuração}

\begin{lstlisting}[language=Python]
from statsmodels.stats.multitest import multipletests

def correcao_multiplas_comparacoes(p_valores, metodos=['bonferroni', 'holm', 'fdr_bh']):
    """
    Aplica diferentes metodos de correcao para multiplas comparacoes
    
    Parametros:
    -----------
    p_valores : array-like
        Lista de p-valores nao corrigidos
    metodos : list
        Metodos de correcao a aplicar
    """
    
    print("CORRECAO PARA MULTIPLAS COMPARACOES")
    print("=" * 40)
    
    p_valores = np.array(p_valores)
    n_testes = len(p_valores)
    
    print(f"Numero de testes: {n_testes}")
    print(f"P-valores originais: {p_valores}")
    print(f"Significativos sem correcao (alpha=0.05): {np.sum(p_valores < 0.05)}")
    
    # Calcular inflacao do erro Tipo I
    erro_familywise = 1 - (1 - 0.05)**n_testes
    print(f"Erro Tipo I familywise sem correcao: {erro_familywise:.3f}")
    
    print(f"\nCORRECOES APLICADAS:")
    print("-" * 20)
    
    return p_valores, n_testes
\end{lstlisting}
\end{examplebox}

\begin{examplebox}
\textbf{Correções para Múltiplas Comparações - Aplicação dos Métodos}

\begin{lstlisting}[language=Python]
def aplicar_correcoes(p_valores, metodos):
    """Aplica diferentes metodos de correcao"""
    resultados = {}
    
    for metodo in metodos:
        reject, p_corrigidos, alpha_sidak, alpha_bonf = multipletests(
            p_valores, alpha=0.05, method=metodo, returnsorted=False)
        
        n_significativos = np.sum(reject)
        
        print(f"\n{metodo.upper()}:")
        print(f"  P-valores corrigidos: {p_corrigidos.round(4)}")
        print(f"  Significativos apos correcao: {n_significativos}")
        print(f"  Indices significativos: {np.where(reject)[0]}")
        
        resultados[metodo] = {
            'p_corrigidos': p_corrigidos,
            'significativos': reject,
            'n_significativos': n_significativos
        }
    
    return resultados
\end{lstlisting}
\end{examplebox}

\begin{examplebox}
\textbf{Correções para Múltiplas Comparações - Visualizações}

\begin{lstlisting}[language=Python]
def visualizar_correcoes(p_valores, resultados, metodos):
    """Cria visualizacoes das correcoes"""
    # Visualizacao
    fig, (ax1, ax2) = plt.subplots(1, 2, figsize=(14, 6))
    
    # Comparacao dos p-valores
    x = np.arange(len(p_valores))
    largura = 0.2
    
    ax1.bar(x - largura, p_valores, largura, label='Original', alpha=0.8)
    
    for i, metodo in enumerate(metodos):
        offset = largura * (i - len(metodos)/2 + 1)
        ax1.bar(x + offset, resultados[metodo]['p_corrigidos'], largura, 
               label=metodo.replace('_', ' ').title(), alpha=0.8)
    
    ax1.axhline(y=0.05, color='red', linestyle='--', alpha=0.7, label='alpha = 0.05')
    ax1.set_xlabel('Indice do Teste')
    ax1.set_ylabel('P-valor')
    ax1.set_title('Comparacao de P-valores')
    ax1.legend()
    ax1.grid(True, alpha=0.3)
    
    # Numero de descobertas significativas
    metodos_plot = ['Original'] + metodos
    n_sig_plot = [np.sum(p_valores < 0.05)] + [resultados[m]['n_significativos'] for m in metodos]
    
    cores = ['red'] + ['blue', 'green', 'orange'][:len(metodos)]
    bars = ax2.bar(metodos_plot, n_sig_plot, color=cores, alpha=0.7)
    
    # Adicionar valores nas barras
    for bar in bars:
        height = bar.get_height()
        ax2.text(bar.get_x() + bar.get_width()/2., height + 0.1,
                f'{int(height)}', ha='center', va='bottom')
    
    ax2.set_xlabel('Metodo de Correcao')
    ax2.set_ylabel('Numero de Testes Significativos')
    ax2.set_title('Impacto das Correcoes')
    ax2.grid(True, alpha=0.3)
    
    plt.tight_layout()
    plt.show()
    
    return resultados
\end{lstlisting}
\end{examplebox}

\begin{examplebox}
\textbf{Correções para Múltiplas Comparações - Exemplo Completo}

\begin{lstlisting}[language=Python]
# Simular exemplo com multiplas comparacoes
np.random.seed(123)

# Simular 15 testes: 3 com efeito real, 12 sem efeito
p_valores_exemplo = []

# 3 testes com efeito real (p-valores baixos)
for _ in range(3):
    # Simular dados com diferenca real
    grupo1 = np.random.normal(50, 10, 30)
    grupo2 = np.random.normal(58, 10, 30)  # Diferenca real
    _, p = stats.ttest_ind(grupo1, grupo2)
    p_valores_exemplo.append(p)

# 12 testes sem efeito (apenas ruido)
for _ in range(12):
    # Simular dados sem diferenca
    grupo1 = np.random.normal(50, 10, 30)
    grupo2 = np.random.normal(50, 10, 30)  # Sem diferenca
    _, p = stats.ttest_ind(grupo1, grupo2)
    p_valores_exemplo.append(p)

print("EXEMPLO: 15 TESTES (3 com efeito real, 12 sem efeito)")
print("=" * 55)

# Executar correcoes
p_valores, n_testes = correcao_multiplas_comparacoes(p_valores_exemplo)
resultados = aplicar_correcoes(p_valores, ['bonferroni', 'holm', 'fdr_bh'])
resultado_final = visualizar_correcoes(p_valores, resultados, ['bonferroni', 'holm', 'fdr_bh'])
\end{lstlisting}
\end{examplebox}

\section{Exercícios Práticos Avançados}

\begin{examplebox}
\textbf{Exercício 1: Meta-Análise Simples}

Implemente uma função que combine resultados de múltiplos estudos usando meta-análise de efeitos fixos e aleatórios. Use dados simulados de 8 estudos com tamanhos de amostra diferentes.

\textbf{Exercício 2: Análise de Mediação}

Crie uma análise completa de mediação usando bootstrap para testar se uma variável medeia a relação entre X e Y. Inclua teste de Sobel e intervalos de confiança bootstrap.

\textbf{Exercício 3: Análise de Regressão Logística}

Desenvolva uma função que execute regressão logística completa com diagnósticos, incluindo teste de Hosmer-Lemeshow, curva ROC, e validação cruzada.

\textbf{Exercício 4: Análise de Sobrevivência Básica}

Implemente análise de Kaplan-Meier com teste log-rank para comparar curvas de sobrevivência entre grupos.
\end{examplebox}

Este capítulo forneceu ferramentas estatísticas avançadas essenciais para pesquisa rigorosa. No próximo capítulo, exploraremos machine learning aplicado à pesquisa acadêmica, incluindo classificação, regressão e métodos de aprendizado não supervisionado para descoberta de padrões em dados complexos.
% =============================================================================
% CAPÍTULO 5: METODOLOGIA DE PESQUISA COM PYTHON
% =============================================================================

\chapter{Metodologia de Pesquisa com Python}

\lettrine{A}{pesquisa científica moderna} exige não apenas rigor metodológico, mas também eficiência computacional e reprodutibilidade. Python oferece um framework completo para estruturar projetos de pesquisa de forma organizada, desde o planejamento inicial até a publicação dos resultados. Este capítulo apresenta metodologias e melhores práticas para conduzir pesquisa acadêmica utilizando Python como ferramenta central.

\section{Planejamento de Projetos de Pesquisa}

Um projeto de pesquisa bem estruturado em Python começa com planejamento cuidadoso. A organização inicial determina a facilidade de manutenção, colaboração e reprodução dos resultados ao longo de todo o ciclo de vida da pesquisa.

\subsection{Estrutura de Diretórios para Projetos de Pesquisa}
\begin{pythonbox}
\begin{lstlisting}[language=bash]
projeto_pesquisa/
 README.md                 # Descrição do projeto
 requirements.txt          # Dependências Python
 environment.yml          # Ambiente conda (alternativa)
 .gitignore              # Arquivos ignorados pelo Git
 setup.py                # Configuração do pacote
 data/                   # Dados do projeto
    raw/               # Dados brutos (nunca modificar)
    interim/           # Dados em processamento
    processed/         # Dados finais para análise
    external/          # Dados de fontes externas
 notebooks/             # Jupyter notebooks
    exploratory/       # Análise exploratória
    reports/           # Notebooks finais
    sandbox/           # Testes e experimentos
 src/                   # Código fonte
    __init__.py
    data/              # Scripts de coleta/processamento
    features/          # Scripts de feature engineering
    models/            # Scripts de modelagem
    visualization/     # Scripts de visualização
    utils/             # Utilitários gerais
 tests/                 # Testes automatizados
 docs/                  # Documentação
 results/               # Resultados e outputs
    figures/           # Gráficos e visualizações
    tables/            # Tabelas de resultados
    models/            # Modelos treinados
 references/            # Literatura e referências
\end{lstlisting}
\end{pythonbox}

\begin{researchbox}
\textbf{Caso Real - Projeto de Análise Longitudinal:}

Um pesquisador em psicologia estuda o desenvolvimento cognitivo em crianças ao longo de 5 anos:

\begin{lstlisting}[language=Python]
# src/data/longitudinal_processor.py
import pandas as pd
import numpy as np
from datetime import datetime
import logging

class LongitudinalDataProcessor:
    """Processador para dados longitudinais de desenvolvimento cognitivo"""
    
    def __init__(self, config):
        self.config = config
        self.logger = self._setup_logger()
        
    def _setup_logger(self):
        logging.basicConfig(
            filename=f'logs/processing_{datetime.now().strftime("%Y%m%d")}.log',
            level=logging.INFO,
            format='%(asctime)s - %(levelname)s - %(message)s'
        )
        return logging.getLogger(__name__)
    
    def load_raw_data(self, wave):
        """Carrega dados de uma onda específica do estudo"""
        try:
            filepath = f"data/raw/wave_{wave}_cognitive_tests.csv"
            data = pd.read_csv(filepath)
            self.logger.info(f"Carregados {len(data)} registros da onda {wave}")
            return data
        except FileNotFoundError:
            self.logger.error(f"Arquivo da onda {wave} não encontrado")
            raise
\end{lstlisting}
\end{researchbox}

\newpage

\begin{pythonbox}
\begin{lstlisting}[language=Python]
# Continuação: src/data/longitudinal_processor.py
    
    def validate_participant_ids(self, data, wave):
        """Valida consistência dos IDs de participantes"""
        expected_participants = self.config['participants_per_wave'][wave]
        actual_participants = len(data['participant_id'].unique())
        
        if actual_participants != expected_participants:
            self.logger.warning(
                f"Onda {wave}: esperados {expected_participants}, "
                f"encontrados {actual_participants} participantes"
            )
        
        return data
    
    def harmonize_test_scores(self, data):
        """Harmoniza escores de testes entre diferentes versões"""
        data['cognitive_score_std'] = data.groupby('age_group')['cognitive_score'].transform(
            lambda x: (x - x.mean()) / x.std()
        )
        return data
    
    def process_wave_data(self, wave):
        """Processa dados completos de uma onda"""
        # Carregar dados brutos
        data = self.load_raw_data(wave)
        
        # Validar participantes
        data = self.validate_participant_ids(data, wave)
        
        # Harmonizar escores
        data = self.harmonize_test_scores(data)
        
        # Salvar dados processados
        output_path = f"data/processed/wave_{wave}_processed.csv"
        data.to_csv(output_path, index=False)
        
        self.logger.info(f"Processamento da onda {wave} concluído")
        return data
\end{lstlisting}
\end{pythonbox}

\subsection{Configuração e Parametrização}

\begin{pythonbox}
\begin{lstlisting}
# config/research_config.yaml
project:
  name: "Desenvolvimento Cognitivo Longitudinal"
  version: "1.0.0"
  start_date: "2019-01-15"
  
data:
  waves: [1, 2, 3, 4, 5]
  participants_per_wave:
    1: 450
    2: 425
    3: 410
    4: 395
    5: 380
  
analysis:
  significance_level: 0.05
  minimum_effect_size: 0.3
  bootstrap_samples: 1000
  
models:
  growth_curve:
    method: "hierarchical_linear"
    random_effects: ["intercept", "slope"]
  
output:
  figures_dpi: 300
  table_format: "latex"
  decimal_places: 3
\end{lstlisting}
\end{pythonbox}

\begin{pythonbox}
\begin{lstlisting}[language=Python]
# src/config/config_loader.py
import yaml
from pathlib import Path

def load_config(config_path="config/research_config.yaml"):
    """Carrega configurações do projeto"""
    with open(config_path, 'r', encoding='utf-8') as file:
        config = yaml.safe_load(file)
    return config

def validate_config(config):
    """Valida configurações obrigatórias"""
    required_keys = ['project', 'data', 'analysis']
    for key in required_keys:
        if key not in config:
            raise ValueError(f"Configuração '{key}' obrigatória não encontrada")
    return True

def get_analysis_params(config):
    """Extrai parâmetros para análise"""
    return {
        'alpha': config['analysis']['significance_level'],
        'min_effect_size': config['analysis']['minimum_effect_size'],
        'bootstrap_n': config['analysis']['bootstrap_samples']
    }

# Exemplo de uso
config = load_config()
validate_config(config)
analysis_params = get_analysis_params(config)
\end{lstlisting}
\end{pythonbox}

\section{Reprodutibilidade e Documentação}

A reprodutibilidade é um pilar fundamental da pesquisa científica. Python oferece diversas ferramentas para garantir que análises possam ser replicadas por outros pesquisadores ou pelo próprio autor em momentos futuros.

\subsection{Controle de Versões para Pesquisa}

\begin{pythonbox}
\begin{lstlisting}[language=Python]
# src/utils/reproducibility.py
import random
import numpy as np
import pandas as pd
import torch
from datetime import datetime
import hashlib
import os

class ReproducibilityManager:
    """Gerencia reprodutibilidade de experimentos"""
    
    def __init__(self, seed=42):
        self.seed = seed
        self.experiment_log = []
        
    def set_seeds(self):
        """Define seeds para todos os geradores de números aleatórios"""
        random.seed(self.seed)
        np.random.seed(self.seed)
        torch.manual_seed(self.seed)
        torch.cuda.manual_seed_all(self.seed)
        os.environ['PYTHONHASHSEED'] = str(self.seed)
        
        # Para reprodutibilidade completa no PyTorch
        torch.backends.cudnn.deterministic = True
        torch.backends.cudnn.benchmark = False
        
    def log_experiment(self, experiment_name, parameters):
        """Registra parâmetros de um experimento"""
        experiment_record = {
            'timestamp': datetime.now().isoformat(),
            'experiment': experiment_name,
            'seed': self.seed,
            'parameters': parameters,
            'hash': self._calculate_hash(parameters)
        }
        self.experiment_log.append(experiment_record)
\end{lstlisting}
\end{pythonbox}

\newpage

\begin{pythonbox}
\begin{lstlisting}[language=Python]
# Continuação: src/utils/reproducibility.py
        
    def _calculate_hash(self, parameters):
        """Calcula hash dos parâmetros para identificação única"""
        param_str = str(sorted(parameters.items()))
        return hashlib.md5(param_str.encode()).hexdigest()[:8]
    
    def save_experiment_log(self, filepath="results/experiment_log.json"):
        """Salva log de experimentos"""
        import json
        with open(filepath, 'w') as f:
            json.dump(self.experiment_log, f, indent=2)

# Exemplo de uso
repro = ReproducibilityManager(seed=42)
repro.set_seeds()

# Log do experimento
experiment_params = {
    'model_type': 'random_forest',
    'n_estimators': 100,
    'max_depth': 10,
    'feature_selection': 'mutual_info'
}
repro.log_experiment('cognitive_prediction_v1', experiment_params)
\end{lstlisting}
\end{pythonbox}

\section{Conclusão do Capítulo}

A metodologia de pesquisa com Python vai muito além do simples uso de bibliotecas para análise de dados. Ela engloba uma abordagem sistemática para estruturar projetos, garantir reprodutibilidade, facilitar colaboração e manter a qualidade científica ao longo de todo o ciclo de vida da pesquisa.

Os elementos fundamentais desta metodologia incluem:

\textbf{Estruturação e Planejamento:} Uma organização clara de diretórios e arquivos facilita não apenas o desenvolvimento atual, mas também a manutenção futura e a colaboração com outros pesquisadores.

\textbf{Reprodutibilidade como Prioridade:} O uso de seeds, controle de versões, documentação automatizada e ambientes virtuais garante que resultados possam ser replicados e verificados.

\textbf{Automação Inteligente:} Pipelines automatizados e sistemas de monitoramento reduzem erros humanos e aumentam a eficiência, especialmente em estudos longitudinais ou com múltiplas condições experimentais.

\textbf{Colaboração Efetiva:} Metodologias para projetos multi-site e ferramentas de harmonização de dados permitem que pesquisadores trabalhem juntos de forma produtiva, mesmo com diferenças culturais e técnicas.

\textbf{Qualidade Assegurada:} Testes automatizados e validação contínua identificam problemas precocemente, mantendo a integridade dos dados e análises.

No próximo capítulo, aplicaremos esses princípios metodológicos ao explorar técnicas avançadas de coleta e aquisição de dados, construindo sobre esta base sólida de boas práticas para desenvolver competências técnicas específicas na obtenção de dados de múltiplas fontes.

\begin{examplebox}
\textbf{Principais Aprendizados do Capítulo:}
\begin{itemize}
    \item Estruturação profissional de projetos de pesquisa
    \item Implementação de reprodutibilidade completa
    \item Uso efetivo de Git para pesquisa colaborativa
    \item Gestão robusta de ambientes e dependências
    \item Criação de pipelines automatizados de análise
    \item Desenvolvimento de testes para código científico
    \item Metodologias para projetos multi-institucionais
\end{itemize}
\end{examplebox}

A metodologia apresentada neste capítulo forma a espinha dorsal de qualquer projeto de pesquisa moderno com Python. Dominar esses conceitos não apenas melhora a qualidade técnica do trabalho, mas também facilita a colaboração, acelera descobertas e garante que os resultados possam contribuir efetivamente para o avanço do conhecimento científico.
% =============================================================================
% CAPÍTULO 6: COLETA E AQUISIÇÃO DE DADOS
% =============================================================================

\chapter{Coleta e Aquisição de Dados}

\lettrine{A}{era digital} transformou fundamentalmente a forma como pesquisadores coletam dados. Onde antes dependíamos de questionários impressos e coleta manual, hoje temos acesso a vastas fontes de dados online, APIs de plataformas digitais, sensores IoT e bases de dados públicas. Python oferece um arsenal completo de ferramentas para automatizar e sistematizar a coleta de dados, permitindo que pesquisadores acessem informações antes inimagináveis em escala e velocidade.

\section{Web Scraping para Pesquisa}

Web scraping é a técnica de extrair dados automaticamente de páginas web. Para pesquisadores, essa ferramenta abre possibilidades de coletar dados de redes sociais, sites de notícias, repositórios acadêmicos, bases governamentais e muito mais.

\subsection{Fundamentos do Web Scraping Ético}

Antes de começar a coletar dados da web, é crucial entender os aspectos éticos e legais envolvidos.

\begin{warningbox}
\textbf{Princípios Éticos do Web Scraping:}
\begin{itemize}
    \item \textbf{Respeite o robots.txt:} Sempre verifique as diretrizes de scraping do site
    \item \textbf{Limite a frequência:} Use delays entre requisições para não sobrecarregar servidores
    \item \textbf{Identifique-se:} Use User-Agent apropriado e contato para transparência
    \item \textbf{Dados pessoais:} Nunca colete informações pessoais identificáveis sem consentimento
    \item \textbf{Termos de uso:} Sempre leia e respeite os termos de serviço dos sites
\end{itemize}
\end{warningbox}

\begin{pythonbox}
\begin{lstlisting}[language=Python]
# src/data_collection/ethical_scraper.py
import requests
from bs4 import BeautifulSoup
import time
import random
from urllib.robotparser import RobotFileParser
import logging
from typing import List, Dict, Optional

class EthicalScraper:
    """Scraper que segue princípios éticos de coleta"""
    
    def __init__(self, user_agent="Research Bot", contact_email="researcher@university.edu"):
        self.session = requests.Session()
        self.session.headers.update({
            'User-Agent': f'{user_agent} ({contact_email})',
            'Accept': 'text/html,application/xhtml+xml,application/xml;q=0.9,*/*;q=0.8',
            'Accept-Language': 'en-US,en;q=0.5',
            'Accept-Encoding': 'gzip, deflate',
            'Connection': 'keep-alive',
        })
        self.logger = logging.getLogger(__name__)
        self.delay_range = (1, 3)  # Segundos entre requisições
        
    def can_fetch(self, url: str) -> bool:
        """Verifica se é permitido fazer scraping da URL"""
        try:
            rp = RobotFileParser()
            rp.set_url(f"{url.split('/')[0]}//{url.split('/')[2]}/robots.txt")
            rp.read()
            return rp.can_fetch(self.session.headers['User-Agent'], url)
        except:
            # Se não conseguir acessar robots.txt, assume que é permitido
            return True
    
    def respectful_request(self, url: str, **kwargs) -> Optional[requests.Response]:
        """Faz requisição respeitando delays e robots.txt"""
        if not self.can_fetch(url):
            self.logger.warning(f"Robots.txt proíbe acesso a {url}")
            return None

            \end{lstlisting}
\end{pythonbox}

\begin{pythonbox}
\begin{lstlisting}[language=Python]
        # Delay aleatório entre requisições
        delay = random.uniform(*self.delay_range)
        time.sleep(delay)
        
        try:
            response = self.session.get(url, **kwargs)
            response.raise_for_status()
            return response
        except requests.RequestException as e:
            self.logger.error(f"Erro ao acessar {url}: {e}")
            return None
\end{lstlisting}
\end{pythonbox}

\newpage

\begin{researchbox}
\textbf{Caso Real - Análise de Cobertura Midiática:}

Uma pesquisadora em comunicação social estuda a cobertura de mudanças climáticas em portais de notícias:

\begin{lstlisting}[language=Python]
# src/data_collection/news_scraper.py
from ethical_scraper import EthicalScraper
import pandas as pd
from datetime import datetime
import re
from urllib.parse import urljoin

class NewsArticleScraper(EthicalScraper):
    """Coletor especializado para artigos de notícias"""
    
    def __init__(self, **kwargs):
        super().__init__(**kwargs)
        self.articles = []
        
    def extract_article_data(self, soup, url: str) -> Dict:
        """Extrai dados estruturados de um artigo"""
        article_data = {
            'url': url,
            'collected_at': datetime.now(),
            'title': '',
            'content': '',
            'author': '',
            'publish_date': ''
        }
        
        # Título do artigo
        title_selectors = ['h1', '.article-title', '.post-title', 'title']
        for selector in title_selectors:
            title_elem = soup.select_one(selector)
            if title_elem:
                article_data['title'] = title_elem.get_text().strip()
                break
                
        # Conteúdo principal
        content_selectors = ['.article-content', '.post-content', 'article']
        for selector in content_selectors:
            content_elem = soup.select_one(selector)
            if content_elem:
                # Remove scripts e estilos
                for script in content_elem(["script", "style"]):
                    script.decompose()
                article_data['content'] = content_elem.get_text().strip()
                break
        
        return article_data
\end{lstlisting}
\end{researchbox}

\begin{pythonbox}
\begin{lstlisting}[language=Python]
# Continuação: src/data_collection/news_scraper.py
    
    def search_climate_articles(self, base_urls: List[str], 
                               keywords: List[str], 
                               max_articles: int = 100) -> pd.DataFrame:
        """Busca artigos sobre mudanças climáticas"""
        self.logger.info(f"Iniciando coleta de até {max_articles} artigos")
        
        for base_url in base_urls:
            self.logger.info(f"Processando {base_url}")
            
            # Buscar URLs de artigos
            article_urls = self._find_article_urls(base_url)
            
            for url in article_urls[:max_articles]:
                response = self.respectful_request(url)
                if response:
                    soup = BeautifulSoup(response.content, 'html.parser')
                    article_data = self.extract_article_data(soup, url)
                    
                    # Verificar se contém palavras-chave sobre clima
                    if self._contains_climate_keywords(article_data['content'], keywords):
                        self.articles.append(article_data)
                        self.logger.info(f"Artigo coletado: {article_data['title'][:50]}...")
                        
                if len(self.articles) >= max_articles:
                    break
                    
        return pd.DataFrame(self.articles)
    
    def _contains_climate_keywords(self, text: str, keywords: List[str]) -> bool:
        """Verifica se o texto contém palavras-chave relacionadas ao clima"""
        text_lower = text.lower()
        return any(keyword.lower() in text_lower for keyword in keywords)
    
    def _find_article_urls(self, base_url: str) -> List[str]:
        """Encontra URLs de artigos no site"""
        response = self.respectful_request(base_url)
        if not response:
            return []
\end{lstlisting}
\end{pythonbox}

\begin{pythonbox}
\begin{lstlisting}[language=Python]            
        soup = BeautifulSoup(response.content, 'html.parser')
        links = soup.find_all('a', href=True)
        
        article_urls = []
        for link in links:
            href = link['href']
            if href.startswith('/'):
                href = urljoin(base_url, href)
            
            # Filtrar apenas URLs que parecem ser artigos
            if '/article/' in href or '/news/' in href:
                article_urls.append(href)
                
        return list(set(article_urls))  # Remove duplicatas
\end{lstlisting}
\end{pythonbox}

\section{APIs e Acesso a Bancos de Dados Públicos}

APIs (Application Programming Interfaces) oferecem uma forma estruturada e eficiente de acessar dados. Muitas instituições governamentais, organizações internacionais e plataformas digitais disponibilizam APIs públicas para pesquisadores.

\subsection{Cliente Genérico para APIs REST}

\begin{pythonbox}
\begin{lstlisting}[language=Python]
# src/data_collection/api_client.py
import requests
import pandas as pd
from typing import Dict, List, Optional
import time
from datetime import datetime

class APIClient:
    """Cliente genérico para APIs REST"""
    
    def __init__(self, base_url: str, api_key: Optional[str] = None, 
                 rate_limit: float = 1.0):
        self.base_url = base_url.rstrip('/')
        self.api_key = api_key
        self.rate_limit = rate_limit
        self.session = requests.Session()
        
        # Headers padrão
        self.session.headers.update({
            'User-Agent': 'Research-Python-Client/1.0',
            'Accept': 'application/json',
        })
        
        if api_key:
            self.session.headers.update({
                'Authorization': f'Bearer {api_key}'
            })
            
        self.last_request_time = 0
        
    def _respect_rate_limit(self):
        """Respeita limite de taxa da API"""
        now = time.time()
        time_since_last = now - self.last_request_time
        
        if time_since_last < self.rate_limit:
            time.sleep(self.rate_limit - time_since_last)
            
        self.last_request_time = time.time()
\end{lstlisting}
\end{pythonbox}

\begin{pythonbox}
\begin{lstlisting}[language=Python]        
    def get(self, endpoint: str, params: Dict = None) -> Dict:
        """Faz requisição GET para a API"""
        self._respect_rate_limit()
        
        url = f"{self.base_url}/{endpoint.lstrip('/')}"
        
        try:
            response = self.session.get(url, params=params)
            response.raise_for_status()
            return response.json()
        except requests.RequestException as e:
            print(f"Erro na requisição para {url}: {e}")
            return {}
\end{lstlisting}
\end{pythonbox}

\newpage

\begin{researchbox}
\textbf{Caso Real - Dados Econômicos do Banco Mundial:}

Um economista estuda indicadores de desenvolvimento econômico usando a API do Banco Mundial:

\begin{lstlisting}[language=Python]
# src/data_collection/world_bank_api.py
from api_client import APIClient
import pandas as pd
from typing import List

class WorldBankAPI(APIClient):
    """Cliente especializado para API do Banco Mundial"""
    
    def __init__(self):
        super().__init__("http://api.worldbank.org/v2", rate_limit=0.5)
        
    def get_indicators(self, country_codes: List[str], 
                      indicator_codes: List[str], 
                      start_year: int = 2000, 
                      end_year: int = 2023) -> pd.DataFrame:
        """Obtém indicadores econômicos para países específicos"""
        all_data = []
        
        for country in country_codes:
            for indicator in indicator_codes:
                endpoint = f"countries/{country}/indicators/{indicator}"
                params = {
                    'format': 'json',
                    'date': f'{start_year}:{end_year}',
                    'per_page': 100
                }
                
                data = self.get(endpoint, params)
                
                if len(data) > 1 and data[1]:
                    for record in data[1]:
                        if record['value'] is not None:
                            all_data.append({
                                'country_code': country,
                                'indicator_code': indicator,
                                'year': int(record['date']),
                                'value': float(record['value']),
                                'country_name': record['country']['value'],
                                'indicator_name': record['indicator']['value']
                            })
        
        return pd.DataFrame(all_data)

\end{lstlisting}
\end{researchbox}

\begin{researchbox}
\begin{lstlisting}
# Exemplo de uso
wb_api = WorldBankAPI()

# Países de interesse
countries = ['BRA', 'USA', 'CHN', 'DEU', 'JPN']

# Indicadores econômicos
indicators = [
    'NY.GDP.PCAP.CD',    # PIB per capita
    'SP.POP.TOTL',       # População total
    'SL.UEM.TOTL.ZS',    # Taxa de desemprego
]

# Coletar dados
economic_data = wb_api.get_indicators(countries, indicators, 2010, 2023)
economic_data.to_csv('data/raw/world_bank_indicators.csv', index=False)
\end{lstlisting}
\end{researchbox}

\section{Integração com Redes Sociais}

Plataformas de mídia social são fontes ricas de dados para pesquisa em ciências sociais, comunicação, psicologia e marketing.

\subsection{Coleta de Dados do Twitter/X}

\begin{pythonbox}
\begin{lstlisting}[language=Python]
# src/data_collection/twitter_collector.py
import tweepy
import pandas as pd
from datetime import datetime, timedelta
from typing import List, Dict

class TwitterDataCollector:
    """Coletor de dados do Twitter usando API v2"""
    
    def __init__(self, bearer_token: str):
        self.client = tweepy.Client(bearer_token=bearer_token, wait_on_rate_limit=True)
        
    def search_tweets(self, query: str, max_results: int = 100, 
                     days_back: int = 7) -> pd.DataFrame:
        """Busca tweets recentes baseado em query"""
        end_time = datetime.utcnow()
        start_time = end_time - timedelta(days=days_back)
        
        tweets_data = []
        
        # Configurar campos para coletar
        tweet_fields = ['created_at', 'author_id', 'public_metrics', 'lang']
        
        try:
            tweets = tweepy.Paginator(
                self.client.search_recent_tweets,
                query=query,
                tweet_fields=tweet_fields,
                start_time=start_time,
                end_time=end_time,
                max_results=min(100, max_results)
            ).flatten(limit=max_results)
            
            for tweet in tweets:
                tweet_data = {
                    'id': tweet.id,
                    'text': tweet.text,
                    'created_at': tweet.created_at,
                    'author_id': tweet.author_id,
                    'retweet_count': tweet.public_metrics['retweet_count'],
                    'like_count': tweet.public_metrics['like_count'],
                    'lang': tweet.lang if hasattr(tweet, 'lang') else None
                }
                tweets_data.append(tweet_data)
\end{lstlisting}
\end{pythonbox}

\begin{pythonbox}
\begin{lstlisting}[language=Python]
                
        except Exception as e:
            print(f"Erro ao coletar tweets: {e}")
            
        return pd.DataFrame(tweets_data)
    
    def analyze_sentiment_trends(self, tweets_df: pd.DataFrame) -> pd.DataFrame:
        """Analisa tendências de sentimento nos tweets coletados"""
        if tweets_df.empty:
            return pd.DataFrame()
        
        from textblob import TextBlob
        
        # Análise de sentimento
        tweets_df['sentiment'] = tweets_df['text'].apply(
            lambda x: TextBlob(x).sentiment.polarity
        )
        
        # Agregar por hora
        tweets_df['hour'] = tweets_df['created_at'].dt.floor('H')
        hourly_sentiment = tweets_df.groupby('hour').agg({
            'sentiment': 'mean',
            'like_count': 'mean',
            'id': 'count'
        }).rename(columns={'id': 'tweet_count'})
        
        return hourly_sentiment
\end{lstlisting}
\end{pythonbox}

\section{Monitoramento Contínuo}

Para estudos longitudinais, é essencial implementar sistemas de coleta contínua.

\subsection{Agendamento de Coletas}

\begin{pythonbox}
\begin{lstlisting}[language=Python]
# src/data_collection/scheduled_collector.py
import schedule
import time
import logging
from datetime import datetime
from typing import Callable, Dict

class ScheduledDataCollector:
    """Sistema para coleta automatizada e agendada de dados"""
    
    def __init__(self, base_config: Dict):
        self.config = base_config
        self.collectors = {}
        self.logger = logging.getLogger(__name__)
        
    def register_collector(self, name: str, collector_func: Callable, 
                          schedule_config: Dict):
        """Registra um coletor com sua configuração de agendamento"""
        self.collectors[name] = {
            'function': collector_func,
            'schedule': schedule_config,
            'last_run': None,
            'total_runs': 0
        }
        
        # Configurar agendamento
        if schedule_config['type'] == 'interval':
            if schedule_config['unit'] == 'hours':
                schedule.every(schedule_config['value']).hours.do(
                    self._run_collector, name, collector_func
                )
            elif schedule_config['unit'] == 'days':
                schedule.every(schedule_config['value']).days.do(
                    self._run_collector, name, collector_func
                )
                
    def _run_collector(self, name: str, func: Callable):
        """Executa um coletor específico"""
        try:
            self.logger.info(f"Iniciando coleta: {name}")
            result = func()
            
            self.collectors[name]['last_run'] = datetime.now()
            self.collectors[name]['total_runs'] += 1
\end{lstlisting}
\end{pythonbox}

\begin{pythonbox}
\begin{lstlisting}[language=Python]
            
            # Salvar dados se especificado
            if 'output_path' in self.config:
                timestamp = datetime.now().strftime('%Y%m%d_%H%M%S')
                output_file = f"{self.config['output_path']}/{name}_{timestamp}.csv"
                if hasattr(result, 'to_csv'):
                    result.to_csv(output_file, index=False)
                    
            self.logger.info(f"Coleta {name} concluída")
            
        except Exception as e:
            self.logger.error(f"Erro na coleta {name}: {e}")
    
    def start_monitoring(self):
        """Inicia monitoramento contínuo"""
        self.logger.info("Iniciando sistema de coleta automatizada")
        while True:
            schedule.run_pending()
            time.sleep(60)
\end{lstlisting}
\end{pythonbox}

\section{Validação e Qualidade}

A coleta automatizada exige sistemas robustos de validação.

\subsection{Framework de Validação}

\begin{pythonbox}
\begin{lstlisting}[language=Python]
# src/data_collection/data_validator.py
import pandas as pd
from datetime import datetime
from typing import Dict, Callable

class DataQualityValidator:
    """Sistema de validação de qualidade de dados coletados"""
    
    def __init__(self):
        self.validation_rules = {}
        
    def add_validation_rule(self, rule_name: str, validator_func: Callable):
        """Adiciona regra de validação personalizada"""
        self.validation_rules[rule_name] = validator_func
        
    def validate_dataset(self, data: pd.DataFrame, dataset_name: str = "unnamed") -> Dict:
        """Executa todas as validações em um dataset"""
        report = {
            'dataset_name': dataset_name,
            'timestamp': datetime.now(),
            'total_rows': len(data),
            'total_columns': len(data.columns),
            'issues': [],
            'quality_score': 100.0
        }
        
        # Validações básicas
        missing_data = data.isnull().sum()
        duplicates = data.duplicated().sum()
        
        # Penalizar qualidade por problemas
        missing_penalty = (missing_data.sum() / (len(data) * len(data.columns))) * 50
        duplicate_penalty = (duplicates / len(data)) * 30
        
        report['quality_score'] -= (missing_penalty + duplicate_penalty)

\end{lstlisting}
\end{pythonbox}

\begin{pythonbox}
\begin{lstlisting}[language=Python]        
        # Executar validações customizadas
        for rule_name, rule_func in self.validation_rules.items():
            try:
                result = rule_func(data)
                if not result['passed']:
                    report['issues'].append({
                        'rule': rule_name,
                        'message': result['message']
                    })
                    report['quality_score'] -= 10
            except Exception as e:
                report['issues'].append({
                    'rule': rule_name,
                    'message': f"Erro na validação: {str(e)}"
                })
        
        return report

# Exemplo de uso
validator = DataQualityValidator()

def validate_date_range(data):
    """Valida se datas estão em range aceitável"""
    if 'created_at' not in data.columns:
        return {'passed': True, 'message': 'Sem coluna de data'}
    
    dates = pd.to_datetime(data['created_at'], errors='coerce')
    future_dates = (dates > datetime.now()).sum()
    
    if future_dates > 0:
        return {'passed': False, 'message': f'{future_dates} datas futuras encontradas'}
    
    return {'passed': True, 'message': 'Datas válidas'}

validator.add_validation_rule('date_range', validate_date_range)
\end{lstlisting}
\end{pythonbox}

\section{Considerações Éticas e Legais}

\subsection{Framework Ético para Coleta}

\begin{pythonbox}
\begin{lstlisting}[language=Python]
# src/data_collection/ethical_framework.py
import hashlib
import pandas as pd
import re
from typing import List

class EthicalDataProcessor:
    """Processador que aplica princípios éticos na coleta de dados"""
    
    def __init__(self, research_purpose: str, institution: str):
        self.research_purpose = research_purpose
        self.institution = institution
        
    def anonymize_identifiers(self, data: pd.DataFrame, 
                             identifier_columns: List[str]) -> pd.DataFrame:
        """Remove ou anonimiza identificadores pessoais"""
        anonymized_data = data.copy()
        
        for col in identifier_columns:
            if col in anonymized_data.columns:
                # Criar hash irreversível para IDs únicos
                anonymized_data[f"{col}_hash"] = anonymized_data[col].apply(
                    lambda x: hashlib.sha256(str(x).encode()).hexdigest()[:12]
                )
                # Remover coluna original
                anonymized_data = anonymized_data.drop(columns=[col])
                
        return anonymized_data
    
    def remove_sensitive_content(self, text_series: pd.Series) -> pd.Series:
        """Remove conteúdo potencialmente sensível de textos"""
        cleaned_text = text_series.copy()
        
        # Padrões para remoção
        patterns = {
            'email': r'\b[A-Za-z0-9._%+-]+@[A-Za-z0-9.-]+\.[A-Z|a-z]{2,}\b',
            'phone': r'\b\d{2,3}[-.\s]?\d{4,5}[-.\s]?\d{4}\b',
            'cpf': r'\b\d{3}\.?\d{3}\.?\d{3}[-.]?\d{2}\b'
        }
        
        for pattern_type, pattern in patterns.items():
            cleaned_text = cleaned_text.str.replace(
                pattern, f'[{pattern_type.upper()}_REMOVED]', regex=True
            )
            
        return cleaned_text
    \end{lstlisting}
\end{pythonbox}

\begin{pythonbox}
\begin{lstlisting}[language=Python]
    def generate_privacy_report(self, processed_data: pd.DataFrame) -> Dict:
        """Gera relatório de privacidade"""
        return {
            'timestamp': datetime.now(),
            'processed_rows': len(processed_data),
            'research_purpose': self.research_purpose,
            'institution': self.institution,
            'privacy_measures': [
                'Identificadores removidos',
                'Conteúdo sensível filtrado',
                'Dados anonimizados'
            ]
        }

# Exemplo de uso
processor = EthicalDataProcessor(
    research_purpose="Análise de sentimento político",
    institution="Universidade XYZ"
)

# Processar dados de forma ética
sample_data = pd.DataFrame({
    'user_id': ['user123', 'user456'],
    'text': ['Meu email é joao@email.com', 'Política é importante'],
    'timestamp': pd.date_range('2024-01-01', periods=2)
})

anonymized_data = processor.anonymize_identifiers(sample_data, ['user_id'])
anonymized_data['text'] = processor.remove_sensitive_content(anonymized_data['text'])

print("Dados processados de forma ética:")
print(anonymized_data)
\end{lstlisting}
\end{pythonbox}

\section{Conclusão do Capítulo}

A coleta e aquisição de dados representa uma das fases mais críticas da pesquisa moderna. Python oferece um ecossistema rico de ferramentas que permitem aos pesquisadores acessar dados de múltiplas fontes de forma eficiente, ética e sistematizada.

Os elementos essenciais abordados neste capítulo incluem:

\textbf{Web Scraping Ético:} Técnicas para extrair dados da web respeitando diretrizes técnicas e éticas fundamentais.

\textbf{APIs e Integrações:} Acesso estruturado a dados através de interfaces programáticas, incluindo autenticação e gestão de limites.

\textbf{Redes Sociais:} Métodos especializados para coletar dados de plataformas digitais com análise de sentimento.

\textbf{Monitoramento Contínuo:} Sistemas automatizados para coleta longitudinal, essenciais para estudos dinâmicos.

\textbf{Validação de Qualidade:} Frameworks robustos para garantir integridade e confiabilidade dos dados.

\textbf{Ética e Privacidade:} Princípios e práticas para coleta responsável com proteção de dados pessoais.

\begin{examplebox}
\textbf{Principais Competências Desenvolvidas:}
\begin{itemize}
    \item Implementação de scrapers éticos e eficientes
    \item Integração com APIs públicas e privadas
    \item Coleta automatizada e agendada de dados
    \item Análise de redes sociais e padrões de interação
    \item Validação automática de qualidade de dados
    \item Aplicação de princípios éticos na coleta
    \item Processamento e anonimização de dados sensíveis
\end{itemize}
\end{examplebox}

No próximo capítulo, exploraremos como processar, limpar e preparar os dados coletados para análise, aplicando técnicas avançadas de processamento de linguagem natural que transformam dados textuais brutos em informações estruturadas prontas para investigação científica.
% =============================================================================
% CAPÍTULO 7: PROCESSAMENTO DE LINGUAGEM NATURAL (NLP)
% =============================================================================

\chapter{Processamento de Linguagem Natural (NLP)}

\lettrine{A}{linguagem humana} é uma das fontes mais ricas de dados para pesquisa acadêmica. Textos de redes sociais, entrevistas transcritas, documentos históricos, artigos científicos e corpus literários contêm insights valiosos sobre comportamento humano, tendências sociais, desenvolvimento de ideias e muito mais. Python oferece um ecossistema robusto de bibliotecas de NLP que permitem aos pesquisadores extrair significado, padrões e estruturas de dados textuais de forma sistemática e escalável.

\section{Fundamentos do Processamento de Texto}

Antes de realizar análises sofisticadas, é essencial dominar as técnicas fundamentais de preprocessamento e limpeza de texto. Estas etapas determinam a qualidade de todas as análises subsequentes.

\subsection{Limpeza e Normalização de Texto}

\begin{pythonbox}
\begin{lstlisting}[language=Python]
# src/nlp/text_preprocessor.py
import re
import string
import unicodedata
from typing import List, Dict, Optional
import pandas as pd

class TextPreprocessor:
    """Processador para limpeza e normalização de texto"""
    
    def __init__(self, language='portuguese'):
        self.language = language
        self.stopwords = self._load_stopwords()
        self.contractions = self._load_contractions()
        
    def _load_stopwords(self) -> set:
        """Carrega stopwords para o idioma especificado"""
        try:
            import nltk
            nltk.download('stopwords', quiet=True)
            from nltk.corpus import stopwords
            
            if self.language == 'portuguese':
                return set(stopwords.words('portuguese'))
            elif self.language == 'english':
                return set(stopwords.words('english'))
            else:
                return set(stopwords.words('english'))  # fallback
                
        except ImportError:
            # Stopwords básicas se NLTK não estiver disponível
            if self.language == 'portuguese':
                return {'de', 'a', 'o', 'que', 'e', 'do', 'da', 'em', 'um', 'para', 
                       'é', 'com', 'não', 'uma', 'os', 'no', 'se', 'na', 'por', 'mais'}
            else:
                return {'the', 'a', 'an', 'and', 'or', 'but', 'in', 'on', 'at', 
                       'to', 'for', 'of', 'with', 'by', 'is', 'are', 'was', 'were'}
    
    def _load_contractions(self) -> Dict[str, str]:
        """Carrega mapeamento de contrações"""
        if self.language == 'portuguese':
            return {
                'não': 'não', 'nao': 'não', 'voce': 'você', 'vc': 'você',
                'pq': 'porque', 'pra': 'para', 'pro': 'para o'
            }
\end{lstlisting}
\end{pythonbox}
\begin{pythonbox}
\begin{lstlisting}[language=Python]            
            
        else:  # English
            return {
                "can't": "cannot", "won't": "will not", "n't": " not",
                "'re": " are", "'ve": " have", "'ll": " will", "'d": " would",
                "'m": " am", "it's": "it is", "that's": "that is"
            }
\end{lstlisting}
\end{pythonbox}

\newpage

\begin{pythonbox}
\begin{lstlisting}[language=Python]
# Continuação: src/nlp/text_preprocessor.py
    
    def clean_text(self, text: str, 
                   remove_urls: bool = True,
                   remove_mentions: bool = True,
                   remove_hashtags: bool = False,
                   remove_numbers: bool = False,
                   remove_punctuation: bool = True,
                   lowercase: bool = True) -> str:
        """Aplica limpeza básica ao texto"""
        
        if not isinstance(text, str):
            return ""
            
        # Normalizar unicode
        text = unicodedata.normalize('NFKD', text)
        
        # Remover URLs
        if remove_urls:
            text = re.sub(r'http[s]?://(?:[a-zA-Z]|[0-9]|[$-_@.&+]|[!*\\(\\),]|(?:%[0-9a-fA-F][0-9a-fA-F]))+', '', text)
        
        # Remover menções (@usuario)
        if remove_mentions:
            text = re.sub(r'@\w+', '', text)
            
        # Remover hashtags (mas manter o texto)
        if remove_hashtags:
            text = re.sub(r'#(\w+)', r'\1', text)
            
        # Remover números
        if remove_numbers:
            text = re.sub(r'\d+', '', text)
            
        # Expandir contrações
        for contraction, expansion in self.contractions.items():
            text = re.sub(r'\b' + contraction + r'\b', expansion, text, flags=re.IGNORECASE)
            
        # Converter para minúsculas
        if lowercase:
            text = text.lower()
            
        # Remover pontuação
        if remove_punctuation:
            text = text.translate(str.maketrans('', '', string.punctuation))
            
        # Remover espaços extras
        text = re.sub(r'\s+', ' ', text).strip()
            
        return text
        \end{lstlisting}
\end{pythonbox}
\begin{pythonbox}
\begin{lstlisting}[language=Python]   
    def tokenize(self, text: str, remove_stopwords: bool = True) -> List[str]:
        """Tokeniza texto em palavras"""
        # Limpeza básica
        cleaned_text = self.clean_text(text)
        
        # Tokenização simples por espaços
        tokens = cleaned_text.split()
        
        # Remover stopwords
        if remove_stopwords:
            tokens = [token for token in tokens if token not in self.stopwords]
            
        # Filtrar tokens muito curtos
        tokens = [token for token in tokens if len(token) > 2]
        
        return tokens
\end{lstlisting}
\end{pythonbox}

\begin{researchbox}
\textbf{Caso Real - Análise de Comentários de Políticas Públicas:}

Uma pesquisadora em políticas públicas analisa comentários de consultas públicas governamentais:

\begin{lstlisting}[language=Python]
# src/nlp/public_consultation_analyzer.py
from text_preprocessor import TextPreprocessor
import pandas as pd
from collections import Counter
import matplotlib.pyplot as plt

class PublicConsultationAnalyzer:
    """Analisador para comentários de consultas públicas"""
    
    def __init__(self):
        self.preprocessor = TextPreprocessor(language='portuguese')
        self.comments_df = None
        
    def load_and_clean_comments(self, filepath: str) -> pd.DataFrame:
        """Carrega e limpa comentários de consulta pública"""
        # Carregar dados
        self.comments_df = pd.read_csv(filepath)
        
        # Limpar textos
        self.comments_df['comment_cleaned'] = self.comments_df['comment'].apply(
            lambda x: self.preprocessor.clean_text(x, remove_hashtags=False)
        )
        
        # Tokenizar
        self.comments_df['tokens'] = self.comments_df['comment_cleaned'].apply(
            lambda x: self.preprocessor.tokenize(x)
        )
        
        # Filtrar comentários muito curtos
        self.comments_df = self.comments_df[
            self.comments_df['tokens'].apply(len) >= 3
        ]
        
        return self.comments_df
    
    def extract_key_themes(self, min_frequency: int = 5) -> Dict[str, int]:
        """Extrai temas principais dos comentários"""
        # Combinar todos os tokens
        all_tokens = []
        for tokens in self.comments_df['tokens']:
            all_tokens.extend(tokens)
\end{lstlisting}
\end{researchbox}
\begin{researchbox}
\begin{lstlisting}[language=Python]         
        # Contar frequências
        token_freq = Counter(all_tokens)
        
        # Filtrar por frequência mínima
        key_themes = {token: freq for token, freq in token_freq.items() 
                     if freq >= min_frequency}
        
        return dict(sorted(key_themes.items(), key=lambda x: x[1], reverse=True))
\end{lstlisting}
\end{researchbox}

\section{Análise de Sentimento}

A análise de sentimento é uma das aplicações mais comuns de NLP em pesquisa, permitindo quantificar atitudes, opiniões e emoções expressas em texto.

\subsection{Abordagens Básicas para Análise de Sentimento}

\begin{pythonbox}
\begin{lstlisting}[language=Python]
# src/nlp/sentiment_analyzer.py
import pandas as pd
from typing import Dict, List, Union
import numpy as np

class SentimentAnalyzer:
    """Analisador de sentimento com múltiplas abordagens"""
    
    def __init__(self, language='portuguese'):
        self.language = language
        self.lexicon = self._load_sentiment_lexicon()
        
    def _load_sentiment_lexicon(self) -> Dict[str, float]:
        """Carrega léxico de sentimento para o idioma"""
        if self.language == 'portuguese':
            # Léxico básico em português
            positive_words = {
                'bom': 1.0, 'ótimo': 2.0, 'excelente': 2.0, 'maravilhoso': 2.0,
                'legal': 1.0, 'bacana': 1.0, 'perfeito': 2.0, 'incrível': 2.0,
                'positivo': 1.0, 'feliz': 1.5, 'alegre': 1.5, 'satisfeito': 1.0,
                'aprovado': 1.0, 'concordo': 1.0, 'apoio': 1.5, 'gosto': 1.0
            }
            
            negative_words = {
                'ruim': -1.0, 'péssimo': -2.0, 'terrível': -2.0, 'horrível': -2.0,
                'negativo': -1.0, 'triste': -1.5, 'raiva': -1.5, 'ódio': -2.0,
                'contra': -1.0, 'discordo': -1.0, 'odeio': -2.0, 'detesto': -2.0,
                'reprovado': -1.0, 'rejeitado': -1.5, 'problema': -1.0
            }
            
            lexicon = {**positive_words, **negative_words}
            
        else:  # English fallback
            lexicon = {
                'good': 1.0, 'great': 2.0, 'excellent': 2.0, 'amazing': 2.0,
                'bad': -1.0, 'terrible': -2.0, 'awful': -2.0, 'horrible': -2.0,
                'love': 2.0, 'like': 1.0, 'hate': -2.0, 'dislike': -1.0
            }
            
        return lexicon
    \end{lstlisting}
\end{pythonbox}
\begin{pythonbox}
\begin{lstlisting}[language=Python]   
    def lexicon_based_sentiment(self, text: str) -> Dict[str, Union[float, str]]:
        """Análise de sentimento baseada em léxico"""
        # Preprocessar texto
        words = text.lower().split()
        
        # Calcular score
        scores = [self.lexicon.get(word, 0) for word in words]
        
        if not scores:
            return {'score': 0.0, 'label': 'neutral', 'confidence': 0.0}
            
        avg_score = np.mean(scores)
        
        # Classificar sentimento
        if avg_score > 0.1:
            label = 'positive'
        elif avg_score < -0.1:
            label = 'negative'
        else:
            label = 'neutral'
            
        # Confiança baseada na magnitude do score
        confidence = min(abs(avg_score), 1.0)
        
        return {
            'score': round(avg_score, 3),
            'label': label,
            'confidence': round(confidence, 3)
        }
\end{lstlisting}
\end{pythonbox}

\newpage

\begin{pythonbox}
\begin{lstlisting}[language=Python]
# Continuação: src/nlp/sentiment_analyzer.py
    
    def analyze_sentiment_dataset(self, texts: List[str], 
                                 method: str = 'lexicon') -> pd.DataFrame:
        """Analisa sentimento de um dataset de textos"""
        results = []
        
        for text in texts:
            if method == 'lexicon':
                sentiment = self.lexicon_based_sentiment(text)
            else:
                sentiment = self.lexicon_based_sentiment(text)
                
            results.append({
                'text': text,
                **sentiment
            })
        
        return pd.DataFrame(results)
    
    def sentiment_trends_over_time(self, df: pd.DataFrame, 
                                  text_col: str, 
                                  date_col: str,
                                  freq: str = 'D') -> pd.DataFrame:
        """Analisa tendências de sentimento ao longo do tempo"""
        # Analisar sentimentos
        sentiments = self.analyze_sentiment_dataset(df[text_col].tolist())
        
        # Adicionar datas
        sentiments['date'] = pd.to_datetime(df[date_col])
        
        # Agrupar por período
        trends = sentiments.set_index('date').resample(freq).agg({
            'score': 'mean',
            'label': lambda x: x.value_counts().index[0] if len(x) > 0 else 'neutral'
        })
        
        return trends

# Exemplo de uso
analyzer = SentimentAnalyzer(language='portuguese')

# Textos exemplo
sample_texts = [
    "Esta política pública é excelente e vai ajudar muito!",
    "Discordo totalmente desta proposta, é muito ruim",
    "A medida tem pontos positivos e negativos",
    "Não entendi direito a proposta"
]

    \end{lstlisting}
\end{pythonbox}
\begin{pythonbox}
\begin{lstlisting}[language=Python]   

# Analisar sentimentos
results = analyzer.analyze_sentiment_dataset(sample_texts)
print("Análise de Sentimento:")
print(results[['text', 'score', 'label']].head())
\end{lstlisting}
\end{pythonbox}

\section{Extração de Entidades e Modelagem de Tópicos}

A extração de informações estruturadas de texto não estruturado é fundamental para muitas análises de pesquisa.

\subsection{Reconhecimento de Entidades Nomeadas (NER)}

\begin{pythonbox}
\begin{lstlisting}[language=Python]
# src/nlp/entity_extractor.py
import re
import pandas as pd
from typing import List, Dict
from collections import defaultdict

class EntityExtractor:
    """Extrator de entidades nomeadas de texto"""
    
    def __init__(self, language='portuguese'):
        self.language = language
        self.patterns = self._load_patterns()
        
    def _load_patterns(self) -> Dict[str, str]:
        """Carrega padrões regex para diferentes tipos de entidades"""
        return {
            'email': r'\b[A-Za-z0-9._%+-]+@[A-Za-z0-9.-]+\.[A-Z|a-z]{2,}\b',
            'phone_br': r'\b(?:\(\d{2}\)\s?)?\d{4,5}[-\s]?\d{4}\b',
            'cpf': r'\b\d{3}\.?\d{3}\.?\d{3}[-.]?\d{2}\b',
            'date_br': r'\b\d{1,2}[\/\-\.]\d{1,2}[\/\-\.]\d{2,4}\b',
            'money_br': r'R\$\s*\d{1,3}(?:\.\d{3})*(?:,\d{2})?',
            'hashtag': r'#\w+',
            'mention': r'@\w+',
        }
        \end{lstlisting}
\end{pythonbox}
\begin{pythonbox}
\begin{lstlisting}[language=Python]   
    def extract_entities(self, text: str, 
                        entity_types: List[str] = None) -> Dict[str, List[str]]:
        """Extrai entidades específicas do texto"""
        if entity_types is None:
            entity_types = list(self.patterns.keys())
            
        entities = defaultdict(list)
        
        for entity_type in entity_types:
            if entity_type in self.patterns:
                pattern = self.patterns[entity_type]
                matches = re.findall(pattern, text, re.IGNORECASE)
                entities[entity_type].extend(matches)
                
        # Remover duplicatas mantendo ordem
        for entity_type in entities:
            entities[entity_type] = list(dict.fromkeys(entities[entity_type]))
            
        return dict(entities)
    
    def extract_named_entities_spacy(self, text: str) -> Dict[str, List[Dict]]:
        """Extrai entidades usando spaCy (se disponível)"""
        try:
            import spacy
            
            # Tentar carregar modelo em português
            try:
                nlp = spacy.load("pt_core_news_sm")
            except OSError:
                return self._fallback_entity_extraction(text)
            
            doc = nlp(text)
            entities = defaultdict(list)
            
            for ent in doc.ents:
                entities[ent.label_].append({
                    'text': ent.text,
                    'start': ent.start_char,
                    'end': ent.end_char
                })
                
            return dict(entities)
            
        except ImportError:
            return self._fallback_entity_extraction(text)
\end{lstlisting}
\end{pythonbox}

\newpage

\begin{pythonbox}
\begin{lstlisting}[language=Python]
# Continuação: src/nlp/entity_extractor.py
    
    def _fallback_entity_extraction(self, text: str) -> Dict[str, List[Dict]]:
        """Extração básica quando spaCy não está disponível"""
        basic_patterns = {
            'PERSON': r'\b[A-ZÁÉÍÓÚÃÕÇ][a-záéíóúãõç]+(?:\s+[A-ZÁÉÍÓÚÃÕÇ][a-záéíóúãõç]+)+\b',
            'ORG': r'\b(?:Empresa|Instituto|Universidade)\s+[A-ZÁÉÍÓÚÃÕÇ][a-záéíóúãõç\s]+\b',
            'LOC': r'\b(?:São Paulo|Rio de Janeiro|Brasil|Brasília)\b',
        }
        
        entities = defaultdict(list)
        
        for entity_type, pattern in basic_patterns.items():
            matches = re.finditer(pattern, text)
            for match in matches:
                entities[entity_type].append({
                    'text': match.group(),
                    'start': match.start(),
                    'end': match.end()
                })
                
        return dict(entities)

# Exemplo de uso
extractor = EntityExtractor(language='portuguese')

sample_text = """
João Silva trabalhou na Empresa ABC de São Paulo.
Seu email é joao@empresa.com e telefone (11) 98765-4321.
A empresa investiu R$ 1.500.000,00 em tecnologia.
"""

# Extrair entidades
basic_entities = extractor.extract_entities(sample_text)
named_entities = extractor.extract_named_entities_spacy(sample_text)

print("Entidades encontradas:")
for entity_type, entities in basic_entities.items():
    if entities:
        print(f"{entity_type}: {entities}")
\end{lstlisting}
\end{pythonbox}

\subsection{Modelagem de Tópicos}

\begin{pythonbox}
\begin{lstlisting}[language=Python]
# src/nlp/topic_modeling.py
import pandas as pd
import numpy as np
from typing import List, Dict
from sklearn.feature_extraction.text import TfidfVectorizer
from sklearn.decomposition import LatentDirichletAllocation

class TopicModeler:
    """Modelador de tópicos para análise de grandes corpus"""
    
    def __init__(self, language='portuguese'):
        self.language = language
        self.vectorizer = None
        self.lda_model = None
        self.feature_names = None
        
    def prepare_documents(self, documents: List[str], 
                         min_df: int = 2, 
                         max_df: float = 0.8,
                         max_features: int = 1000) -> np.ndarray:
        """Prepara documentos para modelagem de tópicos"""
        if self.language == 'portuguese':
            stop_words = ['de', 'a', 'o', 'que', 'e', 'do', 'da', 'em', 'um', 'para', 
                         'é', 'com', 'não', 'uma', 'os', 'no', 'se', 'na', 'por', 'mais']
        else:
            stop_words = 'english'
        
        self.vectorizer = TfidfVectorizer(
            max_df=max_df,
            min_df=min_df,
            max_features=max_features,
            stop_words=stop_words,
            lowercase=True
        )
        
        doc_term_matrix = self.vectorizer.fit_transform(documents)
        self.feature_names = self.vectorizer.get_feature_names_out()
        
        return doc_term_matrix
        \end{lstlisting}
\end{pythonbox}
\begin{pythonbox}
\begin{lstlisting}[language=Python]   
    def fit_lda_model(self, doc_term_matrix: np.ndarray, 
                     n_topics: int = 5) -> LatentDirichletAllocation:
        """Treina modelo LDA para extração de tópicos"""
        self.lda_model = LatentDirichletAllocation(
            n_components=n_topics,
            random_state=42,
            max_iter=100
        )
        
        self.lda_model.fit(doc_term_matrix)
        return self.lda_model
    
    def get_topic_words(self, n_words: int = 10) -> List[List[str]]:
        """Extrai palavras mais importantes para cada tópico"""
        if not self.lda_model or self.feature_names is None:
            return []
            
        topics = []
        for topic_idx, topic in enumerate(self.lda_model.components_):
            top_words_idx = topic.argsort()[-n_words:][::-1]
            top_words = [self.feature_names[i] for i in top_words_idx]
            topics.append(top_words)
            
        return topics
    
    def analyze_corpus_topics(self, documents: List[str], 
                             n_topics: int = 5) -> Dict:
        """Análise completa de tópicos em um corpus"""
        doc_term_matrix = self.prepare_documents(documents)
        self.fit_lda_model(doc_term_matrix, n_topics)
        topic_words = self.get_topic_words()
        
        return {
            'topic_words': topic_words,
            'n_topics': n_topics,
            'vocab_size': len(self.feature_names)
        }

# Exemplo de uso
modeler = TopicModeler(language='portuguese')

sample_documents = [
    "A política de saúde pública precisa de mais investimento",
    "Educação é fundamental para o desenvolvimento do país",
    "O transporte público está em crise",
    "Segurança pública é prioridade",
]
    \end{lstlisting}
\end{pythonbox}
\begin{pythonbox}
\begin{lstlisting}[language=Python]   
results = modeler.analyze_corpus_topics(sample_documents, n_topics=2)

print("Tópicos encontrados:")
for i, words in enumerate(results['topic_words']):
    print(f"Tópico {i}: {', '.join(words[:5])}")
\end{lstlisting}
\end{pythonbox}

\section{Conclusão do Capítulo}

O processamento de linguagem natural representa uma das áreas mais dinâmicas e impactantes da ciência de dados aplicada à pesquisa acadêmica. Python oferece um ecossistema abrangente que permite aos pesquisadores extrair insights profundos de dados textuais, desde análises básicas até técnicas sofisticadas de modelagem.

Os elementos fundamentais abordados neste capítulo incluem:

\textbf{Preprocessamento Robusto:} Técnicas essenciais de limpeza, normalização e tokenização que formam a base de todas as análises subsequentes.

\textbf{Análise de Sentimento:} Métodos para quantificar atitudes e emoções expressas em texto, fundamentais para estudos de opinião pública e comportamento social.

\textbf{Extração de Entidades:} Identificação automática de pessoas, organizações, locais e conceitos importantes em grandes volumes de texto.

\textbf{Modelagem de Tópicos:} Descoberta automática de temas latentes em corpus extensos, revelando estruturas temáticas ocultas.

\begin{examplebox}
\textbf{Principais Competências Desenvolvidas:}
\begin{itemize}
    \item Implementação de pipelines robustos de preprocessamento
    \item Análise de sentimento com múltiplas abordagens
    \item Extração automática de entidades nomeadas
    \item Modelagem de tópicos com LDA
    \item Processamento de texto em português e inglês
    \item Integração de técnicas de NLP em fluxos de pesquisa
    \item Visualização de resultados de análise textual
\end{itemize}
\end{examplebox}

As técnicas apresentadas neste capítulo transformam texto não estruturado em dados quantitativos analisáveis, abrindo novas possibilidades para pesquisa em humanidades digitais, ciências sociais computacionais e estudos interdisciplinares que dependem de análise textual em larga escala.

No próximo capítulo, exploraremos técnicas de machine learning aplicadas à pesquisa acadêmica, construindo sobre os fundamentos de NLP para desenvolver modelos preditivos e classificadores que podem automatizar tarefas de análise e descobrir padrões complexos em dados multidimensionais.
% =============================================================================
% CAPÍTULO 8: MACHINE LEARNING PARA PESQUISADORES
% =============================================================================

\chapter{Machine Learning para Pesquisadores}

\lettrine{O}{machine learning} revolucionou a forma como pesquisadores abordam problemas complexos em diversas disciplinas acadêmicas. Desde a predição de resultados em experimentos até a descoberta de padrões ocultos em grandes datasets, as técnicas de aprendizado de máquina oferecem ferramentas poderosas para automatizar análises, fazer predições e extrair insights que seriam impossíveis de obter através de métodos tradicionais. Python, com seu ecossistema robusto de bibliotecas como scikit-learn, TensorFlow e PyTorch, democratizou o acesso a essas técnicas avançadas.

\section{Fundamentos do Machine Learning em Pesquisa}

Antes de implementar algoritmos complexos, é essencial compreender os fundamentos teóricos e práticos do machine learning aplicado à pesquisa acadêmica.

\subsection{Tipos de Problemas e Algoritmos}

\begin{pythonbox}
\begin{lstlisting}[language=Python]
# src/ml/ml_framework.py
import pandas as pd
import numpy as np
from typing import Dict, List, Tuple, Optional, Union
from sklearn.model_selection import train_test_split
from sklearn.preprocessing import StandardScaler, LabelEncoder
from sklearn.metrics import classification_report, mean_squared_error, r2_score
import matplotlib.pyplot as plt
import seaborn as sns

class ResearchMLFramework:
    """Framework para aplicação de ML em pesquisa acadêmica"""
    
    def __init__(self, random_state: int = 42):
        self.random_state = random_state
        self.models = {}
        self.results = {}
        self.preprocessors = {}
        
    def identify_problem_type(self, target: pd.Series) -> str:
        """Identifica automaticamente o tipo de problema ML"""
        # Verificar se é variável categórica
        if target.dtype == 'object' or target.nunique() < 10:
            unique_values = target.nunique()
            if unique_values == 2:
                return 'binary_classification'
            elif unique_values <= 10:
                return 'multiclass_classification'
            else:
                return 'regression'
        
        # Verificar se é contínua
        elif np.issubdtype(target.dtype, np.number):
            # Se muitos valores únicos, provavelmente regressão
            if target.nunique() > target.count() * 0.1:
                return 'regression'
            else:
                return 'multiclass_classification'
        
        return 'regression'  # fallback
    
    def recommend_algorithms(self, problem_type: str, 
                           dataset_size: int) -> List[str]:
        """Recomenda algoritmos baseado no tipo de problema"""
        recommendations = {
            'binary_classification': {
                'small': ['logistic_regression', 'svm', 'naive_bayes'],
                'medium': ['random_forest', 'gradient_boosting', 'svm'],
                'large': ['logistic_regression', 'gradient_boosting', 'neural_network']
            },
            'multiclass_classification': {
                'small': ['random_forest', 'svm', 'knn'],
                'medium': ['random_forest', 'gradient_boosting', 'svm'],
                'large': ['gradient_boosting', 'neural_network', 'random_forest']
            },
            'regression': {
                'small': ['linear_regression', 'random_forest', 'svm'],
                'medium': ['random_forest', 'gradient_boosting', 'svm'],
                'large': ['gradient_boosting', 'neural_network', 'linear_regression']
            }
        }
        
        # Classificar tamanho do dataset
        if dataset_size < 1000:
            size_category = 'small'
        elif dataset_size < 10000:
            size_category = 'medium'
        else:
            size_category = 'large'
            
        return recommendations.get(problem_type, {}).get(size_category, ['random_forest'])
\end{lstlisting>
\end{pythonbox>

\newpage

\begin{pythonbox}
\begin{lstlisting}[language=Python]
# Continuação: src/ml/ml_framework.py
    
    def prepare_data(self, df: pd.DataFrame, 
                    target_column: str,
                    test_size: float = 0.2) -> Tuple[pd.DataFrame, pd.DataFrame, pd.Series, pd.Series]:
        """Prepara dados para machine learning"""
        
        # Separar features e target
        X = df.drop(columns=[target_column])
        y = df[target_column]
        
        # Tratar valores faltantes
        X = self._handle_missing_values(X)
        
        # Codificar variáveis categóricas
        X = self._encode_categorical_variables(X)
        
        # Dividir em treino e teste
        X_train, X_test, y_train, y_test = train_test_split(
            X, y, test_size=test_size, random_state=self.random_state,
            stratify=y if self.identify_problem_type(y) != 'regression' else None
        )
        
        # Escalar features numéricas
        X_train, X_test = self._scale_features(X_train, X_test)
        
        return X_train, X_test, y_train, y_test
    
    def _handle_missing_values(self, X: pd.DataFrame) -> pd.DataFrame:
        """Trata valores faltantes"""
        X_processed = X.copy()
        
        for column in X_processed.columns:
            if X_processed[column].isnull().any():
                if X_processed[column].dtype == 'object':
                    # Variáveis categóricas: usar moda
                    mode_value = X_processed[column].mode()[0] if len(X_processed[column].mode()) > 0 else 'Unknown'
                    X_processed[column].fillna(mode_value, inplace=True)
                else:
                    # Variáveis numéricas: usar mediana
                    median_value = X_processed[column].median()
                    X_processed[column].fillna(median_value, inplace=True)
        
        return X_processed
    
    def _encode_categorical_variables(self, X: pd.DataFrame) -> pd.DataFrame:
        """Codifica variáveis categóricas"""
        X_processed = X.copy()
        
        for column in X_processed.columns:
            if X_processed[column].dtype == 'object':
                # Para variáveis com muitas categorias, usar target encoding
                if X_processed[column].nunique() > 10:
                    # Por simplicidade, usar label encoding
                    le = LabelEncoder()
                    X_processed[column] = le.fit_transform(X_processed[column].astype(str))
                    self.preprocessors[f'{column}_encoder'] = le
                else:
                    # Para poucas categorias, usar one-hot encoding
                    dummies = pd.get_dummies(X_processed[column], prefix=column)
                    X_processed = pd.concat([X_processed.drop(column, axis=1), dummies], axis=1)
        
        return X_processed
    
    def _scale_features(self, X_train: pd.DataFrame, 
                       X_test: pd.DataFrame) -> Tuple[pd.DataFrame, pd.DataFrame]:
        """Escala features numéricas"""
        scaler = StandardScaler()
        
        # Identificar colunas numéricas
        numeric_columns = X_train.select_dtypes(include=[np.number]).columns
        
        if len(numeric_columns) > 0:
            X_train_scaled = X_train.copy()
            X_test_scaled = X_test.copy()
            
            X_train_scaled[numeric_columns] = scaler.fit_transform(X_train[numeric_columns])
            X_test_scaled[numeric_columns] = scaler.transform(X_test[numeric_columns])
            
            self.preprocessors['scaler'] = scaler
            
            return X_train_scaled, X_test_scaled
        
        return X_train, X_test
\end{lstlisting}
\end{pythonbox}

\begin{researchbox}
\textbf{Caso Real - Predição de Sucesso Acadêmico:}

Uma pesquisadora em educação quer prever o sucesso acadêmico de estudantes baseado em variáveis socioeconômicas e de desempenho:

\begin{lstlisting}[language=Python]
# src/ml/academic_success_predictor.py
from ml_framework import ResearchMLFramework
import pandas as pd
from sklearn.ensemble import RandomForestClassifier
from sklearn.linear_model import LogisticRegression
from sklearn.metrics import classification_report, confusion_matrix

class AcademicSuccessPredictor:
    """Preditor de sucesso acadêmico usando ML"""
    
    def __init__(self):
        self.ml_framework = ResearchMLFramework()
        self.models = {}
        
    def load_and_analyze_data(self, filepath: str) -> pd.DataFrame:
        """Carrega e analisa dados acadêmicos"""
        df = pd.read_csv(filepath)
        
        # Análise exploratória básica
        print("Análise Exploratória dos Dados:")
        print(f"Dimensões: {df.shape}")
        print(f"Valores faltantes: {df.isnull().sum().sum()}")
        print(f"Variáveis categóricas: {df.select_dtypes(include=['object']).columns.tolist()}")
        
        return df
    
    def create_success_variable(self, df: pd.DataFrame) -> pd.DataFrame:
        """Cria variável binária de sucesso acadêmico"""
        # Assumindo que temos uma coluna 'nota_final'
        df_processed = df.copy()
        
        # Definir sucesso como nota >= 7.0
        df_processed['sucesso_academico'] = (df_processed['nota_final'] >= 7.0).astype(int)
        
        return df_processed
    
    def train_models(self, df: pd.DataFrame) -> Dict:
        """Treina múltiplos modelos para comparação"""
        # Preparar dados
        X_train, X_test, y_train, y_test = self.ml_framework.prepare_data(
            df, 'sucesso_academico'
        )
        
        # Modelos a serem testados
        models_to_test = {
            'logistic_regression': LogisticRegression(random_state=42),
            'random_forest': RandomForestClassifier(random_state=42, n_estimators=100)
        }
        
        results = {}
        
        for name, model in models_to_test.items():
            # Treinar modelo
            model.fit(X_train, y_train)
            
            # Fazer predições
            y_pred = model.predict(X_test)
            
            # Avaliar performance
            report = classification_report(y_test, y_pred, output_dict=True)
            
            results[name] = {
                'model': model,
                'predictions': y_pred,
                'classification_report': report,
                'accuracy': report['accuracy']
            }
            
            print(f"\nResultados - {name}:")
            print(f"Acurácia: {report['accuracy']:.3f}")
            print(f"F1-Score: {report['macro avg']['f1-score']:.3f}")
        
        self.models = results
        return results
\end{lstlisting}
\end{researchbox}

\section{Classificação para Problemas de Pesquisa}

A classificação é uma das tarefas mais comuns em machine learning aplicado à pesquisa, permitindo categorizar observações em grupos predefinidos.

\subsection{Classificação Binária e Multiclasse}

\begin{pythonbox}
\begin{lstlisting}[language=Python]
# src/ml/classification_toolkit.py
import pandas as pd
import numpy as np
from sklearn.ensemble import RandomForestClassifier, GradientBoostingClassifier
from sklearn.linear_model import LogisticRegression
from sklearn.svm import SVC
from sklearn.naive_bayes import GaussianNB
from sklearn.metrics import accuracy_score, precision_recall_fscore_support
from sklearn.model_selection import cross_val_score, GridSearchCV
import matplotlib.pyplot as plt

class ClassificationToolkit:
    """Toolkit para problemas de classificação em pesquisa"""
    
    def __init__(self, random_state: int = 42):
        self.random_state = random_state
        self.models = self._initialize_models()
        self.best_model = None
        self.best_params = None
        
    def _initialize_models(self) -> Dict:
        """Inicializa modelos de classificação"""
        return {
            'logistic_regression': LogisticRegression(random_state=self.random_state),
            'random_forest': RandomForestClassifier(random_state=self.random_state),
            'gradient_boosting': GradientBoostingClassifier(random_state=self.random_state),
            'svm': SVC(random_state=self.random_state),
            'naive_bayes': GaussianNB()
        }
    
    def compare_algorithms(self, X_train: pd.DataFrame, 
                          y_train: pd.Series,
                          cv_folds: int = 5) -> pd.DataFrame:
        """Compara performance de diferentes algoritmos"""
        results = []
        
        for name, model in self.models.items():
            try:
                # Validação cruzada
                cv_scores = cross_val_score(
                    model, X_train, y_train, 
                    cv=cv_folds, scoring='accuracy'
                )


\end{lstlisting}
\end{pythonbox}
\begin{pythonbox}
\begin{lstlisting}[language=Python]                
                results.append({
                    'algorithm': name,
                    'mean_accuracy': cv_scores.mean(),
                    'std_accuracy': cv_scores.std(),
                    'min_accuracy': cv_scores.min(),
                    'max_accuracy': cv_scores.max()
                })
                
            except Exception as e:
                print(f"Erro ao treinar {name}: {e}")
                continue
        
        results_df = pd.DataFrame(results)
        results_df = results_df.sort_values('mean_accuracy', ascending=False)
        
        return results_df
    
    def hyperparameter_tuning(self, X_train: pd.DataFrame, 
                             y_train: pd.Series,
                             algorithm: str = 'random_forest') -> Dict:
        """Otimização de hiperparâmetros"""
        param_grids = {
            'random_forest': {
                'n_estimators': [50, 100, 200],
                'max_depth': [None, 10, 20],
                'min_samples_split': [2, 5, 10]
            },
            'logistic_regression': {
                'C': [0.1, 1.0, 10.0],
                'penalty': ['l1', 'l2'],
                'solver': ['liblinear']
            },
            'svm': {
                'C': [0.1, 1.0, 10.0],
                'kernel': ['rbf', 'linear'],
                'gamma': ['scale', 'auto']
            }
        }
        
        if algorithm not in param_grids:
            raise ValueError(f"Algoritmo {algorithm} não suportado")
        
        model = self.models[algorithm]
        param_grid = param_grids[algorithm]

\end{lstlisting}
\end{pythonbox}
\begin{pythonbox}
\begin{lstlisting}[language=Python]          
        # Grid search com validação cruzada

        
        grid_search = GridSearchCV(
            model, param_grid, cv=5, 
            scoring='accuracy', n_jobs=-1
        )
        
        grid_search.fit(X_train, y_train)
        
        self.best_model = grid_search.best_estimator_
        self.best_params = grid_search.best_params_
        
        return {
            'best_params': grid_search.best_params_,
            'best_score': grid_search.best_score_,
            'best_model': grid_search.best_estimator_
        }
\end{lstlisting}
\end{pythonbox}

\newpage

\begin{pythonbox}
\begin{lstlisting}[language=Python]
# Continuação: src/ml/classification_toolkit.py
    
    def evaluate_model(self, model, X_test: pd.DataFrame, 
                      y_test: pd.Series) -> Dict:
        """Avalia performance detalhada do modelo"""
        y_pred = model.predict(X_test)
        
        # Métricas básicas
        accuracy = accuracy_score(y_test, y_pred)
        precision, recall, f1, support = precision_recall_fscore_support(
            y_test, y_pred, average='weighted'
        )
        
        # Métricas por classe
        precision_per_class, recall_per_class, f1_per_class, _ = \
            precision_recall_fscore_support(y_test, y_pred, average=None)
        
        # Matriz de confusão
        from sklearn.metrics import confusion_matrix
        cm = confusion_matrix(y_test, y_pred)
        
        return {
            'accuracy': accuracy,
            'precision': precision,
            'recall': recall,
            'f1_score': f1,
            'precision_per_class': precision_per_class,
            'recall_per_class': recall_per_class,
            'f1_per_class': f1_per_class,
            'confusion_matrix': cm,
            'predictions': y_pred
        }
    
    def feature_importance_analysis(self, model, 
                                   feature_names: List[str]) -> pd.DataFrame:
        """Analisa importância das features"""
        if hasattr(model, 'feature_importances_'):
            importances = model.feature_importances_
        elif hasattr(model, 'coef_'):
            # Para modelos lineares, usar valor absoluto dos coeficientes
            importances = np.abs(model.coef_[0] if len(model.coef_.shape) > 1 else model.coef_)
        else:
            print("Modelo não suporta análise de importância")
            return pd.DataFrame()
\end{lstlisting}
\end{pythonbox}
\begin{pythonbox}
\begin{lstlisting}[language=Python]          
        importance_df = pd.DataFrame({
            'feature': feature_names,
            'importance': importances
        }).sort_values('importance', ascending=False)
        
        return importance_df
    
    def plot_performance_comparison(self, results_df: pd.DataFrame):
        """Visualiza comparação de performance"""
        plt.figure(figsize=(12, 6))
        
        # Gráfico de barras com erro
        plt.errorbar(
            range(len(results_df)), 
            results_df['mean_accuracy'],
            yerr=results_df['std_accuracy'],
            fmt='o', capsize=5, capthick=2
        )
        
        plt.bar(range(len(results_df)), results_df['mean_accuracy'], 
                alpha=0.7, color='skyblue')
        
        plt.xlabel('Algoritmos')
        plt.ylabel('Acurácia Média')
        plt.title('Comparação de Performance entre Algoritmos')
        plt.xticks(range(len(results_df)), results_df['algorithm'], rotation=45)
        plt.grid(True, alpha=0.3)
        plt.tight_layout()
        plt.show()

# Exemplo de uso
toolkit = ClassificationToolkit()

# Dados exemplo (substituir por dados reais)
X_train = pd.DataFrame(np.random.randn(1000, 5), 
                      columns=['feature1', 'feature2', 'feature3', 'feature4', 'feature5'])
y_train = pd.Series(np.random.choice([0, 1], 1000))

# Comparar algoritmos
comparison_results = toolkit.compare_algorithms(X_train, y_train)
print("Comparação de Algoritmos:")
print(comparison_results)

# Otimizar melhor algoritmo
best_algorithm = comparison_results.iloc[0]['algorithm']
tuning_results = toolkit.hyperparameter_tuning(X_train, y_train, best_algorithm)
print(f"\nMelhores parâmetros para {best_algorithm}:")
print(tuning_results['best_params'])
\end{lstlisting}
\end{pythonbox}

\section{Regressão para Predição Quantitativa}

A regressão permite prever valores contínuos, essencial para estudos que envolvem medições quantitativas e análises preditivas.

\subsection{Regressão Linear e Não-Linear}

\begin{pythonbox}
\begin{lstlisting}[language=Python]
# src/ml/regression_toolkit.py
import pandas as pd
import numpy as np
from sklearn.linear_model import LinearRegression, Ridge, Lasso, ElasticNet
from sklearn.ensemble import RandomForestRegressor, GradientBoostingRegressor
from sklearn.svm import SVR
from sklearn.metrics import mean_squared_error, r2_score, mean_absolute_error
from sklearn.model_selection import cross_val_score
import matplotlib.pyplot as plt

class RegressionToolkit:
    """Toolkit para problemas de regressão em pesquisa"""
    
    def __init__(self, random_state: int = 42):
        self.random_state = random_state
        self.models = self._initialize_models()
        self.trained_models = {}
        
    def _initialize_models(self) -> Dict:
        """Inicializa modelos de regressão"""
        return {
            'linear_regression': LinearRegression(),
            'ridge_regression': Ridge(random_state=self.random_state),
            'lasso_regression': Lasso(random_state=self.random_state),
            'elastic_net': ElasticNet(random_state=self.random_state),
            'random_forest': RandomForestRegressor(random_state=self.random_state),
            'gradient_boosting': GradientBoostingRegressor(random_state=self.random_state),
            'svr': SVR()
        }
    
    def compare_regression_models(self, X_train: pd.DataFrame, 
                                 y_train: pd.Series) -> pd.DataFrame:
        """Compara diferentes modelos de regressão"""
        results = []
        
        for name, model in self.models.items():
            try:
                # Validação cruzada para R2
                r2_scores = cross_val_score(
                    model, X_train, y_train, 
                    cv=5, scoring='r2'
                )
\end{lstlisting}
\end{pythonbox}
\begin{pythonbox}
\begin{lstlisting}[language=Python]                  
                # Validação cruzada para MSE (negativo)
                mse_scores = -cross_val_score(
                    model, X_train, y_train, 
                    cv=5, scoring='neg_mean_squared_error'
                )
                
                results.append({
                    'algorithm': name,
                    'mean_r2': r2_scores.mean(),
                    'std_r2': r2_scores.std(),
                    'mean_mse': mse_scores.mean(),
                    'std_mse': mse_scores.std(),
                    'mean_rmse': np.sqrt(mse_scores.mean())
                })
                
                # Treinar modelo para análises posteriores
                model.fit(X_train, y_train)
                self.trained_models[name] = model
                
            except Exception as e:
                print(f"Erro ao treinar {name}: {e}")
                continue
        
        results_df = pd.DataFrame(results)
        results_df = results_df.sort_values('mean_r2', ascending=False)
        
        return results_df
    
    def detailed_evaluation(self, model, X_test: pd.DataFrame, 
                           y_test: pd.Series, model_name: str) -> Dict:
        """Avaliação detalhada de modelo de regressão"""
        y_pred = model.predict(X_test)
        
        # Métricas
        r2 = r2_score(y_test, y_pred)
        mse = mean_squared_error(y_test, y_pred)
        rmse = np.sqrt(mse)
        mae = mean_absolute_error(y_test, y_pred)
        
        # Erro percentual absoluto médio
        mape = np.mean(np.abs((y_test - y_pred) / y_test)) * 100
        
        return {
            'model_name': model_name,
            'r2_score': r2,
            'mse': mse,
            'rmse': rmse,
            'mae': mae,
            'mape': mape,
            'predictions': y_pred,
            'residuals': y_test - y_pred
        }
\end{lstlisting}
\end{pythonbox}

\newpage

\begin{pythonbox}
\begin{lstlisting}[language=Python]
# Continuação: src/ml/regression_toolkit.py
    
    def plot_regression_diagnostics(self, evaluation_results: Dict):
        """Cria gráficos de diagnóstico para regressão"""
        y_true = evaluation_results['predictions'] + evaluation_results['residuals']
        y_pred = evaluation_results['predictions']
        residuals = evaluation_results['residuals']
        
        fig, axes = plt.subplots(2, 2, figsize=(15, 10))
        
        # 1. Valores preditos vs valores reais
        axes[0,0].scatter(y_true, y_pred, alpha=0.6)
        axes[0,0].plot([y_true.min(), y_true.max()], [y_true.min(), y_true.max()], 'r--', lw=2)
        axes[0,0].set_xlabel('Valores Reais')
        axes[0,0].set_ylabel('Valores Preditos')
        axes[0,0].set_title('Predições vs Valores Reais')
        
        # 2. Resíduos vs valores preditos
        axes[0,1].scatter(y_pred, residuals, alpha=0.6)
        axes[0,1].axhline(y=0, color='r', linestyle='--')
        axes[0,1].set_xlabel('Valores Preditos')
        axes[0,1].set_ylabel('Resíduos')
        axes[0,1].set_title('Resíduos vs Valores Preditos')
        
        # 3. Histograma dos resíduos
        axes[1,0].hist(residuals, bins=30, alpha=0.7, color='skyblue')
        axes[1,0].set_xlabel('Resíduos')
        axes[1,0].set_ylabel('Frequência')
        axes[1,0].set_title('Distribuição dos Resíduos')
        
        # 4. Q-Q plot dos resíduos
        from scipy import stats
        stats.probplot(residuals, dist="norm", plot=axes[1,1])
        axes[1,1].set_title('Q-Q Plot dos Resíduos')
        
        plt.tight_layout()
        plt.show()
        
        # Estatísticas dos resíduos
        print(f"Estatísticas dos Resíduos:")
        print(f"Média: {residuals.mean():.4f}")
        print(f"Desvio Padrão: {residuals.std():.4f}")
        print(f"Teste de Normalidade (Shapiro-Wilk): {stats.shapiro(residuals)[1]:.4f}")
    \end{lstlisting}
\end{pythonbox}
\begin{pythonbox}
\begin{lstlisting}[language=Python]  
    def feature_importance_regression(self, model, 
                                    feature_names: List[str]) -> pd.DataFrame:
        """Analisa importância das features em regressão"""
        if hasattr(model, 'feature_importances_'):
            # Modelos baseados em árvores
            importances = model.feature_importances_
        elif hasattr(model, 'coef_'):
            # Modelos lineares
            importances = np.abs(model.coef_)
        else:
            print("Modelo não suporta análise de importância")
            return pd.DataFrame()
        
        importance_df = pd.DataFrame({
            'feature': feature_names,
            'importance': importances
        }).sort_values('importance', ascending=False)
        
        return importance_df

# Exemplo de uso
regression_toolkit = RegressionToolkit()

# Dados exemplo (substituir por dados reais)
X_train = pd.DataFrame(np.random.randn(1000, 5), 
                      columns=['var1', 'var2', 'var3', 'var4', 'var5'])
y_train = pd.Series(2*X_train['var1'] + X_train['var2'] + np.random.randn(1000)*0.5)

# Comparar modelos de regressão
regression_comparison = regression_toolkit.compare_regression_models(X_train, y_train)
print("Comparação de Modelos de Regressão:")
print(regression_comparison[['algorithm', 'mean_r2', 'mean_rmse']])

# Avaliar melhor modelo
best_model_name = regression_comparison.iloc[0]['algorithm']
best_model = regression_toolkit.trained_models[best_model_name]

# Criar dados de teste
X_test = pd.DataFrame(np.random.randn(200, 5), 
                     columns=['var1', 'var2', 'var3', 'var4', 'var5'])
y_test = pd.Series(2*X_test['var1'] + X_test['var2'] + np.random.randn(200)*0.5)

# Avaliação detalhada
evaluation = regression_toolkit.detailed_evaluation(best_model, X_test, y_test, best_model_name)
print(f"\nAvaliação do {best_model_name}:")
print(f"R2 Score: {evaluation['r2_score']:.4f}")
print(f"RMSE: {evaluation['rmse']:.4f}")
print(f"MAE: {evaluation['mae']:.4f}")
\end{lstlisting}
\end{pythonbox}

\section{Aprendizado Não Supervisionado}

O aprendizado não supervisionado permite descobrir padrões ocultos em dados sem variáveis target predefinidas, essencial para análise exploratória e descoberta de insights.

\subsection{Clustering e Análise de Componentes}

\begin{pythonbox}
\begin{lstlisting}[language=Python]
# src/ml/unsupervised_toolkit.py
import pandas as pd
import numpy as np
from sklearn.cluster import KMeans, DBSCAN, AgglomerativeClustering
from sklearn.decomposition import PCA, FactorAnalysis
from sklearn.manifold import TSNE
from sklearn.preprocessing import StandardScaler
from sklearn.metrics import silhouette_score, adjusted_rand_score
import matplotlib.pyplot as plt
import seaborn as sns

class UnsupervisedToolkit:
    """Toolkit para aprendizado não supervisionado em pesquisa"""
    
    def __init__(self, random_state: int = 42):
        self.random_state = random_state
        self.clustering_models = {}
        self.dimensionality_models = {}
        
    def optimal_clusters_analysis(self, X: pd.DataFrame, 
                                 max_clusters: int = 10) -> Dict:
        """Determina número ótimo de clusters usando múltiplos métodos"""
        X_scaled = StandardScaler().fit_transform(X)
        
        # Método do cotovelo (inércia)
        inertias = []
        silhouette_scores = []
        k_range = range(2, max_clusters + 1)
        
        for k in k_range:
            kmeans = KMeans(n_clusters=k, random_state=self.random_state, n_init=10)
            cluster_labels = kmeans.fit_predict(X_scaled)
            
            inertias.append(kmeans.inertia_)
            silhouette_scores.append(silhouette_score(X_scaled, cluster_labels))
        
        # Encontrar cotovelo
        optimal_k_elbow = self._find_elbow_point(k_range, inertias)
        optimal_k_silhouette = k_range[np.argmax(silhouette_scores)]
        
        return {
            'k_range': list(k_range),
            'inertias': inertias,
            'silhouette_scores': silhouette_scores,
            'optimal_k_elbow': optimal_k_elbow,
            'optimal_k_silhouette': optimal_k_silhouette,
            'max_silhouette_score': max(silhouette_scores)
        }
    \end{lstlisting}
\end{pythonbox}
\begin{pythonbox}
\begin{lstlisting}[language=Python]  
    def _find_elbow_point(self, k_range: range, inertias: List[float]) -> int:
        """Encontra o ponto do cotovelo na curva de inércia"""
        # Método da segunda derivada
        if len(inertias) < 3:
            return k_range[0]
            
        # Calcular diferenças
        first_diff = np.diff(inertias)
        second_diff = np.diff(first_diff)
        
        # Encontrar ponto onde a segunda derivada é máxima
        elbow_idx = np.argmax(second_diff) + 2  # +2 devido aos diffs
        return list(k_range)[min(elbow_idx, len(k_range) - 1)]
    
    def apply_clustering_algorithms(self, X: pd.DataFrame, 
                                   n_clusters: int = 3) -> Dict:
        """Aplica diferentes algoritmos de clustering"""
        X_scaled = StandardScaler().fit_transform(X)
        
        algorithms = {
            'kmeans': KMeans(n_clusters=n_clusters, random_state=self.random_state),
            'agglomerative': AgglomerativeClustering(n_clusters=n_clusters),
            'dbscan': DBSCAN(eps=0.5, min_samples=5)
        }
        
        results = {}
        
        for name, algorithm in algorithms.items():
            cluster_labels = algorithm.fit_predict(X_scaled)
            
            # Calcular métricas (se houver mais de 1 cluster)
            if len(np.unique(cluster_labels)) > 1:
                silhouette = silhouette_score(X_scaled, cluster_labels)
            else:
                silhouette = -1
            
            results[name] = {
                'labels': cluster_labels,
                'n_clusters_found': len(np.unique(cluster_labels)),
                'silhouette_score': silhouette,
                'model': algorithm
            }
            
            self.clustering_models[name] = algorithm
        
        return results
\end{lstlisting}
\end{pythonbox}

\newpage

\begin{pythonbox}
\begin{lstlisting}[language=Python]
# Continuação: src/ml/unsupervised_toolkit.py
    
    def dimensionality_reduction(self, X: pd.DataFrame, 
                                n_components: int = 2) -> Dict:
        """Aplica técnicas de redução de dimensionalidade"""
        X_scaled = StandardScaler().fit_transform(X)
        
        # PCA
        pca = PCA(n_components=n_components, random_state=self.random_state)
        X_pca = pca.fit_transform(X_scaled)
        
        # t-SNE
        tsne = TSNE(n_components=n_components, random_state=self.random_state)
        X_tsne = tsne.fit_transform(X_scaled)
        
        # Factor Analysis
        fa = FactorAnalysis(n_components=n_components, random_state=self.random_state)
        X_fa = fa.fit_transform(X_scaled)
        
        results = {
            'pca': {
                'transformed_data': X_pca,
                'explained_variance_ratio': pca.explained_variance_ratio_,
                'cumulative_variance': np.cumsum(pca.explained_variance_ratio_),
                'model': pca
            },
            'tsne': {
                'transformed_data': X_tsne,
                'model': tsne
            },
            'factor_analysis': {
                'transformed_data': X_fa,
                'model': fa
            }
        }
        
        self.dimensionality_models = {
            'pca': pca,
            'tsne': tsne,
            'factor_analysis': fa
        }
        
        return results
\end{lstlisting}
\end{pythonbox}
\begin{pythonbox}
\begin{lstlisting}[language=Python]      
    def plot_clustering_results(self, X: pd.DataFrame, 
                               clustering_results: Dict,
                               reduction_results: Dict):
        """Visualiza resultados de clustering"""
        n_algorithms = len(clustering_results)
        fig, axes = plt.subplots(1, n_algorithms, figsize=(5*n_algorithms, 4))
        
        if n_algorithms == 1:
            axes = [axes]
        
        # Usar PCA para visualização 2D
        X_pca = reduction_results['pca']['transformed_data']
        
        for i, (name, result) in enumerate(clustering_results.items()):
            labels = result['labels']
            n_clusters = result['n_clusters_found']
            silhouette = result['silhouette_score']
            
            scatter = axes[i].scatter(X_pca[:, 0], X_pca[:, 1], 
                                    c=labels, cmap='viridis', alpha=0.6)
            axes[i].set_title(f'{name.title()}\n'
                            f'Clusters: {n_clusters}, Silhouette: {silhouette:.3f}')
            axes[i].set_xlabel('PC1')
            axes[i].set_ylabel('PC2')
            
            # Adicionar colorbar
            plt.colorbar(scatter, ax=axes[i])
        
        plt.tight_layout()
        plt.show()
    
    def analyze_cluster_characteristics(self, X: pd.DataFrame, 
                                      cluster_labels: np.ndarray) -> pd.DataFrame:
        """Analisa características de cada cluster"""
        df_with_clusters = X.copy()
        df_with_clusters['cluster'] = cluster_labels
        
        # Estatísticas por cluster
        cluster_stats = df_with_clusters.groupby('cluster').agg([
            'mean', 'std', 'median', 'count'
        ]).round(3)
        
        return cluster_stats
\end{lstlisting}
\end{pythonbox}
\begin{pythonbox}
\begin{lstlisting}[language=Python]  
# Exemplo de uso
unsupervised_toolkit = UnsupervisedToolkit()

# Dados exemplo
np.random.seed(42)
X_example = pd.DataFrame({
    'feature1': np.concatenate([np.random.normal(0, 1, 100), np.random.normal(3, 1, 100)]),
    'feature2': np.concatenate([np.random.normal(0, 1, 100), np.random.normal(3, 1, 100)]),
    'feature3': np.concatenate([np.random.normal(1, 0.5, 100), np.random.normal(2, 0.5, 100)])
})

# Análise do número ótimo de clusters
optimal_analysis = unsupervised_toolkit.optimal_clusters_analysis(X_example)
print(f"Número ótimo de clusters (cotovelo): {optimal_analysis['optimal_k_elbow']}")
print(f"Número ótimo de clusters (silhouette): {optimal_analysis['optimal_k_silhouette']}")

# Aplicar algoritmos de clustering
clustering_results = unsupervised_toolkit.apply_clustering_algorithms(X_example, n_clusters=2)

# Redução de dimensionalidade
reduction_results = unsupervised_toolkit.dimensionality_reduction(X_example)

print(f"Variância explicada pelo PCA: {reduction_results['pca']['explained_variance_ratio']}")
\end{lstlisting}
\end{pythonbox}

\section{Validação e Interpretação de Modelos}

A validação rigorosa e interpretação adequada dos modelos são fundamentais para garantir a confiabilidade dos resultados em pesquisa acadêmica.

\subsection{Técnicas de Validação Cruzada}

\begin{pythonbox}
\begin{lstlisting}[language=Python]
# src/ml/model_validation.py
import pandas as pd
import numpy as np
from sklearn.model_selection import (
    cross_val_score, StratifiedKFold, TimeSeriesSplit,
    validation_curve, learning_curve
)
from sklearn.metrics import classification_report, confusion_matrix
import matplotlib.pyplot as plt
import seaborn as sns
from typing import Dict, List, Tuple

class ModelValidation:
    """Classe para validação rigorosa de modelos ML"""
    
    def __init__(self, random_state: int = 42):
        self.random_state = random_state
        
    def comprehensive_cross_validation(self, model, X: pd.DataFrame, 
                                     y: pd.Series, 
                                     problem_type: str = 'classification') -> Dict:
        """Validação cruzada abrangente"""
        results = {}
        
        # Escolher estratégia de validação
        if problem_type == 'classification':
            cv_strategy = StratifiedKFold(n_splits=5, shuffle=True, 
                                        random_state=self.random_state)
            scoring_metrics = ['accuracy', 'precision_macro', 'recall_macro', 'f1_macro']
        else:  # regression
            cv_strategy = 5  # KFold padrão
            scoring_metrics = ['r2', 'neg_mean_squared_error', 'neg_mean_absolute_error']
        
        # Calcular métricas
        for metric in scoring_metrics:
            scores = cross_val_score(model, X, y, cv=cv_strategy, scoring=metric)
            results[metric] = {
                'scores': scores,
                'mean': scores.mean(),
                'std': scores.std(),
                'ci_95': (scores.mean() - 1.96*scores.std(), 
                         scores.mean() + 1.96*scores.std())
            }
        
        return results
\end{lstlisting}
\end{pythonbox}
\begin{pythonbox}
\begin{lstlisting}[language=Python]      
    def temporal_validation(self, model, X: pd.DataFrame, 
                           y: pd.Series, n_splits: int = 5) -> Dict:
        """Validação para dados temporais"""
        tscv = TimeSeriesSplit(n_splits=n_splits)
        
        fold_results = []
        for fold, (train_idx, test_idx) in enumerate(tscv.split(X)):
            X_train, X_test = X.iloc[train_idx], X.iloc[test_idx]
            y_train, y_test = y.iloc[train_idx], y.iloc[test_idx]
            
            # Treinar e avaliar
            model.fit(X_train, y_train)
            score = model.score(X_test, y_test)
            
            fold_results.append({
                'fold': fold,
                'train_size': len(train_idx),
                'test_size': len(test_idx),
                'score': score
            })
        
        return {
            'fold_results': fold_results,
            'mean_score': np.mean([r['score'] for r in fold_results]),
            'std_score': np.std([r['score'] for r in fold_results])
        }
    
    def learning_curve_analysis(self, model, X: pd.DataFrame, 
                               y: pd.Series) -> Dict:
        """Análise de curva de aprendizado"""
        train_sizes = np.linspace(0.1, 1.0, 10)
        
        train_sizes_abs, train_scores, val_scores = learning_curve(
            model, X, y, train_sizes=train_sizes, cv=5,
            random_state=self.random_state, n_jobs=-1
        )
        
        return {
            'train_sizes': train_sizes_abs,
            'train_scores_mean': train_scores.mean(axis=1),
            'train_scores_std': train_scores.std(axis=1),
            'val_scores_mean': val_scores.mean(axis=1),
            'val_scores_std': val_scores.std(axis=1)
        }
\end{lstlisting}
\end{pythonbox}
\begin{pythonbox}
\begin{lstlisting}[language=Python]      
    def plot_learning_curves(self, learning_results: Dict, 
                            title: str = "Curvas de Aprendizado"):
        """Visualiza curvas de aprendizado"""
        plt.figure(figsize=(10, 6))
        
        train_sizes = learning_results['train_sizes']
        train_mean = learning_results['train_scores_mean']
        train_std = learning_results['train_scores_std']
        val_mean = learning_results['val_scores_mean']
        val_std = learning_results['val_scores_std']
        
        # Curvas de treino
        plt.plot(train_sizes, train_mean, 'o-', color='blue', label='Score de Treino')
        plt.fill_between(train_sizes, train_mean - train_std, train_mean + train_std, 
                        alpha=0.2, color='blue')
        
        # Curvas de validação
        plt.plot(train_sizes, val_mean, 'o-', color='red', label='Score de Validação')
        plt.fill_between(train_sizes, val_mean - val_std, val_mean + val_std, 
                        alpha=0.2, color='red')
        
        plt.xlabel('Tamanho do Conjunto de Treino')
        plt.ylabel('Score')
        plt.title(title)
        plt.legend(loc='best')
        plt.grid(True, alpha=0.3)
        plt.show()
\end{lstlisting}
\end{pythonbox}

\newpage

\begin{pythonbox}
\begin{lstlisting}[language=Python]
# Continuação: src/ml/model_validation.py
    
    def statistical_significance_test(self, model1_scores: np.ndarray,
                                    model2_scores: np.ndarray,
                                    alpha: float = 0.05) -> Dict:
        """Teste de significância estatística entre modelos"""
        from scipy import stats
        
        # Teste t pareado
        t_stat, p_value = stats.ttest_rel(model1_scores, model2_scores)
        
        # Teste de Wilcoxon (não-paramétrico)
        wilcoxon_stat, wilcoxon_p = stats.wilcoxon(model1_scores, model2_scores)
        
        return {
            't_statistic': t_stat,
            't_test_p_value': p_value,
            't_test_significant': p_value < alpha,
            'wilcoxon_statistic': wilcoxon_stat,
            'wilcoxon_p_value': wilcoxon_p,
            'wilcoxon_significant': wilcoxon_p < alpha,
            'mean_difference': np.mean(model1_scores - model2_scores)
        }
    
    def bootstrap_confidence_interval(self, model, X: pd.DataFrame, 
                                    y: pd.Series, 
                                    n_bootstrap: int = 1000,
                                    confidence: float = 0.95) -> Dict:

\end{lstlisting}
\end{pythonbox}
\begin{pythonbox}
\begin{lstlisting}[language=Python]                                      
        """Calcula intervalo de confiança via bootstrap"""
        bootstrap_scores = []
        
        for _ in range(n_bootstrap):
            # Amostragem com reposição
            indices = np.random.choice(len(X), size=len(X), replace=True)
            X_boot = X.iloc[indices]
            y_boot = y.iloc[indices]
            
            # Dividir em treino e teste
            split_idx = int(0.8 * len(X_boot))
            X_train, X_test = X_boot[:split_idx], X_boot[split_idx:]
            y_train, y_test = y_boot[:split_idx], y_boot[split_idx:]
            
            # Treinar e avaliar
            model.fit(X_train, y_train)
            score = model.score(X_test, y_test)
            bootstrap_scores.append(score)
        
        bootstrap_scores = np.array(bootstrap_scores)
        
        # Calcular intervalo de confiança
        alpha = 1 - confidence
        lower_percentile = (alpha/2) * 100
        upper_percentile = (1 - alpha/2) * 100
        
        ci_lower = np.percentile(bootstrap_scores, lower_percentile)
        ci_upper = np.percentile(bootstrap_scores, upper_percentile)
        
        return {
            'bootstrap_scores': bootstrap_scores,
            'mean_score': bootstrap_scores.mean(),
            'std_score': bootstrap_scores.std(),
            'confidence_interval': (ci_lower, ci_upper),
            'confidence_level': confidence
        }

# Exemplo de uso
validator = ModelValidation()

# Dados exemplo
from sklearn.datasets import make_classification
from sklearn.ensemble import RandomForestClassifier

X, y = make_classification(n_samples=1000, n_features=10, n_classes=2, random_state=42)
X_df = pd.DataFrame(X, columns=[f'feature_{i}' for i in range(10)])
y_series = pd.Series(y)

model = RandomForestClassifier(random_state=42)
\end{lstlisting}
\end{pythonbox}
\begin{pythonbox}
\begin{lstlisting}[language=Python]  
# Validação cruzada abrangente
cv_results = validator.comprehensive_cross_validation(model, X_df, y_series)
print("Resultados da Validação Cruzada:")
for metric, results in cv_results.items():
    print(f"{metric}: {results['mean']:.3f} +/- {results['std']:.3f}")

# Análise de curva de aprendizado
learning_results = validator.learning_curve_analysis(model, X_df, y_series)
validator.plot_learning_curves(learning_results)

# Bootstrap confidence interval
bootstrap_results = validator.bootstrap_confidence_interval(model, X_df, y_series)
print(f"\nIntervalo de Confiança (Bootstrap):")
print(f"Score médio: {bootstrap_results['mean_score']:.3f}")
print(f"95% CI: {bootstrap_results['confidence_interval']}")
\end{lstlisting}
\end{pythonbox}

\section{Conclusão do Capítulo}

O machine learning transformou fundamentalmente a paisagem da pesquisa acadêmica, oferecendo ferramentas poderosas para análise de dados complexos, predição de resultados e descoberta de padrões ocultos. Python, através de seu ecossistema robusto de bibliotecas, democratizou o acesso a essas técnicas avançadas.

Os elementos fundamentais abordados neste capítulo incluem:

\textbf{Framework Metodológico:} Estrutura sistemática para identificação de problemas, seleção de algoritmos e preparação de dados adaptada às necessidades da pesquisa acadêmica.

\textbf{Técnicas de Classificação:} Métodos supervisionados para categorização de observações, essenciais para estudos em ciências sociais, medicina e psicologia.

\textbf{Modelos de Regressão:} Abordagens para predição quantitativa, fundamentais para análises econômicas, físicas e experimentais.

\textbf{Aprendizado Não Supervisionado:} Técnicas para descoberta de padrões em dados sem rótulos, cruciais para análise exploratória e segmentação.

\textbf{Validação Rigorosa:} Métodos estatisticamente sólidos para avaliação de modelos, garantindo confiabilidade e reprodutibilidade dos resultados.

\begin{examplebox}
\textbf{Principais Competências Desenvolvidas:}
\begin{itemize}
    \item Identificação automática de tipos de problemas ML
    \item Implementação de pipelines completos de machine learning
    \item Comparação sistemática de algoritmos
    \item Otimização de hiperparâmetros
    \item Técnicas de clustering e redução de dimensionalidade
    \item Validação estatisticamente rigorosa de modelos
    \item Interpretação e comunicação de resultados
\end{itemize}
\end{examplebox}

As técnicas apresentadas neste capítulo capacitam pesquisadores a extrair insights sophisticados de dados complexos, automatizar processos de análise e fazer predições confiáveis. O machine learning não substitui o rigor científico tradicional, mas o amplifica, permitindo análises em escalas e complexidades antes inimagináveis.

No próximo capítulo, exploraremos técnicas avançadas de visualização de dados, construindo sobre os fundamentos de análise para criar representações visuais que comuniquem efetivamente os insights descobertos através das técnicas de machine learning e análise estatística apresentadas até aqui.
% =============================================================================
% CAPÍTULO 9: ANÁLISE DE SÉRIES TEMPORAIS
% =============================================================================

\chapter{Análise de Séries Temporais}

\lettrine{A}{análise de séries temporais} é fundamental para compreender fenômenos que evoluem ao longo do tempo. Na pesquisa acadêmica, encontramos dados temporais em diversas áreas: desde variações climáticas até indicadores econômicos, de padrões de comportamento animal até dados epidemiológicos. Python oferece ferramentas poderosas para extrair insights significativos desses dados temporais.

\section{Dados Temporais em Pesquisa}

\subsection{Características dos Dados Temporais}

Os dados de séries temporais possuem características únicas que os distinguem de outros tipos de dados. A dependência temporal significa que observações próximas no tempo tendem a ser correlacionadas, violando a suposição de independência de muitos métodos estatísticos tradicionais.

\begin{pythonbox}
\begin{lstlisting}[language=Python]
import pandas as pd
import numpy as np
import matplotlib.pyplot as plt
import seaborn as sns
from datetime import datetime, timedelta
import warnings
warnings.filterwarnings('ignore')

# Configuração para gráficos
plt.style.use('seaborn-v0_8')
sns.set_palette("husl")
\end{lstlisting}
\end{pythonbox}

\begin{pythonbox}
\begin{lstlisting}[language=Python]
# Criando dados simulados de temperatura para pesquisa climática
np.random.seed(42)
datas = pd.date_range(start='2020-01-01', end='2023-12-31', freq='D')

# Simulando temperatura com tendência, sazonalidade e ruído
tendencia = np.linspace(15, 17, len(datas))  # Aquecimento gradual
sazonalidade = 10 * np.sin(2 * np.pi * np.arange(len(datas)) / 365.25)
ruido = np.random.normal(0, 2, len(datas))

temperatura = tendencia + sazonalidade + ruido
\end{lstlisting}
\end{pythonbox}

\begin{pythonbox}
\begin{lstlisting}[language=Python]
# Criando DataFrame
df_clima = pd.DataFrame({
    'data': datas,
    'temperatura': temperatura
})

print("Estrutura dos dados temporais:")
print(df_clima.head())
print(f"Período: {df_clima['data'].min()} a {df_clima['data'].max()}")
print(f"Número de observações: {len(df_clima)}")
\end{lstlisting}
\end{pythonbox}

\subsection{Importação e Preparação de Dados Temporais}

\begin{examplebox}
Vamos trabalhar com dados reais de pesquisa acadêmica. Considere um estudo sobre padrões de atividade acadêmica em publicações científicas:
\end{examplebox}

\begin{pythonbox}
\begin{lstlisting}[language=Python]
# Simulando dados de um estudo sobre produtividade acadêmica
np.random.seed(123)

# Dados de publicações científicas por mês
datas_pub = pd.date_range(start='2015-01-01', end='2023-12-31', freq='M')

# Simulando padrões realistas de publicação
base_pub = 50
crescimento = np.linspace(0, 30, len(datas_pub))
sazon_acad = 15 * np.sin(2 * np.pi * np.arange(len(datas_pub)) / 12 + np.pi/2)
eventos_esp = np.zeros(len(datas_pub))
\end{lstlisting}
\end{pythonbox}

\begin{pythonbox}
\begin{lstlisting}[language=Python]
# Simulando impacto da pandemia (2020-2021)
pandemia_inicio = ((datas_pub >= '2020-03-01') & 
                   (datas_pub <= '2020-12-31'))
pandemia_recup = ((datas_pub >= '2021-01-01') & 
                  (datas_pub <= '2021-12-31'))

eventos_esp[pandemia_inicio] = -20
eventos_esp[pandemia_recup] = 10

ruido = np.random.normal(0, 8, len(datas_pub))
\end{lstlisting}
\end{pythonbox}

\begin{pythonbox}
\begin{lstlisting}[language=Python]
publicacoes = (base_pub + crescimento + sazon_acad + 
               eventos_esp + ruido).astype(int)
publicacoes = np.maximum(publicacoes, 10)  # Mínimo de 10 publicações

# Criando DataFrame principal
df_academico = pd.DataFrame({
    'data': datas_pub,
    'publicacoes': publicacoes,
    'ano': datas_pub.year,
    'mes': datas_pub.month,
    'trimestre': datas_pub.quarter
})

# Configurando índice temporal
df_academico.set_index('data', inplace=True)
\end{lstlisting}
\end{pythonbox}

\begin{pythonbox}
\begin{lstlisting}[language=Python]
print("Dataset de produtividade acadêmica:")
print(df_academico.head(10))
print(f"\nEstatísticas descritivas:")
print(df_academico['publicacoes'].describe())
\end{lstlisting}
\end{pythonbox}

\subsection{Visualização Inicial dos Dados}

A visualização é o primeiro passo para compreender uma série temporal. Devemos observar tendências, padrões sazonais, outliers e possíveis quebras estruturais.

\begin{pythonbox}
\begin{lstlisting}[language=Python]
# Criando visualizações exploratórias
fig, axes = plt.subplots(2, 2, figsize=(15, 10))

# Gráfico principal da série
axes[0, 0].plot(df_academico.index, df_academico['publicacoes'], 
                linewidth=1.5, color='steelblue')
axes[0, 0].set_title('Série Temporal: Publicações Científicas por Mês', 
                     fontsize=12, fontweight='bold')
axes[0, 0].set_ylabel('Número de Publicações')
axes[0, 0].grid(True, alpha=0.3)
\end{lstlisting}
\end{pythonbox}

\begin{pythonbox}
\begin{lstlisting}[language=Python]
# Destacando período da pandemia
pandemia_mask = ((df_academico.index >= '2020-03-01') & 
                 (df_academico.index <= '2021-12-31'))
axes[0, 0].fill_between(df_academico.index[pandemia_mask], 
                        df_academico['publicacoes'][pandemia_mask],
                        alpha=0.3, color='red', label='Período Pandemia')
axes[0, 0].legend()

# Padrão sazonal por mês
publicacoes_por_mes = df_academico.groupby('mes')['publicacoes'].mean()
axes[0, 1].bar(range(1, 13), publicacoes_por_mes.values, 
               color='lightcoral', alpha=0.7)
axes[0, 1].set_title('Padrão Sazonal Médio por Mês', 
                     fontsize=12, fontweight='bold')
axes[0, 1].set_xlabel('Mês')
axes[0, 1].set_ylabel('Publicações Médias')
axes[0, 1].set_xticks(range(1, 13))
axes[0, 1].grid(True, alpha=0.3)
\end{lstlisting}
\end{pythonbox}

\begin{pythonbox}
\begin{lstlisting}[language=Python]
# Tendência anual
publicacoes_por_ano = df_academico.groupby('ano')['publicacoes'].sum()
axes[1, 0].plot(publicacoes_por_ano.index, publicacoes_por_ano.values, 
                marker='o', linewidth=2, markersize=6, color='darkgreen')
axes[1, 0].set_title('Tendência Anual: Total de Publicações', 
                     fontsize=12, fontweight='bold')
axes[1, 0].set_xlabel('Ano')
axes[1, 0].set_ylabel('Total de Publicações')
axes[1, 0].grid(True, alpha=0.3)

# Distribuição dos valores
axes[1, 1].hist(df_academico['publicacoes'], bins=20, 
                color='orange', alpha=0.7, edgecolor='black')
axes[1, 1].set_title('Distribuição dos Valores', 
                     fontsize=12, fontweight='bold')
axes[1, 1].set_xlabel('Número de Publicações')
axes[1, 1].set_ylabel('Frequência')
axes[1, 1].grid(True, alpha=0.3)

plt.tight_layout()
plt.show()
\end{lstlisting}
\end{pythonbox}

\section{Decomposição e Tendências}

\subsection{Decomposição de Séries Temporais}

A decomposição permite separar uma série temporal em seus componentes fundamentais: tendência, sazonalidade e ruído. Isso facilita a compreensão dos padrões subjacentes.

\begin{pythonbox}
\begin{lstlisting}[language=Python]
from statsmodels.tsa.seasonal import seasonal_decompose
from statsmodels.tsa.holtwinters import ExponentialSmoothing

# Decomposição da série temporal
decomposicao = seasonal_decompose(df_academico['publicacoes'], 
                                  model='additive', period=12)
\end{lstlisting}
\end{pythonbox}

\begin{pythonbox}
\begin{lstlisting}[language=Python]
# Visualizando a decomposição
fig, axes = plt.subplots(4, 1, figsize=(15, 12))

# Série original
axes[0].plot(df_academico.index, df_academico['publicacoes'], 
             color='black', linewidth=1.5)
axes[0].set_title('Série Original', fontsize=12, fontweight='bold')
axes[0].set_ylabel('Publicações')
axes[0].grid(True, alpha=0.3)

# Tendência
axes[1].plot(df_academico.index, decomposicao.trend, 
             color='red', linewidth=2)
axes[1].set_title('Componente de Tendência', fontsize=12, fontweight='bold')
axes[1].set_ylabel('Tendência')
axes[1].grid(True, alpha=0.3)
\end{lstlisting}
\end{pythonbox}

\begin{pythonbox}
\begin{lstlisting}[language=Python]
# Sazonalidade
axes[2].plot(df_academico.index, decomposicao.seasonal, 
             color='blue', linewidth=1.5)
axes[2].set_title('Componente Sazonal', fontsize=12, fontweight='bold')
axes[2].set_ylabel('Sazonalidade')
axes[2].grid(True, alpha=0.3)

# Resíduo
axes[3].plot(df_academico.index, decomposicao.resid, 
             color='green', linewidth=1)
axes[3].set_title('Resíduo (Ruído)', fontsize=12, fontweight='bold')
axes[3].set_ylabel('Resíduo')
axes[3].set_xlabel('Data')
axes[3].grid(True, alpha=0.3)

plt.tight_layout()
plt.show()
\end{lstlisting}
\end{pythonbox}

\begin{pythonbox}
\begin{lstlisting}[language=Python]
# Análise dos componentes
print("Análise da Decomposição:")
print(f"Variabilidade da tendência: {decomposicao.trend.std():.2f}")
print(f"Variabilidade sazonal: {decomposicao.seasonal.std():.2f}")
print(f"Variabilidade do ruído: {decomposicao.resid.std():.2f}")
\end{lstlisting}
\end{pythonbox}

\subsection{Detecção de Mudanças Estruturais}

\begin{researchbox}
Em pesquisa acadêmica, é crucial identificar quando ocorrem mudanças significativas nos padrões dos dados. Por exemplo, o impacto de políticas públicas ou eventos externos como a pandemia.
\end{researchbox}

\begin{pythonbox}
\begin{lstlisting}[language=Python]
from scipy import stats
import ruptures as rpt

# Detecção de pontos de mudança usando Ruptures
# Preparando os dados
serie_valores = df_academico['publicacoes'].values

# Detectando mudanças na média usando Pelt
modelo = rpt.Pelt(model="rbf").fit(serie_valores)
pontos_mudanca = modelo.predict(pen=10)

# Removendo o último ponto (que é sempre o final da série)
pontos_mudanca = pontos_mudanca[:-1]
\end{lstlisting}
\end{pythonbox}

\begin{pythonbox}
\begin{lstlisting}[language=Python]
# Convertendo índices para datas
datas_mudanca = [df_academico.index[i] for i in pontos_mudanca]

print("Pontos de mudança detectados:")
for i, data in enumerate(datas_mudanca):
    print(f"Mudança {i+1}: {data.strftime('%Y-%m')}")
\end{lstlisting}
\end{pythonbox}

\begin{pythonbox}
\begin{lstlisting}[language=Python]
# Visualizando os pontos de mudança
plt.figure(figsize=(15, 6))
plt.plot(df_academico.index, df_academico['publicacoes'], 
         linewidth=1.5, color='steelblue', label='Série Original')

# Marcando os pontos de mudança
for data in datas_mudanca:
    plt.axvline(x=data, color='red', linestyle='--', alpha=0.8)

plt.title('Detecção de Mudanças Estruturais na Série', 
          fontsize=14, fontweight='bold')
plt.xlabel('Data')
plt.ylabel('Número de Publicações')
plt.legend()
plt.grid(True, alpha=0.3)
plt.xticks(rotation=45)
plt.tight_layout()
plt.show()
\end{lstlisting}
\end{pythonbox}

\begin{pythonbox}
\begin{lstlisting}[language=Python]
# Análise estatística dos períodos
print("\nAnálise estatística por período:")
pontos_completos = [0] + pontos_mudanca + [len(df_academico)]

for i in range(len(pontos_completos)-1):
    inicio = pontos_completos[i]
    fim = pontos_completos[i+1]
    periodo_dados = df_academico.iloc[inicio:fim]['publicacoes']
    
    data_inicio = df_academico.index[inicio].strftime('%Y-%m')
    data_fim = df_academico.index[fim-1].strftime('%Y-%m')
    
    print(f"Período {i+1} ({data_inicio} a {data_fim}):")
    print(f"  Média: {periodo_dados.mean():.2f}")
    print(f"  Desvio padrão: {periodo_dados.std():.2f}")
    print(f"  Mediana: {periodo_dados.median():.2f}")
    print()
\end{lstlisting}
\end{pythonbox}

\section{Modelos ARIMA e Previsão}

\subsection{Fundamentos dos Modelos ARIMA}

Os modelos ARIMA (AutoRegressive Integrated Moving Average) são fundamentais para análise e previsão de séries temporais. Eles combinam três componentes: autorregressão (AR), diferenciação (I) e média móvel (MA).

\begin{pythonbox}
\begin{lstlisting}[language=Python]
from statsmodels.tsa.arima.model import ARIMA
from statsmodels.tsa.stattools import adfuller, acf, pacf
from statsmodels.graphics.tsaplots import plot_acf, plot_pacf
from sklearn.metrics import mean_absolute_error, mean_squared_error
\end{lstlisting}
\end{pythonbox}

\begin{pythonbox}
\begin{lstlisting}[language=Python]
# Teste de estacionariedade (Teste de Dickey-Fuller Aumentado)
def teste_estacionariedade(serie, titulo='Série'):
    """Realiza teste de estacionariedade e exibe resultados"""
    resultado = adfuller(serie.dropna())
    
    print(f'Teste de Estacionariedade - {titulo}:')
    print(f'Estatística ADF: {resultado[0]:.6f}')
    print(f'p-valor: {resultado[1]:.6f}')
    print(f'Valores críticos:')
    for chave, valor in resultado[4].items():
        print(f'\t{chave}: {valor:.3f}')
    
    if resultado[1] <= 0.05:
        print("Resultado: A série é estacionária (rejeita H0)")
    else:
        print("Resultado: A série não é estacionária (não rejeita H0)")
    print('-' * 50)
\end{lstlisting}
\end{pythonbox}

\begin{pythonbox}
\begin{lstlisting}[language=Python]
# Testando estacionariedade da série original
teste_estacionariedade(df_academico['publicacoes'], 'Original')

# Aplicando diferenciação se necessário
df_academico['pub_diff1'] = df_academico['publicacoes'].diff()
teste_estacionariedade(df_academico['pub_diff1'], 'Primeira Diferença')
\end{lstlisting}
\end{pythonbox}

\begin{pythonbox}
\begin{lstlisting}[language=Python]
# Visualizando ACF e PACF para identificar parâmetros ARIMA
fig, axes = plt.subplots(2, 2, figsize=(15, 10))

# Série original
axes[0, 0].plot(df_academico.index, df_academico['publicacoes'])
axes[0, 0].set_title('Série Original')
axes[0, 0].grid(True, alpha=0.3)

# Série diferenciada
axes[0, 1].plot(df_academico.index, df_academico['pub_diff1'])
axes[0, 1].set_title('Primeira Diferença')
axes[0, 1].grid(True, alpha=0.3)

# ACF da série diferenciada
plot_acf(df_academico['pub_diff1'].dropna(), ax=axes[1, 0], lags=24)
axes[1, 0].set_title('Função de Autocorrelação (ACF)')

# PACF da série diferenciada
plot_pacf(df_academico['pub_diff1'].dropna(), ax=axes[1, 1], lags=24)
axes[1, 1].set_title('Função de Autocorrelação Parcial (PACF)')

plt.tight_layout()
plt.show()
\end{lstlisting}
\end{pythonbox}

\subsection{Seleção e Ajuste do Modelo ARIMA}

\begin{pythonbox}
\begin{lstlisting}[language=Python]
from itertools import product

# Função para seleção automática de parâmetros ARIMA
def selecionar_arima(serie, max_p=3, max_d=2, max_q=3):
    """
    Seleciona os melhores parâmetros ARIMA usando AIC
    """
    melhor_aic = np.inf
    melhor_params = None
    melhor_modelo = None
    
    # Testando diferentes combinações de parâmetros
    for p in range(max_p + 1):
        for d in range(max_d + 1):
            for q in range(max_q + 1):
                try:
                    modelo = ARIMA(serie, order=(p, d, q))
                    resultado = modelo.fit()
                    
                    if resultado.aic < melhor_aic:
                        melhor_aic = resultado.aic
                        melhor_params = (p, d, q)
                        melhor_modelo = resultado
                        
                except:
                    continue
    
    return melhor_params, melhor_modelo, melhor_aic
\end{lstlisting}
\end{pythonbox}

\begin{pythonbox}
\begin{lstlisting}[language=Python]
# Dividindo os dados em treino e teste (80%-20%)
tamanho_treino = int(len(df_academico) * 0.8)
treino = df_academico['publicacoes'][:tamanho_treino]
teste = df_academico['publicacoes'][tamanho_treino:]

print(f"Período de treino: {treino.index[0]} a {treino.index[-1]}")
print(f"Período de teste: {teste.index[0]} a {teste.index[-1]}")

# Selecionando o melhor modelo
melhor_params, modelo_ajustado, melhor_aic = selecionar_arima(treino)

print(f"\nMelhor modelo ARIMA{melhor_params}")
print(f"AIC: {melhor_aic:.2f}")
\end{lstlisting}
\end{pythonbox}

\begin{pythonbox}
\begin{lstlisting}[language=Python]
print("\nResumo do modelo:")
print(modelo_ajustado.summary())
\end{lstlisting}
\end{pythonbox}

\subsection{Previsão e Validação}

\begin{pythonbox}
\begin{lstlisting}[language=Python]
# Fazendo previsões
previsoes = modelo_ajustado.forecast(steps=len(teste))
intervalo_confianca = modelo_ajustado.get_forecast(steps=len(teste))
ic = intervalo_confianca.conf_int()

# Calculando métricas de erro
mae = mean_absolute_error(teste, previsoes)
rmse = np.sqrt(mean_squared_error(teste, previsoes))
mape = np.mean(np.abs((teste - previsoes) / teste)) * 100

print(f"Métricas de Avaliação:")
print(f"MAE (Erro Absoluto Médio): {mae:.2f}")
print(f"RMSE (Raiz do Erro Quadrático Médio): {rmse:.2f}")
print(f"MAPE (Erro Percentual Absoluto Médio): {mape:.2f}%")
\end{lstlisting}
\end{pythonbox}

\begin{pythonbox}
\begin{lstlisting}[language=Python]
# Visualizando previsões
plt.figure(figsize=(15, 8))

# Dados de treino
plt.plot(treino.index, treino.values, label='Dados de Treino', 
         color='blue', linewidth=1.5)

# Dados de teste (valores reais)
plt.plot(teste.index, teste.values, label='Valores Reais', 
         color='green', linewidth=2)

# Previsões
plt.plot(teste.index, previsoes, label='Previsões ARIMA', 
         color='red', linewidth=2, linestyle='--')

# Intervalo de confiança
plt.fill_between(teste.index, ic.iloc[:, 0], ic.iloc[:, 1], 
                 color='red', alpha=0.2, label='IC 95%')
\end{lstlisting}
\end{pythonbox}

\begin{pythonbox}
\begin{lstlisting}[language=Python]
plt.title(f'Previsão ARIMA{melhor_params} - Publicações Científicas', 
          fontsize=14, fontweight='bold')
plt.xlabel('Data')
plt.ylabel('Número de Publicações')
plt.legend()
plt.grid(True, alpha=0.3)
plt.xticks(rotation=45)
plt.tight_layout()
plt.show()
\end{lstlisting}
\end{pythonbox}

\begin{pythonbox}
\begin{lstlisting}[language=Python]
# Análise dos resíduos
residuos = modelo_ajustado.resid

fig, axes = plt.subplots(2, 2, figsize=(15, 10))

# Resíduos ao longo do tempo
axes[0, 0].plot(residuos.index, residuos.values)
axes[0, 0].set_title('Resíduos ao Longo do Tempo')
axes[0, 0].set_ylabel('Resíduos')
axes[0, 0].grid(True, alpha=0.3)

# Histograma dos resíduos
axes[0, 1].hist(residuos.dropna(), bins=20, alpha=0.7, color='orange')
axes[0, 1].set_title('Distribuição dos Resíduos')
axes[0, 1].set_xlabel('Resíduos')
axes[0, 1].set_ylabel('Frequência')
\end{lstlisting}
\end{pythonbox}

\begin{pythonbox}
\begin{lstlisting}[language=Python]
# Q-Q plot
from scipy import stats
stats.probplot(residuos.dropna(), dist="norm", plot=axes[1, 0])
axes[1, 0].set_title('Q-Q Plot dos Resíduos')

# ACF dos resíduos
plot_acf(residuos.dropna(), ax=axes[1, 1], lags=20)
axes[1, 1].set_title('ACF dos Resíduos')

plt.tight_layout()
plt.show()
\end{lstlisting}
\end{pythonbox}

\section{Análise de Sazonalidade}

\subsection{Identificação de Padrões Sazonais}

A sazonalidade é um componente crucial em muitas séries temporais acadêmicas. Vamos explorar métodos para identificar e modelar esses padrões.

\begin{pythonbox}
\begin{lstlisting}[language=Python]
# Análise sazonal detalhada
from statsmodels.tsa.seasonal import STL

# Decomposição STL (Seasonal and Trend decomposition using Loess)
stl = STL(df_academico['publicacoes'], seasonal=13)  # seasonal=13 para dados mensais
resultado_stl = stl.fit()

# Visualizando decomposição STL
fig = resultado_stl.plot()
fig.suptitle('Decomposição STL - Publicações Científicas', 
             fontsize=14, fontweight='bold')
plt.tight_layout()
plt.show()
\end{lstlisting}
\end{pythonbox}

\begin{pythonbox}
\begin{lstlisting}[language=Python]
# Análise de sazonalidade por diferentes períodos
fig, axes = plt.subplots(2, 2, figsize=(15, 10))

# Padrão mensal
monthly_pattern = df_academico.groupby(df_academico.index.month)['publicacoes'].agg(['mean', 'std'])
axes[0, 0].bar(monthly_pattern.index, monthly_pattern['mean'], 
               yerr=monthly_pattern['std'], capsize=5, alpha=0.7, color='skyblue')
axes[0, 0].set_title('Padrão Sazonal Mensal')
axes[0, 0].set_xlabel('Mês')
axes[0, 0].set_ylabel('Publicações Médias')
axes[0, 0].set_xticks(range(1, 13))
axes[0, 0].grid(True, alpha=0.3)
\end{lstlisting}
\end{pythonbox}

\begin{pythonbox}
\begin{lstlisting}[language=Python]
# Padrão trimestral
quarterly_pattern = df_academico.groupby('trimestre')['publicacoes'].agg(['mean', 'std'])
axes[0, 1].bar(quarterly_pattern.index, quarterly_pattern['mean'], 
               yerr=quarterly_pattern['std'], capsize=5, alpha=0.7, color='lightgreen')
axes[0, 1].set_title('Padrão Sazonal Trimestral')
axes[0, 1].set_xlabel('Trimestre')
axes[0, 1].set_ylabel('Publicações Médias')
axes[0, 1].grid(True, alpha=0.3)

# Heatmap sazonal
pivot_sazonal = df_academico.pivot_table(values='publicacoes', 
                                         index=df_academico.index.year, 
                                         columns=df_academico.index.month, 
                                         aggfunc='mean')

sns.heatmap(pivot_sazonal, annot=True, fmt='.0f', cmap='YlOrRd', 
            ax=axes[1, 0], cbar_kws={'label': 'Publicações'})
axes[1, 0].set_title('Heatmap Sazonal (Ano x Mês)')
axes[1, 0].set_xlabel('Mês')
axes[1, 0].set_ylabel('Ano')
\end{lstlisting}
\end{pythonbox}

\begin{pythonbox}
\begin{lstlisting}[language=Python]
# Boxplot por mês
df_academico.boxplot(column='publicacoes', by=df_academico.index.month, 
                     ax=axes[1, 1])
axes[1, 1].set_title('Distribuição por Mês')
axes[1, 1].set_xlabel('Mês')
axes[1, 1].set_ylabel('Publicações')

plt.tight_layout()
plt.show()
\end{lstlisting}
\end{pythonbox}

\begin{pythonbox}
\begin{lstlisting}[language=Python]
# Teste estatístico de sazonalidade
from scipy.stats import kruskal

# Teste de Kruskal-Wallis para sazonalidade mensal
grupos_mensais = [df_academico[df_academico.index.month == m]['publicacoes'].values 
                  for m in range(1, 13)]
stat_kruskal, p_valor = kruskal(*grupos_mensais)

print(f"Teste de Kruskal-Wallis para Sazonalidade Mensal:")
print(f"Estatística: {stat_kruskal:.4f}")
print(f"p-valor: {p_valor:.6f}")

if p_valor < 0.05:
    print("Resultado: Há evidência significativa de sazonalidade mensal")
else:
    print("Resultado: Não há evidência significativa de sazonalidade mensal")
\end{lstlisting}
\end{pythonbox}

\subsection{Modelagem Sazonal com SARIMA}

\begin{researchbox}
Para séries com sazonalidade clara, os modelos SARIMA (Seasonal ARIMA) são mais apropriados, pois incluem componentes sazonais explícitos.
\end{researchbox}

\begin{pythonbox}
\begin{lstlisting}[language=Python]
from statsmodels.tsa.statespace.sarimax import SARIMAX

# Função para seleção de parâmetros SARIMA
def selecionar_sarima(serie, max_p=2, max_d=1, max_q=2, 
                     max_P=1, max_D=1, max_Q=1, s=12):
    """
    Seleciona os melhores parâmetros SARIMA usando AIC
    """
    melhor_aic = np.inf
    melhor_params = None
    melhor_modelo = None
    
    for p in range(max_p + 1):
        for d in range(max_d + 1):
            for q in range(max_q + 1):
                for P in range(max_P + 1):
                    for D in range(max_D + 1):
                        for Q in range(max_Q + 1):
                            try:
                                modelo = SARIMAX(serie, 
                                                order=(p, d, q),
                                                seasonal_order=(P, D, Q, s))
                                resultado = modelo.fit(disp=False)
                                
                                if resultado.aic < melhor_aic:
                                    melhor_aic = resultado.aic
                                    melhor_params = ((p, d, q), (P, D, Q, s))
                                    melhor_modelo = resultado
                                    
                            except:
                                continue
    
    return melhor_params, melhor_modelo, melhor_aic
\end{lstlisting}
\end{pythonbox}

\begin{pythonbox}
\begin{lstlisting}[language=Python]
# Selecionando o melhor modelo SARIMA
print("Selecionando modelo SARIMA... (pode demorar alguns minutos)")
melhor_params_sarima, modelo_sarima, melhor_aic_sarima = selecionar_sarima(treino)

print(f"\nMelhor modelo SARIMA{melhor_params_sarima[0]}x{melhor_params_sarima[1]}")
print(f"AIC: {melhor_aic_sarima:.2f}")

# Fazendo previsões com SARIMA
previsoes_sarima = modelo_sarima.forecast(steps=len(teste))
ic_sarima = modelo_sarima.get_forecast(steps=len(teste)).conf_int()
\end{lstlisting}
\end{pythonbox}

\begin{pythonbox}
\begin{lstlisting}[language=Python]
# Métricas de avaliação SARIMA
mae_sarima = mean_absolute_error(teste, previsoes_sarima)
rmse_sarima = np.sqrt(mean_squared_error(teste, previsoes_sarima))
mape_sarima = np.mean(np.abs((teste - previsoes_sarima) / teste)) * 100

print(f"\nMétricas SARIMA:")
print(f"MAE: {mae_sarima:.2f}")
print(f"RMSE: {rmse_sarima:.2f}")
print(f"MAPE: {mape_sarima:.2f}%")

print(f"\nComparação com ARIMA:")
print(f"ARIMA - MAE: {mae:.2f}, RMSE: {rmse:.2f}, MAPE: {mape:.2f}%")
print(f"SARIMA - MAE: {mae_sarima:.2f}, RMSE: {rmse_sarima:.2f}, MAPE: {mape_sarima:.2f}%")
\end{lstlisting}
\end{pythonbox}

\begin{pythonbox}
\begin{lstlisting}[language=Python]
# Visualizando comparação entre modelos
plt.figure(figsize=(15, 8))

# Dados históricos
plt.plot(treino.index, treino.values, label='Dados de Treino', 
         color='blue', linewidth=1.5)
plt.plot(teste.index, teste.values, label='Valores Reais', 
         color='green', linewidth=2)

# Previsões ARIMA
plt.plot(teste.index, previsoes, label=f'ARIMA{melhor_params}', 
         color='red', linewidth=2, linestyle='--', alpha=0.8)

# Previsões SARIMA
plt.plot(teste.index, previsoes_sarima, 
         label=f'SARIMA{melhor_params_sarima[0]}x{melhor_params_sarima[1]}', 
         color='purple', linewidth=2, linestyle=':', alpha=0.8)
\end{lstlisting}
\end{pythonbox}

\begin{pythonbox}
\begin{lstlisting}[language=Python]
# Intervalo de confiança SARIMA
plt.fill_between(teste.index, ic_sarima.iloc[:, 0], ic_sarima.iloc[:, 1], 
                 color='purple', alpha=0.2, label='IC 95% SARIMA')

plt.title('Comparação: ARIMA vs SARIMA - Publicações Científicas', 
          fontsize=14, fontweight='bold')
plt.xlabel('Data')
plt.ylabel('Número de Publicações')
plt.legend()
plt.grid(True, alpha=0.3)
plt.xticks(rotation=45)
plt.tight_layout()
plt.show()
\end{lstlisting}
\end{pythonbox}

\subsection{Análise de Ciclos Acadêmicos}

\begin{examplebox}
Na pesquisa acadêmica, é importante compreender os ciclos específicos da área. Por exemplo, períodos de submissão de artigos, calendário de conferências, e ciclos de financiamento.
\end{examplebox}

\begin{pythonbox}
\begin{lstlisting}[language=Python]
# Análise de ciclos específicos da área acadêmica
from scipy.fft import fft, fftfreq
from scipy.signal import find_peaks

# Análise espectral para identificar periodicidades
def analise_espectral(serie, freq_amostragem=12):
    """
    Realiza análise espectral para identificar periodicidades dominantes
    """
    # Removendo tendência (detrending)
    serie_detrend = serie - np.polyval(np.polyfit(range(len(serie)), 
                                                  serie, 1), range(len(serie)))
    
    # Transformada de Fourier
    fft_valores = fft(serie_detrend)
    frequencias = fftfreq(len(serie_detrend), 1/freq_amostragem)
    
    # Magnitude do espectro (apenas frequências positivas)
    n = len(frequencias) // 2
    magnitude = np.abs(fft_valores[:n])
    freq_positivas = frequencias[:n]
    
    # Convertendo frequências para períodos
    periodos = 1 / freq_positivas[1:]  # Excluindo frequência zero
    magnitude = magnitude[1:]
    
    return periodos, magnitude
\end{lstlisting}
\end{pythonbox}

\begin{pythonbox}
\begin{lstlisting}[language=Python]
# Aplicando análise espectral
periodos, magnitude = analise_espectral(df_academico['publicacoes'].values)

# Identificando picos no espectro
picos, propriedades = find_peaks(magnitude, height=np.max(magnitude)*0.1)
\end{lstlisting}
\end{pythonbox}

\begin{pythonbox}
\begin{lstlisting}[language=Python]
# Visualizando análise espectral
fig, axes = plt.subplots(2, 2, figsize=(15, 10))

# Espectro de potência
axes[0, 0].plot(periodos, magnitude, linewidth=1.5, color='navy')
axes[0, 0].scatter(periodos[picos], magnitude[picos], 
                   color='red', s=50, zorder=5)
axes[0, 0].set_title('Espectro de Potência - Identificação de Ciclos')
axes[0, 0].set_xlabel('Período (meses)')
axes[0, 0].set_ylabel('Magnitude')
axes[0, 0].set_xlim(0, 24)
axes[0, 0].grid(True, alpha=0.3)

# Anotando os principais períodos detectados
for i, pico in enumerate(picos[:5]):  # Mostrando apenas os 5 principais
    periodo_detectado = periodos[pico]
    axes[0, 0].annotate(f'{periodo_detectado:.1f}m', 
                        xy=(periodo_detectado, magnitude[pico]),
                        xytext=(periodo_detectado, magnitude[pico] + magnitude[pico]*0.1),
                        ha='center', fontsize=9, color='red')
\end{lstlisting}
\end{pythonbox}

\begin{pythonbox}
\begin{lstlisting}[language=Python]
# Análise por semestre acadêmico (assumindo início em março e setembro)
df_academico['semestre'] = np.where(df_academico.index.month.isin([3,4,5,6,7,8]), 1, 2)
semestre_stats = df_academico.groupby(['ano', 'semestre'])['publicacoes'].agg(['mean', 'sum', 'std'])

# Visualizando padrões semestrais
sem1_values = semestre_stats[semestre_stats.index.get_level_values(1) == 1]['mean'].values
sem2_values = semestre_stats[semestre_stats.index.get_level_values(1) == 2]['mean'].values

anos = semestre_stats.index.get_level_values(0).unique()
x = np.arange(len(anos))

axes[0, 1].bar(x - 0.2, sem1_values, 0.4, label='1º Semestre', alpha=0.7, color='lightblue')
axes[0, 1].bar(x + 0.2, sem2_values, 0.4, label='2º Semestre', alpha=0.7, color='orange')
axes[0, 1].set_title('Publicações Médias por Semestre Acadêmico')
axes[0, 1].set_xlabel('Ano')
axes[0, 1].set_ylabel('Publicações Médias')
axes[0, 1].set_xticks(x)
axes[0, 1].set_xticklabels(anos, rotation=45)
axes[0, 1].legend()
axes[0, 1].grid(True, alpha=0.3)
\end{lstlisting}
\end{pythonbox}

\begin{pythonbox}
\begin{lstlisting}[language=Python]
# Análise de concentração de publicações por período do ano
trimestre_concentracao = df_academico.groupby(['ano', 'trimestre'])['publicacoes'].sum().unstack()
trimestre_pct = trimestre_concentracao.div(trimestre_concentracao.sum(axis=1), axis=0) * 100

# Heatmap de concentração por trimestre
im = axes[1, 0].imshow(trimestre_pct.values, cmap='YlOrRd', aspect='auto')
axes[1, 0].set_title('Concentração de Publicações por Trimestre (%)')
axes[1, 0].set_xlabel('Trimestre')
axes[1, 0].set_ylabel('Ano')
axes[1, 0].set_xticks(range(4))
axes[1, 0].set_xticklabels(['Q1', 'Q2', 'Q3', 'Q4'])
axes[1, 0].set_yticks(range(len(trimestre_pct.index)))
axes[1, 0].set_yticklabels(trimestre_pct.index)

# Adicionando colorbar
cbar = plt.colorbar(im, ax=axes[1, 0])
cbar.set_label('Concentração (%)')
\end{lstlisting}
\end{pythonbox}

\begin{pythonbox}
\begin{lstlisting}[language=Python]
# Padrão de crescimento anual
crescimento_anual = df_academico.groupby('ano')['publicacoes'].sum().pct_change() * 100
axes[1, 1].bar(crescimento_anual.index[1:], crescimento_anual.values[1:], 
               color=['red' if x < 0 else 'green' for x in crescimento_anual.values[1:]], 
               alpha=0.7)
axes[1, 1].axhline(y=0, color='black', linestyle='-', alpha=0.8)
axes[1, 1].set_title('Taxa de Crescimento Anual (%)')
axes[1, 1].set_xlabel('Ano')
axes[1, 1].set_ylabel('Crescimento (%)')
axes[1, 1].grid(True, alpha=0.3)

plt.tight_layout()
plt.show()
\end{lstlisting}
\end{pythonbox}

\begin{pythonbox}
\begin{lstlisting}[language=Python]
# Relatório de análise de sazonalidade
print("=" * 60)
print("RELATÓRIO DE ANÁLISE DE SAZONALIDADE")
print("=" * 60)

print(f"\n1. PERÍODOS DOMINANTES DETECTADOS:")
for i, pico in enumerate(picos[:3]):
    periodo = periodos[pico]
    print(f"   Período {i+1}: {periodo:.1f} meses")

print(f"\n2. ANÁLISE SEMESTRAL:")
print(f"   Média 1º Semestre: {np.mean(sem1_values):.1f} publicações")
print(f"   Média 2º Semestre: {np.mean(sem2_values):.1f} publicações")
diferenca_rel = ((np.mean(sem2_values) - np.mean(sem1_values))/np.mean(sem1_values)*100)
print(f"   Diferença relativa: {diferenca_rel:.1f}%")

print(f"\n3. CONCENTRAÇÃO TRIMESTRAL MÉDIA:")
concentracao_media = trimestre_pct.mean()
for i, trimestre in enumerate(['Q1', 'Q2', 'Q3', 'Q4']):
    print(f"   {trimestre}: {concentracao_media.iloc[i]:.1f}%")

print(f"\n4. CRESCIMENTO MÉDIO ANUAL: {np.mean(crescimento_anual.dropna()):.1f}%")
\end{lstlisting}
\end{pythonbox}

\begin{pythonbox}
\begin{lstlisting}[language=Python]
# Teste de sazonalidade mais rigoroso
from statsmodels.stats.diagnostic import acorr_ljungbox

# Teste de Ljung-Box nos resíduos dessazonalizados
serie_dessazonalizada = df_academico['publicacoes'] - resultado_stl.seasonal
residuos_dessaz = serie_dessazonalizada - serie_dessazonalizada.rolling(12).mean()

ljung_box = acorr_ljungbox(residuos_dessaz.dropna(), lags=12, return_df=True)
print(f"\n5. TESTE DE LJUNG-BOX (Autocorrelação Residual):")
print(f"   p-valor mínimo: {ljung_box['lb_pvalue'].min():.6f}")

if ljung_box['lb_pvalue'].min() > 0.05:
    print("   Resultado: Não há autocorrelação significativa nos resíduos")
else:
    print("   Resultado: Ainda há autocorrelação nos resíduos")
\end{lstlisting}
\end{pythonbox}

\subsection{Previsão com Incerteza Sazonal}

\begin{warningbox}
Ao fazer previsões de séries sazonais, é crucial considerar que a incerteza aumenta em períodos tradicionalmente mais voláteis. Sempre forneça intervalos de confiança junto com as previsões pontuais.
\end{warningbox}

\begin{pythonbox}
\begin{lstlisting}[language=Python]
# Previsão estendida com análise de incerteza sazonal
horizonte_previsao = 12  # 12 meses à frente

# Previsão com modelo SARIMA
previsao_futura = modelo_sarima.forecast(steps=horizonte_previsao)
ic_futuro = modelo_sarima.get_forecast(steps=horizonte_previsao).conf_int()

# Criando datas futuras
ultima_data = df_academico.index[-1]
datas_futuras = pd.date_range(start=ultima_data + pd.DateOffset(months=1), 
                              periods=horizonte_previsao, freq='M')
\end{lstlisting}
\end{pythonbox}

\begin{pythonbox}
\begin{lstlisting}[language=Python]
# Análise de incerteza por período sazonal
incerteza_sazonal = []
for i, data_futura in enumerate(datas_futuras):
    mes = data_futura.month
    
    # Histórico de variabilidade para o mesmo mês
    dados_mes = df_academico[df_academico.index.month == mes]['publicacoes']
    variabilidade_historica = dados_mes.std()
    
    # Incerteza da previsão
    ic_amplitude = ic_futuro.iloc[i, 1] - ic_futuro.iloc[i, 0]
    
    incerteza_sazonal.append({
        'mes': mes,
        'data': data_futura,
        'previsao': previsao_futura.iloc[i],
        'ic_inferior': ic_futuro.iloc[i, 0],
        'ic_superior': ic_futuro.iloc[i, 1],
        'variabilidade_historica': variabilidade_historica,
        'amplitude_ic': ic_amplitude
    })

df_incerteza = pd.DataFrame(incerteza_sazonal)
\end{lstlisting}
\end{pythonbox}

\begin{pythonbox}
\begin{lstlisting}[language=Python]
# Visualizando previsões com análise de incerteza
fig, axes = plt.subplots(2, 1, figsize=(15, 12))

# Gráfico principal de previsão
ultimos_24_meses = df_academico.tail(24)
axes[0].plot(ultimos_24_meses.index, ultimos_24_meses['publicacoes'], 
             'o-', label='Dados Históricos', color='blue', linewidth=2)

axes[0].plot(datas_futuras, previsao_futura, 'o-', 
             label='Previsão SARIMA', color='red', linewidth=2)

axes[0].fill_between(datas_futuras, df_incerteza['ic_inferior'], 
                     df_incerteza['ic_superior'], 
                     alpha=0.3, color='red', label='IC 95%')
\end{lstlisting}
\end{pythonbox}

\begin{pythonbox}
\begin{lstlisting}[language=Python]
# Destacando períodos de maior incerteza
alta_incerteza = df_incerteza['amplitude_ic'] > df_incerteza['amplitude_ic'].median()
axes[0].scatter(df_incerteza[alta_incerteza]['data'], 
                df_incerteza[alta_incerteza]['previsao'],
                s=100, color='orange', marker='*', 
                label='Alta Incerteza', zorder=5)

axes[0].set_title('Previsão com Análise de Incerteza Sazonal', 
                  fontsize=14, fontweight='bold')
axes[0].set_ylabel('Número de Publicações')
axes[0].legend()
axes[0].grid(True, alpha=0.3)
axes[0].tick_params(axis='x', rotation=45)
\end{lstlisting}
\end{pythonbox}

\begin{pythonbox}
\begin{lstlisting}[language=Python]
# Gráfico de incerteza por mês
meses_nomes = ['Jan', 'Fev', 'Mar', 'Abr', 'Mai', 'Jun',
               'Jul', 'Ago', 'Set', 'Out', 'Nov', 'Dez']

incerteza_por_mes = df_incerteza.groupby('mes')['amplitude_ic'].mean()
variabilidade_por_mes = df_incerteza.groupby('mes')['variabilidade_historica'].mean()

x = range(1, 13)
axes[1].bar([i-0.2 for i in x], [incerteza_por_mes.get(i, 0) for i in x], 
            0.4, label='Incerteza da Previsão', alpha=0.7, color='red')
axes[1].bar([i+0.2 for i in x], [variabilidade_por_mes.get(i, 0) for i in x], 
            0.4, label='Variabilidade Histórica', alpha=0.7, color='blue')

axes[1].set_title('Incerteza da Previsão vs Variabilidade Histórica por Mês')
axes[1].set_xlabel('Mês')
axes[1].set_ylabel('Amplitude da Incerteza')
axes[1].set_xticks(x)
axes[1].set_xticklabels(meses_nomes)
axes[1].legend()
axes[1].grid(True, alpha=0.3)

plt.tight_layout()
plt.show()
\end{lstlisting}
\end{pythonbox}

\begin{pythonbox}
\begin{lstlisting}[language=Python]
# Relatório de previsão
print("=" * 70)
print("RELATÓRIO DE PREVISÃO - PRÓXIMOS 12 MESES")
print("=" * 70)

print(f"\nPREVISÃO MENSAL:")
for _, row in df_incerteza.iterrows():
    mes_nome = meses_nomes[row['mes']-1]
    print(f"{row['data'].strftime('%Y-%m')} ({mes_nome}): "
          f"{row['previsao']:.0f} publicações "
          f"[{row['ic_inferior']:.0f} - {row['ic_superior']:.0f}]")

print(f"\nRESUMO ESTATÍSTICO:")
print(f"Total previsto (12 meses): {df_incerteza['previsao'].sum():.0f} publicações")
print(f"Média mensal: {df_incerteza['previsao'].mean():.1f} publicações")

max_idx = df_incerteza['previsao'].idxmax()
min_idx = df_incerteza['previsao'].idxmin()
print(f"Mês com maior previsão: {meses_nomes[df_incerteza.loc[max_idx, 'mes']-1]} "
      f"({df_incerteza['previsao'].max():.0f} publicações)")
print(f"Mês com menor previsão: {meses_nomes[df_incerteza.loc[min_idx, 'mes']-1]} "
      f"({df_incerteza['previsao'].min():.0f} publicações)")
\end{lstlisting}
\end{pythonbox}

\begin{pythonbox}
\begin{lstlisting}[language=Python]
print(f"\nANÁLISE DE INCERTEZA:")
max_inc_idx = df_incerteza['amplitude_ic'].idxmax()
min_inc_idx = df_incerteza['amplitude_ic'].idxmin()
print(f"Período de maior incerteza: {meses_nomes[df_incerteza.loc[max_inc_idx, 'mes']-1]}")
print(f"Período de menor incerteza: {meses_nomes[df_incerteza.loc[min_inc_idx, 'mes']-1]}")
print(f"Incerteza média: ±{df_incerteza['amplitude_ic'].mean()/2:.1f} publicações")

# Comparação com tendência histórica
crescimento_historico = df_academico.tail(12)['publicacoes'].sum()
crescimento_previsto = df_incerteza['previsao'].sum()
variacao_percentual = ((crescimento_previsto - crescimento_historico) / crescimento_historico) * 100

print(f"\nCOMPARAÇÃO COM PERÍODO ANTERIOR:")
print(f"Últimos 12 meses: {crescimento_historico:.0f} publicações")
print(f"Próximos 12 meses (previsto): {crescimento_previsto:.0f} publicações")
print(f"Variação esperada: {variacao_percentual:+.1f}%")
\end{lstlisting}
\end{pythonbox}

\section{Considerações Finais do Capítulo}

A análise de séries temporais em contextos acadêmicos requer cuidado especial com as características específicas dos dados de pesquisa. Fatores como sazonalidade acadêmica, eventos externos (pandemias, mudanças de política), e ciclos de financiamento podem ter impactos significativos nas séries.

Os métodos apresentados neste capítulo fornecem uma base sólida para:

\begin{itemize}
\item Identificar e modelar padrões temporais em dados de pesquisa
\item Fazer previsões robustas considerando incertezas sazonais
\item Detectar mudanças estruturais em séries temporais
\item Avaliar o impacto de eventos externos nos dados
\end{itemize}

É importante lembrar que a qualidade das previsões depende fundamentalmente da qualidade e representatividade dos dados históricos. Sempre valide os modelos com dados externos e considere fatores qualitativos que podem não estar capturados nos dados quantitativos.

\begin{researchbox}
\textbf{Aplicações Práticas:} Os métodos deste capítulo podem ser aplicados em estudos longitudinais, análise de tendências de citações, previsão de demanda por recursos de pesquisa, monitoramento de indicadores acadêmicos, e avaliação de impacto de políticas científicas.
\end{researchbox}
% =============================================================================
% CAPÍTULO 10: VISUALIZAÇÕES INTERATIVAS E DASHBOARDS
% =============================================================================

\chapter{Visualizações Interativas e Dashboards}

\lettrine{A}{comunicação} efetiva de resultados científicos vai além de gráficos estáticos. Na era digital, visualizações interativas e dashboards permitem que pesquisadores explorem dados dinamicamente, facilitando a descoberta de padrões e a comunicação de insights complexos. Este capítulo explora ferramentas modernas para criar interfaces interativas que transformam dados em experiências envolventes.

\section{Plotly e Visualizações Dinâmicas}

\subsection{Fundamentos do Plotly}

O Plotly é uma biblioteca poderosa para criação de visualizações interativas que podem ser incorporadas em notebooks, aplicações web ou relatórios. Oferece interatividade nativa com zoom, pan, hover e seleção de dados.

\begin{pythonbox}
\begin{lstlisting}[language=Python]
import plotly.express as px
import plotly.graph_objects as go
from plotly.subplots import make_subplots
import plotly.offline as pyo
import pandas as pd
import numpy as np
from datetime import datetime, timedelta
\end{lstlisting}
\end{pythonbox}

\begin{pythonbox}
\begin{lstlisting}[language=Python]
# Configuração para exibição em notebooks
pyo.init_notebook_mode(connected=True)

# Criando dados de exemplo para pesquisa em ciências sociais
np.random.seed(42)
n_participantes = 500

# Simulando dados de um estudo sobre bem-estar e produtividade
dados_estudo = pd.DataFrame({
    'participante_id': range(1, n_participantes + 1),
    'idade': np.random.normal(35, 12, n_participantes).astype(int),
    'bem_estar': np.random.normal(7, 1.5, n_participantes),
    'produtividade': np.random.normal(75, 15, n_participantes),
    'horas_trabalho': np.random.normal(8, 2, n_participantes),
    'satisfacao_trabalho': np.random.normal(6.5, 1.8, n_participantes),
    'area': np.random.choice(['Tecnologia', 'Educação', 'Saúde', 'Finanças'], n_participantes),
    'experiencia': np.random.randint(1, 20, n_participantes)
})

# Adicionando correlações realistas
dados_estudo['produtividade'] += dados_estudo['bem_estar'] * 3 + np.random.normal(0, 5, n_participantes)
dados_estudo['satisfacao_trabalho'] += dados_estudo['bem_estar'] * 0.3 + np.random.normal(0, 0.5, n_participantes)

# Limitando valores aos ranges apropriados
dados_estudo['bem_estar'] = np.clip(dados_estudo['bem_estar'], 1, 10)
dados_estudo['satisfacao_trabalho'] = np.clip(dados_estudo['satisfacao_trabalho'], 1, 10)
dados_estudo['produtividade'] = np.clip(dados_estudo['produtividade'], 30, 100)
dados_estudo['horas_trabalho'] = np.clip(dados_estudo['horas_trabalho'], 4, 12)

print("Dataset criado com sucesso!")
print(dados_estudo.head())
print(f"\nEstatísticas descritivas:")
print(dados_estudo.describe())
\end{lstlisting}
\end{pythonbox}

\subsection{Gráficos Interativos Básicos}

\begin{examplebox}
Vamos criar visualizações interativas que permitem explorar as relações entre bem-estar, produtividade e satisfação no trabalho:
\end{examplebox}

\begin{pythonbox}
\begin{lstlisting}[language=Python]
# Gráfico de dispersão interativo
fig_scatter = px.scatter(
    dados_estudo, 
    x='bem_estar', 
    y='produtividade',
    color='area',
    size='experiencia',
    hover_data=['idade', 'horas_trabalho', 'satisfacao_trabalho'],
    title='Relação entre Bem-estar e Produtividade por Área de Atuação',
    labels={
        'bem_estar': 'Índice de Bem-estar (1-10)',
        'produtividade': 'Índice de Produtividade (30-100)',
        'area': 'Área de Atuação',
        'experiencia': 'Anos de Experiência'
    }
)

# Personalizando o layout
fig_scatter.update_layout(
    width=800,
    height=600,
    font=dict(size=12),
    hovermode='closest'
)

# Adicionando linha de tendência
fig_scatter.add_scatter(
    x=dados_estudo['bem_estar'],
    y=np.poly1d(np.polyfit(dados_estudo['bem_estar'], dados_estudo['produtividade'], 1))(dados_estudo['bem_estar']),
    mode='lines',
    name='Tendência Linear',
    line=dict(color='red', dash='dash')
)

fig_scatter.show()
\end{lstlisting}
\end{pythonbox}

\begin{pythonbox}
\begin{lstlisting}[language=Python]
# Histograma interativo com seleção de variável
from ipywidgets import interact, Dropdown

def criar_histograma_interativo(variavel='bem_estar'):
    fig = px.histogram(
        dados_estudo, 
        x=variavel,
        color='area',
        nbins=20,
        title=f'Distribuição de {variavel.replace("_", " ").title()}',
        barmode='overlay',
        opacity=0.7
    )
    
    fig.update_layout(
        width=700,
        height=500,
        bargap=0.1
    )
    
    return fig.show()

# Lista de variáveis disponíveis
variaveis_numericas = ['bem_estar', 'produtividade', 'horas_trabalho', 
                      'satisfacao_trabalho', 'idade', 'experiencia']

print("Histograma Interativo - selecione uma variável:")
print("Disponível em ambiente Jupyter com widgets")
\end{lstlisting}
\end{pythonbox}

\subsection{Visualizações Multidimensionais}

\begin{pythonbox}
\begin{lstlisting}[language=Python]
# Matriz de correlação interativa
correlacoes = dados_estudo[variaveis_numericas].corr()

fig_heatmap = go.Figure(data=go.Heatmap(
    z=correlacoes.values,
    x=correlacoes.columns,
    y=correlacoes.columns,
    colorscale='RdBu',
    zmid=0,
    text=correlacoes.round(2).values,
    texttemplate="%{text}",
    textfont={"size": 10},
    hoverongaps=False
))

fig_heatmap.update_layout(
    title='Matriz de Correlação Interativa - Variáveis do Estudo',
    width=600,
    height=600,
    font=dict(size=12)
)

fig_heatmap.show()
\end{lstlisting}
\end{pythonbox}

\begin{pythonbox}
\begin{lstlisting}[language=Python]
# Gráfico de coordenadas paralelas
fig_parallel = go.Figure(data=go.Parcoords(
    line=dict(
        color=dados_estudo['bem_estar'],
        colorscale='Viridis',
        showscale=True,
        colorbar=dict(title="Bem-estar")
    ),
    dimensions=[
        dict(
            range=[dados_estudo['idade'].min(), dados_estudo['idade'].max()],
            label='Idade',
            values=dados_estudo['idade']
        ),
        dict(
            range=[dados_estudo['experiencia'].min(), dados_estudo['experiencia'].max()],
            label='Experiência',
            values=dados_estudo['experiencia']
        ),
        dict(
            range=[dados_estudo['horas_trabalho'].min(), dados_estudo['horas_trabalho'].max()],
            label='Horas Trabalho',
            values=dados_estudo['horas_trabalho']
        ),
        dict(
            range=[dados_estudo['bem_estar'].min(), dados_estudo['bem_estar'].max()],
            label='Bem-estar',
            values=dados_estudo['bem_estar']
        ),
        dict(
            range=[dados_estudo['produtividade'].min(), dados_estudo['produtividade'].max()],
            label='Produtividade',
            values=dados_estudo['produtividade']
        ),
        dict(
            range=[dados_estudo['satisfacao_trabalho'].min(), dados_estudo['satisfacao_trabalho'].max()],
            label='Satisfação',
            values=dados_estudo['satisfacao_trabalho']
        )
    ]
))

fig_parallel.update_layout(
    title='Coordenadas Paralelas - Análise Multivariada',
    font=dict(size=12)
)

fig_parallel.show()
\end{lstlisting}
\end{pythonbox}

\subsection{Animações e Séries Temporais}

\begin{pythonbox}
\begin{lstlisting}[language=Python]
# Criando dados temporais para demonstrar animações
datas = pd.date_range(start='2020-01-01', end='2023-12-31', freq='M')
np.random.seed(42)

# Simulando evolução temporal de indicadores por área
dados_temporais = []
for area in dados_estudo['area'].unique():
    for data in datas:
        # Simulando tendências e sazonalidade
        base_bem_estar = 6 + np.sin(data.month * 2 * np.pi / 12) * 0.5
        base_produtividade = 70 + np.sin(data.month * 2 * np.pi / 12) * 5
        
        # Adicionando variação por área
        if area == 'Tecnologia':
            base_bem_estar += 0.5
            base_produtividade += 5
        elif area == 'Saúde':
            base_bem_estar += 0.3
            base_produtividade += 3
        elif area == 'Educação':
            base_bem_estar += 0.1
            base_produtividade -= 2
        
        # Adicionando ruído e tendência temporal
        tendencia = (data.year - 2020) * 0.1
        ruido_bem_estar = np.random.normal(0, 0.3)
        ruido_produtividade = np.random.normal(0, 3)
        
        dados_temporais.append({
            'data': data,
            'area': area,
            'bem_estar_medio': base_bem_estar + tendencia + ruido_bem_estar,
            'produtividade_media': base_produtividade + tendencia * 2 + ruido_produtividade,
            'ano': data.year,
            'mes': data.month
        })

df_temporal = pd.DataFrame(dados_temporais)

print("Dados temporais criados:")
print(df_temporal.head())
\end{lstlisting}
\end{pythonbox}

\begin{pythonbox}
\begin{lstlisting}[language=Python]
# Gráfico animado mostrando evolução temporal
fig_animado = px.scatter(
    df_temporal,
    x='bem_estar_medio',
    y='produtividade_media',
    color='area',
    animation_frame='ano',
    animation_group='area',
    size_max=20,
    range_x=[5, 8],
    range_y=[60, 85],
    title='Evolução Temporal: Bem-estar vs Produtividade por Área',
    labels={
        'bem_estar_medio': 'Bem-estar Médio',
        'produtividade_media': 'Produtividade Média',
        'area': 'Área de Atuação'
    }
)

fig_animado.update_layout(
    width=800,
    height=600,
    font=dict(size=12)
)

fig_animado.show()
\end{lstlisting}
\end{pythonbox}

\section{Streamlit para Aplicações de Pesquisa}

\subsection{Fundamentos do Streamlit}

O Streamlit permite criar aplicações web interativas com Python puro, sem necessidade de conhecimento em HTML, CSS ou JavaScript. É ideal para criar dashboards de pesquisa e ferramentas analíticas.

\begin{pythonbox}
\begin{lstlisting}[language=Python]
import streamlit as st
import pandas as pd
import numpy as np
import plotly.express as px
import plotly.graph_objects as go
from datetime import datetime, timedelta

# Nota: Este código deve ser salvo em um arquivo .py separado
# e executado com: streamlit run nome_do_arquivo.py
\end{lstlisting}
\end{pythonbox}

\begin{examplebox}
Vamos criar uma aplicação Streamlit para análise interativa de dados de pesquisa. Salve o código a seguir como \texttt{app\_pesquisa.py}:
\end{examplebox}

\begin{pythonbox}
\begin{lstlisting}[language=Python]
# app_pesquisa.py - Aplicação Streamlit para Análise de Dados de Pesquisa

def main():
    st.set_page_config(
        page_title="Dashboard de Pesquisa",
        page_icon="��",
        layout="wide",
        initial_sidebar_state="expanded"
    )
    
    st.title("�� Dashboard Interativo de Análise de Pesquisa")
    st.markdown("---")
    
    # Sidebar para controles
    st.sidebar.header("�� Controles de Análise")
    
    # Upload de dados ou usar dados de exemplo
    opcao_dados = st.sidebar.radio(
        "Fonte de Dados:",
        ["Usar dados de exemplo", "Upload de arquivo CSV"]
    )
    
    if opcao_dados == "Upload de arquivo CSV":
        uploaded_file = st.sidebar.file_uploader(
            "Escolha um arquivo CSV",
            type="csv"
        )
        
        if uploaded_file is not None:
            dados = pd.read_csv(uploaded_file)
            st.sidebar.success("Arquivo carregado com sucesso!")
        else:
            st.warning("Por favor, faça upload de um arquivo CSV.")
            return
    else:
        # Usando dados de exemplo (mesmo dataset anterior)
        dados = carregar_dados_exemplo()
    
    # Exibindo informações básicas
    col1, col2, col3, col4 = st.columns(4)
    
    with col1:
        st.metric("Total de Participantes", len(dados))
    with col2:
        st.metric("Bem-estar Médio", f"{dados['bem_estar'].mean():.1f}")
    with col3:
        st.metric("Produtividade Média", f"{dados['produtividade'].mean():.1f}")
    with col4:
        st.metric("Satisfação Média", f"{dados['satisfacao_trabalho'].mean():.1f}")
    
    st.markdown("---")

if __name__ == "__main__":
    main()
\end{lstlisting}
\end{pythonbox}

\section{Jupyter Widgets e Interfaces Interativas}

\subsection{Fundamentos dos Jupyter Widgets}

Os Jupyter Widgets (ipywidgets) permitem criar interfaces interativas diretamente nos notebooks Jupyter, facilitando a exploração de dados e parâmetros de modelos.

\begin{pythonbox}
\begin{lstlisting}[language=Python]
import ipywidgets as widgets
from ipywidgets import interact, interactive, fixed, interact_manual
from IPython.display import display, clear_output
import matplotlib.pyplot as plt
import seaborn as sns
\end{lstlisting}
\end{pythonbox}

\begin{pythonbox}
\begin{lstlisting}[language=Python]
# Widget básico para exploração de dados
def explorar_distribuicoes(dados):
    """Cria widgets para explorar distribuições de variáveis"""
    
    # Definindo opções
    variaveis = ['bem_estar', 'produtividade', 'satisfacao_trabalho', 'idade', 'experiencia']
    areas = ['Todas'] + list(dados['area'].unique())
    tipos_grafico = ['Histograma', 'Box Plot', 'Density Plot']
    
    # Criando widgets
    widget_variavel = widgets.Dropdown(
        options=variaveis,
        value='bem_estar',
        description='Variável:'
    )
    
    widget_area = widgets.Dropdown(
        options=areas,
        value='Todas',
        description='Área:'
    )
    
    widget_tipo = widgets.Dropdown(
        options=tipos_grafico,
        value='Histograma',
        description='Tipo:'
    )
    
    widget_bins = widgets.IntSlider(
        value=20,
        min=5,
        max=50,
        step=5,
        description='Bins:'
    )
    
    # Interface interativa
    interface = interactive(
        plotar_distribuicao,
        variavel=widget_variavel,
        area=widget_area,
        tipo_grafico=widget_tipo,
        bins=widget_bins
    )
    
    return interface
\end{lstlisting}
\end{pythonbox}

\section{Dashboards com Dash}

\subsection{Introdução ao Dash}

O Dash é um framework Python para criar aplicações web analíticas interativas. É especialmente útil para dashboards corporativos e painéis de monitoramento científico.

\begin{pythonbox}
\begin{lstlisting}[language=Python]
import dash
from dash import dcc, html, Input, Output, State, callback_context
import dash_bootstrap_components as dbc
import plotly.express as px
import plotly.graph_objects as go
from datetime import datetime, timedelta
import json
\end{lstlisting}
\end{pythonbox}

\begin{examplebox}
Vamos criar um dashboard completo para monitoramento de indicadores de pesquisa. Salve o código a seguir como \texttt{dashboard\_pesquisa.py}:
\end{examplebox}

\begin{pythonbox}
\begin{lstlisting}[language=Python]
# dashboard_pesquisa.py - Dashboard completo com Dash

# Configuração da aplicação
app = dash.Dash(__name__, external_stylesheets=[dbc.themes.BOOTSTRAP])
app.title = "Dashboard de Pesquisa Acadêmica"

# Carregando dados (mesmo dataset anterior)
def carregar_dados():
    np.random.seed(42)
    n_participantes = 1000
    
    dados = pd.DataFrame({
        'participante_id': range(1, n_participantes + 1),
        'idade': np.random.normal(35, 12, n_participantes).astype(int),
        'bem_estar': np.random.normal(7, 1.5, n_participantes),
        'produtividade': np.random.normal(75, 15, n_participantes),
        'horas_trabalho': np.random.normal(8, 2, n_participantes),
        'satisfacao_trabalho': np.random.normal(6.5, 1.8, n_participantes),
        'area': np.random.choice(['Tecnologia', 'Educação', 'Saúde', 'Finanças'], n_participantes),
        'experiencia': np.random.randint(1, 20, n_participantes),
        'data_coleta': pd.date_range(start='2023-01-01', periods=n_participantes, freq='H')
    })
    
    # Correlações realistas
    dados['produtividade'] += dados['bem_estar'] * 3 + np.random.normal(0, 5, n_participantes)
    dados['satisfacao_trabalho'] += dados['bem_estar'] * 0.3 + np.random.normal(0, 0.5, n_participantes)
    
    # Limitando valores
    dados['bem_estar'] = np.clip(dados['bem_estar'], 1, 10)
    dados['satisfacao_trabalho'] = np.clip(dados['satisfacao_trabalho'], 1, 10)
    dados['produtividade'] = np.clip(dados['produtividade'], 30, 100)
    dados['horas_trabalho'] = np.clip(dados['horas_trabalho'], 4, 12)
    
    return dados

dados_globais = carregar_dados()
\end{lstlisting}
\end{pythonbox}

\begin{pythonbox}
\begin{lstlisting}[language=Python]
# Layout básico do dashboard
app.layout = html.Div([
    html.H1("Dashboard de Pesquisa Acadêmica"),
    dcc.Graph(id="exemplo-grafico"),
    html.P("Dashboard funcional com Dash")
])

# Executando a aplicação
if __name__ == '__main__':
    app.run_server(debug=True, port=8050)
\end{lstlisting}
\end{pythonbox}

\section{Considerações Finais do Capítulo}

As visualizações interativas e dashboards representam uma evolução natural na comunicação científica. Elas permitem que pesquisadores e stakeholders explorem dados de forma intuitiva, descobrindo padrões que poderiam passar despercebidos em análises estáticas.

### Principais Benefícios:

\begin{itemize}
\item \textbf{Engajamento}: Interfaces interativas mantêm o usuário engajado na exploração dos dados
\item \textbf{Flexibilidade}: Permitem análises sob diferentes perspectivas sem necessidade de novo código
\item \textbf{Acessibilidade}: Tornam análises complexas acessíveis a não-programadores
\item \textbf{Colaboração}: Facilitam o compartilhamento e discussão de resultados
\item \textbf{Iteração}: Permitem refinamento rápido de hipóteses e análises
\end{itemize}

### Escolha da Ferramenta:

\begin{itemize}
\item \textbf{Plotly}: Ideal para gráficos interativos em notebooks e relatórios
\item \textbf{Streamlit}: Perfeito para protótipos rápidos e aplicações simples
\item \textbf{Jupyter Widgets}: Excelente para exploração paramétrica em notebooks
\item \textbf{Dash}: Melhor opção para dashboards corporativos e aplicações robustas
\end{itemize}

\begin{warningbox}
\textbf{Boas Práticas:} Mantenha a simplicidade, teste a usabilidade com usuários reais, documente as funcionalidades, implemente controle de acesso quando necessário, e sempre valide os dados antes da visualização.
\end{warningbox}

\begin{researchbox}
\textbf{Aplicações Práticas:} Dashboards de monitoramento de experimentos, interfaces para coleta de dados de pesquisa, painéis de acompanhamento de indicadores acadêmicos, ferramentas de análise colaborativa, e sistemas de visualização para apresentações interativas.
\end{researchbox}
% =============================================================================
% CAPÍTULO 11: COMUNICAÇÃO CIENTÍFICA COM PYTHON
% =============================================================================

\chapter{Comunicação Científica com Python}

\lettrine{A}{comunicação} efetiva de resultados científicos é tão importante quanto a própria pesquisa. Python oferece ferramentas poderosas para automatizar a geração de relatórios, criar apresentações dinâmicas e produzir visualizações de qualidade para publicação. Este capítulo explora como transformar análises Python em comunicação científica profissional e impactante.

% =============================================================================
\section{Relatórios Automatizados}
% =============================================================================

\subsection{Fundamentos da Geração Automatizada}

A automação de relatórios permite criar documentos consistentes, reproduzíveis e atualizáveis dinamicamente. Isso é especialmente valioso em pesquisas longitudinais ou quando é necessário gerar relatórios periódicos.

\begin{pythonbox}
\begin{lstlisting}[language=Python]
import pandas as pd
import numpy as np
import matplotlib.pyplot as plt
import seaborn as sns
from datetime import datetime, timedelta
import jinja2
from reportlab.lib import colors
from reportlab.lib.pagesizes import letter, A4
from reportlab.platypus import SimpleDocTemplate, Table, TableStyle, Paragraph, Spacer
from reportlab.lib.styles import getSampleStyleSheet, ParagraphStyle
from reportlab.lib.units import inch
import warnings
warnings.filterwarnings('ignore')
\end{lstlisting}
\end{pythonbox}

\begin{pythonbox}
\begin{lstlisting}[language=Python]
# Configuração de estilo para gráficos
plt.style.use('seaborn-v0_8-whitegrid')
sns.set_palette("husl")

# Criando dados de exemplo para relatório de pesquisa
np.random.seed(42)
n_participantes = 200

# Simulando dados de um estudo longitudinal sobre aprendizado
dados_estudo = pd.DataFrame({
    'participante_id': range(1, n_participantes + 1),
    'grupo': np.random.choice(['Controle', 'Experimental'], n_participantes),
    'idade': np.random.normal(22, 3, n_participantes).astype(int),
    'genero': np.random.choice(['M', 'F', 'NB'], n_participantes, p=[0.45, 0.45, 0.1]),
    'pre_teste': np.random.normal(65, 15, n_participantes),
    'pos_teste': np.random.normal(70, 18, n_participantes),
    'satisfacao': np.random.normal(7.5, 1.8, n_participantes),
    'tempo_estudo': np.random.normal(4.5, 2, n_participantes),
    'data_coleta': pd.date_range(start='2024-01-15', periods=n_participantes, freq='D')
})

# Adicionando efeito realista do grupo experimental
mask_experimental = dados_estudo['grupo'] == 'Experimental'
dados_estudo.loc[mask_experimental, 'pos_teste'] += np.random.normal(8, 3, mask_experimental.sum())
dados_estudo.loc[mask_experimental, 'satisfacao'] += np.random.normal(0.8, 0.4, mask_experimental.sum())

# Limitando valores aos ranges apropriados
dados_estudo['pre_teste'] = np.clip(dados_estudo['pre_teste'], 0, 100)
dados_estudo['pos_teste'] = np.clip(dados_estudo['pos_teste'], 0, 100)
dados_estudo['satisfacao'] = np.clip(dados_estudo['satisfacao'], 1, 10)
dados_estudo['tempo_estudo'] = np.clip(dados_estudo['tempo_estudo'], 1, 10)

# Calculando variáveis derivadas
dados_estudo['ganho_aprendizado'] = dados_estudo['pos_teste'] - dados_estudo['pre_teste']
dados_estudo['ganho_percentual'] = (dados_estudo['ganho_aprendizado'] / dados_estudo['pre_teste']) * 100

print("Dataset para relatório criado:")
print(dados_estudo.head())
print(f"\nEstatísticas básicas:")
print(dados_estudo.groupby('grupo')[['pre_teste', 'pos_teste', 'ganho_aprendizado']].mean())
\end{lstlisting}
\end{pythonbox}

\subsection{Classe para Geração de Relatórios}

\begin{examplebox}
Vamos criar uma classe abrangente para automatizar a geração de relatórios científicos:
\end{examplebox}

\begin{pythonbox}
\begin{lstlisting}[language=Python]
class GeradorRelatorio:
    """Classe para automatizar a geração de relatórios científicos"""
    
    def __init__(self, dados, titulo="Relatório de Pesquisa", autor="Pesquisador"):
        self.dados = dados
        self.titulo = titulo
        self.autor = autor
        self.data_relatorio = datetime.now()
        self.figuras = []
        self.estatisticas = {}
        
    def calcular_estatisticas_descritivas(self):
        """Calcula estatísticas descritivas básicas"""
        stats = {}
        
        # Estatísticas por grupo
        for grupo in self.dados['grupo'].unique():
            dados_grupo = self.dados[self.dados['grupo'] == grupo]
            stats[grupo] = {
                'n': len(dados_grupo),
                'pre_teste_media': dados_grupo['pre_teste'].mean(),
                'pre_teste_dp': dados_grupo['pre_teste'].std(),
                'pos_teste_media': dados_grupo['pos_teste'].mean(),
                'pos_teste_dp': dados_grupo['pos_teste'].std(),
                'ganho_media': dados_grupo['ganho_aprendizado'].mean(),
                'ganho_dp': dados_grupo['ganho_aprendizado'].std(),
                'satisfacao_media': dados_grupo['satisfacao'].mean(),
                'satisfacao_dp': dados_grupo['satisfacao'].std()
            }
        
        # Estatísticas gerais
        stats['geral'] = {
            'total_participantes': len(self.dados),
            'idade_media': self.dados['idade'].mean(),
            'idade_dp': self.dados['idade'].std(),
            'periodo_coleta': f"{self.dados['data_coleta'].min().strftime('%d/%m/%Y')} a {self.dados['data_coleta'].max().strftime('%d/%m/%Y')}"
        }
        
        self.estatisticas = stats
        return stats


    \end{lstlisting}
\end{pythonbox}

\begin{pythonbox}
\begin{lstlisting}[language=Python]
    def realizar_testes_estatisticos(self):
        """Realiza testes estatísticos principais"""
        from scipy import stats as scipy_stats
        
        # Separando grupos
        controle = self.dados[self.dados['grupo'] == 'Controle']
        experimental = self.dados[self.dados['grupo'] == 'Experimental']
        
        # Teste t para diferenças entre grupos no pós-teste
        t_stat, p_valor = scipy_stats.ttest_ind(
            controle['pos_teste'], 
            experimental['pos_teste']
        )
        
        # Teste t pareado para ganho de aprendizado no grupo experimental
        t_stat_pareado, p_valor_pareado = scipy_stats.ttest_rel(
            experimental['pre_teste'], 
            experimental['pos_teste']
        )
        
        # Tamanho do efeito (Cohen's d)
        pooled_std = np.sqrt(((len(controle)-1)*controle['pos_teste'].std()**2 + 
                             (len(experimental)-1)*experimental['pos_teste'].std()**2) / 
                            (len(controle)+len(experimental)-2))
        cohens_d = (experimental['pos_teste'].mean() - controle['pos_teste'].mean()) / pooled_std
        
        testes = {
            'teste_independente': {
                't_statistic': t_stat,
                'p_valor': p_valor,
                'cohens_d': cohens_d,
                'significativo': p_valor < 0.05
            },
            'teste_pareado_experimental': {
                't_statistic': t_stat_pareado,
                'p_valor': p_valor_pareado,
                'significativo': p_valor_pareado < 0.05
            }
        }
        
        self.testes_estatisticos = testes
        return testes
\end{lstlisting}
\end{pythonbox}

\begin{pythonbox}
\begin{lstlisting}[language=Python]
    def gerar_graficos(self, salvar=True):
        """Gera gráficos para o relatório"""
        
        # Configuração de figura
        fig, axes = plt.subplots(2, 2, figsize=(15, 12))
        
        # Gráfico 1: Comparação pré vs pós por grupo
        grupos = self.dados['grupo'].unique()
        x = np.arange(len(grupos))
        width = 0.35
        
        pre_medias = [self.dados[self.dados['grupo'] == g]['pre_teste'].mean() for g in grupos]
        pos_medias = [self.dados[self.dados['grupo'] == g]['pos_teste'].mean() for g in grupos]
        pre_erros = [self.dados[self.dados['grupo'] == g]['pre_teste'].std() for g in grupos]
        pos_erros = [self.dados[self.dados['grupo'] == g]['pos_teste'].std() for g in grupos]
        
        axes[0, 0].bar(x - width/2, pre_medias, width, label='Pré-teste', 
                       yerr=pre_erros, capsize=5, alpha=0.8)
        axes[0, 0].bar(x + width/2, pos_medias, width, label='Pós-teste', 
                       yerr=pos_erros, capsize=5, alpha=0.8)
        axes[0, 0].set_xlabel('Grupo')
        axes[0, 0].set_ylabel('Pontuação')
        axes[0, 0].set_title('Comparação Pré vs Pós-teste por Grupo')
        axes[0, 0].set_xticks(x)
        axes[0, 0].set_xticklabels(grupos)
        axes[0, 0].legend()
        axes[0, 0].grid(True, alpha=0.3)
        
        # Gráfico 2: Distribuição dos ganhos de aprendizado
        self.dados.boxplot(column='ganho_aprendizado', by='grupo', ax=axes[0, 1])
        axes[0, 1].set_title('Distribuição dos Ganhos de Aprendizado')
        axes[0, 1].set_xlabel('Grupo')
        axes[0, 1].set_ylabel('Ganho (Pós - Pré)')
        
        # Gráfico 3: Correlação entre satisfação e ganho
        cores = {'Controle': 'blue', 'Experimental': 'red'}
        for grupo in grupos:
            dados_grupo = self.dados[self.dados['grupo'] == grupo]
            axes[1, 0].scatter(dados_grupo['satisfacao'], dados_grupo['ganho_aprendizado'],
                              c=cores[grupo], alpha=0.6, label=grupo)

   \end{lstlisting}
\end{pythonbox}

\begin{pythonbox}
\begin{lstlisting}[language=Python]               
        axes[1, 0].set_xlabel('Satisfação')
        axes[1, 0].set_ylabel('Ganho de Aprendizado')
        axes[1, 0].set_title('Satisfação vs Ganho de Aprendizado')
        axes[1, 0].legend()
        axes[1, 0].grid(True, alpha=0.3)
        
        # Gráfico 4: Evolução temporal (últimos 30 dias)
        dados_recentes = self.dados.tail(30)
        dados_temporal = dados_recentes.groupby(['data_coleta', 'grupo'])['ganho_aprendizado'].mean().unstack()
        
        if 'Controle' in dados_temporal.columns:
            axes[1, 1].plot(dados_temporal.index, dados_temporal['Controle'], 
                           marker='o', label='Controle', linewidth=2)
        if 'Experimental' in dados_temporal.columns:
            axes[1, 1].plot(dados_temporal.index, dados_temporal['Experimental'], 
                           marker='s', label='Experimental', linewidth=2)

  
        axes[1, 1].set_xlabel('Data')
        axes[1, 1].set_ylabel('Ganho Médio')
        axes[1, 1].set_title('Evolução Temporal dos Ganhos (Últimos 30 dias)')
        axes[1, 1].legend()
        axes[1, 1].grid(True, alpha=0.3)
        axes[1, 1].tick_params(axis='x', rotation=45)
        
        plt.tight_layout()
        
        if salvar:
            nome_arquivo = f"graficos_relatorio_{self.data_relatorio.strftime('%Y%m%d_%H%M%S')}.png"
            plt.savefig(nome_arquivo, dpi=300, bbox_inches='tight')
            self.figuras.append(nome_arquivo)
            print(f"Gráficos salvos em: {nome_arquivo}")
        
        plt.show()
        return fig
\end{lstlisting}
\end{pythonbox}

\begin{pythonbox}
\begin{lstlisting}[language=Python]
    def gerar_relatorio_html(self):
        """Gera relatório em HTML usando Jinja2"""
        
        # Template HTML
        template_html = """
        <!DOCTYPE html>
        <html>
        <head>
            <meta charset="UTF-8">
            <title>{{ titulo }}</title>
            <style>
                body { font-family: Arial, sans-serif; margin: 40px; line-height: 1.6; }
                .header { text-align: center; border-bottom: 2px solid #333; padding-bottom: 20px; }
                .section { margin: 30px 0; }
                .stats-table { border-collapse: collapse; width: 100%; margin: 20px 0; }
                .stats-table th, .stats-table td { border: 1px solid #ddd; padding: 12px; text-align: left; }
                .stats-table th { background-color: #f2f2f2; font-weight: bold; }
                .highlight { background-color: #ffffcc; }
                .significant { color: #d63031; font-weight: bold; }
                .footer { margin-top: 50px; text-align: center; font-size: 0.9em; color: #666; }
            </style>
        </head>
        <body>
            <div class="header">
                <h1>{{ titulo }}</h1>
                <p><strong>Autor:</strong> {{ autor }}</p>
                <p><strong>Data:</strong> {{ data_relatorio }}</p>
                <p><strong>Período de Coleta:</strong> {{ periodo_coleta }}</p>
            </div>
            
            <div class="section">
                <h2>1. Resumo Executivo</h2>
                <p>Este relatório apresenta os resultados de um estudo experimental sobre aprendizado, 
                envolvendo {{ total_participantes }} participantes divididos entre grupo controle e experimental. 
                O objetivo foi avaliar a eficácia de uma nova metodologia de ensino.</p>
   \end{lstlisting}
\end{pythonbox}

\begin{pythonbox}
\begin{lstlisting}[language=Python]                       
                <h3>Principais Achados:</h3>
                <ul>
                    <li>Diferença significativa entre grupos no pós-teste: <span class="{{ 'significant' if teste_significativo else '' }}">{{ 'SIM' if teste_significativo else 'NÃO' }} (p = {{ p_valor | round(4) }})</span></li>
                    <li>Tamanho do efeito (Cohen's d): {{ cohens_d | round(3) }}</li>
                    <li>Ganho médio grupo experimental: {{ ganho_experimental | round(2) }} pontos</li>
                    <li>Ganho médio grupo controle: {{ ganho_controle | round(2) }} pontos</li>
                </ul>
            </div>
            
            <div class="section">
                <h2>2. Características da Amostra</h2>
                <table class="stats-table">
                    <tr>
                        <th>Característica</th>
                        <th>Valor</th>
                    </tr>
                    <tr>
                        <td>Total de Participantes</td>
                        <td>{{ total_participantes }}</td>
                    </tr>
                    <tr>
                        <td>Idade Média (DP)</td>
                        <td>{{ idade_media | round(1) }} ({{ idade_dp | round(1) }})</td>
                    </tr>
                    <tr>
                        <td>Grupo Controle</td>
                        <td>{{ n_controle }} participantes</td>
                    </tr>
                    <tr>
                        <td>Grupo Experimental</td>
                        <td>{{ n_experimental }} participantes</td>
                    </tr>
                </table>
            </div>
            
            <div class="section">
                <h2>3. Resultados por Grupo</h2>
                <table class="stats-table">
                    <tr>
                        <th>Medida</th>
                        <th>Controle</th>
                        <th>Experimental</th>
                    </tr>
   \end{lstlisting}
\end{pythonbox}

\begin{pythonbox}
\begin{lstlisting}[language=Python]                           
                    <tr>
                        <td>Pré-teste (M ± DP)</td>
                        <td>{{ pre_controle | round(2) }} ± {{ pre_controle_dp | round(2) }}</td>
                        <td>{{ pre_experimental | round(2) }} ± {{ pre_experimental_dp | round(2) }}</td>
                    </tr>
                    <tr class="highlight">
                        <td>Pós-teste (M ± DP)</td>
                        <td>{{ pos_controle | round(2) }} ± {{ pos_controle_dp | round(2) }}</td>
                        <td>{{ pos_experimental | round(2) }} ± {{ pos_experimental_dp | round(2) }}</td>
                    </tr>
                    <tr class="highlight">
                        <td>Ganho de Aprendizado</td>
                        <td>{{ ganho_controle | round(2) }} ± {{ ganho_controle_dp | round(2) }}</td>
                        <td>{{ ganho_experimental | round(2) }} ± {{ ganho_experimental_dp | round(2) }}</td>
                    </tr>
                    <tr>
                        <td>Satisfação</td>
                        <td>{{ sat_controle | round(2) }} ± {{ sat_controle_dp | round(2) }}</td>
                        <td>{{ sat_experimental | round(2) }} ± {{ sat_experimental_dp | round(2) }}</td>
                    </tr>
                </table>
            </div>
            
            <div class="section">
                <h2>4. Análises Estatísticas</h2>
                <h3>Teste t para Amostras Independentes (Pós-teste)</h3>
                <ul>
                    <li>t({{ graus_liberdade }}) = {{ t_statistic | round(3) }}</li>
                    <li>p = {{ p_valor | round(4) }}</li>
                    <li>Cohen's d = {{ cohens_d | round(3) }}</li>
                    <li><strong>Interpretação:</strong> {{ interpretacao_resultado }}</li>
                </ul>
   \end{lstlisting}
\end{pythonbox}

\begin{pythonbox}
\begin{lstlisting}[language=Python]                       
                <h3>Teste t Pareado (Grupo Experimental: Pré vs Pós)</h3>
                <ul>
                    <li>t = {{ t_pareado | round(3) }}</li>
                    <li>p = {{ p_pareado | round(4) }}</li>
                    <li><strong>Resultado:</strong> {{ resultado_pareado }}</li>
                </ul>
            </div>
            
            <div class="footer">
                <p>Relatório gerado automaticamente em {{ data_relatorio }}</p>
                <p>Sistema de Análise de Dados - Python</p>
            </div>
        </body>
        </html>
        """
        
        # Preparando dados para o template
        stats = self.estatisticas
        testes = self.testes_estatisticos
        
        dados_template = {
            'titulo': self.titulo,
            'autor': self.autor,
            'data_relatorio': self.data_relatorio.strftime('%d/%m/%Y %H:%M'),
            'periodo_coleta': stats['geral']['periodo_coleta'],
            'total_participantes': stats['geral']['total_participantes'],
            'idade_media': stats['geral']['idade_media'],
            'idade_dp': stats['geral']['idade_dp'],
            
            # Dados por grupo
            'n_controle': stats['Controle']['n'],
            'n_experimental': stats['Experimental']['n'],
            'pre_controle': stats['Controle']['pre_teste_media'],
            'pre_controle_dp': stats['Controle']['pre_teste_dp'],
            'pre_experimental': stats['Experimental']['pre_teste_media'],
            'pre_experimental_dp': stats['Experimental']['pre_teste_dp'],
            'pos_controle': stats['Controle']['pos_teste_media'],
            'pos_controle_dp': stats['Controle']['pos_teste_dp'],

   \end{lstlisting}
\end{pythonbox}

\begin{pythonbox}
\begin{lstlisting}[language=Python]                   
            'pos_experimental': stats['Experimental']['pos_teste_media'],
            'pos_experimental_dp': stats['Experimental']['pos_teste_dp'],
            'ganho_controle': stats['Controle']['ganho_media'],
            'ganho_controle_dp': stats['Controle']['ganho_dp'],
            'ganho_experimental': stats['Experimental']['ganho_media'],
            'ganho_experimental_dp': stats['Experimental']['ganho_dp'],
            'sat_controle': stats['Controle']['satisfacao_media'],
            'sat_controle_dp': stats['Controle']['satisfacao_dp'],
            'sat_experimental': stats['Experimental']['satisfacao_media'],
            'sat_experimental_dp': stats['Experimental']['satisfacao_dp'],
            
            # Testes estatísticos
            't_statistic': testes['teste_independente']['t_statistic'],
            'p_valor': testes['teste_independente']['p_valor'],
            'cohens_d': testes['teste_independente']['cohens_d'],
            'teste_significativo': testes['teste_independente']['significativo'],
            'graus_liberdade': len(self.dados) - 2,
            't_pareado': testes['teste_pareado_experimental']['t_statistic'],
            'p_pareado': testes['teste_pareado_experimental']['p_valor'],
            
            # Interpretações
            'interpretacao_resultado': 'Diferença estatisticamente significativa entre os grupos' if testes['teste_independente']['significativo'] else 'Não há diferença estatisticamente significativa entre os grupos',
            'resultado_pareado': 'Melhora significativa no grupo experimental' if testes['teste_pareado_experimental']['significativo'] else 'Não há melhora significativa no grupo experimental'
        }
        
        # Renderizando template
        template = jinja2.Template(template_html)
        html_content = template.render(**dados_template)
        
        # Salvando arquivo
        nome_arquivo = f"relatorio_{self.data_relatorio.strftime('%Y%m%d_%H%M%S')}.html"
        with open(nome_arquivo, 'w', encoding='utf-8') as f:
            f.write(html_content)
        
        print(f"Relatório HTML gerado: {nome_arquivo}")
        return nome_arquivo, html_content
\end{lstlisting}
\end{pythonbox}

\begin{pythonbox}
\begin{lstlisting}[language=Python]
    def gerar_relatorio_completo(self):
        """Gera relatório completo com todas as análises"""
        print("Gerando relatório completo...")
        print("=" * 50)
        
        # Passo 1: Calcular estatísticas
        print("1. Calculando estatísticas descritivas...")
        self.calcular_estatisticas_descritivas()
        
        # Passo 2: Realizar testes
        print("2. Realizando testes estatísticos...")
        self.realizar_testes_estatisticos()
        
        # Passo 3: Gerar gráficos
        print("3. Gerando gráficos...")
        self.gerar_graficos()
        
        # Passo 4: Gerar relatório HTML
        print("4. Gerando relatório HTML...")
        arquivo_html, _ = self.gerar_relatorio_html()
        
        print("=" * 50)
        print("Relatório completo gerado com sucesso!")
        print(f"Arquivo: {arquivo_html}")
        
        return arquivo_html

# Exemplo de uso
relatorio = GeradorRelatorio(
    dados_estudo, 
    titulo="Eficácia de Nova Metodologia de Ensino",
    autor="Dr. João Silva"
)

arquivo_relatorio = relatorio.gerar_relatorio_completo()
\end{lstlisting}
\end{pythonbox}

\subsection{Relatórios em PDF com ReportLab}

\begin{pythonbox}
\begin{lstlisting}[language=Python]
def gerar_relatorio_pdf(dados, titulo="Relatório de Pesquisa"):
    """Gera relatório profissional em PDF"""
    from reportlab.platypus import SimpleDocTemplate, Paragraph, Spacer, Table, TableStyle, Image
    from reportlab.lib.styles import getSampleStyleSheet, ParagraphStyle
    from reportlab.lib import colors
    from reportlab.lib.units import inch
    from reportlab.lib.pagesizes import letter
    
    # Configuração do documento
    nome_arquivo = f"relatorio_pdf_{datetime.now().strftime('%Y%m%d_%H%M%S')}.pdf"
    doc = SimpleDocTemplate(nome_arquivo, pagesize=letter,
                          rightMargin=72, leftMargin=72,
                          topMargin=72, bottomMargin=18)
    
    # Estilos
    styles = getSampleStyleSheet()
    titulo_style = ParagraphStyle(
        'CustomTitle',
        parent=styles['Heading1'],
        fontSize=18,
        spaceAfter=30,
        alignment=1  # Centralizado
    )
    
    # Conteúdo do documento
    story = []
    
    # Título
    story.append(Paragraph(titulo, titulo_style))
    story.append(Spacer(1, 12))
    
    # Informações gerais
    info_geral = f"""
    <b>Data do Relatório:</b> {datetime.now().strftime('%d/%m/%Y')}<br/>
    <b>Total de Participantes:</b> {len(dados)}<br/>
    <b>Período de Coleta:</b> {dados['data_coleta'].min().strftime('%d/%m/%Y')} a {dados['data_coleta'].max().strftime('%d/%m/%Y')}
    """
    story.append(Paragraph(info_geral, styles['Normal']))
    story.append(Spacer(1, 12))
    
    # Estatísticas por grupo
    story.append(Paragraph("<b>Estatísticas por Grupo</b>", styles['Heading2']))
       \end{lstlisting}
\end{pythonbox}

\begin{pythonbox}
\begin{lstlisting}[language=Python]       
    # Criando tabela de estatísticas
    dados_tabela = [['Medida', 'Controle', 'Experimental']]
    
    for grupo in ['Controle', 'Experimental']:
        dados_grupo = dados[dados['grupo'] == grupo]
        if grupo == 'Controle':
            linha_pre = ['Pré-teste (M ± DP)', 
                        f"{dados_grupo['pre_teste'].mean():.2f} ± {dados_grupo['pre_teste'].std():.2f}",
                        '']
            linha_pos = ['Pós-teste (M ± DP)', 
                        f"{dados_grupo['pos_teste'].mean():.2f} ± {dados_grupo['pos_teste'].std():.2f}",
                        '']
        else:
            linha_pre[2] = f"{dados_grupo['pre_teste'].mean():.2f} ± {dados_grupo['pre_teste'].std():.2f}"
            linha_pos[2] = f"{dados_grupo['pos_teste'].mean():.2f} ± {dados_grupo['pos_teste'].std():.2f}"
    
    dados_tabela.extend([linha_pre, linha_pos])
    
    # Formatando tabela
    tabela = Table(dados_tabela)
    tabela.setStyle(TableStyle([
        ('BACKGROUND', (0, 0), (-1, 0), colors.grey),
        ('TEXTCOLOR', (0, 0), (-1, 0), colors.whitesmoke),
        ('ALIGN', (0, 0), (-1, -1), 'CENTER'),
        ('FONTNAME', (0, 0), (-1, 0), 'Helvetica-Bold'),
        ('FONTSIZE', (0, 0), (-1, 0), 14),
        ('BOTTOMPADDING', (0, 0), (-1, 0), 12),
        ('BACKGROUND', (0, 1), (-1, -1), colors.beige),
        ('GRID', (0, 0), (-1, -1), 1, colors.black)
    ]))
    
    story.append(tabela)
    story.append(Spacer(1, 12))
    
    # Conclusões
    story.append(Paragraph("<b>Conclusões</b>", styles['Heading2']))
    conclusoes = """
    Com base nas análises realizadas, observamos diferenças notáveis entre os grupos controle e experimental. 
    O grupo experimental apresentou maior ganho de aprendizado em comparação ao grupo controle, 
    sugerindo a eficácia da nova metodologia implementada.
    """
    story.append(Paragraph(conclusoes, styles['Normal']))
       \end{lstlisting}
\end{pythonbox}

\begin{pythonbox}
\begin{lstlisting}[language=Python]       
    # Gerando PDF
    doc.build(story)
    print(f"Relatório PDF gerado: {nome_arquivo}")
    return nome_arquivo

# Exemplo de uso
arquivo_pdf = gerar_relatorio_pdf(dados_estudo, "Relatório de Eficácia Educacional")
\end{lstlisting}
\end{pythonbox}

% =============================================================================
\section{Apresentações com Jupyter}
% =============================================================================

\subsection{Jupyter como Ferramenta de Apresentação}

O Jupyter oferece extensões poderosas para criar apresentações científicas interativas, permitindo combinar código, resultados e narrativa em um formato dinâmico.

\begin{pythonbox}
\begin{lstlisting}[language=Python]
# Instalação das dependências para apresentações
# !pip install RISE
# !pip install jupyter_contrib_nbextensions
# !jupyter contrib nbextension install --user
# !jupyter nbextension enable rise --py

import matplotlib.pyplot as plt
import seaborn as sns
import plotly.express as px
import plotly.graph_objects as go
from IPython.display import HTML, display, Markdown
import warnings
warnings.filterwarnings('ignore')
\end{lstlisting}
\end{pythonbox}

\begin{pythonbox}
\begin{lstlisting}[language=Python]
# Configuração para apresentações
plt.rcParams['figure.figsize'] = (12, 8)
plt.rcParams['font.size'] = 14
sns.set_style("whitegrid")
sns.set_palette("husl")

# Função para criar slides com markdown
def criar_slide_titulo(titulo, subtitulo="", autor=""):
    """Cria slide de título para apresentação"""
    html_content = f"""
    <div style="text-align: center; padding: 100px 0;">
        <h1 style="font-size: 48px; color: #2c3e50; margin-bottom: 30px;">{titulo}</h1>
        {f'<h2 style="font-size: 32px; color: #7f8c8d; margin-bottom: 20px;">{subtitulo}</h2>' if subtitulo else ''}
        {f'<h3 style="font-size: 24px; color: #95a5a6;">{autor}</h3>' if autor else ''}
        <p style="font-size: 18px; color: #bdc3c7; margin-top: 40px;">{datetime.now().strftime('%d de %B de %Y')}</p>
    </div>
    """
    return HTML(html_content)

# Função para slides de conteúdo
def criar_slide_conteudo(titulo, conteudo, layout="single"):
    """Cria slide de conteúdo"""
    if layout == "single":
        html_content = f"""
        <div style="padding: 20px;">
            <h2 style="color: #2c3e50; border-bottom: 3px solid #3498db; padding-bottom: 10px; margin-bottom: 30px;">{titulo}</h2>
            <div style="font-size: 18px; line-height: 1.6;">
                {conteudo}
            </div>
        </div>
        """
    elif layout == "two-column":
        conteudo_esq, conteudo_dir = conteudo
        html_content = f"""
        <div style="padding: 20px;">
            <h2 style="color: #2c3e50; border-bottom: 3px solid #3498db; padding-bottom: 10px; margin-bottom: 30px;">{titulo}</h2>
            <div style="display: flex; gap: 40px;">
                <div style="flex: 1; font-size: 16px; line-height: 1.6;">
                    {conteudo_esq}
                </div>
                <div style="flex: 1; font-size: 16px; line-height: 1.6;">
                    {conteudo_dir}
                </div>
            </div>
        </div>
        """
   \end{lstlisting}
\end{pythonbox}

\begin{pythonbox}
\begin{lstlisting}[language=Python]           
    return HTML(html_content)

# Exemplo de slide de título
display(criar_slide_titulo(
    "Análise de Eficácia Educacional",
    "Resultados do Estudo Experimental",
    "Dr. João Silva - Universidade de Pesquisa"
))
\end{lstlisting}
\end{pythonbox}

\subsection{Slides Interativos com Dados}

\begin{pythonbox}
\begin{lstlisting}[language=Python]
# Slide com objetivos
conteudo_objetivos = """
<h3>Objetivos do Estudo:</h3>
<ul style="font-size: 20px; line-height: 2;">
    <li>Avaliar a eficácia de uma nova metodologia de ensino</li>
    <li>Comparar resultados entre grupo controle e experimental</li>
    <li>Analisar fatores que influenciam o aprendizado</li>
    <li>Fornecer recomendações baseadas em evidências</li>
</ul>

<h3 style="margin-top: 40px;">Hipóteses:</h3>
<ul style="font-size: 20px; line-height: 2;">
    <li><strong>H₁:</strong> O grupo experimental apresentará maior ganho de aprendizado</li>
    <li><strong>H₀:</strong> Não há diferença significativa entre os grupos</li>
</ul>
"""

display(criar_slide_conteudo("Objetivos e Hipóteses", conteudo_objetivos))
\end{lstlisting}
\end{pythonbox}

\begin{pythonbox}
\begin{lstlisting}[language=Python]
# Slide com metodologia
conteudo_metodologia = f"""
<h3>Desenho do Estudo:</h3>
<p style="font-size: 18px;">Experimento controlado randomizado com pré e pós-teste</p>

<h3>Participantes:</h3>
<ul style="font-size: 18px; line-height: 1.8;">
    <li><strong>N =</strong> {len(dados_estudo)} participantes</li>
    <li><strong>Grupo Controle:</strong> {len(dados_estudo[dados_estudo['grupo'] == 'Controle'])} participantes</li>
    <li><strong>Grupo Experimental:</strong> {len(dados_estudo[dados_estudo['grupo'] == 'Experimental'])} participantes</li>
    <li><strong>Idade média:</strong> {dados_estudo['idade'].mean():.1f} anos (DP = {dados_estudo['idade'].std():.1f})</li>
</ul>

<h3>Instrumentos:</h3>
<ul style="font-size: 18px; line-height: 1.8;">
    <li>Teste de conhecimento (0-100 pontos)</li>
    <li>Escala de satisfação (1-10 pontos)</li>
    <li>Questionário sociodemográfico</li>
</ul>
"""

display(criar_slide_conteudo("Metodologia", conteudo_metodologia))
\end{lstlisting}
\end{pythonbox}

\begin{pythonbox}
\begin{lstlisting}[language=Python]
# Slide com gráfico interativo
def criar_slide_com_grafico(titulo, dados, tipo_grafico="bar"):
    """Cria slide com gráfico incorporado"""
    
    if tipo_grafico == "bar":
        # Gráfico de barras comparativo
        stats_grupos = dados.groupby('grupo').agg({
            'pre_teste': ['mean', 'std'],
            'pos_teste': ['mean', 'std'],
            'ganho_aprendizado': ['mean', 'std']
        }).round(2)
        
        fig = go.Figure()
        
        grupos = stats_grupos.index
        fig.add_trace(go.Bar(
            name='Pré-teste',
            x=grupos,
            y=stats_grupos['pre_teste']['mean'],
            error_y=dict(type='data', array=stats_grupos['pre_teste']['std']),
            marker_color='lightblue'
        ))
        
        fig.add_trace(go.Bar(
            name='Pós-teste',
            x=grupos,
            y=stats_grupos['pos_teste']['mean'],
            error_y=dict(type='data', array=stats_grupos['pos_teste']['std']),
            marker_color='darkblue'
        ))
        
        fig.update_layout(
            title=f'{titulo}',
            xaxis_title='Grupo',
            yaxis_title='Pontuação',
            barmode='group',
            font=dict(size=16),
            height=500,
            template='plotly_white'
        )
           \end{lstlisting}
\end{pythonbox}

\begin{pythonbox}
\begin{lstlisting}[language=Python]       
    elif tipo_grafico == "scatter":
        # Gráfico de dispersão
        fig = px.scatter(
            dados, 
            x='pre_teste', 
            y='pos_teste',
            color='grupo',
            size='satisfacao',
            title=titulo,
            labels={
                'pre_teste': 'Pré-teste',
                'pos_teste': 'Pós-teste',
                'grupo': 'Grupo'
            }
        )
        fig.update_layout(font=dict(size=16), height=500)
    
    # Exibindo título e gráfico
    display(HTML(f'<h2 style="color: #2c3e50; text-align: center; margin: 20px 0;">{titulo}</h2>'))
    fig.show()

# Criando slides com gráficos
criar_slide_com_grafico("Comparação Pré vs Pós-teste", dados_estudo, "bar")
\end{lstlisting}
\end{pythonbox}

\begin{pythonbox}
\begin{lstlisting}[language=Python]
# Slide com resultados estatísticos
def criar_slide_estatisticas(dados):
    """Cria slide com resultados estatísticos"""
    from scipy import stats
    
    # Realizando testes
    controle = dados[dados['grupo'] == 'Controle']
    experimental = dados[dados['grupo'] == 'Experimental']
    
    # Teste t
    t_stat, p_valor = stats.ttest_ind(experimental['pos_teste'], controle['pos_teste'])
    
    # Cohen's d
    pooled_std = np.sqrt(((len(controle)-1)*controle['pos_teste'].std()**2 + 
                         (len(experimental)-1)*experimental['pos_teste'].std()**2) / 
                        (len(controle)+len(experimental)-2))
    cohens_d = (experimental['pos_teste'].mean() - controle['pos_teste'].mean()) / pooled_std
    
    # Interpretação do tamanho do efeito
    if abs(cohens_d) < 0.2:
        interpretacao_d = "Pequeno"
    elif abs(cohens_d) < 0.5:
        interpretacao_d = "Médio"
    elif abs(cohens_d) < 0.8:
        interpretacao_d = "Grande"
    else:
        interpretacao_d = "Muito Grande"
    
    conteudo_stats = f"""
    <div style="text-align: center;">
        <h3>Teste t para Amostras Independentes</h3>
        <div style="background-color: #ecf0f1; padding: 30px; border-radius: 10px; margin: 20px 0;">
            <p style="font-size: 24px; margin: 10px 0;"><strong>t({len(dados)-2}) = {t_stat:.3f}</strong></p>
            <p style="font-size: 24px; margin: 10px 0; color: {'#e74c3c' if p_valor < 0.05 else '#95a5a6'};"><strong>p = {p_valor:.4f}</strong></p>
            <p style="font-size: 20px; margin: 10px 0;"><strong>Cohen's d = {cohens_d:.3f}</strong> <em>({interpretacao_d})</em></p>
        </div>
        
        <h3 style="margin-top: 40px;">Conclusão:</h3>
        <p style="font-size: 20px; color: {'#27ae60' if p_valor < 0.05 else '#e74c3c'};">
            {'✓ Diferença estatisticamente significativa' if p_valor < 0.05 else '✗ Diferença não significativa'}
        </p>
           \end{lstlisting}
\end{pythonbox}

\begin{pythonbox}
\begin{lstlisting}[language=Python]       
        <div style="margin-top: 30px; padding: 20px; background-color: #d5dbdb; border-radius: 5px;">
            <h4>Ganho Médio de Aprendizado:</h4>
            <p style="font-size: 18px;">
                <strong>Controle:</strong> {controle['ganho_aprendizado'].mean():.2f} ± {controle['ganho_aprendizado'].std():.2f}<br>
                <strong>Experimental:</strong> {experimental['ganho_aprendizado'].mean():.2f} ± {experimental['ganho_aprendizado'].std():.2f}
            </p>
        </div>
    </div>
    """
    
    return HTML(f"""
    <div style="padding: 20px;">
        <h2 style="color: #2c3e50; border-bottom: 3px solid #3498db; padding-bottom: 10px; margin-bottom: 30px; text-align: center;">Resultados Estatísticos</h2>
        {conteudo_stats}
    </div>
    """)

display(criar_slide_estatisticas(dados_estudo))
\end{lstlisting}
\end{pythonbox}

\subsection{Slides de Conclusão e Recomendações}

\begin{pythonbox}
\begin{lstlisting}[language=Python]
# Slide de conclusões
def criar_slide_conclusoes(dados):
    """Cria slide com conclusões do estudo"""
    
    # Calculando estatísticas finais
    experimental = dados[dados['grupo'] == 'Experimental']
    controle = dados[dados['grupo'] == 'Controle']
    
    ganho_experimental = experimental['ganho_aprendizado'].mean()
    ganho_controle = controle['ganho_aprendizado'].mean()
    diferenca_percentual = ((ganho_experimental - ganho_controle) / ganho_controle) * 100
    
    conteudo_conclusoes = f"""
    <h3>Principais Achados:</h3>
    <div style="background-color: #e8f5e8; padding: 20px; border-left: 5px solid #27ae60; margin: 20px 0;">
        <ul style="font-size: 18px; line-height: 2;">
            <li>O grupo experimental apresentou <strong>{diferenca_percentual:.1f}% mais ganho</strong> de aprendizado</li>
            <li>Diferença estatisticamente significativa entre os grupos (p < 0.05)</li>
            <li>Tamanho do efeito considerado <strong>médio a grande</strong></li>
            <li>Alta satisfação no grupo experimental</li>
        </ul>
    </div>
    
    <h3>Implicações Práticas:</h3>
    <ul style="font-size: 18px; line-height: 2;">
        <li>A nova metodologia é <strong>eficaz</strong> para melhorar o aprendizado</li>
        <li>Implementação recomendada em maior escala</li>
        <li>Benefícios observados em diferentes perfis de estudantes</li>
    </ul>
    
    <h3>Limitações:</h3>
    <ul style="font-size: 16px; line-height: 1.8; color: #7f8c8d;">
        <li>Estudo de curta duração</li>
        <li>Amostra específica de uma instituição</li>
        <li>Necessidade de replicação em outros contextos</li>
    </ul>
    """
       \end{lstlisting}
\end{pythonbox}

\begin{pythonbox}
\begin{lstlisting}[language=Python]       
    return HTML(f"""
    <div style="padding: 20px;">
        <h2 style="color: #2c3e50; border-bottom: 3px solid #3498db; padding-bottom: 10px; margin-bottom: 30px;">Conclusões</h2>
        {conteudo_conclusoes}
    </div>
    """)

display(criar_slide_conclusoes(dados_estudo))
\end{lstlisting}
\end{pythonbox}

\begin{pythonbox}
\begin{lstlisting}[language=Python]
# Slide de recomendações
conteudo_recomendacoes = """
<h3>�� Recomendações para Implementação:</h3>
<div style="display: grid; grid-template-columns: 1fr 1fr; gap: 30px; margin: 30px 0;">
    <div style="background-color: #fff3cd; padding: 20px; border-radius: 8px; border-left: 4px solid #ffc107;">
        <h4 style="color: #856404;">Curto Prazo (3-6 meses)</h4>
        <ul style="line-height: 1.8;">
            <li>Treinar educadores na nova metodologia</li>
            <li>Implementar em turmas piloto</li>
            <li>Monitorar resultados iniciais</li>
        </ul>
    </div>
    
    <div style="background-color: #d4edda; padding: 20px; border-radius: 8px; border-left: 4px solid #28a745;">
        <h4 style="color: #155724;">Longo Prazo (6-12 meses)</h4>
        <ul style="line-height: 1.8;">
            <li>Expansão para toda a instituição</li>
            <li>Desenvolvimento de materiais</li>
            <li>Avaliação de impacto em larga escala</li>
        </ul>
    </div>
</div>

<h3>Próximos Estudos:</h3>
<ul style="font-size: 18px; line-height: 2;">
    <li>Estudo longitudinal de 12 meses</li>
    <li>Análise de custo-benefício</li>
    <li>Investigação de fatores moderadores</li>
    <li>Replicação em diferentes contextos</li>
</ul>

<div style="text-align: center; margin-top: 40px; padding: 20px; background-color: #f8f9fa; border-radius: 10px;">
    <h3 style="color: #495057;">Obrigado!</h3>
    <p style="font-size: 18px; color: #6c757d;">Perguntas e Discussão</p>
</div>
"""

display(criar_slide_conteudo("Recomendações e Próximos Passos", conteudo_recomendacoes))
\end{lstlisting}
\end{pythonbox}

% =============================================================================
\section{Geração de Gráficos para Publicações}
% =============================================================================

\subsection{Padrões de Qualidade para Publicação}

Para publicações científicas, os gráficos devem seguir padrões específicos de qualidade, resolução e formatação.

\begin{pythonbox}
\begin{lstlisting}[language=Python]
# Configurações para gráficos de publicação
import matplotlib.pyplot as plt
import seaborn as sns
from matplotlib import rcParams

# Configurações para publicação científica
def config_publicacao():
    """Configura matplotlib para gráficos de publicação"""
    rcParams['figure.figsize'] = (8, 6)
    rcParams['figure.dpi'] = 300
    rcParams['savefig.dpi'] = 300
    rcParams['font.size'] = 12
    rcParams['axes.labelsize'] = 14
    rcParams['axes.titlesize'] = 16
    rcParams['xtick.labelsize'] = 12
    rcParams['ytick.labelsize'] = 12
    rcParams['legend.fontsize'] = 12
    rcParams['font.family'] = 'serif'
    rcParams['font.serif'] = ['Times New Roman', 'Times', 'serif']
    rcParams['text.usetex'] = False  # Definir como True se LaTeX estiver disponível
    rcParams['axes.linewidth'] = 1.5
    rcParams['grid.linewidth'] = 0.5
    rcParams['lines.linewidth'] = 2
    rcParams['patch.linewidth'] = 0.5
    rcParams['xtick.major.width'] = 1.5
    rcParams['ytick.major.width'] = 1.5
    rcParams['xtick.minor.width'] = 1
    rcParams['ytick.minor.width'] = 1

# Aplicando configurações
config_publicacao()

# Paleta de cores para publicação (colorblind-friendly)
cores_publicacao = ['#1f77b4', '#ff7f0e', '#2ca02c', '#d62728', '#9467bd', '#8c564b']
sns.set_palette(cores_publicacao)
\end{lstlisting}
\end{pythonbox}

\begin{pythonbox}
\begin{lstlisting}[language=Python]
# Função para criar gráficos de publicação
def criar_grafico_publicacao(dados, tipo='barras', titulo='', 
                            xlabel='', ylabel='', filename=None):
    """
    Cria gráficos adequados para publicação científica
    """
    fig, ax = plt.subplots(figsize=(8, 6))
    
    if tipo == 'barras':
        # Gráfico de barras com erro padrão
        stats = dados.groupby('grupo').agg({
            'pre_teste': ['mean', 'sem'],
            'pos_teste': ['mean', 'sem']
        })
        
        x = np.arange(len(stats.index))
        width = 0.35
        
        bars1 = ax.bar(x - width/2, stats['pre_teste']['mean'], width, 
                      yerr=stats['pre_teste']['sem'], 
                      label='Pré-teste', capsize=5, alpha=0.8)
        bars2 = ax.bar(x + width/2, stats['pos_teste']['mean'], width,
                      yerr=stats['pos_teste']['sem'],
                      label='Pós-teste', capsize=5, alpha=0.8)
        
        ax.set_xlabel(xlabel if xlabel else 'Grupo')
        ax.set_ylabel(ylabel if ylabel else 'Pontuação')
        ax.set_xticks(x)
        ax.set_xticklabels(stats.index)
        ax.legend()
        
    elif tipo == 'boxplot':
        # Box plot com pontos individuais
        box_data = [dados[dados['grupo'] == g]['ganho_aprendizado'] for g in dados['grupo'].unique()]
        box_labels = dados['grupo'].unique()
        
        bp = ax.boxplot(box_data, labels=box_labels, patch_artist=True, showmeans=True)
        
        # Personalizando cores
        for i, patch in enumerate(bp['boxes']):
            patch.set_facecolor(cores_publicacao[i])
            patch.set_alpha(0.7)
        
        ax.set_xlabel(xlabel if xlabel else 'Grupo')
        ax.set_ylabel(ylabel if ylabel else 'Ganho de Aprendizado')
           \end{lstlisting}
\end{pythonbox}

\begin{pythonbox}
\begin{lstlisting}[language=Python]       
    elif tipo == 'scatter':
        # Scatter plot com linha de regressão
        for i, grupo in enumerate(dados['grupo'].unique()):
            subset = dados[dados['grupo'] == grupo]
            ax.scatter(subset['pre_teste'], subset['pos_teste'], 
                      c=cores_publicacao[i], label=grupo, alpha=0.7, s=50)
        
        # Adicionando linha de identidade
        lims = [
            np.min([ax.get_xlim(), ax.get_ylim()]),
            np.max([ax.get_xlim(), ax.get_ylim()]),
        ]
        ax.plot(lims, lims, 'k--', alpha=0.5, zorder=0, label='y = x')
        
        ax.set_xlabel(xlabel if xlabel else 'Pré-teste')
        ax.set_ylabel(ylabel if ylabel else 'Pós-teste')
        ax.legend()
    
    # Formatação final
    ax.set_title(titulo, pad=20)
    ax.grid(True, alpha=0.3)
    ax.spines['top'].set_visible(False)
    ax.spines['right'].set_visible(False)
    
    plt.tight_layout()
    
    if filename:
        plt.savefig(filename, dpi=300, bbox_inches='tight', 
                   facecolor='white', edgecolor='none')
        print(f"Gráfico salvo como: {filename}")
    
    plt.show()
    return fig, ax

# Exemplos de gráficos para publicação
criar_grafico_publicacao(
    dados_estudo, 
    tipo='barras',
    titulo='Comparação entre Grupos no Pré e Pós-teste',
    ylabel='Pontuação (0-100)',
    filename='figura1_comparacao_grupos.png'
)
\end{lstlisting}
\end{pythonbox}

\begin{pythonbox}
\begin{lstlisting}[language=Python]
# Gráfico de ganhos de aprendizado
criar_grafico_publicacao(
    dados_estudo,
    tipo='boxplot', 
    titulo='Distribuição dos Ganhos de Aprendizado por Grupo',
    ylabel='Ganho (Pós-teste - Pré-teste)',
    filename='figura2_ganhos_aprendizado.png'
)
\end{lstlisting}
\end{pythonbox}

\begin{pythonbox}
\begin{lstlisting}[language=Python]
# Scatter plot pré vs pós
criar_grafico_publicacao(
    dados_estudo,
    tipo='scatter',
    titulo='Relação entre Pré-teste e Pós-teste por Grupo', 
    xlabel='Pontuação Pré-teste',
    ylabel='Pontuação Pós-teste',
    filename='figura3_correlacao_pre_pos.png'
)
\end{lstlisting}
\end{pythonbox}

\subsection{Gráficos Multivariados Avançados}

\begin{pythonbox}
\begin{lstlisting}[language=Python]
# Figura composta para publicação
def criar_figura_multipla(dados, filename=None):
    """Cria figura com múltiplos painéis para publicação"""
    
    fig, axes = plt.subplots(2, 2, figsize=(12, 10))
    
    # Painel A: Barras comparativas
    stats = dados.groupby('grupo').agg({
        'pre_teste': ['mean', 'sem'],
        'pos_teste': ['mean', 'sem']
    })
    
    x = np.arange(len(stats.index))
    width = 0.35
    
    axes[0, 0].bar(x - width/2, stats['pre_teste']['mean'], width, 
                   yerr=stats['pre_teste']['sem'], 
                   label='Pré-teste', capsize=5, alpha=0.8)
    axes[0, 0].bar(x + width/2, stats['pos_teste']['mean'], width,
                   yerr=stats['pos_teste']['sem'],
                   label='Pós-teste', capsize=5, alpha=0.8)
    
    axes[0, 0].set_title('A) Comparação Pré vs Pós-teste', fontweight='bold', loc='left')
    axes[0, 0].set_ylabel('Pontuação')
    axes[0, 0].set_xticks(x)
    axes[0, 0].set_xticklabels(stats.index)
    axes[0, 0].legend()
    axes[0, 0].grid(True, alpha=0.3)
    
    # Painel B: Box plot dos ganhos
    box_data = [dados[dados['grupo'] == g]['ganho_aprendizado'] for g in dados['grupo'].unique()]
    bp = axes[0, 1].boxplot(box_data, labels=dados['grupo'].unique(), patch_artist=True)
    
    for i, patch in enumerate(bp['boxes']):
        patch.set_facecolor(cores_publicacao[i])
        patch.set_alpha(0.7)
    
    axes[0, 1].set_title('B) Distribuição dos Ganhos', fontweight='bold', loc='left')
    axes[0, 1].set_ylabel('Ganho de Aprendizado')
    axes[0, 1].grid(True, alpha=0.3)
       \end{lstlisting}
\end{pythonbox}

\begin{pythonbox}
\begin{lstlisting}[language=Python]       
    # Painel C: Correlação satisfação vs ganho
    for i, grupo in enumerate(dados['grupo'].unique()):
        subset = dados[dados['grupo'] == grupo]
        axes[1, 0].scatter(subset['satisfacao'], subset['ganho_aprendizado'],
                          c=cores_publicacao[i], label=grupo, alpha=0.7)
    
    axes[1, 0].set_title('C) Satisfação vs Ganho', fontweight='bold', loc='left')
    axes[1, 0].set_xlabel('Satisfação')
    axes[1, 0].set_ylabel('Ganho de Aprendizado')
    axes[1, 0].legend()
    axes[1, 0].grid(True, alpha=0.3)
    
    # Painel D: Histograma idade por grupo
    for i, grupo in enumerate(dados['grupo'].unique()):
        subset = dados[dados['grupo'] == grupo]
        axes[1, 1].hist(subset['idade'], alpha=0.7, label=grupo, 
                       bins=15, color=cores_publicacao[i])
    
    axes[1, 1].set_title('D) Distribuição de Idade', fontweight='bold', loc='left')
    axes[1, 1].set_xlabel('Idade')
    axes[1, 1].set_ylabel('Frequência')
    axes[1, 1].legend()
    axes[1, 1].grid(True, alpha=0.3)
    
    # Removendo spines superiores e direitas
    for ax in axes.flat:
        ax.spines['top'].set_visible(False)
        ax.spines['right'].set_visible(False)
    
    plt.tight_layout()
    
    if filename:
        plt.savefig(filename, dpi=300, bbox_inches='tight', 
                   facecolor='white', edgecolor='none')
        Continuando o LaTeX do Capítulo 11:
        print(f"Figura múltipla salva como: {filename}")
    
    return fig

# Gerando figura múltipla
figura_multipla = criar_figura_multipla(dados_estudo, 'figura4_analise_completa.png')
\end{lstlisting}
\end{pythonbox}

\subsection{Gráficos com Anotações Estatísticas}

\begin{pythonbox}
\begin{lstlisting}[language=Python]
def criar_grafico_com_anotacoes(dados):
    """Cria gráfico com anotações estatísticas para publicação"""
    from scipy import stats as scipy_stats
    
    fig, ax = plt.subplots(figsize=(10, 8))
    
    # Preparando dados
    controle = dados[dados['grupo'] == 'Controle']['pos_teste']
    experimental = dados[dados['grupo'] == 'Experimental']['pos_teste']
    
    # Violin plot
    positions = [1, 2]
    vp = ax.violinplot([controle, experimental], positions=positions, 
                       widths=0.5, showmeans=True, showmedians=True)
    
    # Personalizando cores
    for i, pc in enumerate(vp['bodies']):
        pc.set_facecolor(cores_publicacao[i])
        pc.set_alpha(0.7)
    
    # Teste estatístico
    t_stat, p_valor = scipy_stats.ttest_ind(controle, experimental)
    
    # Adicionando anotação de significância
    if p_valor < 0.001:
        sig_text = '***'
    elif p_valor < 0.01:
        sig_text = '**'
    elif p_valor < 0.05:
        sig_text = '*'
    else:
        sig_text = 'ns'
    
    # Linha conectando grupos
    y_max = max(controle.max(), experimental.max())
    y_line = y_max + 5
    ax.plot([1, 2], [y_line, y_line], 'k-', linewidth=1)
    ax.text(1.5, y_line + 1, sig_text, ha='center', fontsize=14)
    ax.text(1.5, y_line + 4, f'p = {p_valor:.4f}', ha='center', fontsize=10)
    
    # Adicionando médias como texto
    ax.text(1, controle.mean() - 5, f'M = {controle.mean():.1f}', 
            ha='center', fontsize=10, fontweight='bold')
    ax.text(2, experimental.mean() - 5, f'M = {experimental.mean():.1f}', 
            ha='center', fontsize=10, fontweight='bold')
       \end{lstlisting}
\end{pythonbox}

\begin{pythonbox}
\begin{lstlisting}[language=Python]       
    # Formatação
    ax.set_xticks(positions)
    ax.set_xticklabels(['Controle', 'Experimental'])
    ax.set_ylabel('Pontuação Pós-teste')
    ax.set_title('Comparação de Desempenho entre Grupos com Análise Estatística')
    ax.grid(True, alpha=0.3, axis='y')
    ax.spines['top'].set_visible(False)
    ax.spines['right'].set_visible(False)
    
    plt.tight_layout()
    plt.savefig('figura5_violin_anotado.png', dpi=300, bbox_inches='tight')
    plt.show()
    
    return fig

# Criando gráfico com anotações
grafico_anotado = criar_grafico_com_anotacoes(dados_estudo)
\end{lstlisting}
\end{pythonbox}

% =============================================================================
\section{Templates de Documentação}
% =============================================================================

\subsection{Sistema de Templates para Documentação Científica}

\begin{pythonbox}
\begin{lstlisting}[language=Python]
class TemplatesCientificos:
    """Sistema de templates para documentação científica"""
    
    def __init__(self):
        self.templates = {}
        self.carregar_templates()
    
    def carregar_templates(self):
        """Carrega templates predefinidos"""
        
        # Template de artigo científico
        self.templates['artigo'] = """
# {{ titulo }}

**{{ autores }}**  
*{{ afiliacao }}*

## Resumo
{{ resumo }}

**Palavras-chave:** {{ palavras_chave }}

## 1. Introdução
{{ introducao }}

## 2. Metodologia

### 2.1 Participantes
{{ participantes }}

### 2.2 Procedimentos
{{ procedimentos }}

### 2.3 Análise de Dados
{{ analise_dados }}

## 3. Resultados
{{ resultados }}

## 4. Discussão
{{ discussao }}
   \end{lstlisting}
\end{pythonbox}

\begin{pythonbox}
\begin{lstlisting}[language=Python]       
## 5. Conclusão
{{ conclusao }}

## Referências
{{ referencias }}
"""
        
        # Template de protocolo de pesquisa
        self.templates['protocolo'] = """
# Protocolo de Pesquisa: {{ titulo }}

## Informações Gerais
- **Pesquisador Principal:** {{ pesquisador_principal }}
- **Instituição:** {{ instituicao }}
- **Data de Início:** {{ data_inicio }}
- **Duração Prevista:** {{ duracao }}

## Objetivos
### Objetivo Geral
{{ objetivo_geral }}

### Objetivos Específicos
{{ objetivos_especificos }}

## Hipóteses
{{ hipoteses }}

## Metodologia
### Desenho do Estudo
{{ desenho_estudo }}

### Critérios de Inclusão
{{ criterios_inclusao }}

### Critérios de Exclusão
{{ criterios_exclusao }}

### Tamanho da Amostra
{{ tamanho_amostra }}

## Considerações Éticas
{{ consideracoes_eticas }}

## Cronograma
{{ cronograma }}
   \end{lstlisting}
\end{pythonbox}

\begin{pythonbox}
\begin{lstlisting}[language=Python]       
## Orçamento
{{ orcamento }}
"""
        
        # Template de relatório técnico
        self.templates['relatorio_tecnico'] = """
<!DOCTYPE html>
<html>
<head>
    <title>{{ titulo }}</title>
    <style>
        body { font-family: 'Arial', sans-serif; margin: 40px; }
        .header { background-color: #2c3e50; color: white; padding: 20px; }
        .section { margin: 30px 0; }
        .table { width: 100%; border-collapse: collapse; }
        .table th, .table td { border: 1px solid #ddd; padding: 8px; }
        .table th { background-color: #f2f2f2; }
        .highlight { background-color: #ffffcc; padding: 10px; }
    </style>
</head>
<body>
    <div class="header">
        <h1>{{ titulo }}</h1>
        <p>{{ data }} | {{ autor }}</p>
    </div>
    
    <div class="section">
        <h2>Sumário Executivo</h2>
        <div class="highlight">{{ sumario_executivo }}</div>
    </div>
    
    <div class="section">
        <h2>Metodologia</h2>
        {{ metodologia }}
    </div>
    
    <div class="section">
        <h2>Resultados</h2>
        {{ resultados }}
    </div>
    
    <div class="section">
        <h2>Recomendações</h2>
        {{ recomendacoes }}
    </div>
       \end{lstlisting}
\end{pythonbox}

\begin{pythonbox}
\begin{lstlisting}[language=Python]       
    <div class="section">
        <h2>Anexos</h2>
        {{ anexos }}
    </div>
</body>
</html>
"""
    
    def gerar_documento(self, tipo_template, dados):
        """Gera documento a partir de template"""
        if tipo_template not in self.templates:
            raise ValueError(f"Template '{tipo_template}' não encontrado")
        
        template = jinja2.Template(self.templates[tipo_template])
        return template.render(**dados)
    
    def salvar_documento(self, conteudo, filename, formato='md'):
        """Salva documento gerado"""
        with open(filename, 'w', encoding='utf-8') as f:
            f.write(conteudo)
        print(f"Documento salvo: {filename}")
        return filename

# Exemplo de uso
templates = TemplatesCientificos()

# Dados para artigo
dados_artigo = {
    'titulo': 'Eficácia de Nova Metodologia de Ensino: Um Estudo Experimental',
    'autores': 'Silva, J.; Santos, M.; Oliveira, P.',
    'afiliacao': 'Universidade de Pesquisa, Departamento de Educação',
    'resumo': 'Este estudo investigou a eficácia de uma nova metodologia de ensino...',
    'palavras_chave': 'educação, aprendizado, metodologia, experimento',
    'introducao': 'O desenvolvimento de metodologias eficazes de ensino...',
    'participantes': 'Participaram do estudo 200 estudantes universitários...',
    'procedimentos': 'Os participantes foram randomicamente alocados...',
    'analise_dados': 'As análises foram conduzidas usando Python 3.9...',
    'resultados': 'Os resultados indicaram diferença significativa...',
    'discussao': 'Os achados sugerem que a nova metodologia...',
    'conclusao': 'Este estudo fornece evidências preliminares...',
    'referencias': '1. Smith et al. (2023)...'
}
   \end{lstlisting}
\end{pythonbox}

\begin{pythonbox}
\begin{lstlisting}[language=Python]       
# Gerando artigo
artigo = templates.gerar_documento('artigo', dados_artigo)
templates.salvar_documento(artigo, 'artigo_cientifico.md')
\end{lstlisting}
\end{pythonbox}

\subsection{Gerador de Documentação de Código}

\begin{pythonbox}
\begin{lstlisting}[language=Python]
class GeradorDocumentacao:
    """Gera documentação automática para projetos científicos"""
    
    def __init__(self, nome_projeto, versao="1.0.0"):
        self.nome_projeto = nome_projeto
        self.versao = versao
        self.secoes = []
        
    def adicionar_secao(self, titulo, conteudo):
        """Adiciona seção à documentação"""
        self.secoes.append({'titulo': titulo, 'conteudo': conteudo})
    
    def documentar_funcao(self, funcao):
        """Extrai documentação de uma função"""
        import inspect
        
        doc = {
            'nome': funcao.__name__,
            'docstring': inspect.getdoc(funcao) or 'Sem documentação',
            'assinatura': str(inspect.signature(funcao)),
            'codigo': inspect.getsource(funcao) if hasattr(funcao, '__code__') else 'N/A'
        }
        return doc
    
    def documentar_dataset(self, df, nome="Dataset"):
        """Documenta estrutura de um dataset"""
        doc = f"""
### {nome}

**Dimensões:** {df.shape[0]} linhas × {df.shape[1]} colunas

**Colunas:**
"""
        for col in df.columns:
            dtype = str(df[col].dtype)
            nulls = df[col].isnull().sum()
            unique = df[col].nunique()
            doc += f"\n- **{col}** ({dtype}): {unique} valores únicos, {nulls} valores nulos"
        
        doc += f"\n\n**Estatísticas Básicas:**\n```\n{df.describe().to_string()}\n```"
        
        return doc
    
    def gerar_readme(self):
        """Gera arquivo README.md"""
        readme = f"""# {self.nome_projeto}
   \end{lstlisting}
\end{pythonbox}

\begin{pythonbox}
\begin{lstlisting}[language=Python]       
**Versão:** {self.versao}  
**Data:** {datetime.now().strftime('%d/%m/%Y')}

## Descrição
Projeto de análise de dados científicos desenvolvido em Python.

## Instalação

```bash
pip install -r requirements.txt
```

## Uso

```python
from analise import GeradorRelatorio

relatorio = GeradorRelatorio(dados)
relatorio.gerar_relatorio_completo()
```

## Estrutura do Projeto

```
{self.nome_projeto}/
│
├── data/              # Dados brutos e processados
├── notebooks/         # Jupyter notebooks
├── src/              # Código fonte
├── reports/          # Relatórios gerados
├── figures/          # Gráficos e visualizações
└── docs/             # Documentação
```

## Dependências

- Python >= 3.8
- pandas >= 1.3.0
- numpy >= 1.21.0
- matplotlib >= 3.4.0
- seaborn >= 0.11.0
- scipy >= 1.7.0

## Contribuidores

- [Nome do Pesquisador]

## Licença
   \end{lstlisting}
\end{pythonbox}

\begin{pythonbox}
\begin{lstlisting}[language=Python]       
Este projeto está sob licença MIT.
"""
        return readme
    
    def gerar_documentacao_completa(self, output_dir='docs'):
        """Gera documentação completa do projeto"""
        import os
        
        # Criar diretório se não existir
        if not os.path.exists(output_dir):
            os.makedirs(output_dir)
        
        # Gerar README
        readme = self.gerar_readme()
        with open(f'{output_dir}/README.md', 'w', encoding='utf-8') as f:
            f.write(readme)
        
        # Gerar documentação das seções
        doc_completa = f"# Documentação Completa - {self.nome_projeto}\n\n"
        
        for secao in self.secoes:
            doc_completa += f"## {secao['titulo']}\n\n{secao['conteudo']}\n\n"
        
        with open(f'{output_dir}/documentacao.md', 'w', encoding='utf-8') as f:
            f.write(doc_completa)
        
        print(f"Documentação gerada em: {output_dir}/")
        return output_dir

# Exemplo de uso
doc = GeradorDocumentacao("Análise de Eficácia Educacional", "1.0.0")

# Documentando dataset
doc.adicionar_secao("Dataset Principal", doc.documentar_dataset(dados_estudo))

# Documentando função
doc.adicionar_secao("Funções de Análise", doc.documentar_funcao(GeradorRelatorio.calcular_estatisticas_descritivas))

# Gerando documentação completa
doc.gerar_documentacao_completa()
\end{lstlisting}
\end{pythonbox}

% =============================================================================
\section{Exercícios Práticos}
% =============================================================================

\begin{exercisebox}
\textbf{Exercício 1: Relatório Automatizado Personalizado}

Crie um sistema que:
1. Leia dados de múltiplas fontes (CSV, Excel, API)
2. Realize análises estatísticas apropriadas
3. Gere relatório em HTML e PDF
4. Envie por e-mail automaticamente
\end{exercisebox}

\begin{exercisebox}
\textbf{Exercício 2: Dashboard Interativo}

Desenvolva um dashboard que:
1. Atualize em tempo real
2. Permita filtros dinâmicos
3. Exporte gráficos em alta resolução
4. Gere relatórios sob demanda
\end{exercisebox}

\begin{exercisebox}
\textbf{Exercício 3: Apresentação Automatizada}

Implemente um sistema que:
1. Analise resultados de pesquisa
2. Crie slides automaticamente
3. Inclua gráficos interativos
4. Exporte para PowerPoint
\end{exercisebox}

% =============================================================================
\section{Conclusão}
% =============================================================================

A comunicação científica efetiva com Python transforma dados complexos em insights acessíveis. As ferramentas apresentadas neste capítulo permitem automatizar a geração de relatórios, criar apresentações dinâmicas e produzir visualizações de qualidade para publicação. A automação não apenas economiza tempo, mas também garante consistência e reprodutibilidade na comunicação de resultados científicos.

\vspace{1cm}
\begin{center}
\rule{0.8\textwidth}{0.4pt}
\end{center}


\end{document}
